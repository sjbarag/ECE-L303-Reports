\subsection{Rectifier}
The first circuit built was a half-wave rectifier, utilizing a 1N4006 PN
junction diode as the primary element because of its directional current flow
property.  When forward biased (acheived when the voltage on the p-doped side
is higher than the voltage on the n-doped side), a standard diode will allow
current to flow; a reverse biased, however, no current will flow.  If said
diode is connected in series with a load resistor between the two inputs of an
AC signal as shown in Figure~\ref{fig:schem1}, a half-wave rectifier results.
%
\begin{figure}[H]
	\centering
	\begin{circuitikz}
	\draw (0,0) to [vsourcesin, l=\SI{12.2}{\volt} (RMS) @ \SI{60}{\hertz}] ++(0,2)
	to [short] ++(1,0)
	to [Do, v^=$V_d \approx$ \SI{0.7}{\volt}, i_=$i$, l_=1N4006] ++(2, 0)
	to [short] ++(1,0)
	to [R, v^=$V_o$, l_=\SI{1}{\kilo\ohm}] ++(0, -2)
	to [short] (0, 0);
\end{circuitikz}

	\caption{Half-wave rectifier circuit, as originally suggested by the lab one instructions.}
	\label{fig:schem1}
\end{figure}
%
During phases where the input signal is positive, the voltage across the
resistor will be roughly equal to the input.  When the signal becomes negative
Ohm's Law states that the voltage across the resistor drops to zero, as no
current flows.  The above circuit uses an input signal of \SI{12.2}{\volt}RMS
and outputs a half-rectified signal with an average value of \SI{5.6}{\volt}DC.

\subsection{Logic Indicator}
Circuit number two allowed students to experiment with light-emitting diodes
(LEDs).  LEDs behave nearly identically to regular diodes, in that current can
only flow through them in the forward biased configuration.  As their name
suggests, LEDs also emit light (at a wavelength determined by the chemicals
used during construction) while current flows through them.  A functioning
circuit to light an LED is shown in Figure~\ref{fig:schem2}.
%
\begin{figure}[H]
	\centering
	\begin{circuitikz}
	\draw (0,0) to [V, l=$0 \rightarrow$ \SI{5}{\volt}DC] ++(0,2)
	to [short] ++(1,0)
	to [R, i_=$i_d$, l_=\SI{300}{\ohm}] ++(2, 0)
	to [short] ++(1,0)
	to [led, l^=Red LED] ++(0, -2)
	to [short] (0, 0);
\end{circuitikz}

	\caption{Circuit to light an LED, as provided by the instructions.}
	\label{fig:schem2}
\end{figure}
%
In the above circuit, the input voltage is stepped from \SI{0}{\volt}DC to
\SI{5}{\volt}DC in increments of one Volt.  This allows students to see the
effects of input voltage to an LEDs brightness.  The provided LED ``turns on''
(emits light) when roughly \SI{15}{\milli\ampere} flows through the element, so
input voltage and resistor pairs that provide at least this much current result
in the diode turning on.  As the input voltage increases to provide more
current, the brightness of the diode should increase.


\subsection{Voltage Regulator}
Students were also tasked with creating a voltage regulator based on a simple
zener diode --- a diode specially designed to be used in a reverse biased
configuration near the voltage at which its semiconductor breaks down.
Figure~\ref{fig:schem3} shows a functioning zener diode voltage regulator ($V_z$).
%
\begin{figure}[H]
	\centering
	\begin{circuitikz}
	\draw (0,0) to [V, l=$5 \rightarrow$ \SI{25}{\volt}DC] ++(0,2)
	to [short] ++(1,0)
	to [R, i^=I, l_=\SI{400}{\ohm}] ++(2, 0)
	to [short] ++(1,0)
	to [open, v^=$V_o$] ++(0,-2);
	\draw (0,0) to [short] (4,0)
	to [zDo, l^=1N4740] ++(0, 2);
\end{circuitikz}

	\caption{Voltage regulataor schematic.  This design was initially presented in the lab instructions.}
	\label{fig:schem3}
\end{figure}
%
For voltages below $V_z$, the diode acts as a standard reverse-biased diode,
allowing only the reverse-saturation current to flow: typically, only a few
microamps.  Because such a small amount of current is flowing, none is dropped
across the resistor, thus the output is equal to the input voltage.  When the
input exceeds $V_z$ current begins to flow, and the voltage drop across the
resistor is equal to the difference between the voltages at the input and
output.  This results in a nearly constant voltage of $V_z$ across the diode,
and thus the output is constant.  The voltage was stepped from \SI{5}{\volt}DC
to \SI{25}{\volt}{DC} in increments of one Volt in this test, while measuring
the voltage across the diode at each value.

\subsection{Constant Current Source}
The fourth and final circuit utilized a JFET to create a constant current
source.  By shorting the gate and source pins on the device as shown in
Figure~\ref{fig:schem4}, the current through the drain pin becomes a constant
based on the manufacturing process of the device.  This is significant because,
regardless of the load, the current will change very little.
%
\begin{figure}[H]
	\centering
	\begin{circuitikz}
	\draw(0,0) node[njfet,yscale=-1] (fet) {};
	\draw(fet.gate) |- (fet.drain);
	\draw(fet.source) to [vR, l_=Decade Box, mirror] ++(0,2)
	to [short] ++(1,0)
	to [ammeter] ++(0,-2) node[] (end) {};
	\draw(fet.drain) to [short] ++(1,0)
	to [V, l_=\SI{16}{\volt}DC \& \SI{32}{\volt}DC] ++(0, 2)
	to [short] ($(end)$);
	\draw (fet.gate) node[anchor=north east] {2N5459};
\end{circuitikz}

	\caption{Schematic provided to create a constant current source using a JFET.}
	\label{fig:schem4}
\end{figure}
%
In the performed experiment, the DC source was set first to \SI{16}{\volt},
while the load resistor was varied from \SI{50}{\ohm} to \SI{2}{\kilo\ohm} in
steps of \SI{100}{\ohm} via a decade box.  The same test was repeated with the
voltage set to \SI{32}{\volt}.  At each resistance value, the current through
the drain was measured with a DC Ammeter.
