\subsection{Design}
Students designed the appropriate circuit during the pre-lab phase of the
experiment.  Given a target corner frequency~$f_c$ of~\SI{363}{\hertz} and
a~\SI{100}{\milli\henry} inductor, students calculated values for the
capacitors~$C_1$ and~$C_2$ and the resistor~$R_0$ from the provided layout
shown in Figure~\ref{f:prelabSchem}.
%
\begin{figure}[H]
	\centering
	\begin{circuitikz}
	\draw
	(0, 0) to [R, l=$R_0$, o-] ++(2.0, 0) coordinate (j1)
	to [L, l=$L:$ $\frac{4 R_0}{3 \omega_c}$] ++(2.0, 0) coordinate (j2)
	to [short, -o] ++(2.0, 0)
	% yeah, it's the golden ratio on the next uncommented line.  What are you
	% gonna do about it?
	to [open, v^=$V_o$, o-o] ++(0, -1.53)

	(0, 0) to [open, v=$V_i$, o-o] ++(0, -1.53)
	to [short, o-o] ++(6, 0)

	(j1) to [C, l_=\small$C_1$: $\frac{1}{2 R_0 \omega_c}$] ++(0, -1.53)
	(j2) to [C, l^=\small$C_2$: $\frac{3}{2 R_0 \omega_c}$] ++(0, -1.53);
\end{circuitikz}

	\parbox{.6\textwidth}{
	\caption[Prelab Circuit Diagram]{Provided circuit schematic for a third-order low pass filter.}
	\label{f:prelabSchem}}
\end{figure}
%
After students completed their computation,~$R_0$ was found to
be~\SI{513}{\ohm}, $C_1$ to be~\SI{142}{\nano\farad}, and~$C_2$ to
be~\SI{427}{\nano\farad}.

\subsection{Simulation}
Students next obtained circuit components that were as close to the required
values as possible.  The measured and required values are tabulated below in
Table~\ref{t:components}.
%
\begin{table}[H]
	\centering
	\begin{tabular}{|c|c|c|}
	\hline
	\tbf{Label} & \tbf{Required Value} & \tbf{Measured Value} \\ \hline
	$R_0$       & \SI{513}{\ohm}       & \SI{502}{\ohm} \\ \hline
	$C_1$       & \SI{142}{\nano\farad}& \SI{147.24}{\nano\farad} \\ \hline
	$C_2$       & \SI{420}{\nano\farad}& \SI{420.2}{\nano\farad}  \\ \hline
	$L$         & \SI{100}{\milli\henry}&\SI{97.6}{\milli\henry}  \\ \hline
	$R_L$       & ---                  & \SI{104.4}{\ohm} \\ \hline
\end{tabular}

	\parbox{.6\textwidth}{
	\caption[Required and Measured Element Values]{Measured and required
	element values for the circuit shown in Figure~\ref{f:prelabSchem}.  The
	element~$R_L$ refers to the intrinsic resistance of the inductor.}
	\label{t:components}}
\end{table}
%
These element values were used to create a PSpice simulation, the schematic for
which is shown in Figure~\ref{f:pspiceSchem}.  Note the presence of the
element~$R_L$, which was used in PSpice to simulate the intrinsic resistance of
the inductor.  This is reflected in the circuit schematic.
%
\begin{figure}[H]
	\centering
	\begin{circuitikz}
	\draw
	(0, 0) to [R, l=$R_0$, o-] ++(2.0, 0) coordinate (j1)
	to [L, l=$L:$ $\frac{4 R_0}{3 \omega_c}$] ++(2.0, 0)
	to [R, l=$R_L$] ++(2, 0) coordinate (j2)
	to [short, -o] ++(2.0, 0)
	% yeah, it's the golden ratio on the next uncommented line.  What are you
	% gonna do about it?
	to [open, v^=$V_o$, o-o] ++(0, -1.53)

	(0, 0) to [open, v=$V_i$, o-o] ++(0, -1.53)
	to [short, o-o] ++(8, 0)

	(j1) to [C, l_=\small$C_1$: $\frac{1}{2 R_0 \omega_c}$] ++(0, -1.53)
	(j2) to [C, l^=\small$C_2$: $\frac{3}{2 R_0 \omega_c}$] ++(0, -1.53);
\end{circuitikz}

	\parbox{.6\textwidth}{
	\caption[PSpice Circuit Diagram]{Circuit diagram used to produce the PSpice
	simulations for the system's pulse response and its frequency response.}
	\label{f:pspiceSchem}}
\end{figure}
%
This circuit was tested with two inputs: a five volt square wave with a
frequency equal to the target corner frequency~(\SI{363}{\hertz}), and a
sinusoidal signal of varying frequency.

\subsection{Measurement}
When the simulation was complete, students first supplied a five-volt square
wave at~\SI{363}{\hertz} to the system while measuring the input and output
waveforms on the oscilloscope.  A~\SI{50}{\ohm} resistor was placed across the
input terminals as shown in Figure~\ref{f:realSchem} in order to improve the
input waveform, and it remained in place for the remaining tests.
%
\begin{figure}[H]
	\centering
	\begin{circuitikz}
	\draw
	(-1.25, 0) to [short, o-] (0, 0)
	to [R, l=$R_0$] ++(2.25, 0) coordinate (j1)
	to [L, l=$L:$ $\frac{4 R_0}{3 \omega_c}$] ++(2.0, 0) coordinate (j2)
	to [short, -o] ++(2.0, 0)
	% yeah, it's the golden ratio on the next uncommented line.  What are you
	% gonna do about it?
	to [open, v^=$V_o$, o-o] ++(0, -1.53)

	(-1.25, 0) to [open, v=$V_i$, o-o] ++(0, -1.53)
	to [short, o-o] ++(7.25, 0)

	(j1) to [C, l_=\small$C_1$: $\frac{1}{2 R_0 \omega_c}$] ++(0, -1.53)
	(j2) to [C, l^=\small$C_2$: $\frac{3}{2 R_0 \omega_c}$] ++(0, -1.53);

	\draw[dashed, color=gray] (-.25, 0) to [R, l_=$R_\text{in}$] ++(0, -1.53);
	% want more dashes?  Uncomment the next line and comment the previous one.
%	\draw[densely dashed, color=gray] (-.25, 0) to [R, l_=$R_\text{in}$] ++(0, -1.53);
\end{circuitikz}

	\label{f:realSchem}
\end{figure}
%
Next, a one-volt sinusoidal wave was placed at the input terminals.  First with
a frequency equal to~$f_c$, then with a frequency of~$10 \times f_c$ and
finally with a frequency of~$\frac{1}{10} \times f_c$.  At each of the input
frequencies, the input and output waveforms were measured and the gain of the
system was calculated.  Finally, the frequency response of the low-pass filter
was measured using the LabView Virtual Instrument~(VI) on the in-lab PC.
