Each of the following circuits was captured using Cadence PSpice.  They are
redrawn here for clarity and ease of understanding, but screenshots of the
PSpice circuits are shown in
the appendix.

\subsection{Rectifier}
The first circuit simulated was the  half-wave rectifier from lab one, which
utilized a~1N4006~PN~junction diode as the primary element because of its
directional current flow property.  When forward biased (achieved when the
voltage on the p-doped side is higher than the voltage on the n-doped side), a
standard diode will allow current to flow; a reverse biased, however, no
current will flow.  If said diode is connected in series with a load resistor
between the two inputs of an AC signal as shown in Figure~\ref{fig:schem1}, a
half-wave rectifier results.
%
\begin{figure}[H]
	\centering
	\begin{circuitikz}
	\draw (0,0) to [vsourcesin, l=\SI{12.2}{\volt} (RMS) @ \SI{60}{\hertz}] ++(0,2)
	to [short] ++(1,0)
	to [Do, v^=$V_d \approx$ \SI{0.7}{\volt}, i_=$i$, l_=1N4006] ++(2, 0)
	to [short] ++(1,0)
	to [R, v^=$V_o$, l_=\SI{1}{\kilo\ohm}] ++(0, -2)
	to [short] (0, 0);
\end{circuitikz}

	\parbox{3.5in}{
	\caption[Schematic --- Half-wave Rectifier]{Half-wave rectifier circuit, as
		originally suggested by the lab one instructions.}
	\label{fig:schem1}}
\end{figure}
%
During phases where the input signal is positive, the voltage across the
resistor will be roughly equal to the input.  When the signal becomes negative
Ohm's Law states that the voltage across the resistor drops to zero, as no
current flows.  The above circuit uses an input signal of~\SI{12.2}{\volt}RMS
and outputs a half-rectified signal with an average value of~\SI{5.6}{\volt}DC.

\subsection{Voltage Regulator}
Students were also tasked with modeling a voltage regulator based on a simple
zener diode --- a diode specially designed to be used in a reverse biased
configuration near the voltage at which its semiconductor breaks down.
Figure~\ref{fig:schem3} shows a functioning zener diode voltage regulator
($V_z$), the design of which is identical to the one built in lab one.
%
\begin{figure}[H]
	\centering
	\begin{circuitikz}
	\draw (0,0) to [V, l=$5 \rightarrow$ \SI{25}{\volt}DC] ++(0,2)
	to [short] ++(1,0)
	to [R, i^=I, l_=\SI{400}{\ohm}] ++(2, 0)
	to [short] ++(1,0)
	to [open, v^=$V_o$] ++(0,-2);
	\draw (0,0) to [short] (4,0)
	to [zDo, l^=1N4740] ++(0, 2);
\end{circuitikz}
\\
	\parbox{3.5in}{
	\caption[Schematic --- Voltage Regulator]{Voltage regulator schematic.
		This design was initially presented in the lab one instructions.}
	\label{fig:schem3}}
\end{figure}
%
For voltages below $V_z$, the diode acts as a standard reverse-biased diode,
allowing only the reverse-saturation current to flow: typically, only a few
microamps.  Because such a small amount of current is flowing, none is dropped
across the resistor, thus the output is equal to the input voltage.  When the
input exceeds $V_z$ current begins to flow, and the voltage drop across the
resistor is equal to the difference between the voltages at the input and
output.  This results in a nearly constant voltage of $V_z$ across the diode,
and thus the output is constant.  The voltage was stepped from~\SI{5}{\volt}DC
to~\SI{25}{\volt}{DC} in increments of one Volt in this simulation, while
recording the voltage across the diode at each value.

\subsection{Constant Current Source}
The third circuit utilized a JFET to create a constant current source, as was
initially required in the instructions for experiment one.  By shorting the
gate and source pins on the device as shown in Figure~\ref{fig:schem4}, the
current through the drain pin becomes a constant based on the manufacturing
process of the device.  This is significant because, regardless of the load,
the current will change very little.
%
\begin{figure}[H]
	\centering
	\begin{circuitikz}
	\draw(0,0) node[njfet,yscale=-1] (fet) {};
	\draw(fet.gate) |- (fet.drain);
	\draw(fet.source) to [vR, l_=Decade Box, mirror] ++(0,2)
	to [short] ++(1,0)
	to [ammeter] ++(0,-2) node[] (end) {};
	\draw(fet.drain) to [short] ++(1,0)
	to [V, l_=\SI{16}{\volt}DC \& \SI{32}{\volt}DC] ++(0, 2)
	to [short] ($(end)$);
	\draw (fet.gate) node[anchor=north east] {2N5459};
\end{circuitikz}
\\
	\parbox{3.5in}{
	\caption[Schematic --- Constant Current Source]{Schematic provided in
		experiment one to create a constant current source using a JFET.}
	\label{fig:schem4}}
\end{figure}
%
In the performed experiment, the DC source was set first to~\SI{16}{\volt},
while the load resistor was varied from~\SI{50}{\ohm} to~\SI{2}{\kilo\ohm} in
steps of~\SI{100}{\ohm} by changing a global parameter \texttt{RD}.  The same
test was repeated with the voltage set to~\SI{32}{\volt}.  At each resistance
value, the current through the drain was measured with a current probe.

\subsection{High Input Resistance Buffer Amplifier}
The final circuit simulated was not built in experiment one, but is still
extremely important in modern electronics.  A buffer amplifier with high input
impedance was created using a BJT according to the provided specifications, and
is shown in Figure~\ref{fig:schem5}.
%
\begin{figure}[H]
	\centering
	\begin{circuitikz}
	\draw(0,0) node[npn] (bjt) {2N2222}

	(bjt.emitter) to [R, v^=$V_\text{out}$, l_=$R_E$: \SI{3}{\kilo\ohm}] ++(0,-2) node[ground] {}

	(bjt.base) to [R, l_=$R_B$: \SI{20}{\kilo\ohm}] ++(-2,0)
	to [open] ++(0,-2) node[ground] {}
	to [V, l^=$V_\text{in}$] ++(0,2)

	(bjt.collector) to [short] ++(2,0)
	to [open] ++(0, -2) node[ground] {}
	to [V, l_=$V_{CC}$] ++(0,2);
\end{circuitikz}
\\
	\parbox{3.5in}{
	\caption[Schematic --- Buffer Amplifier]{Circuit diagram for the BJT-based
		high input resistance buffer amplifier, originally presented to the student
		in experiment one.}
	\label{fig:schem5}}
\end{figure}
%
The presence of an emitter resistor as the load resistor in this circuit makes
this an ``emitter follower'' circuit.  Such circuits tend to exhibit very low
(near-unity) gain.  The input impedance, however, tends to be rather high, as
it is defined by
%
\begin{equation}
	R_\text{in} = R_B + \beta R_E \quad \text{,}
	\label{eq:rin}
\end{equation}
%
where~$\beta$ is the BJT's DC current gain, and~$R_B$ and~$R_E$ are the base
and emitter resistors, respectively.  As the DC current gain is usually on the
order of~100, even a small load resistor at the emitter will cause the input
impedance to be very large.  This provides useful applications when dealing
with various sensors, many of which should be isolated from the microcontroller
monitoring their output.  The high input resistance amplifier was simulated by
sweeping the input voltage from~\SI{6}{\volt} through~\SI{26}{\volt} in
one-volt steps, while monitoring the output voltage.  The input resistance was
also monitored as~$V_\text{in}$ varied, to ensure that it remains high.
