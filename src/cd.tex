\subsection{Free Running Oscillator}
A free-running oscillator was first created by using the schematic shown in
Figure~\ref{fig:free_run}.
%
\begin{figure}[H]
	\centering
	%\includegraphics[width=.6\textwidth]{img/shot/free_run.pdf}
	A free-running oscillator can be created by using the schematic shown in
Figure~\ref{fig:free_run}, where the switch is fixed closed.
%
\begin{figure}[H]
	\centering
	\includegraphics[width=.6\textwidth]{img/shot/free_run.pdf}
	\caption{}
	\label{fig:free_run}
\end{figure}
%
Using this circuit, the period of the output signal can be calculated by using~\eqref{eq:t0}:
%
\begin{equation}
	T_0 = \frac{3}{2} R_A C_T \left[ 1 + \frac{R_B}{2R_A - R_B} \right] \quad \text{,}
	\label{eq:t0}
\end{equation}
%
where only~$R_A$ and~$C_T$ are external to the provided circuit and available
to modify.  In the case that~$R_A = R_B$, the duty cycle of the square wave
output becomes exactly~\SI{50}{\percent}, and the frequency calculation
simiplifies to the~\eqref{eq:t0_even}.
%
\begin{equation}
	T_0 = 3 R_A C_T
	\label{eq:t0_even}
\end{equation}
%
Testing the possible output frequencies is possible by fixing~$C_T$ at the suggested value of~\SI{3300}{\pico\farad} and varying~$R_A$.

	\parbox{.6\textwidth}{
	\caption{Free-running oscilloscope circuit schematic, as provided in Figure~1 of the
	Intersil datasheet for an ICL8038 integrated circuit [1].}
	\label{fig:free_run}}
\end{figure}
%
Using this circuit, the period of the output signal can be calculated by using~\eqref{eq:t0}:
%
\begin{equation}
	T_0 = \frac{3}{2} R_A C_T \left[ 1 + \frac{R_B}{2R_A - R_B} \right] \quad \text{,}
	\label{eq:t0}
\end{equation}
%
where only~$R_A$ and~$C_T$ are external to the provided circuit and available
to modify.  In the case that~$R_A = R_B$, the duty cycle of the square wave
output becomes exactly~\SI{50}{\percent}, and the frequency calculation
simplifies to that shown in~\eqref{eq:t0_even}.
%
\begin{equation}
	T_0 = 3 R_A C_T
	\label{eq:t0_even}
\end{equation}
%
Testing the possible output frequencies is possible by fixing~$C_T$ at the
suggested value of~\SI{3.3}{\nano\farad} and varying~$R_A$.  In order to view
all three cases for $R_A$~(where~$R_A < R_B$,~$R_A = R_B$, and~$R_A > R_B$), it
should be varied from~\SI{6}{\kilo\ohm} to~\SI{20}{\kilo\ohm} in steps
of~\SI{2}{\kilo\ohm} while measuring the output frequency and duty cycle at
each stage.  This allows several steps before and after the critical point
of~$R_A = R_B$, so that students may view the effects of a varying resistance.

Similarly, $R_A$ can be fixed to~\SI{10}{\kilo\ohm}~(the same as~$R_B$) while
varying~$C_T$.  This ensures a constant duty cycle of~\SI{50}{\percent} over
different values for the timing capacitor.  It is suggested that~$C_T$ is
varied from~\SI{100}{\pico\farad} to~\SI{3.3}{\micro\farad} in three steps per
order of magnitude, while using an oscilloscope to measure the output frequency
at each step.  By~\eqref{eq:t0_even}, the frequency will vary linearly with
changing capacitance.

\subsection{DC Sweep}
Next, a DC sweep was performed on the circuit to view the effects of input
voltage on the duty cycle.  This circuit was based on the one shown in
Figure~\ref{fig:dc_sweep}.
%
\begin{figure}[H]
	\centering
	Next, a DC sweep will be performed on the circuit to view the effects of input
voltage on the duty cycle.  This circuit will be based on the one shown in
Figure~\ref{fig:dc_sweep}.
%
\begin{figure}[H]
	\centering
	Next, a DC sweep will be performed on the circuit to view the effects of input
voltage on the duty cycle.  This circuit will be based on the one shown in
Figure~\ref{fig:dc_sweep}.
%
\begin{figure}[H]
	\centering
	Next, a DC sweep will be performed on the circuit to view the effects of input
voltage on the duty cycle.  This circuit will be based on the one shown in
Figure~\ref{fig:dc_sweep}.
%
\begin{figure}[H]
	\centering
	\input{schem/dc_sweep.tex}
	\caption{DC sweep schematic shown in Figure~5B of the Intersil ICL8038 datasheet [1].  All component values are the same as in Figure~\ref{fig:free_run}}
	\label{fig:dc_sweep}
\end{figure}
%
In this configuration, the negative lead of a DC voltage is connected to pin~8.
With all components as defined as in Figure~\ref{fig:dc_sweep}, the DC voltage
is varied from~\SI{1}{\volt} to~\SI{8}{\volt} in steps of
roughly~\SI{1}{\volt}.  When the input approaches eight volts, the output
quality will degrade.

\begin{figure}[H]
	\centering
	\input{img/plot/dc_sweep.tex}
\end{figure}

	\caption{DC sweep schematic shown in Figure~5B of the Intersil ICL8038 datasheet [1].  All component values are the same as in Figure~\ref{fig:free_run}}
	\label{fig:dc_sweep}
\end{figure}
%
In this configuration, the negative lead of a DC voltage is connected to pin~8.
With all components as defined as in Figure~\ref{fig:dc_sweep}, the DC voltage
is varied from~\SI{1}{\volt} to~\SI{8}{\volt} in steps of
roughly~\SI{1}{\volt}.  When the input approaches eight volts, the output
quality will degrade.

\begin{figure}[H]
	\centering
	Next, a DC sweep will be performed on the circuit to view the effects of input
voltage on the duty cycle.  This circuit will be based on the one shown in
Figure~\ref{fig:dc_sweep}.
%
\begin{figure}[H]
	\centering
	\input{schem/dc_sweep.tex}
	\caption{DC sweep schematic shown in Figure~5B of the Intersil ICL8038 datasheet [1].  All component values are the same as in Figure~\ref{fig:free_run}}
	\label{fig:dc_sweep}
\end{figure}
%
In this configuration, the negative lead of a DC voltage is connected to pin~8.
With all components as defined as in Figure~\ref{fig:dc_sweep}, the DC voltage
is varied from~\SI{1}{\volt} to~\SI{8}{\volt} in steps of
roughly~\SI{1}{\volt}.  When the input approaches eight volts, the output
quality will degrade.

\begin{figure}[H]
	\centering
	\input{img/plot/dc_sweep.tex}
\end{figure}

\end{figure}

	\caption{DC sweep schematic shown in Figure~5B of the Intersil ICL8038 datasheet [1].  All component values are the same as in Figure~\ref{fig:free_run}}
	\label{fig:dc_sweep}
\end{figure}
%
In this configuration, the negative lead of a DC voltage is connected to pin~8.
With all components as defined as in Figure~\ref{fig:dc_sweep}, the DC voltage
is varied from~\SI{1}{\volt} to~\SI{8}{\volt} in steps of
roughly~\SI{1}{\volt}.  When the input approaches eight volts, the output
quality will degrade.

\begin{figure}[H]
	\centering
	Next, a DC sweep will be performed on the circuit to view the effects of input
voltage on the duty cycle.  This circuit will be based on the one shown in
Figure~\ref{fig:dc_sweep}.
%
\begin{figure}[H]
	\centering
	Next, a DC sweep will be performed on the circuit to view the effects of input
voltage on the duty cycle.  This circuit will be based on the one shown in
Figure~\ref{fig:dc_sweep}.
%
\begin{figure}[H]
	\centering
	\input{schem/dc_sweep.tex}
	\caption{DC sweep schematic shown in Figure~5B of the Intersil ICL8038 datasheet [1].  All component values are the same as in Figure~\ref{fig:free_run}}
	\label{fig:dc_sweep}
\end{figure}
%
In this configuration, the negative lead of a DC voltage is connected to pin~8.
With all components as defined as in Figure~\ref{fig:dc_sweep}, the DC voltage
is varied from~\SI{1}{\volt} to~\SI{8}{\volt} in steps of
roughly~\SI{1}{\volt}.  When the input approaches eight volts, the output
quality will degrade.

\begin{figure}[H]
	\centering
	\input{img/plot/dc_sweep.tex}
\end{figure}

	\caption{DC sweep schematic shown in Figure~5B of the Intersil ICL8038 datasheet [1].  All component values are the same as in Figure~\ref{fig:free_run}}
	\label{fig:dc_sweep}
\end{figure}
%
In this configuration, the negative lead of a DC voltage is connected to pin~8.
With all components as defined as in Figure~\ref{fig:dc_sweep}, the DC voltage
is varied from~\SI{1}{\volt} to~\SI{8}{\volt} in steps of
roughly~\SI{1}{\volt}.  When the input approaches eight volts, the output
quality will degrade.

\begin{figure}[H]
	\centering
	Next, a DC sweep will be performed on the circuit to view the effects of input
voltage on the duty cycle.  This circuit will be based on the one shown in
Figure~\ref{fig:dc_sweep}.
%
\begin{figure}[H]
	\centering
	\input{schem/dc_sweep.tex}
	\caption{DC sweep schematic shown in Figure~5B of the Intersil ICL8038 datasheet [1].  All component values are the same as in Figure~\ref{fig:free_run}}
	\label{fig:dc_sweep}
\end{figure}
%
In this configuration, the negative lead of a DC voltage is connected to pin~8.
With all components as defined as in Figure~\ref{fig:dc_sweep}, the DC voltage
is varied from~\SI{1}{\volt} to~\SI{8}{\volt} in steps of
roughly~\SI{1}{\volt}.  When the input approaches eight volts, the output
quality will degrade.

\begin{figure}[H]
	\centering
	\input{img/plot/dc_sweep.tex}
\end{figure}

\end{figure}

\end{figure}

	\parbox{.6\textwidth}{
	\caption{DC sweep schematic shown in Figure~5B of the Intersil ICL8038
	datasheet [1].  All component values are the same as in
	Figure~\ref{fig:free_run}}
	\label{fig:dc_sweep}}
\end{figure}
%
In this configuration, the negative lead of a DC voltage was connected to pin~8.
With all components as defined as in Figure~\ref{fig:dc_sweep}, the DC voltage
was varied from~\SI{0}{\volt} to~\SI{8}{\volt} in steps of
roughly~\SI{1}{\volt}.  When the input approaches eight volts, the output
quality will degrade.

\subsection{FM Modulation}
Finally, the voltage-controlled oscillator was fed a frequency modulated
sinusoid.  The results of the DC sweep imply the expected result for
this test, in which the duty cycle of the output oscillates.
%
\begin{figure}[H]
	\centering
	\begin{circuitikz}
	\draw[ultra thick] (-2, -1.5) rectangle (2, 1.5);
	\draw[/tikz/circuitikz/bipoles/length=1cm]
	(-1.5, 1.5) node[below] {4} to [R, l=\SI{10}{\kilo\ohm}] ++(0, 1.5)
		to [short, -o] (4, 3) node[right] {$V+$}
	(0, 1.5) node[below] {5} to [R, l=\SI{+10}{\kilo\ohm}] ++(0, 1.5)
	(1.5, 1.5) node[below] {6} to [short] ++(0, 1.5)
	(3, 3) to [R, l_=\SI{10}{\kilo\ohm}] ++(0, -1.5) to [short] ++(0, -.5)

	(1.5, -1.5) node[above] {12} to [R, l=\SI{82}{\kilo\ohm}] ++(0, -1.5)
	(0, -1.5) node[above] {11} to [short] ++(0, -1.5)
	(-1.5, -1.5) node[above] {10} to [C, l=\SI{3.3}{\micro\farad}] ++(0, -1.5)
		to [short, -o] (4, -3) node[right] {$V-$};



%	(-2, 0) node[right] {8} to [short] ++(-.5, 0)
%		to [short] ++(0, 1) to [short] ++(.5, 0) node[right] {7};

	% short stuff on the right side
	\draw[/tikz/circuitikz/bipoles/length=.75cm]
	(2, 1) node[left] {9} to [short, -o] ++(2, 0)
	(2, .5) node[left] {3} to [short, -o] ++(2, 0)
		(3, .5) to [R, l=$R_\text{TRI}$] ++(0, -1) node[ground] {}
	(2, -1) node[left] {2} to [short, -o] ++(2, 0)
		(3, -1) to [R, l=$R_\text{SINE}$] ++(0, -1) node[ground] {}

	% short stuff on the left side
	(-2, 1) node[right] {7} to [short] ++(-0.5, 0)
		to [R, l_=\SI{10}{\kilo\ohm}] ++(0, -1) to [C, l_=\SI{10}{\micro\farad}, -o] ++(0, -.75) node[below] {FM}
	(-2, 0) node[right] {8} to [short] ++(-.5, 0);

	% square wave
	\draw[thick]
	(4.2, .85) -- ++(.1, 0) -- ++(0, .3)
		-- ++(.2, 0) -- ++(0, -.3) -- ++(.2, 0) -- ++(0, .3)
		-- ++(.2, 0) -- ++(0, -.3) -- ++(.1, 0);

	% triangle
	\draw[thick]
	(4.2, .5) -- ++(.1, .15) -- ++(.2, -.3) -- ++(.2, .3) -- ++(.2, -.3) --  ++(.1, .15);

	% sinusoid
	\draw[thick]
	(4.2, -1) sin ++(.1, .15) cos ++(.1, -.15) sin ++(.1, -.15) cos ++(.1, .15)
		sin ++(.1, .15) cos ++(.1, -.15) sin ++(.1, -.15) cos ++(.1, .15);
\end{circuitikz}

	\parbox{.6\textwidth}{
	\caption{Schematic for a frequency-modulated voltage-controlled oscillator
	as provided in the Intersil datasheet for an ICL8038 IC [1].  All
	previously-used components are unchanged, and the new resistor and
	capacitor are~\SI{10}{\kilo\ohm} and~\SI{10}{\micro\farad}, respectively.}
	\label{fig:freq_mod}}
\end{figure}
%
For this experiment all previously-used components were unchanged, and the new
resistor and capacitor are~\SI{10}{\kilo\ohm} and~\SI{10}{\micro\farad},
respectively.  In this case, the amplitude of the input signal determines the
frequency of the output, which will vary from~\SI{1}{\volt} to~\SI{3}{\volt} in
one-volt increments.  For each of these amplitudes, the input frequency
varied from~\SI{1}{\hertz} to~\SI{2}{\hertz} in half-hertz steps.  This allowed
students to view the effects of a varying amplitude on the output frequency,
while observing the effects of a varying input frequency on the rate of change
of the output's duty cycle.
