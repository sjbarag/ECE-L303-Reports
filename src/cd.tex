\subsection{Rectifier}
The first circuit built was a half-wave rectifier, utilizing a 1N4006 PN
junction diode as the primary element because of its directional current flow
property.  When forward biased (acheived when the voltage on the p-doped side
is higher than the voltage on the n-doped side), a standard diode will allow
current to flow; a reverse biased, however, no current will flow.  If said
diode is connected in series with a load resistor between the two inputs of an
AC signal as shown in Figure~\ref{fig:schem1}, a half-wave rectifier results.
%
\begin{figure}[H]
	\centering
	\begin{circuitikz}
	\draw (0,0) to [vsourcesin, l=\SI{12.2}{\volt} (RMS) @ \SI{60}{\hertz}] ++(0,2)
	to [short] ++(1,0)
	to [Do, v^=$V_d \approx$ \SI{0.7}{\volt}, i_=$i$, l_=1N4006] ++(2, 0)
	to [short] ++(1,0)
	to [R, v^=$V_o$, l_=\SI{1}{\kilo\ohm}] ++(0, -2)
	to [short] (0, 0);
\end{circuitikz}

	\caption{Half-wave rectifier circuit, as originally suggested by the lab one instructions.}
	\label{fig:schem1}
\end{figure}
%
During phases where the input signal is positive, the voltage across the
resistor will be roughly equal to the input.  When the signal becomes negative
Ohm's Law states that the voltage across the resistor drops to zero, as no
current flows.  The above circuit uses an input signal of \SI{12.2}{\volt}RMS
and outputs a half-rectified signal with an average value of \SI{5.6}{\volt}DC.

\begin{figure}[H]
\begin{circuitikz}
	\draw (0,0) to [V, l=$0 \rightarrow$ \SI{5}{\volt}DC] ++(0,2)
	to [short] ++(1,0)
	to [R, i_=$i_d$, l_=\SI{300}{\ohm}] ++(2, 0)
	to [short] ++(1,0)
	to [led, l^=Red LED] ++(0, -2)
	to [short] (0, 0);
\end{circuitikz}

\end{figure}

\begin{figure}[H]
\begin{circuitikz}
	\draw (0,0) to [V, l=$5 \rightarrow$ \SI{25}{\volt}DC] ++(0,2)
	to [short] ++(1,0)
	to [R, i^=I, l_=\SI{400}{\ohm}] ++(2, 0)
	to [short] ++(1,0)
	to [open, v^=$V_o$] ++(0,-2);
	\draw (0,0) to [short] (4,0)
	to [zDo, l^=1N4740] ++(0, 2);
\end{circuitikz}

\end{figure}

\begin{figure}[H]
\begin{circuitikz}
	\draw(0,0) node[njfet,yscale=-1] (fet) {};
	\draw(fet.gate) |- (fet.drain);
	\draw(fet.source) to [vR, l_=Decade Box, mirror] ++(0,2)
	to [short] ++(1,0)
	to [ammeter] ++(0,-2) node[] (end) {};
	\draw(fet.drain) to [short] ++(1,0)
	to [V, l_=\SI{16}{\volt}DC \& \SI{32}{\volt}DC] ++(0, 2)
	to [short] ($(end)$);
	\draw (fet.gate) node[anchor=north east] {Part Number};
\end{circuitikz}

\end{figure}

