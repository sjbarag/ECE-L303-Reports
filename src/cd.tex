\subsection{Free Running Oscillator}
A free-running oscillator can be created by using the schematic shown in
Figure~\ref{fig:free_run}, where the switch is fixed closed.
%
\begin{figure}[H]
	\centering
	%\includegraphics[width=.6\textwidth]{img/shot/free_run.pdf}
	A free-running oscillator can be created by using the schematic shown in
Figure~\ref{fig:free_run}, where the switch is fixed closed.
%
\begin{figure}[H]
	\centering
	%\includegraphics[width=.6\textwidth]{img/shot/free_run.pdf}
	A free-running oscillator can be created by using the schematic shown in
Figure~\ref{fig:free_run}, where the switch is fixed closed.
%
\begin{figure}[H]
	\centering
	%\includegraphics[width=.6\textwidth]{img/shot/free_run.pdf}
	A free-running oscillator can be created by using the schematic shown in
Figure~\ref{fig:free_run}, where the switch is fixed closed.
%
\begin{figure}[H]
	\centering
	%\includegraphics[width=.6\textwidth]{img/shot/free_run.pdf}
	\input{schem/free_run.tex}
	\parbox{.6\textwidth}{
	\caption{Free-running oscilloscope circuit schematic, as provided in Figure~1 of the
	Intersil datasheet for an ICL8038 integrated circuit [1].}
	\label{fig:free_run}}
\end{figure}
%
Using this circuit, the period of the output signal can be calculated by using~\eqref{eq:t0}:
%
\begin{equation}
	T_0 = \frac{3}{2} R_A C_T \left[ 1 + \frac{R_B}{2R_A - R_B} \right] \quad \text{,}
	\label{eq:t0}
\end{equation}
%
where only~$R_A$ and~$C_T$ are external to the provided circuit and available
to modify.  In the case that~$R_A = R_B$, the duty cycle of the square wave
output becomes exactly~\SI{50}{\percent}, and the frequency calculation
simiplifies to the~\eqref{eq:t0_even}.
%
\begin{equation}
	T_0 = 3 R_A C_T
	\label{eq:t0_even}
\end{equation}
%
Testing the possible output frequencies is possible by fixing~$C_T$ at the
suggested value of~\SI{3300}{\pico\farad} and varying~$R_A$.  In order to view
all three cases for $R_A$~(where~$R_A < R_B$,~$R_A = R_B$, and~$R_A > R_B$), it
should be varied from~\SI{1}{\kilo\ohm} to~\SI{20}{\kilo\ohm} in steps
of~\SI{2}{\kilo\ohm} while measuring the output frequency and duty cycle at
each stage.  This allows several steps before and after the critical point
of~$R_A = R_B$, so that students may view the effects of a varying resistance.

Similarly, $R_A$ can be fixed to~\SI{10}{\kilo\ohm}~(the same as~$R_B$) while
varying~$C_T$.  This ensures a constant duty cycle of~\SI{50}{\percent} over
different values for the timing capacitor.  It is suggested that~$C_T$ is
varied from~\SI{100}{\pico\farad} to~\SI{100}{\micro\farad} in steps of an
order of magnitude, while using an oscilloscope to measure the output frequency
at each step.  By~\eqref{eq:t0_even}, the frequency will vary linearly with
changing capacitance.

\begin{figure}[H]
	\centering
	\input{img/plot/free_run_c.tex}
\end{figure}

\begin{figure}[H]
	\centering
	\input{img/plot/free_run_r.tex}
\end{figure}


%www.intersil.com/data/FN/FN2864.pdf

	\parbox{.6\textwidth}{
	\caption{Free-running oscilloscope circuit schematic, as provided in Figure~1 of the
	Intersil datasheet for an ICL8038 integrated circuit [1].}
	\label{fig:free_run}}
\end{figure}
%
Using this circuit, the period of the output signal can be calculated by using~\eqref{eq:t0}:
%
\begin{equation}
	T_0 = \frac{3}{2} R_A C_T \left[ 1 + \frac{R_B}{2R_A - R_B} \right] \quad \text{,}
	\label{eq:t0}
\end{equation}
%
where only~$R_A$ and~$C_T$ are external to the provided circuit and available
to modify.  In the case that~$R_A = R_B$, the duty cycle of the square wave
output becomes exactly~\SI{50}{\percent}, and the frequency calculation
simiplifies to the~\eqref{eq:t0_even}.
%
\begin{equation}
	T_0 = 3 R_A C_T
	\label{eq:t0_even}
\end{equation}
%
Testing the possible output frequencies is possible by fixing~$C_T$ at the
suggested value of~\SI{3300}{\pico\farad} and varying~$R_A$.  In order to view
all three cases for $R_A$~(where~$R_A < R_B$,~$R_A = R_B$, and~$R_A > R_B$), it
should be varied from~\SI{1}{\kilo\ohm} to~\SI{20}{\kilo\ohm} in steps
of~\SI{2}{\kilo\ohm} while measuring the output frequency and duty cycle at
each stage.  This allows several steps before and after the critical point
of~$R_A = R_B$, so that students may view the effects of a varying resistance.

Similarly, $R_A$ can be fixed to~\SI{10}{\kilo\ohm}~(the same as~$R_B$) while
varying~$C_T$.  This ensures a constant duty cycle of~\SI{50}{\percent} over
different values for the timing capacitor.  It is suggested that~$C_T$ is
varied from~\SI{100}{\pico\farad} to~\SI{100}{\micro\farad} in steps of an
order of magnitude, while using an oscilloscope to measure the output frequency
at each step.  By~\eqref{eq:t0_even}, the frequency will vary linearly with
changing capacitance.

\begin{figure}[H]
	\centering
	\begin{tikzpicture}[gnuplot]
%% generated with GNUPLOT 4.4p2 (Lua 5.1.4; terminal rev. 97, script rev. 96a)
%% Thu 17 Nov 2011 11:33:29 PM EST
\gpsolidlines
\gpcolor{gp lt color axes}
\gpsetlinetype{gp lt axes}
\gpsetlinewidth{1.00}
\draw[gp path] (1.688,0.985)--(9.014,0.985);
\gpcolor{gp lt color border}
\gpsetlinetype{gp lt border}
\draw[gp path] (1.688,0.985)--(1.868,0.985);
\node[gp node right] at (1.504,0.985) { 0.01};
\draw[gp path] (1.688,1.214)--(1.778,1.214);
\draw[gp path] (1.688,1.517)--(1.778,1.517);
\draw[gp path] (1.688,1.672)--(1.778,1.672);
\gpcolor{gp lt color axes}
\gpsetlinetype{gp lt axes}
\draw[gp path] (1.688,1.746)--(1.872,1.746);
\draw[gp path] (5.180,1.746)--(9.014,1.746);
\gpcolor{gp lt color border}
\gpsetlinetype{gp lt border}
\draw[gp path] (1.688,1.746)--(1.868,1.746);
\node[gp node right] at (1.504,1.746) { 0.1};
\draw[gp path] (1.688,1.975)--(1.778,1.975);
\draw[gp path] (1.688,2.278)--(1.778,2.278);
\draw[gp path] (1.688,2.433)--(1.778,2.433);
\gpcolor{gp lt color axes}
\gpsetlinetype{gp lt axes}
\draw[gp path] (1.688,2.507)--(9.014,2.507);
\gpcolor{gp lt color border}
\gpsetlinetype{gp lt border}
\draw[gp path] (1.688,2.507)--(1.868,2.507);
\node[gp node right] at (1.504,2.507) { 1};
\draw[gp path] (1.688,2.736)--(1.778,2.736);
\draw[gp path] (1.688,3.039)--(1.778,3.039);
\draw[gp path] (1.688,3.194)--(1.778,3.194);
\gpcolor{gp lt color axes}
\gpsetlinetype{gp lt axes}
\draw[gp path] (1.688,3.268)--(9.014,3.268);
\gpcolor{gp lt color border}
\gpsetlinetype{gp lt border}
\draw[gp path] (1.688,3.268)--(1.868,3.268);
\node[gp node right] at (1.504,3.268) { 10};
\draw[gp path] (1.688,3.497)--(1.778,3.497);
\draw[gp path] (1.688,3.800)--(1.778,3.800);
\draw[gp path] (1.688,3.955)--(1.778,3.955);
\gpcolor{gp lt color axes}
\gpsetlinetype{gp lt axes}
\draw[gp path] (1.688,4.029)--(9.014,4.029);
\gpcolor{gp lt color border}
\gpsetlinetype{gp lt border}
\draw[gp path] (1.688,4.029)--(1.868,4.029);
\node[gp node right] at (1.504,4.029) { 100};
\draw[gp path] (1.688,4.258)--(1.778,4.258);
\draw[gp path] (1.688,4.561)--(1.778,4.561);
\draw[gp path] (1.688,4.716)--(1.778,4.716);
\gpcolor{gp lt color axes}
\gpsetlinetype{gp lt axes}
\draw[gp path] (1.688,4.790)--(9.014,4.790);
\gpcolor{gp lt color border}
\gpsetlinetype{gp lt border}
\draw[gp path] (1.688,4.790)--(1.868,4.790);
\node[gp node right] at (1.504,4.790) { 1000};
\gpcolor{gp lt color axes}
\gpsetlinetype{gp lt axes}
\draw[gp path] (1.688,0.985)--(1.688,4.790);
\gpcolor{gp lt color border}
\gpsetlinetype{gp lt border}
\draw[gp path] (1.688,0.985)--(1.688,1.165);
\node[gp node center] at (1.688,0.677) { 0.0001};
\draw[gp path] (2.157,0.985)--(2.157,1.075);
\draw[gp path] (2.778,0.985)--(2.778,1.075);
\draw[gp path] (3.096,0.985)--(3.096,1.075);
\gpcolor{gp lt color axes}
\gpsetlinetype{gp lt axes}
\draw[gp path] (3.247,0.985)--(3.247,1.165);
\draw[gp path] (3.247,2.089)--(3.247,4.790);
\gpcolor{gp lt color border}
\gpsetlinetype{gp lt border}
\draw[gp path] (3.247,0.985)--(3.247,1.165);
\node[gp node center] at (3.247,0.677) { 0.001};
\draw[gp path] (3.716,0.985)--(3.716,1.075);
\draw[gp path] (4.337,0.985)--(4.337,1.075);
\draw[gp path] (4.655,0.985)--(4.655,1.075);
\gpcolor{gp lt color axes}
\gpsetlinetype{gp lt axes}
\draw[gp path] (4.806,0.985)--(4.806,1.165);
\draw[gp path] (4.806,2.089)--(4.806,4.790);
\gpcolor{gp lt color border}
\gpsetlinetype{gp lt border}
\draw[gp path] (4.806,0.985)--(4.806,1.165);
\node[gp node center] at (4.806,0.677) { 0.01};
\draw[gp path] (5.275,0.985)--(5.275,1.075);
\draw[gp path] (5.896,0.985)--(5.896,1.075);
\draw[gp path] (6.214,0.985)--(6.214,1.075);
\gpcolor{gp lt color axes}
\gpsetlinetype{gp lt axes}
\draw[gp path] (6.365,0.985)--(6.365,4.790);
\gpcolor{gp lt color border}
\gpsetlinetype{gp lt border}
\draw[gp path] (6.365,0.985)--(6.365,1.165);
\node[gp node center] at (6.365,0.677) { 0.1};
\draw[gp path] (6.835,0.985)--(6.835,1.075);
\draw[gp path] (7.455,0.985)--(7.455,1.075);
\draw[gp path] (7.773,0.985)--(7.773,1.075);
\gpcolor{gp lt color axes}
\gpsetlinetype{gp lt axes}
\draw[gp path] (7.924,0.985)--(7.924,4.790);
\gpcolor{gp lt color border}
\gpsetlinetype{gp lt border}
\draw[gp path] (7.924,0.985)--(7.924,1.165);
\node[gp node center] at (7.924,0.677) { 1};
\draw[gp path] (8.394,0.985)--(8.394,1.075);
\draw[gp path] (9.014,0.985)--(9.014,1.075);
\draw[gp path] (9.014,0.985)--(8.834,0.985);
\node[gp node left] at (9.198,0.985) { 0};
\draw[gp path] (9.014,1.746)--(8.834,1.746);
\node[gp node left] at (9.198,1.746) { 12};
\draw[gp path] (9.014,2.507)--(8.834,2.507);
\node[gp node left] at (9.198,2.507) { 24};
\draw[gp path] (9.014,3.268)--(8.834,3.268);
\node[gp node left] at (9.198,3.268) { 36};
\draw[gp path] (9.014,4.029)--(8.834,4.029);
\node[gp node left] at (9.198,4.029) { 48};
\draw[gp path] (9.014,4.790)--(8.834,4.790);
\node[gp node left] at (9.198,4.790) { 60};
\draw[gp path] (1.688,4.790)--(1.688,0.985)--(9.014,0.985)--(9.014,4.790)--cycle;
\node[gp node center,rotate=-270] at (0.246,2.887) {Output Frequency, $f$ (\si{\kilo\hertz})};
\node[gp node center,rotate=-270] at (10.087,2.887) {Output Duty Cycle, (\si{\percent})};
\node[gp node center] at (5.351,0.215) {Capacitance, $C_T$ (\si{\micro\farad})};
\node[gp node center] at (5.351,5.252) {Free-running Oscillator: Varying Capacitor};
\draw[gp path] (1.872,1.165)--(1.872,2.089)--(5.180,2.089)--(5.180,1.165)--cycle;
\draw[gp path] (1.872,2.089)--(5.180,2.089);
\node[gp node right] at (3.896,1.781) {$f$};
\gpcolor{gp lt color 0}
\gpsetlinetype{gp lt plot 0}
\gpsetlinewidth{3.00}
\draw[gp path] (4.080,1.781)--(4.996,1.781);
\draw[gp path] (1.688,4.156)--(2.157,4.038)--(2.496,3.946)--(3.247,3.547)--(3.716,3.373)%
  --(4.055,3.293)--(4.806,2.880)--(5.340,2.661)--(5.615,2.537)--(6.365,2.151)--(6.899,1.903)%
  --(7.174,1.783)--(7.924,1.389)--(8.458,1.119)--(8.733,0.985);
\gpsetpointsize{4.00}
\gppoint{gp mark 8}{(1.688,4.156)}
\gppoint{gp mark 8}{(2.157,4.038)}
\gppoint{gp mark 8}{(2.496,3.946)}
\gppoint{gp mark 8}{(3.247,3.547)}
\gppoint{gp mark 8}{(3.716,3.373)}
\gppoint{gp mark 8}{(4.055,3.293)}
\gppoint{gp mark 8}{(4.806,2.880)}
\gppoint{gp mark 8}{(5.340,2.661)}
\gppoint{gp mark 8}{(5.615,2.537)}
\gppoint{gp mark 8}{(6.365,2.151)}
\gppoint{gp mark 8}{(6.899,1.903)}
\gppoint{gp mark 8}{(7.174,1.783)}
\gppoint{gp mark 8}{(7.924,1.389)}
\gppoint{gp mark 8}{(8.458,1.119)}
\gppoint{gp mark 8}{(8.733,0.985)}
\gppoint{gp mark 8}{(4.538,1.781)}
\gpcolor{gp lt color border}
\node[gp node right] at (3.896,1.473) {Duty Cycle};
\gpcolor{gp lt color 1}
\gpsetlinetype{gp lt plot 1}
\draw[gp path] (4.080,1.473)--(4.996,1.473);
\draw[gp path] (1.688,3.084)--(2.157,3.433)--(2.496,3.604)--(3.247,3.985)--(3.716,4.048)%
  --(4.055,4.073)--(4.806,4.149)--(5.340,4.143)--(5.615,4.149)--(6.365,4.156)--(6.899,4.149)%
  --(7.174,4.181)--(7.924,4.156)--(8.458,4.188)--(8.733,4.162);
\gppoint{gp mark 5}{(1.688,3.084)}
\gppoint{gp mark 5}{(2.157,3.433)}
\gppoint{gp mark 5}{(2.496,3.604)}
\gppoint{gp mark 5}{(3.247,3.985)}
\gppoint{gp mark 5}{(3.716,4.048)}
\gppoint{gp mark 5}{(4.055,4.073)}
\gppoint{gp mark 5}{(4.806,4.149)}
\gppoint{gp mark 5}{(5.340,4.143)}
\gppoint{gp mark 5}{(5.615,4.149)}
\gppoint{gp mark 5}{(6.365,4.156)}
\gppoint{gp mark 5}{(6.899,4.149)}
\gppoint{gp mark 5}{(7.174,4.181)}
\gppoint{gp mark 5}{(7.924,4.156)}
\gppoint{gp mark 5}{(8.458,4.188)}
\gppoint{gp mark 5}{(8.733,4.162)}
\gppoint{gp mark 5}{(4.538,1.473)}
\gpcolor{gp lt color border}
\gpsetlinetype{gp lt border}
\gpsetlinewidth{1.00}
\draw[gp path] (1.688,4.790)--(1.688,0.985)--(9.014,0.985)--(9.014,4.790)--cycle;
%% coordinates of the plot area
\gpdefrectangularnode{gp plot 1}{\pgfpoint{1.688cm}{0.985cm}}{\pgfpoint{9.014cm}{4.790cm}}
\end{tikzpicture}
%% gnuplot variables

\end{figure}

\begin{figure}[H]
	\centering
	% R(kOhm)	Vpp(V)	f(kHz)	Duty Cycle (%)
\small
\begin{tabular}{|c|c|c|c|}
\hline
\tbf{Resistor (\si{\kilo\ohm})} &
	\tbf{Output Amplitude (\si{\volt})} &
		\tbf{Output Frequency (\si{\kilo\hertz})} &
			\tbf{Output Duty Cycle (\si{\percent})}\\ \hline
6	&  	9.688	&  	5.97	&  	16.0 \\ \hline
8	&  	9.688	&  	10.08	&  	36.5 \\ \hline
10	&  	9.688	&  	10.72	&  	49.2 \\ \hline
12	&  	9.688	&  	10.36	&  	57.7 \\ \hline
14	&  	9.688	&  	9.780	&  	63.7 \\ \hline
16	&  	9.688	&  	9.112	&  	68.3 \\ \hline
18	&  	9.688	&  	8.457	&  	71.9 \\ \hline
20	&  	9.688	&  	7.850	&  	74.9 \\ \hline
100 &  	9.688	&  	1.770	&  	95.2 \\ \hline
\end{tabular}

\end{figure}


%www.intersil.com/data/FN/FN2864.pdf

	\parbox{.6\textwidth}{
	\caption{Free-running oscilloscope circuit schematic, as provided in Figure~1 of the
	Intersil datasheet for an ICL8038 integrated circuit [1].}
	\label{fig:free_run}}
\end{figure}
%
Using this circuit, the period of the output signal can be calculated by using~\eqref{eq:t0}:
%
\begin{equation}
	T_0 = \frac{3}{2} R_A C_T \left[ 1 + \frac{R_B}{2R_A - R_B} \right] \quad \text{,}
	\label{eq:t0}
\end{equation}
%
where only~$R_A$ and~$C_T$ are external to the provided circuit and available
to modify.  In the case that~$R_A = R_B$, the duty cycle of the square wave
output becomes exactly~\SI{50}{\percent}, and the frequency calculation
simiplifies to the~\eqref{eq:t0_even}.
%
\begin{equation}
	T_0 = 3 R_A C_T
	\label{eq:t0_even}
\end{equation}
%
Testing the possible output frequencies is possible by fixing~$C_T$ at the
suggested value of~\SI{3300}{\pico\farad} and varying~$R_A$.  In order to view
all three cases for $R_A$~(where~$R_A < R_B$,~$R_A = R_B$, and~$R_A > R_B$), it
should be varied from~\SI{1}{\kilo\ohm} to~\SI{20}{\kilo\ohm} in steps
of~\SI{2}{\kilo\ohm} while measuring the output frequency and duty cycle at
each stage.  This allows several steps before and after the critical point
of~$R_A = R_B$, so that students may view the effects of a varying resistance.

Similarly, $R_A$ can be fixed to~\SI{10}{\kilo\ohm}~(the same as~$R_B$) while
varying~$C_T$.  This ensures a constant duty cycle of~\SI{50}{\percent} over
different values for the timing capacitor.  It is suggested that~$C_T$ is
varied from~\SI{100}{\pico\farad} to~\SI{100}{\micro\farad} in steps of an
order of magnitude, while using an oscilloscope to measure the output frequency
at each step.  By~\eqref{eq:t0_even}, the frequency will vary linearly with
changing capacitance.

\begin{figure}[H]
	\centering
	\begin{tikzpicture}[gnuplot]
%% generated with GNUPLOT 4.4p2 (Lua 5.1.4; terminal rev. 97, script rev. 96a)
%% Thu 17 Nov 2011 11:33:29 PM EST
\gpsolidlines
\gpcolor{gp lt color axes}
\gpsetlinetype{gp lt axes}
\gpsetlinewidth{1.00}
\draw[gp path] (1.688,0.985)--(9.014,0.985);
\gpcolor{gp lt color border}
\gpsetlinetype{gp lt border}
\draw[gp path] (1.688,0.985)--(1.868,0.985);
\node[gp node right] at (1.504,0.985) { 0.01};
\draw[gp path] (1.688,1.214)--(1.778,1.214);
\draw[gp path] (1.688,1.517)--(1.778,1.517);
\draw[gp path] (1.688,1.672)--(1.778,1.672);
\gpcolor{gp lt color axes}
\gpsetlinetype{gp lt axes}
\draw[gp path] (1.688,1.746)--(1.872,1.746);
\draw[gp path] (5.180,1.746)--(9.014,1.746);
\gpcolor{gp lt color border}
\gpsetlinetype{gp lt border}
\draw[gp path] (1.688,1.746)--(1.868,1.746);
\node[gp node right] at (1.504,1.746) { 0.1};
\draw[gp path] (1.688,1.975)--(1.778,1.975);
\draw[gp path] (1.688,2.278)--(1.778,2.278);
\draw[gp path] (1.688,2.433)--(1.778,2.433);
\gpcolor{gp lt color axes}
\gpsetlinetype{gp lt axes}
\draw[gp path] (1.688,2.507)--(9.014,2.507);
\gpcolor{gp lt color border}
\gpsetlinetype{gp lt border}
\draw[gp path] (1.688,2.507)--(1.868,2.507);
\node[gp node right] at (1.504,2.507) { 1};
\draw[gp path] (1.688,2.736)--(1.778,2.736);
\draw[gp path] (1.688,3.039)--(1.778,3.039);
\draw[gp path] (1.688,3.194)--(1.778,3.194);
\gpcolor{gp lt color axes}
\gpsetlinetype{gp lt axes}
\draw[gp path] (1.688,3.268)--(9.014,3.268);
\gpcolor{gp lt color border}
\gpsetlinetype{gp lt border}
\draw[gp path] (1.688,3.268)--(1.868,3.268);
\node[gp node right] at (1.504,3.268) { 10};
\draw[gp path] (1.688,3.497)--(1.778,3.497);
\draw[gp path] (1.688,3.800)--(1.778,3.800);
\draw[gp path] (1.688,3.955)--(1.778,3.955);
\gpcolor{gp lt color axes}
\gpsetlinetype{gp lt axes}
\draw[gp path] (1.688,4.029)--(9.014,4.029);
\gpcolor{gp lt color border}
\gpsetlinetype{gp lt border}
\draw[gp path] (1.688,4.029)--(1.868,4.029);
\node[gp node right] at (1.504,4.029) { 100};
\draw[gp path] (1.688,4.258)--(1.778,4.258);
\draw[gp path] (1.688,4.561)--(1.778,4.561);
\draw[gp path] (1.688,4.716)--(1.778,4.716);
\gpcolor{gp lt color axes}
\gpsetlinetype{gp lt axes}
\draw[gp path] (1.688,4.790)--(9.014,4.790);
\gpcolor{gp lt color border}
\gpsetlinetype{gp lt border}
\draw[gp path] (1.688,4.790)--(1.868,4.790);
\node[gp node right] at (1.504,4.790) { 1000};
\gpcolor{gp lt color axes}
\gpsetlinetype{gp lt axes}
\draw[gp path] (1.688,0.985)--(1.688,4.790);
\gpcolor{gp lt color border}
\gpsetlinetype{gp lt border}
\draw[gp path] (1.688,0.985)--(1.688,1.165);
\node[gp node center] at (1.688,0.677) { 0.0001};
\draw[gp path] (2.157,0.985)--(2.157,1.075);
\draw[gp path] (2.778,0.985)--(2.778,1.075);
\draw[gp path] (3.096,0.985)--(3.096,1.075);
\gpcolor{gp lt color axes}
\gpsetlinetype{gp lt axes}
\draw[gp path] (3.247,0.985)--(3.247,1.165);
\draw[gp path] (3.247,2.089)--(3.247,4.790);
\gpcolor{gp lt color border}
\gpsetlinetype{gp lt border}
\draw[gp path] (3.247,0.985)--(3.247,1.165);
\node[gp node center] at (3.247,0.677) { 0.001};
\draw[gp path] (3.716,0.985)--(3.716,1.075);
\draw[gp path] (4.337,0.985)--(4.337,1.075);
\draw[gp path] (4.655,0.985)--(4.655,1.075);
\gpcolor{gp lt color axes}
\gpsetlinetype{gp lt axes}
\draw[gp path] (4.806,0.985)--(4.806,1.165);
\draw[gp path] (4.806,2.089)--(4.806,4.790);
\gpcolor{gp lt color border}
\gpsetlinetype{gp lt border}
\draw[gp path] (4.806,0.985)--(4.806,1.165);
\node[gp node center] at (4.806,0.677) { 0.01};
\draw[gp path] (5.275,0.985)--(5.275,1.075);
\draw[gp path] (5.896,0.985)--(5.896,1.075);
\draw[gp path] (6.214,0.985)--(6.214,1.075);
\gpcolor{gp lt color axes}
\gpsetlinetype{gp lt axes}
\draw[gp path] (6.365,0.985)--(6.365,4.790);
\gpcolor{gp lt color border}
\gpsetlinetype{gp lt border}
\draw[gp path] (6.365,0.985)--(6.365,1.165);
\node[gp node center] at (6.365,0.677) { 0.1};
\draw[gp path] (6.835,0.985)--(6.835,1.075);
\draw[gp path] (7.455,0.985)--(7.455,1.075);
\draw[gp path] (7.773,0.985)--(7.773,1.075);
\gpcolor{gp lt color axes}
\gpsetlinetype{gp lt axes}
\draw[gp path] (7.924,0.985)--(7.924,4.790);
\gpcolor{gp lt color border}
\gpsetlinetype{gp lt border}
\draw[gp path] (7.924,0.985)--(7.924,1.165);
\node[gp node center] at (7.924,0.677) { 1};
\draw[gp path] (8.394,0.985)--(8.394,1.075);
\draw[gp path] (9.014,0.985)--(9.014,1.075);
\draw[gp path] (9.014,0.985)--(8.834,0.985);
\node[gp node left] at (9.198,0.985) { 0};
\draw[gp path] (9.014,1.746)--(8.834,1.746);
\node[gp node left] at (9.198,1.746) { 12};
\draw[gp path] (9.014,2.507)--(8.834,2.507);
\node[gp node left] at (9.198,2.507) { 24};
\draw[gp path] (9.014,3.268)--(8.834,3.268);
\node[gp node left] at (9.198,3.268) { 36};
\draw[gp path] (9.014,4.029)--(8.834,4.029);
\node[gp node left] at (9.198,4.029) { 48};
\draw[gp path] (9.014,4.790)--(8.834,4.790);
\node[gp node left] at (9.198,4.790) { 60};
\draw[gp path] (1.688,4.790)--(1.688,0.985)--(9.014,0.985)--(9.014,4.790)--cycle;
\node[gp node center,rotate=-270] at (0.246,2.887) {Output Frequency, $f$ (\si{\kilo\hertz})};
\node[gp node center,rotate=-270] at (10.087,2.887) {Output Duty Cycle, (\si{\percent})};
\node[gp node center] at (5.351,0.215) {Capacitance, $C_T$ (\si{\micro\farad})};
\node[gp node center] at (5.351,5.252) {Free-running Oscillator: Varying Capacitor};
\draw[gp path] (1.872,1.165)--(1.872,2.089)--(5.180,2.089)--(5.180,1.165)--cycle;
\draw[gp path] (1.872,2.089)--(5.180,2.089);
\node[gp node right] at (3.896,1.781) {$f$};
\gpcolor{gp lt color 0}
\gpsetlinetype{gp lt plot 0}
\gpsetlinewidth{3.00}
\draw[gp path] (4.080,1.781)--(4.996,1.781);
\draw[gp path] (1.688,4.156)--(2.157,4.038)--(2.496,3.946)--(3.247,3.547)--(3.716,3.373)%
  --(4.055,3.293)--(4.806,2.880)--(5.340,2.661)--(5.615,2.537)--(6.365,2.151)--(6.899,1.903)%
  --(7.174,1.783)--(7.924,1.389)--(8.458,1.119)--(8.733,0.985);
\gpsetpointsize{4.00}
\gppoint{gp mark 8}{(1.688,4.156)}
\gppoint{gp mark 8}{(2.157,4.038)}
\gppoint{gp mark 8}{(2.496,3.946)}
\gppoint{gp mark 8}{(3.247,3.547)}
\gppoint{gp mark 8}{(3.716,3.373)}
\gppoint{gp mark 8}{(4.055,3.293)}
\gppoint{gp mark 8}{(4.806,2.880)}
\gppoint{gp mark 8}{(5.340,2.661)}
\gppoint{gp mark 8}{(5.615,2.537)}
\gppoint{gp mark 8}{(6.365,2.151)}
\gppoint{gp mark 8}{(6.899,1.903)}
\gppoint{gp mark 8}{(7.174,1.783)}
\gppoint{gp mark 8}{(7.924,1.389)}
\gppoint{gp mark 8}{(8.458,1.119)}
\gppoint{gp mark 8}{(8.733,0.985)}
\gppoint{gp mark 8}{(4.538,1.781)}
\gpcolor{gp lt color border}
\node[gp node right] at (3.896,1.473) {Duty Cycle};
\gpcolor{gp lt color 1}
\gpsetlinetype{gp lt plot 1}
\draw[gp path] (4.080,1.473)--(4.996,1.473);
\draw[gp path] (1.688,3.084)--(2.157,3.433)--(2.496,3.604)--(3.247,3.985)--(3.716,4.048)%
  --(4.055,4.073)--(4.806,4.149)--(5.340,4.143)--(5.615,4.149)--(6.365,4.156)--(6.899,4.149)%
  --(7.174,4.181)--(7.924,4.156)--(8.458,4.188)--(8.733,4.162);
\gppoint{gp mark 5}{(1.688,3.084)}
\gppoint{gp mark 5}{(2.157,3.433)}
\gppoint{gp mark 5}{(2.496,3.604)}
\gppoint{gp mark 5}{(3.247,3.985)}
\gppoint{gp mark 5}{(3.716,4.048)}
\gppoint{gp mark 5}{(4.055,4.073)}
\gppoint{gp mark 5}{(4.806,4.149)}
\gppoint{gp mark 5}{(5.340,4.143)}
\gppoint{gp mark 5}{(5.615,4.149)}
\gppoint{gp mark 5}{(6.365,4.156)}
\gppoint{gp mark 5}{(6.899,4.149)}
\gppoint{gp mark 5}{(7.174,4.181)}
\gppoint{gp mark 5}{(7.924,4.156)}
\gppoint{gp mark 5}{(8.458,4.188)}
\gppoint{gp mark 5}{(8.733,4.162)}
\gppoint{gp mark 5}{(4.538,1.473)}
\gpcolor{gp lt color border}
\gpsetlinetype{gp lt border}
\gpsetlinewidth{1.00}
\draw[gp path] (1.688,4.790)--(1.688,0.985)--(9.014,0.985)--(9.014,4.790)--cycle;
%% coordinates of the plot area
\gpdefrectangularnode{gp plot 1}{\pgfpoint{1.688cm}{0.985cm}}{\pgfpoint{9.014cm}{4.790cm}}
\end{tikzpicture}
%% gnuplot variables

\end{figure}

\begin{figure}[H]
	\centering
	% R(kOhm)	Vpp(V)	f(kHz)	Duty Cycle (%)
\small
\begin{tabular}{|c|c|c|c|}
\hline
\tbf{Resistor (\si{\kilo\ohm})} &
	\tbf{Output Amplitude (\si{\volt})} &
		\tbf{Output Frequency (\si{\kilo\hertz})} &
			\tbf{Output Duty Cycle (\si{\percent})}\\ \hline
6	&  	9.688	&  	5.97	&  	16.0 \\ \hline
8	&  	9.688	&  	10.08	&  	36.5 \\ \hline
10	&  	9.688	&  	10.72	&  	49.2 \\ \hline
12	&  	9.688	&  	10.36	&  	57.7 \\ \hline
14	&  	9.688	&  	9.780	&  	63.7 \\ \hline
16	&  	9.688	&  	9.112	&  	68.3 \\ \hline
18	&  	9.688	&  	8.457	&  	71.9 \\ \hline
20	&  	9.688	&  	7.850	&  	74.9 \\ \hline
100 &  	9.688	&  	1.770	&  	95.2 \\ \hline
\end{tabular}

\end{figure}


%www.intersil.com/data/FN/FN2864.pdf

	\parbox{.6\textwidth}{
	\caption{Free-running oscilloscope circuit schematic, as provided in Figure~1 of the
	Intersil datasheet for an ICL8038 integrated circuit [1].}
	\label{fig:free_run}}
\end{figure}
%
Using this circuit, the period of the output signal can be calculated by using~\eqref{eq:t0}:
%
\begin{equation}
	T_0 = \frac{3}{2} R_A C_T \left[ 1 + \frac{R_B}{2R_A - R_B} \right] \quad \text{,}
	\label{eq:t0}
\end{equation}
%
where only~$R_A$ and~$C_T$ are external to the provided circuit and available
to modify.  In the case that~$R_A = R_B$, the duty cycle of the square wave
output becomes exactly~\SI{50}{\percent}, and the frequency calculation
simiplifies to the~\eqref{eq:t0_even}.
%
\begin{equation}
	T_0 = 3 R_A C_T
	\label{eq:t0_even}
\end{equation}
%
Testing the possible output frequencies is possible by fixing~$C_T$ at the
suggested value of~\SI{3300}{\pico\farad} and varying~$R_A$.  In order to view
all three cases for $R_A$~(where~$R_A < R_B$,~$R_A = R_B$, and~$R_A > R_B$), it
should be varied from~\SI{1}{\kilo\ohm} to~\SI{20}{\kilo\ohm} in steps
of~\SI{2}{\kilo\ohm} while measuring the output frequency and duty cycle at
each stage.  This allows several steps before and after the critical point
of~$R_A = R_B$, so that students may view the effects of a varying resistance.

Similarly, $R_A$ can be fixed to~\SI{10}{\kilo\ohm}~(the same as~$R_B$) while
varying~$C_T$.  This ensures a constant duty cycle of~\SI{50}{\percent} over
different values for the timing capacitor.  It is suggested that~$C_T$ is
varied from~\SI{100}{\pico\farad} to~\SI{100}{\micro\farad} in steps of an
order of magnitude, while using an oscilloscope to measure the output frequency
at each step.  By~\eqref{eq:t0_even}, the frequency will vary linearly with
changing capacitance.

\subsection{DC Sweep}
Next, a DC sweep will be performed on the circuit to view the effects of input
voltage on the duty cycle.  This circuit will be based on the one shown in
Figure~\ref{fig:dc_sweep}.
%
\begin{figure}[H]
	\centering
	Next, a DC sweep will be performed on the circuit to vivew the effects of input
voltage on the duty cycle.  This circuit will be based on the one shown in
Figure~\ref{fig:dc_sweep}.
%
\begin{figure}[H]
	\centering
	\includegraphics[width=.6\textwidth]{img/shot/dc_sweep.pdf}
	\caption{DC sweep schematic shown in Figure~5B of the Intersil ICL8038 datasheet [1].  All component values are the same as in Figure~\ref{fig:free_run}}
	\label{fig:dc_sweep}
\end{figure}
%
In this configuration, the negative lead of a DC voltage is connected to pin~8.
With all components as defined as in Figure~\ref{fig:dc_sweep}, the DC voltage
is varied from~\SI{1}{\volt} to~\SI{8}{\volt} in steps of
roughly~\SI{1}{\volt}.  When the input approaches eight volts, the output
quality will degrade.

	\caption{DC sweep schematic shown in Figure~5B of the Intersil ICL8038 datasheet [1].  All component values are the same as in Figure~\ref{fig:free_run}}
	\label{fig:dc_sweep}
\end{figure}
%
In this configuration, the negative lead of a DC voltage is connected to pin~8.
With all components as defined as in Figure~\ref{fig:dc_sweep}, the DC voltage
is varied from~\SI{1}{\volt} to~\SI{8}{\volt} in steps of
roughly~\SI{1}{\volt}.  When the input approaches eight volts, the output
quality will degrade.

\subsection{FM Modulation}
Finally, the voltage-controlled oscillator was fed a frequency modulated
sinusoid.  The results of the DC sweep should imply the expected result for
this test, in which the duty cycle of the output oscillates.
%
\begin{figure}[H]
	\centering
	\begin{tikzpicture}[gnuplot]
%% generated with GNUPLOT 4.4p2 (Lua 5.1.4; terminal rev. 97, script rev. 96a)
%% Wed 16 Nov 2011 09:22:45 AM EST
\gpsolidlines
\gpcolor{gp lt color axes}
\gpsetlinetype{gp lt axes}
\gpsetlinewidth{1.00}
\draw[gp path] (1.320,0.985)--(10.242,0.985);
\gpcolor{gp lt color border}
\gpsetlinetype{gp lt border}
\draw[gp path] (1.320,0.985)--(1.500,0.985);
\node[gp node right] at (1.136,0.985) { 0};
\gpcolor{gp lt color axes}
\gpsetlinetype{gp lt axes}
\draw[gp path] (1.320,1.746)--(10.242,1.746);
\gpcolor{gp lt color border}
\gpsetlinetype{gp lt border}
\draw[gp path] (1.320,1.746)--(1.500,1.746);
\node[gp node right] at (1.136,1.746) { 5};
\gpcolor{gp lt color axes}
\gpsetlinetype{gp lt axes}
\draw[gp path] (1.320,2.507)--(7.486,2.507);
\draw[gp path] (10.058,2.507)--(10.242,2.507);
\gpcolor{gp lt color border}
\gpsetlinetype{gp lt border}
\draw[gp path] (1.320,2.507)--(1.500,2.507);
\node[gp node right] at (1.136,2.507) { 10};
\gpcolor{gp lt color axes}
\gpsetlinetype{gp lt axes}
\draw[gp path] (1.320,3.268)--(7.486,3.268);
\draw[gp path] (10.058,3.268)--(10.242,3.268);
\gpcolor{gp lt color border}
\gpsetlinetype{gp lt border}
\draw[gp path] (1.320,3.268)--(1.500,3.268);
\node[gp node right] at (1.136,3.268) { 15};
\gpcolor{gp lt color axes}
\gpsetlinetype{gp lt axes}
\draw[gp path] (1.320,4.029)--(10.242,4.029);
\gpcolor{gp lt color border}
\gpsetlinetype{gp lt border}
\draw[gp path] (1.320,4.029)--(1.500,4.029);
\node[gp node right] at (1.136,4.029) { 20};
\gpcolor{gp lt color axes}
\gpsetlinetype{gp lt axes}
\draw[gp path] (1.320,4.790)--(10.242,4.790);
\gpcolor{gp lt color border}
\gpsetlinetype{gp lt border}
\draw[gp path] (1.320,4.790)--(1.500,4.790);
\node[gp node right] at (1.136,4.790) { 25};
\gpcolor{gp lt color axes}
\gpsetlinetype{gp lt axes}
\draw[gp path] (1.320,0.985)--(1.320,4.790);
\gpcolor{gp lt color border}
\gpsetlinetype{gp lt border}
\draw[gp path] (1.320,0.985)--(1.320,1.165);
\node[gp node center] at (1.320,0.677) { 0};
\gpcolor{gp lt color axes}
\gpsetlinetype{gp lt axes}
\draw[gp path] (2.807,0.985)--(2.807,4.790);
\gpcolor{gp lt color border}
\gpsetlinetype{gp lt border}
\draw[gp path] (2.807,0.985)--(2.807,1.165);
\node[gp node center] at (2.807,0.677) { 0.5};
\gpcolor{gp lt color axes}
\gpsetlinetype{gp lt axes}
\draw[gp path] (4.294,0.985)--(4.294,4.790);
\gpcolor{gp lt color border}
\gpsetlinetype{gp lt border}
\draw[gp path] (4.294,0.985)--(4.294,1.165);
\node[gp node center] at (4.294,0.677) { 1};
\gpcolor{gp lt color axes}
\gpsetlinetype{gp lt axes}
\draw[gp path] (5.781,0.985)--(5.781,4.790);
\gpcolor{gp lt color border}
\gpsetlinetype{gp lt border}
\draw[gp path] (5.781,0.985)--(5.781,1.165);
\node[gp node center] at (5.781,0.677) { 1.5};
\gpcolor{gp lt color axes}
\gpsetlinetype{gp lt axes}
\draw[gp path] (7.268,0.985)--(7.268,4.790);
\gpcolor{gp lt color border}
\gpsetlinetype{gp lt border}
\draw[gp path] (7.268,0.985)--(7.268,1.165);
\node[gp node center] at (7.268,0.677) { 2};
\gpcolor{gp lt color axes}
\gpsetlinetype{gp lt axes}
\draw[gp path] (8.755,0.985)--(8.755,2.271);
\draw[gp path] (8.755,3.503)--(8.755,4.790);
\gpcolor{gp lt color border}
\gpsetlinetype{gp lt border}
\draw[gp path] (8.755,0.985)--(8.755,1.165);
\node[gp node center] at (8.755,0.677) { 2.5};
\gpcolor{gp lt color axes}
\gpsetlinetype{gp lt axes}
\draw[gp path] (10.242,0.985)--(10.242,4.790);
\gpcolor{gp lt color border}
\gpsetlinetype{gp lt border}
\draw[gp path] (10.242,0.985)--(10.242,1.165);
\node[gp node center] at (10.242,0.677) { 3};
\draw[gp path] (1.320,4.790)--(1.320,0.985)--(10.242,0.985)--(10.242,4.790)--cycle;
\node[gp node center,rotate=-270] at (0.246,2.887) {Output Period, $\Delta t$ (\si{\micro\second})};
\node[gp node center] at (5.781,0.215) {Input Frequency, $f$ (\si{\hertz})};
\node[gp node center] at (5.781,5.252) {Frequency Modulation};
\draw[gp path] (7.486,2.271)--(7.486,3.503)--(10.058,3.503)--(10.058,2.271)--cycle;
\draw[gp path] (7.486,3.503)--(10.058,3.503);
\node[gp node right] at (8.774,3.195) {$\SI{1}{\volt}_\text{\footnotesize P-P}$};
\gpcolor{gp lt color 0}
\gpsetlinetype{gp lt plot 0}
\gpsetlinewidth{3.00}
\draw[gp path] (8.958,3.195)--(9.874,3.195);
\draw[gp path] (4.294,1.868)--(5.781,2.050)--(7.268,2.050);
\gpsetpointsize{4.00}
\gppoint{gp mark 8}{(4.294,1.868)}
\gppoint{gp mark 8}{(5.781,2.050)}
\gppoint{gp mark 8}{(7.268,2.050)}
\gppoint{gp mark 8}{(9.416,3.195)}
\gpcolor{gp lt color border}
\node[gp node right] at (8.774,2.887) {$\SI{2}{\volt}_\text{\footnotesize P-P}$};
\gpcolor{gp lt color 1}
\gpsetlinetype{gp lt plot 1}
\draw[gp path] (8.958,2.887)--(9.874,2.887);
\draw[gp path] (4.294,2.842)--(5.781,3.177)--(7.268,3.268);
\gppoint{gp mark 7}{(4.294,2.842)}
\gppoint{gp mark 7}{(5.781,3.177)}
\gppoint{gp mark 7}{(7.268,3.268)}
\gppoint{gp mark 7}{(9.416,2.887)}
\gpcolor{gp lt color border}
\node[gp node right] at (8.774,2.579) {$\SI{3}{\volt}_\text{\footnotesize P-P}$};
\gpcolor{gp lt color 2}
\gpsetlinetype{gp lt plot 2}
\draw[gp path] (8.958,2.579)--(9.874,2.579);
\draw[gp path] (4.294,3.816)--(5.781,4.273)--(7.268,4.516);
\gppoint{gp mark 5}{(4.294,3.816)}
\gppoint{gp mark 5}{(5.781,4.273)}
\gppoint{gp mark 5}{(7.268,4.516)}
\gppoint{gp mark 5}{(9.416,2.579)}
\gpcolor{gp lt color border}
\gpsetlinetype{gp lt border}
\gpsetlinewidth{1.00}
\draw[gp path] (1.320,4.790)--(1.320,0.985)--(10.242,0.985)--(10.242,4.790)--cycle;
%% coordinates of the plot area
\gpdefrectangularnode{gp plot 1}{\pgfpoint{1.320cm}{0.985cm}}{\pgfpoint{10.242cm}{4.790cm}}
\end{tikzpicture}
%% gnuplot variables

	\parbox{.6\textwidth}{
	\caption{Schematic for a frequency-modulated voltage-controlled oscillator
	as provided in the Intersil datasheet for an ICL8038 IC [1].  All
	previously-used components are unchanged, and the new resistor and
	capacitor are~\SI{10}{\kilo\ohm} and~\SI{10}{\micro\farad}, respectively.}
	\label{fig:freq_mod}}
\end{figure}
%
For this experiment all previously-used components are unchanged, and the new
resistor and capacitor are~\SI{10}{\kilo\ohm} and~\SI{10}{\micro\farad},
respectively.  In this case, the amplitude of the input signal determines the
frequency of the output, which will vary from~\SI{1}{\volt} to~\SI{4}{\volt} in
one-volt increments.  For each of these amplitudes, the input frequency will
vary from~\SI{1}{\hertz} to~\SI{4}{\hertz}.  This allows students to view the
effects of a varying amplitude on the output frequency, while observing the
effects of a varying input frequency on the rate of change of the output's duty
cycle.
