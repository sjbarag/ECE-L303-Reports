\begin{table}[H]
	\centering
	\begin{tabular}{|c|c|c|}
	\hline
	$\boldsymbol{V_\mathrm{PS}}$ \tbf{(\si{\volt})} &
		$\boldsymbol{V_\mathrm{P-P}}$ \tbf{(\si{\volt})} &
			$\boldsymbol{f}$ \tbf{(\si{\kilo\hertz})} \\ \hline
	30		& 26.41		& 3.205 \\ \hline
	25		& 21.56		& 3.210 \\ \hline
	20		& 16.88		& 3.236 \\ \hline
	15		& 12.05		& 3.226 \\ \hline
	10		& 7.344		& 3.231 \\ \hline
	5		& 2.656		& 3.273 \\ \hline
\end{tabular}
\\
	\parbox{.6\textwidth}{
	\caption[Voltage Sweep Data]{Data recorded from the voltage sweep,
		measuring the peak-to-peak voltage~($V_\mathrm{P-P}$) and
		frequency~($f$) on an oscilloscope as a function of the power supply
		voltage~($V_\mathrm{PS}$).}
	\label{tab:vSweepData}}
\end{table}

\begin{figure}[H]
	\centering
	\subfloat[Output voltage.  Note that the output voltage increases linearly
	with the supply voltage, as is expected by the characteristics of an
	opamp.]{\begin{tikzpicture}[gnuplot]
%% generated with GNUPLOT 4.4p2 (Lua 5.1.4; terminal rev. 97, script rev. 96a)
%% Sat 15 Oct 2011 09:15:25 PM EDT
\gpsolidlines
\gpcolor{gp lt color axes}
\gpsetlinetype{gp lt axes}
\gpsetlinewidth{1.00}
\draw[gp path] (1.320,0.985)--(5.162,0.985);
\gpcolor{gp lt color border}
\gpsetlinetype{gp lt border}
\draw[gp path] (1.320,0.985)--(1.500,0.985);
\node[gp node right] at (1.136,0.985) { 0};
\gpcolor{gp lt color axes}
\gpsetlinetype{gp lt axes}
\draw[gp path] (1.320,1.619)--(5.162,1.619);
\gpcolor{gp lt color border}
\gpsetlinetype{gp lt border}
\draw[gp path] (1.320,1.619)--(1.500,1.619);
\node[gp node right] at (1.136,1.619) { 5};
\gpcolor{gp lt color axes}
\gpsetlinetype{gp lt axes}
\draw[gp path] (1.320,2.253)--(5.162,2.253);
\gpcolor{gp lt color border}
\gpsetlinetype{gp lt border}
\draw[gp path] (1.320,2.253)--(1.500,2.253);
\node[gp node right] at (1.136,2.253) { 10};
\gpcolor{gp lt color axes}
\gpsetlinetype{gp lt axes}
\draw[gp path] (1.320,2.888)--(5.162,2.888);
\gpcolor{gp lt color border}
\gpsetlinetype{gp lt border}
\draw[gp path] (1.320,2.888)--(1.500,2.888);
\node[gp node right] at (1.136,2.888) { 15};
\gpcolor{gp lt color axes}
\gpsetlinetype{gp lt axes}
\draw[gp path] (1.320,3.522)--(5.162,3.522);
\gpcolor{gp lt color border}
\gpsetlinetype{gp lt border}
\draw[gp path] (1.320,3.522)--(1.500,3.522);
\node[gp node right] at (1.136,3.522) { 20};
\gpcolor{gp lt color axes}
\gpsetlinetype{gp lt axes}
\draw[gp path] (1.320,4.156)--(5.162,4.156);
\gpcolor{gp lt color border}
\gpsetlinetype{gp lt border}
\draw[gp path] (1.320,4.156)--(1.500,4.156);
\node[gp node right] at (1.136,4.156) { 25};
\gpcolor{gp lt color axes}
\gpsetlinetype{gp lt axes}
\draw[gp path] (1.320,4.790)--(5.162,4.790);
\gpcolor{gp lt color border}
\gpsetlinetype{gp lt border}
\draw[gp path] (1.320,4.790)--(1.500,4.790);
\node[gp node right] at (1.136,4.790) { 30};
\gpcolor{gp lt color axes}
\gpsetlinetype{gp lt axes}
\draw[gp path] (1.320,0.985)--(1.320,4.790);
\gpcolor{gp lt color border}
\gpsetlinetype{gp lt border}
\draw[gp path] (1.320,0.985)--(1.320,1.165);
\node[gp node center] at (1.320,0.677) { 0};
\gpcolor{gp lt color axes}
\gpsetlinetype{gp lt axes}
\draw[gp path] (1.960,0.985)--(1.960,4.790);
\gpcolor{gp lt color border}
\gpsetlinetype{gp lt border}
\draw[gp path] (1.960,0.985)--(1.960,1.165);
\node[gp node center] at (1.960,0.677) { 5};
\gpcolor{gp lt color axes}
\gpsetlinetype{gp lt axes}
\draw[gp path] (2.601,0.985)--(2.601,4.790);
\gpcolor{gp lt color border}
\gpsetlinetype{gp lt border}
\draw[gp path] (2.601,0.985)--(2.601,1.165);
\node[gp node center] at (2.601,0.677) { 10};
\gpcolor{gp lt color axes}
\gpsetlinetype{gp lt axes}
\draw[gp path] (3.241,0.985)--(3.241,4.790);
\gpcolor{gp lt color border}
\gpsetlinetype{gp lt border}
\draw[gp path] (3.241,0.985)--(3.241,1.165);
\node[gp node center] at (3.241,0.677) { 15};
\gpcolor{gp lt color axes}
\gpsetlinetype{gp lt axes}
\draw[gp path] (3.881,0.985)--(3.881,4.790);
\gpcolor{gp lt color border}
\gpsetlinetype{gp lt border}
\draw[gp path] (3.881,0.985)--(3.881,1.165);
\node[gp node center] at (3.881,0.677) { 20};
\gpcolor{gp lt color axes}
\gpsetlinetype{gp lt axes}
\draw[gp path] (4.522,0.985)--(4.522,4.790);
\gpcolor{gp lt color border}
\gpsetlinetype{gp lt border}
\draw[gp path] (4.522,0.985)--(4.522,1.165);
\node[gp node center] at (4.522,0.677) { 25};
\gpcolor{gp lt color axes}
\gpsetlinetype{gp lt axes}
\draw[gp path] (5.162,0.985)--(5.162,4.790);
\gpcolor{gp lt color border}
\gpsetlinetype{gp lt border}
\draw[gp path] (5.162,0.985)--(5.162,1.165);
\node[gp node center] at (5.162,0.677) { 30};
\draw[gp path] (1.320,4.790)--(1.320,0.985)--(5.162,0.985)--(5.162,4.790)--cycle;
\node[gp node center,rotate=-270] at (0.246,2.887) {Output Voltage, $V_\mathrm{P-P}$ (V)};
\node[gp node center] at (3.241,0.215) {Power Supply Voltage, $V_\mathrm{PS}$ (V)};
\node[gp node center] at (3.241,5.252) {Voltage Sweep: Output Voltage};
\gpcolor{gp lt color 0}
\gpsetlinetype{gp lt plot 0}
\gpsetlinewidth{3.00}
\draw[gp path] (5.162,4.335)--(4.522,3.720)--(3.881,3.126)--(3.241,2.513)--(2.601,1.916)%
  --(1.960,1.322);
\gpcolor{gp lt color border}
\gpsetlinetype{gp lt border}
\gpsetlinewidth{1.00}
\draw[gp path] (1.320,4.790)--(1.320,0.985)--(5.162,0.985)--(5.162,4.790)--cycle;
%% coordinates of the plot area
\gpdefrectangularnode{gp plot 1}{\pgfpoint{1.320cm}{0.985cm}}{\pgfpoint{5.162cm}{4.790cm}}
\end{tikzpicture}
%% gnuplot variables
\label{fig:vSweepV}}
	\qquad
	\subfloat[Output frequency.  Despite the varying power supply voltage, the
	frequency remains nearly
	constant.]{\begin{tikzpicture}[gnuplot]
%% generated with GNUPLOT 4.4p2 (Lua 5.1.4; terminal rev. 97, script rev. 96a)
%% Sat 15 Oct 2011 09:15:25 PM EDT
\gpsolidlines
\gpcolor{gp lt color axes}
\gpsetlinetype{gp lt axes}
\gpsetlinewidth{1.00}
\draw[gp path] (1.504,0.985)--(5.162,0.985);
\gpcolor{gp lt color border}
\gpsetlinetype{gp lt border}
\draw[gp path] (1.504,0.985)--(1.684,0.985);
\node[gp node right] at (1.320,0.985) { 0};
\gpcolor{gp lt color axes}
\gpsetlinetype{gp lt axes}
\draw[gp path] (1.504,1.529)--(5.162,1.529);
\gpcolor{gp lt color border}
\gpsetlinetype{gp lt border}
\draw[gp path] (1.504,1.529)--(1.684,1.529);
\node[gp node right] at (1.320,1.529) { 0.5};
\gpcolor{gp lt color axes}
\gpsetlinetype{gp lt axes}
\draw[gp path] (1.504,2.072)--(5.162,2.072);
\gpcolor{gp lt color border}
\gpsetlinetype{gp lt border}
\draw[gp path] (1.504,2.072)--(1.684,2.072);
\node[gp node right] at (1.320,2.072) { 1};
\gpcolor{gp lt color axes}
\gpsetlinetype{gp lt axes}
\draw[gp path] (1.504,2.616)--(5.162,2.616);
\gpcolor{gp lt color border}
\gpsetlinetype{gp lt border}
\draw[gp path] (1.504,2.616)--(1.684,2.616);
\node[gp node right] at (1.320,2.616) { 1.5};
\gpcolor{gp lt color axes}
\gpsetlinetype{gp lt axes}
\draw[gp path] (1.504,3.159)--(5.162,3.159);
\gpcolor{gp lt color border}
\gpsetlinetype{gp lt border}
\draw[gp path] (1.504,3.159)--(1.684,3.159);
\node[gp node right] at (1.320,3.159) { 2};
\gpcolor{gp lt color axes}
\gpsetlinetype{gp lt axes}
\draw[gp path] (1.504,3.703)--(5.162,3.703);
\gpcolor{gp lt color border}
\gpsetlinetype{gp lt border}
\draw[gp path] (1.504,3.703)--(1.684,3.703);
\node[gp node right] at (1.320,3.703) { 2.5};
\gpcolor{gp lt color axes}
\gpsetlinetype{gp lt axes}
\draw[gp path] (1.504,4.246)--(5.162,4.246);
\gpcolor{gp lt color border}
\gpsetlinetype{gp lt border}
\draw[gp path] (1.504,4.246)--(1.684,4.246);
\node[gp node right] at (1.320,4.246) { 3};
\gpcolor{gp lt color axes}
\gpsetlinetype{gp lt axes}
\draw[gp path] (1.504,4.790)--(5.162,4.790);
\gpcolor{gp lt color border}
\gpsetlinetype{gp lt border}
\draw[gp path] (1.504,4.790)--(1.684,4.790);
\node[gp node right] at (1.320,4.790) { 3.5};
\gpcolor{gp lt color axes}
\gpsetlinetype{gp lt axes}
\draw[gp path] (1.504,0.985)--(1.504,4.790);
\gpcolor{gp lt color border}
\gpsetlinetype{gp lt border}
\draw[gp path] (1.504,0.985)--(1.504,1.165);
\node[gp node center] at (1.504,0.677) { 0};
\gpcolor{gp lt color axes}
\gpsetlinetype{gp lt axes}
\draw[gp path] (2.114,0.985)--(2.114,4.790);
\gpcolor{gp lt color border}
\gpsetlinetype{gp lt border}
\draw[gp path] (2.114,0.985)--(2.114,1.165);
\node[gp node center] at (2.114,0.677) { 5};
\gpcolor{gp lt color axes}
\gpsetlinetype{gp lt axes}
\draw[gp path] (2.723,0.985)--(2.723,4.790);
\gpcolor{gp lt color border}
\gpsetlinetype{gp lt border}
\draw[gp path] (2.723,0.985)--(2.723,1.165);
\node[gp node center] at (2.723,0.677) { 10};
\gpcolor{gp lt color axes}
\gpsetlinetype{gp lt axes}
\draw[gp path] (3.333,0.985)--(3.333,4.790);
\gpcolor{gp lt color border}
\gpsetlinetype{gp lt border}
\draw[gp path] (3.333,0.985)--(3.333,1.165);
\node[gp node center] at (3.333,0.677) { 15};
\gpcolor{gp lt color axes}
\gpsetlinetype{gp lt axes}
\draw[gp path] (3.943,0.985)--(3.943,4.790);
\gpcolor{gp lt color border}
\gpsetlinetype{gp lt border}
\draw[gp path] (3.943,0.985)--(3.943,1.165);
\node[gp node center] at (3.943,0.677) { 20};
\gpcolor{gp lt color axes}
\gpsetlinetype{gp lt axes}
\draw[gp path] (4.552,0.985)--(4.552,4.790);
\gpcolor{gp lt color border}
\gpsetlinetype{gp lt border}
\draw[gp path] (4.552,0.985)--(4.552,1.165);
\node[gp node center] at (4.552,0.677) { 25};
\gpcolor{gp lt color axes}
\gpsetlinetype{gp lt axes}
\draw[gp path] (5.162,0.985)--(5.162,4.790);
\gpcolor{gp lt color border}
\gpsetlinetype{gp lt border}
\draw[gp path] (5.162,0.985)--(5.162,1.165);
\node[gp node center] at (5.162,0.677) { 30};
\draw[gp path] (1.504,4.790)--(1.504,0.985)--(5.162,0.985)--(5.162,4.790)--cycle;
\node[gp node center,rotate=-270] at (0.246,2.887) {Output Frequency, $f$ (kHz)};
\node[gp node center] at (3.333,0.215) {Power Supply Voltage, $V_\mathrm{PS}$ (V)};
\node[gp node center] at (3.333,5.252) {Voltage Sweep: Output Frequency};
\gpcolor{gp lt color 0}
\gpsetlinetype{gp lt plot 0}
\gpsetlinewidth{3.00}
\draw[gp path] (5.162,4.469)--(4.552,4.475)--(3.943,4.503)--(3.333,4.492)--(2.723,4.498)%
  --(2.114,4.543);
\gpcolor{gp lt color border}
\gpsetlinetype{gp lt border}
\gpsetlinewidth{1.00}
\draw[gp path] (1.504,4.790)--(1.504,0.985)--(5.162,0.985)--(5.162,4.790)--cycle;
%% coordinates of the plot area
\gpdefrectangularnode{gp plot 1}{\pgfpoint{1.504cm}{0.985cm}}{\pgfpoint{5.162cm}{4.790cm}}
\end{tikzpicture}
%% gnuplot variables
\label{fig:vSweepFreq}}

	\parbox{.8\textwidth}{
	\caption[Voltage Sweep Plots]{Plotted measurements from a supply voltage
	sweep.  The measured values from this test is in
	Table~\ref{tab:vSweepData}.}
	\label{fig:vSweepPlots}}
\end{figure}
