\begin{figure}[H]
	\centering
	\includegraphics[width=.6\textwidth]{img/shot/pt2shot.png}
	\parbox{.6\textwidth}{
	\caption[Oscilloscope Screenshot --- Functioning System]{Oscilloscope
	screenshot of a correctly-tuned sine wave generator, based on the system
	shown in Figures~\ref{fig:blockDiag} and~\ref{fig:weinBridgeSchem}.  Note
	the frequency of~\SI{3.205}{\kilo\hertz}, which is just~\SI{2.87}{\percent}
	below the expected value.}
	\label{fig:goodShot}}
\end{figure}

\todo[inline]{talk about pot variations}

\todo[inline]{Intro voltage sweep.  Parallel to line regulation?}
\begin{table}[H]
	\centering
	\begin{tabular}{|c|c|c|}
	\hline
	$\boldsymbol{V_\mathrm{PS}}$ \tbf{(\si{\volt})} &
		$\boldsymbol{V_\mathrm{P-P}}$ \tbf{(\si{\volt})} &
			$\boldsymbol{f}$ \tbf{(\si{\kilo\hertz})} \\ \hline
	30		& 26.41		& 3.205 \\ \hline
	25		& 21.56		& 3.210 \\ \hline
	20		& 16.88		& 3.236 \\ \hline
	15		& 12.05		& 3.226 \\ \hline
	10		& 7.344		& 3.231 \\ \hline
	5		& 2.656		& 3.273 \\ \hline
\end{tabular}
\\
	\parbox{.6\textwidth}{
	\caption[Voltage Sweep Data]{Data recorded from the voltage sweep,
		measuring the peak-to-peak voltage~($V_\mathrm{P-P}$) and
		frequency~($f$) on an oscilloscope as a function of the power supply
		voltage~($V_\mathrm{PS}$).}
	\label{tab:vSweepData}}
\end{table}
\todo[inline]{Explain voltage sweep results}
\begin{figure}[H]
	\centering
	\begin{tikzpicture}[gnuplot]
%% generated with GNUPLOT 4.4p2 (Lua 5.1.4; terminal rev. 97, script rev. 96a)
%% Sat 15 Oct 2011 11:24:37 PM EDT
\gpsolidlines
\gpcolor{gp lt color axes}
\gpsetlinetype{gp lt axes}
\gpsetlinewidth{1.00}
\draw[gp path] (1.320,0.985)--(8.830,0.985);
\gpcolor{gp lt color border}
\gpsetlinetype{gp lt border}
\draw[gp path] (1.320,0.985)--(1.500,0.985);
\node[gp node right] at (1.136,0.985) { 0};
\gpcolor{gp lt color axes}
\gpsetlinetype{gp lt axes}
\draw[gp path] (1.320,1.529)--(8.830,1.529);
\gpcolor{gp lt color border}
\gpsetlinetype{gp lt border}
\draw[gp path] (1.320,1.529)--(1.500,1.529);
\node[gp node right] at (1.136,1.529) { 5};
\gpcolor{gp lt color axes}
\gpsetlinetype{gp lt axes}
\draw[gp path] (1.320,2.072)--(8.830,2.072);
\gpcolor{gp lt color border}
\gpsetlinetype{gp lt border}
\draw[gp path] (1.320,2.072)--(1.500,2.072);
\node[gp node right] at (1.136,2.072) { 10};
\gpcolor{gp lt color axes}
\gpsetlinetype{gp lt axes}
\draw[gp path] (1.320,2.616)--(1.504,2.616);
\draw[gp path] (3.892,2.616)--(8.830,2.616);
\gpcolor{gp lt color border}
\gpsetlinetype{gp lt border}
\draw[gp path] (1.320,2.616)--(1.500,2.616);
\node[gp node right] at (1.136,2.616) { 15};
\gpcolor{gp lt color axes}
\gpsetlinetype{gp lt axes}
\draw[gp path] (1.320,3.159)--(1.504,3.159);
\draw[gp path] (3.892,3.159)--(8.830,3.159);
\gpcolor{gp lt color border}
\gpsetlinetype{gp lt border}
\draw[gp path] (1.320,3.159)--(1.500,3.159);
\node[gp node right] at (1.136,3.159) { 20};
\gpcolor{gp lt color axes}
\gpsetlinetype{gp lt axes}
\draw[gp path] (1.320,3.703)--(8.830,3.703);
\gpcolor{gp lt color border}
\gpsetlinetype{gp lt border}
\draw[gp path] (1.320,3.703)--(1.500,3.703);
\node[gp node right] at (1.136,3.703) { 25};
\gpcolor{gp lt color axes}
\gpsetlinetype{gp lt axes}
\draw[gp path] (1.320,4.246)--(8.830,4.246);
\gpcolor{gp lt color border}
\gpsetlinetype{gp lt border}
\draw[gp path] (1.320,4.246)--(1.500,4.246);
\node[gp node right] at (1.136,4.246) { 30};
\gpcolor{gp lt color axes}
\gpsetlinetype{gp lt axes}
\draw[gp path] (1.320,4.790)--(8.830,4.790);
\gpcolor{gp lt color border}
\gpsetlinetype{gp lt border}
\draw[gp path] (1.320,4.790)--(1.500,4.790);
\node[gp node right] at (1.136,4.790) { 35};
\gpcolor{gp lt color axes}
\gpsetlinetype{gp lt axes}
\draw[gp path] (1.320,0.985)--(1.320,4.790);
\gpcolor{gp lt color border}
\gpsetlinetype{gp lt border}
\draw[gp path] (1.320,0.985)--(1.320,1.165);
\node[gp node center] at (1.320,0.677) { 0};
\gpcolor{gp lt color axes}
\gpsetlinetype{gp lt axes}
\draw[gp path] (2.572,0.985)--(2.572,2.425);
\draw[gp path] (2.572,3.349)--(2.572,4.790);
\gpcolor{gp lt color border}
\gpsetlinetype{gp lt border}
\draw[gp path] (2.572,0.985)--(2.572,1.165);
\node[gp node center] at (2.572,0.677) { 5};
\gpcolor{gp lt color axes}
\gpsetlinetype{gp lt axes}
\draw[gp path] (3.823,0.985)--(3.823,2.425);
\draw[gp path] (3.823,3.349)--(3.823,4.790);
\gpcolor{gp lt color border}
\gpsetlinetype{gp lt border}
\draw[gp path] (3.823,0.985)--(3.823,1.165);
\node[gp node center] at (3.823,0.677) { 10};
\gpcolor{gp lt color axes}
\gpsetlinetype{gp lt axes}
\draw[gp path] (5.075,0.985)--(5.075,4.790);
\gpcolor{gp lt color border}
\gpsetlinetype{gp lt border}
\draw[gp path] (5.075,0.985)--(5.075,1.165);
\node[gp node center] at (5.075,0.677) { 15};
\gpcolor{gp lt color axes}
\gpsetlinetype{gp lt axes}
\draw[gp path] (6.327,0.985)--(6.327,4.790);
\gpcolor{gp lt color border}
\gpsetlinetype{gp lt border}
\draw[gp path] (6.327,0.985)--(6.327,1.165);
\node[gp node center] at (6.327,0.677) { 20};
\gpcolor{gp lt color axes}
\gpsetlinetype{gp lt axes}
\draw[gp path] (7.578,0.985)--(7.578,4.790);
\gpcolor{gp lt color border}
\gpsetlinetype{gp lt border}
\draw[gp path] (7.578,0.985)--(7.578,1.165);
\node[gp node center] at (7.578,0.677) { 25};
\gpcolor{gp lt color axes}
\gpsetlinetype{gp lt axes}
\draw[gp path] (8.830,0.985)--(8.830,4.790);
\gpcolor{gp lt color border}
\gpsetlinetype{gp lt border}
\draw[gp path] (8.830,0.985)--(8.830,1.165);
\node[gp node center] at (8.830,0.677) { 30};
\draw[gp path] (8.830,0.985)--(8.650,0.985);
\node[gp node left] at (9.014,0.985) { 0};
\draw[gp path] (8.830,1.529)--(8.650,1.529);
\node[gp node left] at (9.014,1.529) { 0.5};
\draw[gp path] (8.830,2.072)--(8.650,2.072);
\node[gp node left] at (9.014,2.072) { 1};
\draw[gp path] (8.830,2.616)--(8.650,2.616);
\node[gp node left] at (9.014,2.616) { 1.5};
\draw[gp path] (8.830,3.159)--(8.650,3.159);
\node[gp node left] at (9.014,3.159) { 2};
\draw[gp path] (8.830,3.703)--(8.650,3.703);
\node[gp node left] at (9.014,3.703) { 2.5};
\draw[gp path] (8.830,4.246)--(8.650,4.246);
\node[gp node left] at (9.014,4.246) { 3};
\draw[gp path] (8.830,4.790)--(8.650,4.790);
\node[gp node left] at (9.014,4.790) { 3.5};
\draw[gp path] (1.320,4.790)--(1.320,0.985)--(8.830,0.985)--(8.830,4.790)--cycle;
\node[gp node center,rotate=-270] at (0.246,2.887) {Output Voltage, $V_\mathrm{P-P}$ (V)};
\node[gp node center,rotate=-270] at (10.087,2.887) {Output Frequency, $f$ (kHZ)};
\node[gp node center] at (5.075,0.215) {Power Supply Voltage, $V_\mathrm{PS}$ (V)};
\node[gp node center] at (5.075,5.252) {Voltage Sweep: Output Voltage};
\draw[gp path] (1.504,2.425)--(1.504,3.349)--(3.892,3.349)--(3.892,2.425)--cycle;
\draw[gp path] (1.504,3.349)--(3.892,3.349);
\node[gp node right] at (2.608,3.041) {$V_\mathrm{P-P}$};
\gpcolor{gp lt color 0}
\gpsetlinetype{gp lt plot 0}
\gpsetlinewidth{3.00}
\draw[gp path] (2.792,3.041)--(3.708,3.041);
\draw[gp path] (8.830,3.856)--(7.578,3.329)--(6.327,2.820)--(5.075,2.295)--(3.823,1.783)%
  --(2.572,1.274);
\gpsetpointsize{4.00}
\gppoint{gp mark 1}{(8.830,3.856)}
\gppoint{gp mark 1}{(7.578,3.329)}
\gppoint{gp mark 1}{(6.327,2.820)}
\gppoint{gp mark 1}{(5.075,2.295)}
\gppoint{gp mark 1}{(3.823,1.783)}
\gppoint{gp mark 1}{(2.572,1.274)}
\gppoint{gp mark 1}{(3.250,3.041)}
\gpcolor{gp lt color border}
\node[gp node right] at (2.608,2.733) {$f$};
\gpcolor{gp lt color 1}
\gpsetlinetype{gp lt plot 1}
\draw[gp path] (2.792,2.733)--(3.708,2.733);
\draw[gp path] (8.830,4.469)--(7.578,4.475)--(6.327,4.503)--(5.075,4.492)--(3.823,4.498)%
  --(2.572,4.543);
\gppoint{gp mark 5}{(8.830,4.469)}
\gppoint{gp mark 5}{(7.578,4.475)}
\gppoint{gp mark 5}{(6.327,4.503)}
\gppoint{gp mark 5}{(5.075,4.492)}
\gppoint{gp mark 5}{(3.823,4.498)}
\gppoint{gp mark 5}{(2.572,4.543)}
\gppoint{gp mark 5}{(3.250,2.733)}
\gpcolor{gp lt color border}
\gpsetlinetype{gp lt border}
\gpsetlinewidth{1.00}
\draw[gp path] (1.320,4.790)--(1.320,0.985)--(8.830,0.985)--(8.830,4.790)--cycle;
%% coordinates of the plot area
\gpdefrectangularnode{gp plot 1}{\pgfpoint{1.320cm}{0.985cm}}{\pgfpoint{8.830cm}{4.790cm}}
\end{tikzpicture}
%% gnuplot variables

	\parbox{4.25in}{
	\caption[Voltage Sweep Plot]{Plotted measurements from a supply voltage
	sweep.  The measured values from this test are in
	Table~\ref{tab:vSweepData}.  Note that the output voltage increases
	linearly with the supply voltage, while the frequency remains nearly
	constant.}
	\label{fig:vSweepPlots}}
\end{figure}

\todo[inline,color=green]{Multi-fig of plots: 30V, 15V, 5V in appendix}

\todo[inline]{Describe effects of touching pot}
\missingfigure[figwidth=.6\textwidth]{part6 plots}


