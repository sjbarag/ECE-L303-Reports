\subsection{Step Response}
After completing a simulation of the system's step response, the plot shown
in Figure~\ref{f:stepSim} was produced by PSpice.
%
\begin{figure}[H]
	\centering
	\begin{tikzpicture}[gnuplot]
%% generated with GNUPLOT 4.4p2 (Lua 5.1.4; terminal rev. 97, script rev. 96a)
%% Sun 27 Nov 2011 05:18:37 PM EST
\gpsolidlines
\gpcolor{gp lt color axes}
\gpsetlinetype{gp lt axes}
\gpsetlinewidth{1.00}
\draw[gp path] (1.136,0.985)--(10.242,0.985);
\gpcolor{gp lt color border}
\gpsetlinetype{gp lt border}
\draw[gp path] (1.136,0.985)--(1.316,0.985);
\node[gp node right] at (0.952,0.985) {-1};
\gpcolor{gp lt color axes}
\gpsetlinetype{gp lt axes}
\draw[gp path] (1.136,1.463)--(10.242,1.463);
\gpcolor{gp lt color border}
\gpsetlinetype{gp lt border}
\draw[gp path] (1.136,1.463)--(1.316,1.463);
\node[gp node right] at (0.952,1.463) { 0};
\gpcolor{gp lt color axes}
\gpsetlinetype{gp lt axes}
\draw[gp path] (1.136,1.940)--(10.242,1.940);
\gpcolor{gp lt color border}
\gpsetlinetype{gp lt border}
\draw[gp path] (1.136,1.940)--(1.316,1.940);
\node[gp node right] at (0.952,1.940) { 1};
\gpcolor{gp lt color axes}
\gpsetlinetype{gp lt axes}
\draw[gp path] (1.136,2.418)--(10.242,2.418);
\gpcolor{gp lt color border}
\gpsetlinetype{gp lt border}
\draw[gp path] (1.136,2.418)--(1.316,2.418);
\node[gp node right] at (0.952,2.418) { 2};
\gpcolor{gp lt color axes}
\gpsetlinetype{gp lt axes}
\draw[gp path] (1.136,2.895)--(10.242,2.895);
\gpcolor{gp lt color border}
\gpsetlinetype{gp lt border}
\draw[gp path] (1.136,2.895)--(1.316,2.895);
\node[gp node right] at (0.952,2.895) { 3};
\gpcolor{gp lt color axes}
\gpsetlinetype{gp lt axes}
\draw[gp path] (1.136,3.373)--(10.242,3.373);
\gpcolor{gp lt color border}
\gpsetlinetype{gp lt border}
\draw[gp path] (1.136,3.373)--(1.316,3.373);
\node[gp node right] at (0.952,3.373) { 4};
\gpcolor{gp lt color axes}
\gpsetlinetype{gp lt axes}
\draw[gp path] (1.136,3.850)--(10.242,3.850);
\gpcolor{gp lt color border}
\gpsetlinetype{gp lt border}
\draw[gp path] (1.136,3.850)--(1.316,3.850);
\node[gp node right] at (0.952,3.850) { 5};
\gpcolor{gp lt color axes}
\gpsetlinetype{gp lt axes}
\draw[gp path] (1.136,4.328)--(10.242,4.328);
\gpcolor{gp lt color border}
\gpsetlinetype{gp lt border}
\draw[gp path] (1.136,4.328)--(1.316,4.328);
\node[gp node right] at (0.952,4.328) { 6};
\gpcolor{gp lt color axes}
\gpsetlinetype{gp lt axes}
\draw[gp path] (1.136,0.985)--(1.136,4.328);
\gpcolor{gp lt color border}
\gpsetlinetype{gp lt border}
\draw[gp path] (1.136,0.985)--(1.136,1.165);
\node[gp node center] at (1.136,0.677) { 0};
\gpcolor{gp lt color axes}
\gpsetlinetype{gp lt axes}
\draw[gp path] (2.957,0.985)--(2.957,4.328);
\gpcolor{gp lt color border}
\gpsetlinetype{gp lt border}
\draw[gp path] (2.957,0.985)--(2.957,1.165);
\node[gp node center] at (2.957,0.677) { 2};
\gpcolor{gp lt color axes}
\gpsetlinetype{gp lt axes}
\draw[gp path] (4.778,0.985)--(4.778,4.328);
\gpcolor{gp lt color border}
\gpsetlinetype{gp lt border}
\draw[gp path] (4.778,0.985)--(4.778,1.165);
\node[gp node center] at (4.778,0.677) { 4};
\gpcolor{gp lt color axes}
\gpsetlinetype{gp lt axes}
\draw[gp path] (6.600,0.985)--(6.600,4.328);
\gpcolor{gp lt color border}
\gpsetlinetype{gp lt border}
\draw[gp path] (6.600,0.985)--(6.600,1.165);
\node[gp node center] at (6.600,0.677) { 6};
\gpcolor{gp lt color axes}
\gpsetlinetype{gp lt axes}
\draw[gp path] (8.421,0.985)--(8.421,4.328);
\gpcolor{gp lt color border}
\gpsetlinetype{gp lt border}
\draw[gp path] (8.421,0.985)--(8.421,1.165);
\node[gp node center] at (8.421,0.677) { 8};
\gpcolor{gp lt color axes}
\gpsetlinetype{gp lt axes}
\draw[gp path] (10.242,0.985)--(10.242,4.328);
\gpcolor{gp lt color border}
\gpsetlinetype{gp lt border}
\draw[gp path] (10.242,0.985)--(10.242,1.165);
\node[gp node center] at (10.242,0.677) { 10};
\draw[gp path] (1.136,4.328)--(1.136,0.985)--(10.242,0.985)--(10.242,4.328)--cycle;
\node[gp node center,rotate=-270] at (0.246,2.656) {Voltage (\si{\volt})};
\node[gp node center] at (5.689,0.215) {Time (\si{\milli\second})};
\node[gp node center] at (5.689,4.790) {Simulated Step Response of Third-order Low Pass Filter};
\draw[gp path] (3.301,5.073)--(3.301,5.535)--(8.077,5.535)--(8.077,5.073)--cycle;
\draw[gp path] (3.301,5.535)--(8.077,5.535);
\node[gp node right] at (4.405,5.304) {Input};
\gpcolor{gp lt color 0}
\gpsetlinetype{gp lt plot 0}
\gpsetlinewidth{2.00}
\draw[gp path] (4.589,5.304)--(5.505,5.304);
\draw[gp path] (1.136,1.463)--(1.136,1.510)--(1.136,1.558)--(1.136,1.654)--(1.136,1.845)%
  --(1.136,2.227)--(1.136,2.991)--(1.136,3.850)--(1.137,3.850)--(1.138,3.850)--(1.139,3.850)%
  --(1.140,3.850)--(1.143,3.850)--(1.146,3.850)--(1.153,3.850)--(1.162,3.850)--(1.178,3.850)%
  --(1.203,3.850)--(1.250,3.850)--(1.321,3.850)--(1.402,3.850)--(1.499,3.850)--(1.617,3.850)%
  --(1.731,3.850)--(1.859,3.850)--(2.007,3.850)--(2.088,3.850)--(2.190,3.850)--(2.293,3.850)%
  --(2.363,3.850)--(2.365,3.850)--(2.369,3.850)--(2.377,3.850)--(2.384,3.803)--(2.384,3.569)%
  --(2.384,3.101)--(2.384,2.165)--(2.384,1.463)--(2.385,1.463)--(2.386,1.463)--(2.388,1.463)%
  --(2.393,1.463)--(2.403,1.463)--(2.421,1.463)--(2.455,1.463)--(2.506,1.463)--(2.567,1.463)%
  --(2.629,1.463)--(2.716,1.463)--(2.827,1.463)--(2.942,1.463)--(3.075,1.463)--(3.158,1.463)%
  --(3.267,1.463)--(3.334,1.463)--(3.413,1.463)--(3.484,1.463)--(3.567,1.463)--(3.640,1.510)%
  --(3.640,1.744)--(3.640,2.212)--(3.640,3.148)--(3.640,3.850)--(3.641,3.850)--(3.643,3.850)%
  --(3.645,3.850)--(3.650,3.850)--(3.659,3.850)--(3.678,3.850)--(3.708,3.850)--(3.756,3.850)%
  --(3.828,3.850)--(3.909,3.850)--(4.006,3.850)--(4.126,3.850)--(4.240,3.850)--(4.367,3.850)%
  --(4.515,3.850)--(4.594,3.850)--(4.695,3.850)--(4.798,3.850)--(4.869,3.850)--(4.871,3.850)%
  --(4.875,3.850)--(4.882,3.850)--(4.888,3.803)--(4.888,3.569)--(4.888,3.101)--(4.888,2.165)%
  --(4.888,1.463)--(4.889,1.463)--(4.890,1.463)--(4.892,1.463)--(4.897,1.463)--(4.907,1.463)%
  --(4.925,1.463)--(4.959,1.463)--(5.010,1.463)--(5.071,1.463)--(5.133,1.463)--(5.220,1.463)%
  --(5.331,1.463)--(5.446,1.463)--(5.579,1.463)--(5.662,1.463)--(5.772,1.463)--(5.838,1.463)%
  --(5.917,1.463)--(5.988,1.463)--(6.071,1.463)--(6.144,1.510)--(6.144,1.744)--(6.144,2.212)%
  --(6.144,3.148)--(6.144,3.850)--(6.145,3.850)--(6.147,3.850)--(6.149,3.850)--(6.154,3.850)%
  --(6.163,3.850)--(6.183,3.850)--(6.212,3.850)--(6.260,3.850)--(6.332,3.850)--(6.413,3.850)%
  --(6.510,3.850)--(6.630,3.850)--(6.745,3.850)--(6.872,3.850)--(7.019,3.850)--(7.098,3.850)%
  --(7.199,3.850)--(7.302,3.850)--(7.373,3.850)--(7.375,3.850)--(7.379,3.850)--(7.386,3.850)%
  --(7.392,3.803)--(7.392,3.569)--(7.392,3.101)--(7.392,2.165)--(7.392,1.463)--(7.393,1.463)%
  --(7.394,1.463)--(7.397,1.463)--(7.401,1.463)--(7.411,1.463)--(7.429,1.463)--(7.464,1.463)%
  --(7.514,1.463)--(7.575,1.463)--(7.637,1.463)--(7.724,1.463)--(7.835,1.463)--(7.950,1.463)%
  --(8.083,1.463)--(8.166,1.463)--(8.276,1.463)--(8.342,1.463)--(8.421,1.463)--(8.492,1.463)%
  --(8.575,1.463)--(8.648,1.510)--(8.648,1.744)--(8.648,2.212)--(8.648,3.148)--(8.648,3.850)%
  --(8.649,3.850)--(8.650,3.850)--(8.651,3.850)--(8.653,3.850)--(8.658,3.850)--(8.668,3.850)%
  --(8.687,3.850)--(8.717,3.850)--(8.764,3.850)--(8.836,3.850)--(8.917,3.850)--(9.014,3.850)%
  --(9.134,3.850)--(9.249,3.850)--(9.376,3.850)--(9.523,3.850)--(9.603,3.850)--(9.704,3.850)%
  --(9.806,3.850)--(9.877,3.850)--(9.879,3.850)--(9.883,3.850)--(9.890,3.850)--(9.896,3.803)%
  --(9.896,3.569)--(9.896,3.101)--(9.896,2.165)--(9.896,1.463)--(9.897,1.463)--(9.898,1.463)%
  --(9.901,1.463)--(9.906,1.463)--(9.915,1.463)--(9.933,1.463)--(9.968,1.463)--(10.018,1.463)%
  --(10.079,1.463)--(10.141,1.463)--(10.228,1.463)--(10.242,1.463);
\gpcolor{gp lt color border}
\node[gp node right] at (6.793,5.304) {Output};
\gpcolor{gp lt color 1}
\gpsetlinetype{gp lt plot 1}
\draw[gp path] (6.977,5.304)--(7.893,5.304);
\draw[gp path] (1.136,1.463)--(1.137,1.463)--(1.138,1.463)--(1.139,1.463)--(1.140,1.463)%
  --(1.143,1.463)--(1.146,1.463)--(1.153,1.463)--(1.162,1.465)--(1.178,1.473)--(1.203,1.503)%
  --(1.250,1.620)--(1.321,1.950)--(1.402,2.459)--(1.499,3.070)--(1.617,3.613)--(1.731,3.880)%
  --(1.859,3.944)--(2.007,3.889)--(2.088,3.859)--(2.190,3.835)--(2.293,3.828)--(2.363,3.831)%
  --(2.365,3.832)--(2.369,3.832)--(2.377,3.833)--(2.384,3.833)--(2.385,3.833)--(2.386,3.834)%
  --(2.388,3.834)--(2.393,3.834)--(2.403,3.834)--(2.421,3.829)--(2.455,3.792)--(2.506,3.657)%
  --(2.567,3.371)--(2.629,2.988)--(2.716,2.424)--(2.827,1.845)--(2.942,1.497)--(3.075,1.370)%
  --(3.158,1.376)--(3.267,1.428)--(3.334,1.454)--(3.413,1.475)--(3.484,1.483)--(3.567,1.482)%
  --(3.640,1.477)--(3.641,1.477)--(3.641,1.476)--(3.643,1.476)--(3.645,1.476)--(3.650,1.476)%
  --(3.659,1.476)--(3.678,1.483)--(3.708,1.515)--(3.756,1.636)--(3.828,1.967)--(3.909,2.477)%
  --(4.006,3.087)--(4.126,3.629)--(4.240,3.887)--(4.367,3.944)--(4.515,3.888)--(4.594,3.858)%
  --(4.695,3.834)--(4.798,3.828)--(4.869,3.832)--(4.871,3.832)--(4.875,3.832)--(4.882,3.833)%
  --(4.888,3.833)--(4.889,3.834)--(4.890,3.834)--(4.892,3.834)--(4.897,3.834)--(4.907,3.834)%
  --(4.925,3.829)--(4.959,3.792)--(5.010,3.657)--(5.071,3.371)--(5.133,2.988)--(5.220,2.424)%
  --(5.331,1.845)--(5.446,1.497)--(5.579,1.370)--(5.662,1.376)--(5.772,1.428)--(5.838,1.454)%
  --(5.917,1.475)--(5.988,1.483)--(6.071,1.482)--(6.144,1.477)--(6.145,1.477)--(6.145,1.476)%
  --(6.147,1.476)--(6.149,1.476)--(6.154,1.476)--(6.163,1.476)--(6.183,1.483)--(6.212,1.515)%
  --(6.260,1.636)--(6.332,1.967)--(6.413,2.477)--(6.510,3.087)--(6.630,3.629)--(6.745,3.887)%
  --(6.872,3.944)--(7.019,3.888)--(7.098,3.858)--(7.199,3.834)--(7.302,3.828)--(7.373,3.832)%
  --(7.375,3.832)--(7.379,3.832)--(7.386,3.833)--(7.392,3.833)--(7.393,3.834)--(7.394,3.834)%
  --(7.397,3.834)--(7.401,3.834)--(7.411,3.834)--(7.429,3.829)--(7.464,3.792)--(7.514,3.657)%
  --(7.575,3.371)--(7.637,2.988)--(7.724,2.424)--(7.835,1.845)--(7.950,1.497)--(8.083,1.370)%
  --(8.166,1.376)--(8.276,1.428)--(8.342,1.454)--(8.421,1.475)--(8.492,1.483)--(8.575,1.482)%
  --(8.648,1.477)--(8.649,1.477)--(8.650,1.476)--(8.651,1.476)--(8.653,1.476)--(8.658,1.476)%
  --(8.668,1.476)--(8.687,1.483)--(8.717,1.515)--(8.764,1.636)--(8.836,1.967)--(8.917,2.477)%
  --(9.014,3.087)--(9.134,3.629)--(9.249,3.887)--(9.376,3.944)--(9.523,3.888)--(9.603,3.858)%
  --(9.704,3.834)--(9.806,3.828)--(9.877,3.832)--(9.879,3.832)--(9.883,3.832)--(9.890,3.833)%
  --(9.896,3.833)--(9.897,3.833)--(9.897,3.834)--(9.898,3.834)--(9.901,3.834)--(9.906,3.834)%
  --(9.915,3.834)--(9.933,3.829)--(9.968,3.792)--(10.018,3.657)--(10.079,3.371)--(10.141,2.988)%
  --(10.228,2.424)--(10.242,2.340);
\gpcolor{gp lt color border}
\gpsetlinetype{gp lt border}
\gpsetlinewidth{1.00}
\draw[gp path] (1.136,4.328)--(1.136,0.985)--(10.242,0.985)--(10.242,4.328)--cycle;
%% coordinates of the plot area
\gpdefrectangularnode{gp plot 1}{\pgfpoint{1.136cm}{0.985cm}}{\pgfpoint{10.242cm}{4.328cm}}
\end{tikzpicture}
%% gnuplot variables

	\parbox{.6\textwidth}{
	\caption[Simulated Step Response]{PSpice representation of the step
	response for the circuit shown in Figure~\ref{f:pspiceSchem}.  Note that
	the square signal in red is the input to the system, whereas the green
	curved signal is the output.}
	\label{f:stepSim}}
\end{figure}
%
This same test was repeated for the constructed circuit shown in
Figure~\ref{f:realSchem}, while measuring the output on the in-lab
oscilloscope.  The resulting waveforms are shown in Figure~\ref{f:stepReal}.
%
\begin{figure}[H]
	\centering
	\includegraphics[width=.8\textwidth]{img/shot/squareWaveShot.png}
	\parbox{.6\textwidth}{
	\caption[Measured Step Response]{Results of the step response for a
	square wave input, as measured by an oscilloscope.  In this screenshot, the
	upper signal~(A1) is the input, while the lower~(A2) is the output.}
	\label{f:stepReal}}
\end{figure}
%
While there are similarities between the two resulting output signals, it is
significant to note that the simulated response exhibits Gibbs phenomena on the
leading transition of every wave, while the measured output does not show any
effects of Gibbs' phenomena, in fact failing to exceed the amplitude of the
input.

\subsection{Frequency Response}
\begin{figure}[H]
	\centering
	\begin{tikzpicture}[gnuplot]
%% generated with GNUPLOT 4.4p2 (Lua 5.1.4; terminal rev. 97, script rev. 96a)
%% Mon 28 Nov 2011 10:15:02 AM EST
\gpsolidlines
\gpcolor{gp lt color axes}
\gpsetlinetype{gp lt axes}
\gpsetlinewidth{1.00}
\draw[gp path] (1.504,0.985)--(10.242,0.985);
\gpcolor{gp lt color border}
\gpsetlinetype{gp lt border}
\draw[gp path] (1.504,0.985)--(1.684,0.985);
\node[gp node right] at (1.320,0.985) {-120};
\gpcolor{gp lt color axes}
\gpsetlinetype{gp lt axes}
\draw[gp path] (1.504,1.302)--(10.242,1.302);
\gpcolor{gp lt color border}
\gpsetlinetype{gp lt border}
\draw[gp path] (1.504,1.302)--(1.684,1.302);
\node[gp node right] at (1.320,1.302) {-110};
\gpcolor{gp lt color axes}
\gpsetlinetype{gp lt axes}
\draw[gp path] (1.504,1.619)--(10.242,1.619);
\gpcolor{gp lt color border}
\gpsetlinetype{gp lt border}
\draw[gp path] (1.504,1.619)--(1.684,1.619);
\node[gp node right] at (1.320,1.619) {-100};
\gpcolor{gp lt color axes}
\gpsetlinetype{gp lt axes}
\draw[gp path] (1.504,1.936)--(10.242,1.936);
\gpcolor{gp lt color border}
\gpsetlinetype{gp lt border}
\draw[gp path] (1.504,1.936)--(1.684,1.936);
\node[gp node right] at (1.320,1.936) {-90};
\gpcolor{gp lt color axes}
\gpsetlinetype{gp lt axes}
\draw[gp path] (1.504,2.253)--(10.242,2.253);
\gpcolor{gp lt color border}
\gpsetlinetype{gp lt border}
\draw[gp path] (1.504,2.253)--(1.684,2.253);
\node[gp node right] at (1.320,2.253) {-80};
\gpcolor{gp lt color axes}
\gpsetlinetype{gp lt axes}
\draw[gp path] (1.504,2.570)--(10.242,2.570);
\gpcolor{gp lt color border}
\gpsetlinetype{gp lt border}
\draw[gp path] (1.504,2.570)--(1.684,2.570);
\node[gp node right] at (1.320,2.570) {-70};
\gpcolor{gp lt color axes}
\gpsetlinetype{gp lt axes}
\draw[gp path] (1.504,2.888)--(10.242,2.888);
\gpcolor{gp lt color border}
\gpsetlinetype{gp lt border}
\draw[gp path] (1.504,2.888)--(1.684,2.888);
\node[gp node right] at (1.320,2.888) {-60};
\gpcolor{gp lt color axes}
\gpsetlinetype{gp lt axes}
\draw[gp path] (1.504,3.205)--(10.242,3.205);
\gpcolor{gp lt color border}
\gpsetlinetype{gp lt border}
\draw[gp path] (1.504,3.205)--(1.684,3.205);
\node[gp node right] at (1.320,3.205) {-50};
\gpcolor{gp lt color axes}
\gpsetlinetype{gp lt axes}
\draw[gp path] (1.504,3.522)--(10.242,3.522);
\gpcolor{gp lt color border}
\gpsetlinetype{gp lt border}
\draw[gp path] (1.504,3.522)--(1.684,3.522);
\node[gp node right] at (1.320,3.522) {-40};
\gpcolor{gp lt color axes}
\gpsetlinetype{gp lt axes}
\draw[gp path] (1.504,3.839)--(10.242,3.839);
\gpcolor{gp lt color border}
\gpsetlinetype{gp lt border}
\draw[gp path] (1.504,3.839)--(1.684,3.839);
\node[gp node right] at (1.320,3.839) {-30};
\gpcolor{gp lt color axes}
\gpsetlinetype{gp lt axes}
\draw[gp path] (1.504,4.156)--(10.242,4.156);
\gpcolor{gp lt color border}
\gpsetlinetype{gp lt border}
\draw[gp path] (1.504,4.156)--(1.684,4.156);
\node[gp node right] at (1.320,4.156) {-20};
\gpcolor{gp lt color axes}
\gpsetlinetype{gp lt axes}
\draw[gp path] (1.504,4.473)--(10.242,4.473);
\gpcolor{gp lt color border}
\gpsetlinetype{gp lt border}
\draw[gp path] (1.504,4.473)--(1.684,4.473);
\node[gp node right] at (1.320,4.473) {-10};
\gpcolor{gp lt color axes}
\gpsetlinetype{gp lt axes}
\draw[gp path] (1.504,4.790)--(10.242,4.790);
\gpcolor{gp lt color border}
\gpsetlinetype{gp lt border}
\draw[gp path] (1.504,4.790)--(1.684,4.790);
\node[gp node right] at (1.320,4.790) { 0};
\gpcolor{gp lt color axes}
\gpsetlinetype{gp lt axes}
\draw[gp path] (1.504,0.985)--(1.504,4.790);
\gpcolor{gp lt color border}
\gpsetlinetype{gp lt border}
\draw[gp path] (1.504,0.985)--(1.504,1.165);
\node[gp node center] at (1.504,0.677) { 1};
\gpcolor{gp lt color axes}
\gpsetlinetype{gp lt axes}
\draw[gp path] (2.338,0.985)--(2.338,4.790);
\gpcolor{gp lt color border}
\gpsetlinetype{gp lt border}
\draw[gp path] (2.338,0.985)--(2.338,1.165);
\node[gp node center] at (2.338,0.677) { 3};
\gpcolor{gp lt color axes}
\gpsetlinetype{gp lt axes}
\draw[gp path] (3.252,0.985)--(3.252,4.790);
\gpcolor{gp lt color border}
\gpsetlinetype{gp lt border}
\draw[gp path] (3.252,0.985)--(3.252,1.165);
\node[gp node center] at (3.252,0.677) { 10};
\gpcolor{gp lt color axes}
\gpsetlinetype{gp lt axes}
\draw[gp path] (4.085,0.985)--(4.085,4.790);
\gpcolor{gp lt color border}
\gpsetlinetype{gp lt border}
\draw[gp path] (4.085,0.985)--(4.085,1.165);
\node[gp node center] at (4.085,0.677) { 30};
\gpcolor{gp lt color axes}
\gpsetlinetype{gp lt axes}
\draw[gp path] (4.999,0.985)--(4.999,4.790);
\gpcolor{gp lt color border}
\gpsetlinetype{gp lt border}
\draw[gp path] (4.999,0.985)--(4.999,1.165);
\node[gp node center] at (4.999,0.677) { 100};
\gpcolor{gp lt color axes}
\gpsetlinetype{gp lt axes}
\draw[gp path] (5.833,0.985)--(5.833,4.790);
\gpcolor{gp lt color border}
\gpsetlinetype{gp lt border}
\draw[gp path] (5.833,0.985)--(5.833,1.165);
\node[gp node center] at (5.833,0.677) { 300};
\gpcolor{gp lt color axes}
\gpsetlinetype{gp lt axes}
\draw[gp path] (6.747,0.985)--(6.747,4.790);
\gpcolor{gp lt color border}
\gpsetlinetype{gp lt border}
\draw[gp path] (6.747,0.985)--(6.747,1.165);
\node[gp node center] at (6.747,0.677) { 1000};
\gpcolor{gp lt color axes}
\gpsetlinetype{gp lt axes}
\draw[gp path] (7.581,0.985)--(7.581,4.790);
\gpcolor{gp lt color border}
\gpsetlinetype{gp lt border}
\draw[gp path] (7.581,0.985)--(7.581,1.165);
\node[gp node center] at (7.581,0.677) { 3000};
\gpcolor{gp lt color axes}
\gpsetlinetype{gp lt axes}
\draw[gp path] (8.494,0.985)--(8.494,4.790);
\gpcolor{gp lt color border}
\gpsetlinetype{gp lt border}
\draw[gp path] (8.494,0.985)--(8.494,1.165);
\node[gp node center] at (8.494,0.677) { 10000};
\gpcolor{gp lt color axes}
\gpsetlinetype{gp lt axes}
\draw[gp path] (9.328,0.985)--(9.328,4.790);
\gpcolor{gp lt color border}
\gpsetlinetype{gp lt border}
\draw[gp path] (9.328,0.985)--(9.328,1.165);
\node[gp node center] at (9.328,0.677) { 30000};
\gpcolor{gp lt color axes}
\gpsetlinetype{gp lt axes}
\draw[gp path] (10.242,0.985)--(10.242,4.790);
\gpcolor{gp lt color border}
\gpsetlinetype{gp lt border}
\draw[gp path] (10.242,0.985)--(10.242,1.165);
\node[gp node center] at (10.242,0.677) { 100000};
\draw[gp path] (1.504,4.790)--(1.504,0.985)--(10.242,0.985)--(10.242,4.790)--cycle;
\node[gp node center,rotate=-270] at (0.246,2.887) {Output Gain, $A$, (\si{\decibel})};
\node[gp node center] at (5.873,0.215) {Input Frequency, $f$ (\si{\hertz})};
\node[gp node center] at (5.873,5.252) {Simulated Frequency Response of Third-order Low Pass Filter};
\gpcolor{gp lt color 0}
\gpsetlinetype{gp lt plot 0}
\gpsetlinewidth{3.00}
\draw[gp path] (1.504,4.790)--(1.511,4.790)--(1.518,4.790)--(1.525,4.790)--(1.532,4.790)%
  --(1.539,4.790)--(1.546,4.790)--(1.553,4.790)--(1.560,4.790)--(1.567,4.790)--(1.574,4.790)%
  --(1.581,4.790)--(1.588,4.790)--(1.595,4.790)--(1.602,4.790)--(1.609,4.790)--(1.616,4.790)%
  --(1.623,4.790)--(1.630,4.790)--(1.637,4.790)--(1.644,4.790)--(1.651,4.790)--(1.658,4.790)%
  --(1.665,4.790)--(1.672,4.790)--(1.679,4.790)--(1.686,4.790)--(1.693,4.790)--(1.700,4.790)%
  --(1.707,4.790)--(1.714,4.790)--(1.721,4.790)--(1.728,4.790)--(1.735,4.790)--(1.742,4.790)%
  --(1.749,4.790)--(1.756,4.790)--(1.763,4.790)--(1.770,4.790)--(1.777,4.790)--(1.784,4.790)%
  --(1.791,4.790)--(1.798,4.790)--(1.805,4.790)--(1.812,4.790)--(1.819,4.790)--(1.826,4.790)%
  --(1.833,4.790)--(1.840,4.790)--(1.847,4.790)--(1.854,4.790)--(1.861,4.790)--(1.868,4.790)%
  --(1.874,4.790)--(1.881,4.790)--(1.888,4.790)--(1.895,4.790)--(1.902,4.790)--(1.909,4.790)%
  --(1.916,4.790)--(1.923,4.790)--(1.930,4.790)--(1.937,4.790)--(1.944,4.790)--(1.951,4.790)%
  --(1.958,4.790)--(1.965,4.790)--(1.972,4.790)--(1.979,4.790)--(1.986,4.790)--(1.993,4.790)%
  --(2.000,4.790)--(2.007,4.790)--(2.014,4.790)--(2.021,4.790)--(2.028,4.790)--(2.035,4.790)%
  --(2.042,4.790)--(2.049,4.790)--(2.056,4.790)--(2.063,4.790)--(2.070,4.790)--(2.077,4.790)%
  --(2.084,4.790)--(2.091,4.790)--(2.098,4.790)--(2.105,4.790)--(2.112,4.790)--(2.119,4.790)%
  --(2.126,4.790)--(2.133,4.790)--(2.140,4.790)--(2.147,4.790)--(2.154,4.790)--(2.161,4.790)%
  --(2.168,4.790)--(2.175,4.790)--(2.182,4.790)--(2.189,4.790)--(2.196,4.790)--(2.203,4.790)%
  --(2.210,4.790)--(2.217,4.790)--(2.224,4.790)--(2.231,4.790)--(2.238,4.790)--(2.245,4.790)%
  --(2.252,4.790)--(2.259,4.790)--(2.266,4.790)--(2.273,4.790)--(2.280,4.790)--(2.287,4.790)%
  --(2.294,4.790)--(2.301,4.790)--(2.308,4.790)--(2.315,4.790)--(2.322,4.790)--(2.329,4.790)%
  --(2.336,4.790)--(2.343,4.790)--(2.350,4.790)--(2.357,4.790)--(2.364,4.790)--(2.371,4.790)%
  --(2.378,4.790)--(2.385,4.790)--(2.392,4.790)--(2.399,4.790)--(2.406,4.790)--(2.413,4.790)%
  --(2.420,4.790)--(2.427,4.790)--(2.434,4.790)--(2.441,4.790)--(2.448,4.790)--(2.455,4.790)%
  --(2.462,4.790)--(2.469,4.790)--(2.476,4.790)--(2.483,4.790)--(2.490,4.790)--(2.497,4.790)%
  --(2.504,4.790)--(2.511,4.790)--(2.518,4.790)--(2.525,4.790)--(2.532,4.790)--(2.539,4.790)%
  --(2.546,4.790)--(2.553,4.790)--(2.560,4.790)--(2.567,4.790)--(2.574,4.790)--(2.581,4.790)%
  --(2.588,4.790)--(2.595,4.790)--(2.601,4.790)--(2.608,4.790)--(2.615,4.790)--(2.622,4.790)%
  --(2.629,4.790)--(2.636,4.790)--(2.643,4.790)--(2.650,4.790)--(2.657,4.790)--(2.664,4.790)%
  --(2.671,4.790)--(2.678,4.790)--(2.685,4.790)--(2.692,4.790)--(2.699,4.790)--(2.706,4.790)%
  --(2.713,4.790)--(2.720,4.790)--(2.727,4.790)--(2.734,4.790)--(2.741,4.790)--(2.748,4.790)%
  --(2.755,4.790)--(2.762,4.790)--(2.769,4.790)--(2.776,4.790)--(2.783,4.790)--(2.790,4.790)%
  --(2.797,4.790)--(2.804,4.790)--(2.811,4.790)--(2.818,4.790)--(2.825,4.790)--(2.832,4.790)%
  --(2.839,4.790)--(2.846,4.790)--(2.853,4.790)--(2.860,4.790)--(2.867,4.790)--(2.874,4.790)%
  --(2.881,4.790)--(2.888,4.790)--(2.895,4.790)--(2.902,4.790)--(2.909,4.790)--(2.916,4.790)%
  --(2.923,4.790)--(2.930,4.790)--(2.937,4.790)--(2.944,4.790)--(2.951,4.790)--(2.958,4.790)%
  --(2.965,4.790)--(2.972,4.790)--(2.979,4.790)--(2.986,4.790)--(2.993,4.790)--(3.000,4.790)%
  --(3.007,4.790)--(3.014,4.790)--(3.021,4.790)--(3.028,4.790)--(3.035,4.790)--(3.042,4.790)%
  --(3.049,4.790)--(3.056,4.790)--(3.063,4.790)--(3.070,4.790)--(3.077,4.790)--(3.084,4.790)%
  --(3.091,4.790)--(3.098,4.790)--(3.105,4.790)--(3.112,4.790)--(3.119,4.790)--(3.126,4.790)%
  --(3.133,4.790)--(3.140,4.790)--(3.147,4.790)--(3.154,4.790)--(3.161,4.790)--(3.168,4.790)%
  --(3.175,4.790)--(3.182,4.790)--(3.189,4.790)--(3.196,4.790)--(3.203,4.790)--(3.210,4.790)%
  --(3.217,4.790)--(3.224,4.790)--(3.231,4.790)--(3.238,4.790)--(3.245,4.790)--(3.252,4.790)%
  --(3.259,4.790)--(3.266,4.790)--(3.273,4.790)--(3.280,4.790)--(3.287,4.790)--(3.294,4.790)%
  --(3.301,4.790)--(3.308,4.790)--(3.315,4.790)--(3.322,4.790)--(3.328,4.790)--(3.335,4.790)%
  --(3.342,4.790)--(3.349,4.790)--(3.356,4.790)--(3.363,4.790)--(3.370,4.790)--(3.377,4.790)%
  --(3.384,4.790)--(3.391,4.790)--(3.398,4.790)--(3.405,4.790)--(3.412,4.790)--(3.419,4.790)%
  --(3.426,4.790)--(3.433,4.790)--(3.440,4.790)--(3.447,4.790)--(3.454,4.790)--(3.461,4.790)%
  --(3.468,4.790)--(3.475,4.790)--(3.482,4.790)--(3.489,4.790)--(3.496,4.790)--(3.503,4.790)%
  --(3.510,4.790)--(3.517,4.790)--(3.524,4.790)--(3.531,4.790)--(3.538,4.790)--(3.545,4.790)%
  --(3.552,4.790)--(3.559,4.790)--(3.566,4.790)--(3.573,4.790)--(3.580,4.790)--(3.587,4.790)%
  --(3.594,4.790)--(3.601,4.790)--(3.608,4.790)--(3.615,4.790)--(3.622,4.790)--(3.629,4.790)%
  --(3.636,4.790)--(3.643,4.790)--(3.650,4.790)--(3.657,4.790)--(3.664,4.790)--(3.671,4.790)%
  --(3.678,4.790)--(3.685,4.790)--(3.692,4.790)--(3.699,4.790)--(3.706,4.790)--(3.713,4.790)%
  --(3.720,4.790)--(3.727,4.790)--(3.734,4.790)--(3.741,4.790)--(3.748,4.790)--(3.755,4.790)%
  --(3.762,4.790)--(3.769,4.790)--(3.776,4.790)--(3.783,4.790)--(3.790,4.790)--(3.797,4.790)%
  --(3.804,4.790)--(3.811,4.790)--(3.818,4.790)--(3.825,4.790)--(3.832,4.790)--(3.839,4.790)%
  --(3.846,4.790)--(3.853,4.790)--(3.860,4.790)--(3.867,4.790)--(3.874,4.790)--(3.881,4.790)%
  --(3.888,4.790)--(3.895,4.790)--(3.902,4.790)--(3.909,4.790)--(3.916,4.790)--(3.923,4.790)%
  --(3.930,4.790)--(3.937,4.790)--(3.944,4.790)--(3.951,4.790)--(3.958,4.790)--(3.965,4.790)%
  --(3.972,4.790)--(3.979,4.790)--(3.986,4.790)--(3.993,4.790)--(4.000,4.790)--(4.007,4.790)%
  --(4.014,4.790)--(4.021,4.790)--(4.028,4.790)--(4.035,4.790)--(4.042,4.790)--(4.049,4.790)%
  --(4.055,4.790)--(4.062,4.790)--(4.069,4.790)--(4.076,4.790)--(4.083,4.790)--(4.090,4.790)%
  --(4.097,4.790)--(4.104,4.790)--(4.111,4.790)--(4.118,4.790)--(4.125,4.790)--(4.132,4.790)%
  --(4.139,4.790)--(4.146,4.790)--(4.153,4.790)--(4.160,4.790)--(4.167,4.790)--(4.174,4.790)%
  --(4.181,4.790)--(4.188,4.790)--(4.195,4.790)--(4.202,4.790)--(4.209,4.790)--(4.216,4.790)%
  --(4.223,4.790)--(4.230,4.790)--(4.237,4.790)--(4.244,4.790)--(4.251,4.790)--(4.258,4.790)%
  --(4.265,4.790)--(4.272,4.790)--(4.279,4.790)--(4.286,4.790)--(4.293,4.790)--(4.300,4.790)%
  --(4.307,4.790)--(4.314,4.790)--(4.321,4.790)--(4.328,4.790)--(4.335,4.790)--(4.342,4.790)%
  --(4.349,4.790)--(4.356,4.790)--(4.363,4.790)--(4.370,4.790)--(4.377,4.790)--(4.384,4.790)%
  --(4.391,4.790)--(4.398,4.790)--(4.405,4.790)--(4.412,4.790)--(4.419,4.790)--(4.426,4.790)%
  --(4.433,4.790)--(4.440,4.790)--(4.447,4.790)--(4.454,4.790)--(4.461,4.790)--(4.468,4.790)%
  --(4.475,4.790)--(4.482,4.790)--(4.489,4.790)--(4.496,4.790)--(4.503,4.790)--(4.510,4.790)%
  --(4.517,4.790)--(4.524,4.790)--(4.531,4.790)--(4.538,4.790)--(4.545,4.790)--(4.552,4.790)%
  --(4.559,4.790)--(4.566,4.790)--(4.573,4.790)--(4.580,4.790)--(4.587,4.790)--(4.594,4.790)%
  --(4.601,4.790)--(4.608,4.790)--(4.615,4.790)--(4.622,4.790)--(4.629,4.790)--(4.636,4.790)%
  --(4.643,4.790)--(4.650,4.790)--(4.657,4.790)--(4.664,4.790)--(4.671,4.790)--(4.678,4.790)%
  --(4.685,4.790)--(4.692,4.790)--(4.699,4.790)--(4.706,4.790)--(4.713,4.790)--(4.720,4.789)%
  --(4.727,4.789)--(4.734,4.789)--(4.741,4.789)--(4.748,4.789)--(4.755,4.789)--(4.762,4.789)%
  --(4.769,4.789)--(4.776,4.789)--(4.782,4.789)--(4.789,4.789)--(4.796,4.789)--(4.803,4.789)%
  --(4.810,4.789)--(4.817,4.789)--(4.824,4.789)--(4.831,4.789)--(4.838,4.789)--(4.845,4.789)%
  --(4.852,4.789)--(4.859,4.789)--(4.866,4.789)--(4.873,4.789)--(4.880,4.789)--(4.887,4.789)%
  --(4.894,4.789)--(4.901,4.789)--(4.908,4.789)--(4.915,4.789)--(4.922,4.789)--(4.929,4.789)%
  --(4.936,4.789)--(4.943,4.789)--(4.950,4.789)--(4.957,4.789)--(4.964,4.789)--(4.971,4.789)%
  --(4.978,4.789)--(4.985,4.789)--(4.992,4.789)--(4.999,4.789)--(5.006,4.789)--(5.013,4.789)%
  --(5.020,4.789)--(5.027,4.789)--(5.034,4.789)--(5.041,4.789)--(5.048,4.789)--(5.055,4.789)%
  --(5.062,4.789)--(5.069,4.789)--(5.076,4.789)--(5.083,4.789)--(5.090,4.789)--(5.097,4.789)%
  --(5.104,4.789)--(5.111,4.789)--(5.118,4.789)--(5.125,4.789)--(5.132,4.789)--(5.139,4.788)%
  --(5.146,4.788)--(5.153,4.788)--(5.160,4.788)--(5.167,4.788)--(5.174,4.788)--(5.181,4.788)%
  --(5.188,4.788)--(5.195,4.788)--(5.202,4.788)--(5.209,4.788)--(5.216,4.788)--(5.223,4.788)%
  --(5.230,4.788)--(5.237,4.788)--(5.244,4.788)--(5.251,4.788)--(5.258,4.788)--(5.265,4.788)%
  --(5.272,4.788)--(5.279,4.788)--(5.286,4.788)--(5.293,4.788)--(5.300,4.788)--(5.307,4.788)%
  --(5.314,4.788)--(5.321,4.788)--(5.328,4.787)--(5.335,4.787)--(5.342,4.787)--(5.349,4.787)%
  --(5.356,4.787)--(5.363,4.787)--(5.370,4.787)--(5.377,4.787)--(5.384,4.787)--(5.391,4.787)%
  --(5.398,4.787)--(5.405,4.787)--(5.412,4.787)--(5.419,4.787)--(5.426,4.787)--(5.433,4.787)%
  --(5.440,4.787)--(5.447,4.787)--(5.454,4.787)--(5.461,4.786)--(5.468,4.786)--(5.475,4.786)%
  --(5.482,4.786)--(5.489,4.786)--(5.496,4.786)--(5.503,4.786)--(5.509,4.786)--(5.516,4.786)%
  --(5.523,4.786)--(5.530,4.786)--(5.537,4.786)--(5.544,4.786)--(5.551,4.786)--(5.558,4.785)%
  --(5.565,4.785)--(5.572,4.785)--(5.579,4.785)--(5.586,4.785)--(5.593,4.785)--(5.600,4.785)%
  --(5.607,4.785)--(5.614,4.785)--(5.621,4.785)--(5.628,4.785)--(5.635,4.784)--(5.642,4.784)%
  --(5.649,4.784)--(5.656,4.784)--(5.663,4.784)--(5.670,4.784)--(5.677,4.784)--(5.684,4.784)%
  --(5.691,4.784)--(5.698,4.783)--(5.705,4.783)--(5.712,4.783)--(5.719,4.783)--(5.726,4.783)%
  --(5.733,4.783)--(5.740,4.783)--(5.747,4.783)--(5.754,4.782)--(5.761,4.782)--(5.768,4.782)%
  --(5.775,4.782)--(5.782,4.782)--(5.789,4.782)--(5.796,4.782)--(5.803,4.781)--(5.810,4.781)%
  --(5.817,4.781)--(5.824,4.781)--(5.831,4.781)--(5.838,4.781)--(5.845,4.780)--(5.852,4.780)%
  --(5.859,4.780)--(5.866,4.780)--(5.873,4.780)--(5.880,4.780)--(5.887,4.779)--(5.894,4.779)%
  --(5.901,4.779)--(5.908,4.779)--(5.915,4.779)--(5.922,4.778)--(5.929,4.778)--(5.936,4.778)%
  --(5.943,4.778)--(5.950,4.778)--(5.957,4.777)--(5.964,4.777)--(5.971,4.777)--(5.978,4.777)%
  --(5.985,4.776)--(5.992,4.776)--(5.999,4.776)--(6.006,4.776)--(6.013,4.775)--(6.020,4.775)%
  --(6.027,4.775)--(6.034,4.775)--(6.041,4.774)--(6.048,4.774)--(6.055,4.774)--(6.062,4.773)%
  --(6.069,4.773)--(6.076,4.773)--(6.083,4.773)--(6.090,4.772)--(6.097,4.772)--(6.104,4.772)%
  --(6.111,4.771)--(6.118,4.771)--(6.125,4.771)--(6.132,4.770)--(6.139,4.770)--(6.146,4.769)%
  --(6.153,4.769)--(6.160,4.769)--(6.167,4.768)--(6.174,4.768)--(6.181,4.768)--(6.188,4.767)%
  --(6.195,4.767)--(6.202,4.766)--(6.209,4.766)--(6.216,4.765)--(6.223,4.765)--(6.230,4.764)%
  --(6.237,4.764)--(6.243,4.764)--(6.250,4.763)--(6.257,4.763)--(6.264,4.762)--(6.271,4.762)%
  --(6.278,4.761)--(6.285,4.761)--(6.292,4.760)--(6.299,4.759)--(6.306,4.759)--(6.313,4.758)%
  --(6.320,4.758)--(6.327,4.757)--(6.334,4.756)--(6.341,4.756)--(6.348,4.755)--(6.355,4.755)%
  --(6.362,4.754)--(6.369,4.753)--(6.376,4.752)--(6.383,4.752)--(6.390,4.751)--(6.397,4.750)%
  --(6.404,4.750)--(6.411,4.749)--(6.418,4.748)--(6.425,4.747)--(6.432,4.746)--(6.439,4.745)%
  --(6.446,4.745)--(6.453,4.744)--(6.460,4.743)--(6.467,4.742)--(6.474,4.741)--(6.481,4.740)%
  --(6.488,4.739)--(6.495,4.738)--(6.502,4.737)--(6.509,4.736)--(6.516,4.735)--(6.523,4.733)%
  --(6.530,4.732)--(6.537,4.731)--(6.544,4.730)--(6.551,4.729)--(6.558,4.727)--(6.565,4.726)%
  --(6.572,4.725)--(6.579,4.723)--(6.586,4.722)--(6.593,4.720)--(6.600,4.719)--(6.607,4.717)%
  --(6.614,4.716)--(6.621,4.714)--(6.628,4.712)--(6.635,4.711)--(6.642,4.709)--(6.649,4.707)%
  --(6.656,4.705)--(6.663,4.703)--(6.670,4.701)--(6.677,4.699)--(6.684,4.697)--(6.691,4.695)%
  --(6.698,4.693)--(6.705,4.691)--(6.712,4.689)--(6.719,4.686)--(6.726,4.684)--(6.733,4.681)%
  --(6.740,4.679)--(6.747,4.676)--(6.754,4.674)--(6.761,4.671)--(6.768,4.668)--(6.775,4.665)%
  --(6.782,4.662)--(6.789,4.659)--(6.796,4.656)--(6.803,4.653)--(6.810,4.650)--(6.817,4.647)%
  --(6.824,4.643)--(6.831,4.640)--(6.838,4.636)--(6.845,4.633)--(6.852,4.629)--(6.859,4.625)%
  --(6.866,4.621)--(6.873,4.617)--(6.880,4.614)--(6.887,4.609)--(6.894,4.605)--(6.901,4.601)%
  --(6.908,4.597)--(6.915,4.592)--(6.922,4.588)--(6.929,4.583)--(6.936,4.579)--(6.943,4.574)%
  --(6.950,4.569)--(6.957,4.564)--(6.964,4.559)--(6.970,4.554)--(6.977,4.549)--(6.984,4.544)%
  --(6.991,4.539)--(6.998,4.534)--(7.005,4.528)--(7.012,4.523)--(7.019,4.517)--(7.026,4.512)%
  --(7.033,4.506)--(7.040,4.501)--(7.047,4.495)--(7.054,4.489)--(7.061,4.483)--(7.068,4.477)%
  --(7.075,4.471)--(7.082,4.465)--(7.089,4.459)--(7.096,4.453)--(7.103,4.447)--(7.110,4.441)%
  --(7.117,4.434)--(7.124,4.428)--(7.131,4.421)--(7.138,4.415)--(7.145,4.409)--(7.152,4.402)%
  --(7.159,4.395)--(7.166,4.389)--(7.173,4.382)--(7.180,4.376)--(7.187,4.369)--(7.194,4.362)%
  --(7.201,4.355)--(7.208,4.348)--(7.215,4.342)--(7.222,4.335)--(7.229,4.328)--(7.236,4.321)%
  --(7.243,4.314)--(7.250,4.307)--(7.257,4.300)--(7.264,4.293)--(7.271,4.286)--(7.278,4.279)%
  --(7.285,4.272)--(7.292,4.265)--(7.299,4.257)--(7.306,4.250)--(7.313,4.243)--(7.320,4.236)%
  --(7.327,4.229)--(7.334,4.221)--(7.341,4.214)--(7.348,4.207)--(7.355,4.200)--(7.362,4.192)%
  --(7.369,4.185)--(7.376,4.178)--(7.383,4.170)--(7.390,4.163)--(7.397,4.156)--(7.404,4.148)%
  --(7.411,4.141)--(7.418,4.134)--(7.425,4.126)--(7.432,4.119)--(7.439,4.111)--(7.446,4.104)%
  --(7.453,4.097)--(7.460,4.089)--(7.467,4.082)--(7.474,4.074)--(7.481,4.067)--(7.488,4.059)%
  --(7.495,4.052)--(7.502,4.044)--(7.509,4.037)--(7.516,4.029)--(7.523,4.022)--(7.530,4.014)%
  --(7.537,4.007)--(7.544,3.999)--(7.551,3.992)--(7.558,3.984)--(7.565,3.977)--(7.572,3.969)%
  --(7.579,3.962)--(7.586,3.954)--(7.593,3.947)--(7.600,3.939)--(7.607,3.932)--(7.614,3.924)%
  --(7.621,3.916)--(7.628,3.909)--(7.635,3.901)--(7.642,3.894)--(7.649,3.886)--(7.656,3.879)%
  --(7.663,3.871)--(7.670,3.864)--(7.677,3.856)--(7.684,3.848)--(7.691,3.841)--(7.697,3.833)%
  --(7.704,3.826)--(7.711,3.818)--(7.718,3.811)--(7.725,3.803)--(7.732,3.795)--(7.739,3.788)%
  --(7.746,3.780)--(7.753,3.773)--(7.760,3.765)--(7.767,3.757)--(7.774,3.750)--(7.781,3.742)%
  --(7.788,3.735)--(7.795,3.727)--(7.802,3.719)--(7.809,3.712)--(7.816,3.704)--(7.823,3.697)%
  --(7.830,3.689)--(7.837,3.681)--(7.844,3.674)--(7.851,3.666)--(7.858,3.659)--(7.865,3.651)%
  --(7.872,3.643)--(7.879,3.636)--(7.886,3.628)--(7.893,3.621)--(7.900,3.613)--(7.907,3.605)%
  --(7.914,3.598)--(7.921,3.590)--(7.928,3.583)--(7.935,3.575)--(7.942,3.567)--(7.949,3.560)%
  --(7.956,3.552)--(7.963,3.545)--(7.970,3.537)--(7.977,3.529)--(7.984,3.522)--(7.991,3.514)%
  --(7.998,3.507)--(8.005,3.499)--(8.012,3.491)--(8.019,3.484)--(8.026,3.476)--(8.033,3.469)%
  --(8.040,3.461)--(8.047,3.453)--(8.054,3.446)--(8.061,3.438)--(8.068,3.430)--(8.075,3.423)%
  --(8.082,3.415)--(8.089,3.408)--(8.096,3.400)--(8.103,3.392)--(8.110,3.385)--(8.117,3.377)%
  --(8.124,3.370)--(8.131,3.362)--(8.138,3.354)--(8.145,3.347)--(8.152,3.339)--(8.159,3.332)%
  --(8.166,3.324)--(8.173,3.316)--(8.180,3.309)--(8.187,3.301)--(8.194,3.293)--(8.201,3.286)%
  --(8.208,3.278)--(8.215,3.271)--(8.222,3.263)--(8.229,3.255)--(8.236,3.248)--(8.243,3.240)%
  --(8.250,3.233)--(8.257,3.225)--(8.264,3.217)--(8.271,3.210)--(8.278,3.202)--(8.285,3.194)%
  --(8.292,3.187)--(8.299,3.179)--(8.306,3.172)--(8.313,3.164)--(8.320,3.156)--(8.327,3.149)%
  --(8.334,3.141)--(8.341,3.134)--(8.348,3.126)--(8.355,3.118)--(8.362,3.111)--(8.369,3.103)%
  --(8.376,3.096)--(8.383,3.088)--(8.390,3.080)--(8.397,3.073)--(8.404,3.065)--(8.411,3.057)%
  --(8.418,3.050)--(8.424,3.042)--(8.431,3.035)--(8.438,3.027)--(8.445,3.019)--(8.452,3.012)%
  --(8.459,3.004)--(8.466,2.997)--(8.473,2.989)--(8.480,2.981)--(8.487,2.974)--(8.494,2.966)%
  --(8.501,2.959)--(8.508,2.951)--(8.515,2.943)--(8.522,2.936)--(8.529,2.928)--(8.536,2.920)%
  --(8.543,2.913)--(8.550,2.905)--(8.557,2.898)--(8.564,2.890)--(8.571,2.882)--(8.578,2.875)%
  --(8.585,2.867)--(8.592,2.860)--(8.599,2.852)--(8.606,2.844)--(8.613,2.837)--(8.620,2.829)%
  --(8.627,2.822)--(8.634,2.814)--(8.641,2.806)--(8.648,2.799)--(8.655,2.791)--(8.662,2.783)%
  --(8.669,2.776)--(8.676,2.768)--(8.683,2.761)--(8.690,2.753)--(8.697,2.745)--(8.704,2.738)%
  --(8.711,2.730)--(8.718,2.723)--(8.725,2.715)--(8.732,2.707)--(8.739,2.700)--(8.746,2.692)%
  --(8.753,2.685)--(8.760,2.677)--(8.767,2.669)--(8.774,2.662)--(8.781,2.654)--(8.788,2.646)%
  --(8.795,2.639)--(8.802,2.631)--(8.809,2.624)--(8.816,2.616)--(8.823,2.608)--(8.830,2.601)%
  --(8.837,2.593)--(8.844,2.586)--(8.851,2.578)--(8.858,2.570)--(8.865,2.563)--(8.872,2.555)%
  --(8.879,2.548)--(8.886,2.540)--(8.893,2.532)--(8.900,2.525)--(8.907,2.517)--(8.914,2.509)%
  --(8.921,2.502)--(8.928,2.494)--(8.935,2.487)--(8.942,2.479)--(8.949,2.471)--(8.956,2.464)%
  --(8.963,2.456)--(8.970,2.449)--(8.977,2.441)--(8.984,2.433)--(8.991,2.426)--(8.998,2.418)%
  --(9.005,2.411)--(9.012,2.403)--(9.019,2.395)--(9.026,2.388)--(9.033,2.380)--(9.040,2.372)%
  --(9.047,2.365)--(9.054,2.357)--(9.061,2.350)--(9.068,2.342)--(9.075,2.334)--(9.082,2.327)%
  --(9.089,2.319)--(9.096,2.312)--(9.103,2.304)--(9.110,2.296)--(9.117,2.289)--(9.124,2.281)%
  --(9.131,2.274)--(9.138,2.266)--(9.145,2.258)--(9.151,2.251)--(9.158,2.243)--(9.165,2.235)%
  --(9.172,2.228)--(9.179,2.220)--(9.186,2.213)--(9.193,2.205)--(9.200,2.197)--(9.207,2.190)%
  --(9.214,2.182)--(9.221,2.175)--(9.228,2.167)--(9.235,2.159)--(9.242,2.152)--(9.249,2.144)%
  --(9.256,2.137)--(9.263,2.129)--(9.270,2.121)--(9.277,2.114)--(9.284,2.106)--(9.291,2.099)%
  --(9.298,2.091)--(9.305,2.083)--(9.312,2.076)--(9.319,2.068)--(9.326,2.060)--(9.333,2.053)%
  --(9.340,2.045)--(9.347,2.038)--(9.354,2.030)--(9.361,2.022)--(9.368,2.015)--(9.375,2.007)%
  --(9.382,2.000)--(9.389,1.992)--(9.396,1.984)--(9.403,1.977)--(9.410,1.969)--(9.417,1.962)%
  --(9.424,1.954)--(9.431,1.946)--(9.438,1.939)--(9.445,1.931)--(9.452,1.923)--(9.459,1.916)%
  --(9.466,1.908)--(9.473,1.901)--(9.480,1.893)--(9.487,1.885)--(9.494,1.878)--(9.501,1.870)%
  --(9.508,1.863)--(9.515,1.855)--(9.522,1.847)--(9.529,1.840)--(9.536,1.832)--(9.543,1.825)%
  --(9.550,1.817)--(9.557,1.809)--(9.564,1.802)--(9.571,1.794)--(9.578,1.786)--(9.585,1.779)%
  --(9.592,1.771)--(9.599,1.764)--(9.606,1.756)--(9.613,1.748)--(9.620,1.741)--(9.627,1.733)%
  --(9.634,1.726)--(9.641,1.718)--(9.648,1.710)--(9.655,1.703)--(9.662,1.695)--(9.669,1.688)%
  --(9.676,1.680)--(9.683,1.672)--(9.690,1.665)--(9.697,1.657)--(9.704,1.650)--(9.711,1.642)%
  --(9.718,1.634)--(9.725,1.627)--(9.732,1.619)--(9.739,1.611)--(9.746,1.604)--(9.753,1.596)%
  --(9.760,1.589)--(9.767,1.581)--(9.774,1.573)--(9.781,1.566)--(9.788,1.558)--(9.795,1.551)%
  --(9.802,1.543)--(9.809,1.535)--(9.816,1.528)--(9.823,1.520)--(9.830,1.513)--(9.837,1.505)%
  --(9.844,1.497)--(9.851,1.490)--(9.858,1.482)--(9.865,1.474)--(9.872,1.467)--(9.878,1.459)%
  --(9.885,1.452)--(9.892,1.444)--(9.899,1.436)--(9.906,1.429)--(9.913,1.421)--(9.920,1.414)%
  --(9.927,1.406)--(9.934,1.398)--(9.941,1.391)--(9.948,1.383)--(9.955,1.376)--(9.962,1.368)%
  --(9.969,1.360)--(9.976,1.353)--(9.983,1.345)--(9.990,1.337)--(9.997,1.330)--(10.004,1.322)%
  --(10.011,1.315)--(10.018,1.307)--(10.025,1.299)--(10.032,1.292)--(10.039,1.284)--(10.046,1.277)%
  --(10.053,1.269)--(10.060,1.261)--(10.067,1.254)--(10.074,1.246)--(10.081,1.239)--(10.088,1.231)%
  --(10.095,1.223)--(10.102,1.216)--(10.109,1.208)--(10.116,1.201)--(10.123,1.193)--(10.130,1.185)%
  --(10.137,1.178)--(10.144,1.170)--(10.151,1.162)--(10.158,1.155)--(10.165,1.147)--(10.172,1.140)%
  --(10.179,1.132)--(10.186,1.124)--(10.193,1.117)--(10.200,1.109)--(10.207,1.102)--(10.214,1.094)%
  --(10.221,1.086)--(10.228,1.079)--(10.235,1.071)--(10.242,1.064);
\gpcolor{gp lt color border}
\gpsetlinetype{gp lt border}
\gpsetlinewidth{1.00}
\draw[gp path] (1.504,4.790)--(1.504,0.985)--(10.242,0.985)--(10.242,4.790)--cycle;
%% coordinates of the plot area
\gpdefrectangularnode{gp plot 1}{\pgfpoint{1.504cm}{0.985cm}}{\pgfpoint{10.242cm}{4.790cm}}
\end{tikzpicture}
%% gnuplot variables

\end{figure}

\begin{figure}[H]
	\centering
	\begin{tikzpicture}[gnuplot]
%% generated with GNUPLOT 4.4p2 (Lua 5.1.4; terminal rev. 97, script rev. 96a)
%% Thu 24 Nov 2011 11:30:02 PM EST
\gpsolidlines
\gpcolor{gp lt color axes}
\gpsetlinetype{gp lt axes}
\gpsetlinewidth{1.00}
\draw[gp path] (1.320,0.985)--(10.242,0.985);
\gpcolor{gp lt color border}
\gpsetlinetype{gp lt border}
\draw[gp path] (1.320,0.985)--(1.500,0.985);
\node[gp node right] at (1.136,0.985) {-60};
\gpcolor{gp lt color axes}
\gpsetlinetype{gp lt axes}
\draw[gp path] (1.320,1.619)--(10.242,1.619);
\gpcolor{gp lt color border}
\gpsetlinetype{gp lt border}
\draw[gp path] (1.320,1.619)--(1.500,1.619);
\node[gp node right] at (1.136,1.619) {-50};
\gpcolor{gp lt color axes}
\gpsetlinetype{gp lt axes}
\draw[gp path] (1.320,2.253)--(10.242,2.253);
\gpcolor{gp lt color border}
\gpsetlinetype{gp lt border}
\draw[gp path] (1.320,2.253)--(1.500,2.253);
\node[gp node right] at (1.136,2.253) {-40};
\gpcolor{gp lt color axes}
\gpsetlinetype{gp lt axes}
\draw[gp path] (1.320,2.888)--(10.242,2.888);
\gpcolor{gp lt color border}
\gpsetlinetype{gp lt border}
\draw[gp path] (1.320,2.888)--(1.500,2.888);
\node[gp node right] at (1.136,2.888) {-30};
\gpcolor{gp lt color axes}
\gpsetlinetype{gp lt axes}
\draw[gp path] (1.320,3.522)--(10.242,3.522);
\gpcolor{gp lt color border}
\gpsetlinetype{gp lt border}
\draw[gp path] (1.320,3.522)--(1.500,3.522);
\node[gp node right] at (1.136,3.522) {-20};
\gpcolor{gp lt color axes}
\gpsetlinetype{gp lt axes}
\draw[gp path] (1.320,4.156)--(10.242,4.156);
\gpcolor{gp lt color border}
\gpsetlinetype{gp lt border}
\draw[gp path] (1.320,4.156)--(1.500,4.156);
\node[gp node right] at (1.136,4.156) {-10};
\gpcolor{gp lt color axes}
\gpsetlinetype{gp lt axes}
\draw[gp path] (1.320,4.790)--(10.242,4.790);
\gpcolor{gp lt color border}
\gpsetlinetype{gp lt border}
\draw[gp path] (1.320,4.790)--(1.500,4.790);
\node[gp node right] at (1.136,4.790) { 0};
\gpcolor{gp lt color axes}
\gpsetlinetype{gp lt axes}
\draw[gp path] (2.487,0.985)--(2.487,4.790);
\gpcolor{gp lt color border}
\gpsetlinetype{gp lt border}
\draw[gp path] (2.487,0.985)--(2.487,1.165);
\node[gp node center] at (2.487,0.677) { 100};
\gpcolor{gp lt color axes}
\gpsetlinetype{gp lt axes}
\draw[gp path] (4.337,0.985)--(4.337,4.790);
\gpcolor{gp lt color border}
\gpsetlinetype{gp lt border}
\draw[gp path] (4.337,0.985)--(4.337,1.165);
\node[gp node center] at (4.337,0.677) { 300};
\gpcolor{gp lt color axes}
\gpsetlinetype{gp lt axes}
\draw[gp path] (6.365,0.985)--(6.365,4.790);
\gpcolor{gp lt color border}
\gpsetlinetype{gp lt border}
\draw[gp path] (6.365,0.985)--(6.365,1.165);
\node[gp node center] at (6.365,0.677) { 1000};
\gpcolor{gp lt color axes}
\gpsetlinetype{gp lt axes}
\draw[gp path] (8.215,0.985)--(8.215,4.790);
\gpcolor{gp lt color border}
\gpsetlinetype{gp lt border}
\draw[gp path] (8.215,0.985)--(8.215,1.165);
\node[gp node center] at (8.215,0.677) { 3000};
\gpcolor{gp lt color axes}
\gpsetlinetype{gp lt axes}
\draw[gp path] (10.242,0.985)--(10.242,4.790);
\gpcolor{gp lt color border}
\gpsetlinetype{gp lt border}
\draw[gp path] (10.242,0.985)--(10.242,1.165);
\node[gp node center] at (10.242,0.677) { 10000};
\draw[gp path] (1.320,4.790)--(1.320,0.985)--(10.242,0.985)--(10.242,4.790)--cycle;
\node[gp node center,rotate=-270] at (0.246,2.887) {Output Gain, $A$, (\si{\decibel})};
\node[gp node center] at (5.781,0.215) {Input Frequency, $f$ (\si{\hertz})};
\node[gp node center] at (5.781,5.252) {Measured Frequency Response of Third-order Low Pass Filter};
\gpcolor{gp lt color 0}
\gpsetlinetype{gp lt plot 0}
\gpsetlinewidth{3.00}
\draw[gp path] (1.353,4.783)--(1.535,4.775)--(1.716,4.783)--(1.898,4.776)--(2.079,4.775)%
  --(2.260,4.783)--(2.442,4.768)--(2.623,4.775)--(2.805,4.775)--(2.986,4.768)--(3.167,4.775)%
  --(3.349,4.768)--(3.530,4.775)--(3.712,4.761)--(3.893,4.775)--(4.074,4.760)--(4.256,4.760)%
  --(4.437,4.745)--(4.619,4.737)--(4.800,4.736)--(4.981,4.727)--(5.163,4.727)--(5.344,4.703)%
  --(5.526,4.693)--(5.707,4.682)--(5.888,4.656)--(6.070,4.624)--(6.251,4.587)--(6.433,4.515)%
  --(6.614,4.444)--(6.795,4.346)--(6.977,4.220)--(7.158,4.082)--(7.340,3.933)--(7.521,3.792)%
  --(7.702,3.622)--(7.884,3.499)--(8.065,3.320)--(8.247,3.196)--(8.428,3.077)--(8.609,3.008)%
  --(8.791,2.808)--(8.972,2.820)--(9.154,2.780)--(9.335,2.935)--(9.516,2.787)--(9.698,2.835)%
  --(9.879,2.808)--(10.061,2.750)--(10.242,2.808);
\gpcolor{gp lt color border}
\gpsetlinetype{gp lt border}
\gpsetlinewidth{1.00}
\draw[gp path] (1.320,4.790)--(1.320,0.985)--(10.242,0.985)--(10.242,4.790)--cycle;
%% coordinates of the plot area
\gpdefrectangularnode{gp plot 1}{\pgfpoint{1.320cm}{0.985cm}}{\pgfpoint{10.242cm}{4.790cm}}
\end{tikzpicture}
%% gnuplot variables

\end{figure}

\begin{figure}[H]
	\centering
	\begin{tikzpicture}[gnuplot]
%% generated with GNUPLOT 4.4p2 (Lua 5.1.4; terminal rev. 97, script rev. 96a)
%% Mon 28 Nov 2011 10:15:02 AM EST
\gpsolidlines
\gpcolor{gp lt color axes}
\gpsetlinetype{gp lt axes}
\gpsetlinewidth{1.00}
\draw[gp path] (1.504,0.985)--(10.242,0.985);
\gpcolor{gp lt color border}
\gpsetlinetype{gp lt border}
\draw[gp path] (1.504,0.985)--(1.684,0.985);
\node[gp node right] at (1.320,0.985) {-120};
\gpcolor{gp lt color axes}
\gpsetlinetype{gp lt axes}
\draw[gp path] (1.504,1.302)--(10.242,1.302);
\gpcolor{gp lt color border}
\gpsetlinetype{gp lt border}
\draw[gp path] (1.504,1.302)--(1.684,1.302);
\node[gp node right] at (1.320,1.302) {-110};
\gpcolor{gp lt color axes}
\gpsetlinetype{gp lt axes}
\draw[gp path] (1.504,1.619)--(10.242,1.619);
\gpcolor{gp lt color border}
\gpsetlinetype{gp lt border}
\draw[gp path] (1.504,1.619)--(1.684,1.619);
\node[gp node right] at (1.320,1.619) {-100};
\gpcolor{gp lt color axes}
\gpsetlinetype{gp lt axes}
\draw[gp path] (1.504,1.936)--(10.242,1.936);
\gpcolor{gp lt color border}
\gpsetlinetype{gp lt border}
\draw[gp path] (1.504,1.936)--(1.684,1.936);
\node[gp node right] at (1.320,1.936) {-90};
\gpcolor{gp lt color axes}
\gpsetlinetype{gp lt axes}
\draw[gp path] (1.504,2.253)--(10.242,2.253);
\gpcolor{gp lt color border}
\gpsetlinetype{gp lt border}
\draw[gp path] (1.504,2.253)--(1.684,2.253);
\node[gp node right] at (1.320,2.253) {-80};
\gpcolor{gp lt color axes}
\gpsetlinetype{gp lt axes}
\draw[gp path] (1.504,2.570)--(10.242,2.570);
\gpcolor{gp lt color border}
\gpsetlinetype{gp lt border}
\draw[gp path] (1.504,2.570)--(1.684,2.570);
\node[gp node right] at (1.320,2.570) {-70};
\gpcolor{gp lt color axes}
\gpsetlinetype{gp lt axes}
\draw[gp path] (1.504,2.888)--(10.242,2.888);
\gpcolor{gp lt color border}
\gpsetlinetype{gp lt border}
\draw[gp path] (1.504,2.888)--(1.684,2.888);
\node[gp node right] at (1.320,2.888) {-60};
\gpcolor{gp lt color axes}
\gpsetlinetype{gp lt axes}
\draw[gp path] (1.504,3.205)--(10.242,3.205);
\gpcolor{gp lt color border}
\gpsetlinetype{gp lt border}
\draw[gp path] (1.504,3.205)--(1.684,3.205);
\node[gp node right] at (1.320,3.205) {-50};
\gpcolor{gp lt color axes}
\gpsetlinetype{gp lt axes}
\draw[gp path] (1.504,3.522)--(10.242,3.522);
\gpcolor{gp lt color border}
\gpsetlinetype{gp lt border}
\draw[gp path] (1.504,3.522)--(1.684,3.522);
\node[gp node right] at (1.320,3.522) {-40};
\gpcolor{gp lt color axes}
\gpsetlinetype{gp lt axes}
\draw[gp path] (1.504,3.839)--(10.242,3.839);
\gpcolor{gp lt color border}
\gpsetlinetype{gp lt border}
\draw[gp path] (1.504,3.839)--(1.684,3.839);
\node[gp node right] at (1.320,3.839) {-30};
\gpcolor{gp lt color axes}
\gpsetlinetype{gp lt axes}
\draw[gp path] (1.504,4.156)--(10.242,4.156);
\gpcolor{gp lt color border}
\gpsetlinetype{gp lt border}
\draw[gp path] (1.504,4.156)--(1.684,4.156);
\node[gp node right] at (1.320,4.156) {-20};
\gpcolor{gp lt color axes}
\gpsetlinetype{gp lt axes}
\draw[gp path] (1.504,4.473)--(10.242,4.473);
\gpcolor{gp lt color border}
\gpsetlinetype{gp lt border}
\draw[gp path] (1.504,4.473)--(1.684,4.473);
\node[gp node right] at (1.320,4.473) {-10};
\gpcolor{gp lt color axes}
\gpsetlinetype{gp lt axes}
\draw[gp path] (1.504,4.790)--(10.242,4.790);
\gpcolor{gp lt color border}
\gpsetlinetype{gp lt border}
\draw[gp path] (1.504,4.790)--(1.684,4.790);
\node[gp node right] at (1.320,4.790) { 0};
\gpcolor{gp lt color axes}
\gpsetlinetype{gp lt axes}
\draw[gp path] (1.516,0.985)--(1.516,4.790);
\gpcolor{gp lt color border}
\gpsetlinetype{gp lt border}
\draw[gp path] (1.516,0.985)--(1.516,1.165);
\node[gp node center] at (1.516,0.677) { 10};
\gpcolor{gp lt color axes}
\gpsetlinetype{gp lt axes}
\draw[gp path] (2.559,0.985)--(2.559,4.790);
\gpcolor{gp lt color border}
\gpsetlinetype{gp lt border}
\draw[gp path] (2.559,0.985)--(2.559,1.165);
\node[gp node center] at (2.559,0.677) { 30};
\gpcolor{gp lt color axes}
\gpsetlinetype{gp lt axes}
\draw[gp path] (3.701,0.985)--(3.701,4.790);
\gpcolor{gp lt color border}
\gpsetlinetype{gp lt border}
\draw[gp path] (3.701,0.985)--(3.701,1.165);
\node[gp node center] at (3.701,0.677) { 100};
\gpcolor{gp lt color axes}
\gpsetlinetype{gp lt axes}
\draw[gp path] (4.743,0.985)--(4.743,4.790);
\gpcolor{gp lt color border}
\gpsetlinetype{gp lt border}
\draw[gp path] (4.743,0.985)--(4.743,1.165);
\node[gp node center] at (4.743,0.677) { 300};
\gpcolor{gp lt color axes}
\gpsetlinetype{gp lt axes}
\draw[gp path] (5.885,0.985)--(5.885,4.790);
\gpcolor{gp lt color border}
\gpsetlinetype{gp lt border}
\draw[gp path] (5.885,0.985)--(5.885,1.165);
\node[gp node center] at (5.885,0.677) { 1000};
\gpcolor{gp lt color axes}
\gpsetlinetype{gp lt axes}
\draw[gp path] (6.928,0.985)--(6.928,4.790);
\gpcolor{gp lt color border}
\gpsetlinetype{gp lt border}
\draw[gp path] (6.928,0.985)--(6.928,1.165);
\node[gp node center] at (6.928,0.677) { 3000};
\gpcolor{gp lt color axes}
\gpsetlinetype{gp lt axes}
\draw[gp path] (8.070,0.985)--(8.070,4.790);
\gpcolor{gp lt color border}
\gpsetlinetype{gp lt border}
\draw[gp path] (8.070,0.985)--(8.070,1.165);
\node[gp node center] at (8.070,0.677) { 10000};
\gpcolor{gp lt color axes}
\gpsetlinetype{gp lt axes}
\draw[gp path] (9.112,0.985)--(9.112,4.790);
\gpcolor{gp lt color border}
\gpsetlinetype{gp lt border}
\draw[gp path] (9.112,0.985)--(9.112,1.165);
\node[gp node center] at (9.112,0.677) { 30000};
\draw[gp path] (1.504,4.790)--(1.504,0.985)--(10.242,0.985)--(10.242,4.790)--cycle;
\node[gp node center,rotate=-270] at (0.246,2.887) {Output Gain, $A$, (\si{\decibel})};
\node[gp node center] at (5.873,0.215) {Input Frequency, $f$ (\si{\hertz})};
\node[gp node center] at (5.873,5.252) {Calculated Frequency Response of Third-order Low Pass Filter};
\gpcolor{gp lt color 0}
\gpsetlinetype{gp lt plot 0}
\gpsetlinewidth{3.00}
\draw[gp path] (1.504,4.790)--(1.506,4.790)--(1.508,4.790)--(1.511,4.790)--(1.513,4.790)%
  --(1.515,4.790)--(1.517,4.790)--(1.519,4.790)--(1.521,4.790)--(1.524,4.790)--(1.526,4.790)%
  --(1.528,4.790)--(1.530,4.790)--(1.532,4.790)--(1.535,4.790)--(1.537,4.790)--(1.539,4.790)%
  --(1.541,4.790)--(1.543,4.790)--(1.546,4.790)--(1.548,4.790)--(1.550,4.790)--(1.552,4.790)%
  --(1.554,4.790)--(1.556,4.790)--(1.559,4.790)--(1.561,4.790)--(1.563,4.790)--(1.565,4.790)%
  --(1.567,4.790)--(1.570,4.790)--(1.572,4.790)--(1.574,4.790)--(1.576,4.790)--(1.578,4.790)%
  --(1.580,4.790)--(1.583,4.790)--(1.585,4.790)--(1.587,4.790)--(1.589,4.790)--(1.591,4.790)%
  --(1.594,4.790)--(1.596,4.790)--(1.598,4.790)--(1.600,4.790)--(1.602,4.790)--(1.605,4.790)%
  --(1.607,4.790)--(1.609,4.790)--(1.611,4.790)--(1.613,4.790)--(1.615,4.790)--(1.618,4.790)%
  --(1.620,4.790)--(1.622,4.790)--(1.624,4.790)--(1.626,4.790)--(1.629,4.790)--(1.631,4.790)%
  --(1.633,4.790)--(1.635,4.790)--(1.637,4.790)--(1.639,4.790)--(1.642,4.790)--(1.644,4.790)%
  --(1.646,4.790)--(1.648,4.790)--(1.650,4.790)--(1.653,4.790)--(1.655,4.790)--(1.657,4.790)%
  --(1.659,4.790)--(1.661,4.790)--(1.664,4.790)--(1.666,4.790)--(1.668,4.790)--(1.670,4.790)%
  --(1.672,4.790)--(1.674,4.790)--(1.677,4.790)--(1.679,4.790)--(1.681,4.790)--(1.683,4.790)%
  --(1.685,4.790)--(1.688,4.790)--(1.690,4.790)--(1.692,4.790)--(1.694,4.790)--(1.696,4.790)%
  --(1.698,4.790)--(1.701,4.790)--(1.703,4.790)--(1.705,4.790)--(1.707,4.790)--(1.709,4.790)%
  --(1.712,4.790)--(1.714,4.790)--(1.716,4.790)--(1.718,4.790)--(1.720,4.790)--(1.723,4.790)%
  --(1.725,4.790)--(1.727,4.790)--(1.729,4.790)--(1.731,4.790)--(1.733,4.790)--(1.736,4.790)%
  --(1.738,4.790)--(1.740,4.790)--(1.742,4.790)--(1.744,4.790)--(1.747,4.790)--(1.749,4.790)%
  --(1.751,4.790)--(1.753,4.790)--(1.755,4.790)--(1.757,4.790)--(1.760,4.790)--(1.762,4.790)%
  --(1.764,4.790)--(1.766,4.790)--(1.768,4.790)--(1.771,4.790)--(1.773,4.790)--(1.775,4.790)%
  --(1.777,4.790)--(1.779,4.790)--(1.782,4.790)--(1.784,4.790)--(1.786,4.790)--(1.788,4.790)%
  --(1.790,4.790)--(1.792,4.790)--(1.795,4.790)--(1.797,4.790)--(1.799,4.790)--(1.801,4.790)%
  --(1.803,4.790)--(1.806,4.790)--(1.808,4.790)--(1.810,4.790)--(1.812,4.790)--(1.814,4.790)%
  --(1.816,4.790)--(1.819,4.790)--(1.821,4.790)--(1.823,4.790)--(1.825,4.790)--(1.827,4.790)%
  --(1.830,4.790)--(1.832,4.790)--(1.834,4.790)--(1.836,4.790)--(1.838,4.790)--(1.840,4.790)%
  --(1.843,4.790)--(1.845,4.790)--(1.847,4.790)--(1.849,4.790)--(1.851,4.790)--(1.854,4.790)%
  --(1.856,4.790)--(1.858,4.790)--(1.860,4.790)--(1.862,4.790)--(1.865,4.790)--(1.867,4.790)%
  --(1.869,4.790)--(1.871,4.790)--(1.873,4.790)--(1.875,4.790)--(1.878,4.790)--(1.880,4.790)%
  --(1.882,4.790)--(1.884,4.790)--(1.886,4.790)--(1.889,4.790)--(1.891,4.790)--(1.893,4.790)%
  --(1.895,4.790)--(1.897,4.790)--(1.899,4.790)--(1.902,4.790)--(1.904,4.790)--(1.906,4.790)%
  --(1.908,4.790)--(1.910,4.790)--(1.913,4.790)--(1.915,4.790)--(1.917,4.790)--(1.919,4.790)%
  --(1.921,4.790)--(1.924,4.790)--(1.926,4.790)--(1.928,4.790)--(1.930,4.790)--(1.932,4.790)%
  --(1.934,4.790)--(1.937,4.790)--(1.939,4.790)--(1.941,4.790)--(1.943,4.790)--(1.945,4.790)%
  --(1.948,4.790)--(1.950,4.790)--(1.952,4.790)--(1.954,4.790)--(1.956,4.790)--(1.958,4.790)%
  --(1.961,4.790)--(1.963,4.790)--(1.965,4.790)--(1.967,4.790)--(1.969,4.790)--(1.972,4.790)%
  --(1.974,4.790)--(1.976,4.790)--(1.978,4.790)--(1.980,4.790)--(1.983,4.790)--(1.985,4.790)%
  --(1.987,4.790)--(1.989,4.790)--(1.991,4.790)--(1.993,4.790)--(1.996,4.790)--(1.998,4.790)%
  --(2.000,4.790)--(2.002,4.790)--(2.004,4.790)--(2.007,4.790)--(2.009,4.790)--(2.011,4.790)%
  --(2.013,4.790)--(2.015,4.790)--(2.017,4.790)--(2.020,4.790)--(2.022,4.790)--(2.024,4.790)%
  --(2.026,4.790)--(2.028,4.790)--(2.031,4.790)--(2.033,4.790)--(2.035,4.790)--(2.037,4.790)%
  --(2.039,4.790)--(2.042,4.790)--(2.044,4.790)--(2.046,4.790)--(2.048,4.790)--(2.050,4.790)%
  --(2.052,4.790)--(2.055,4.790)--(2.057,4.790)--(2.059,4.790)--(2.061,4.790)--(2.063,4.790)%
  --(2.066,4.790)--(2.068,4.790)--(2.070,4.790)--(2.072,4.790)--(2.074,4.790)--(2.076,4.790)%
  --(2.079,4.790)--(2.081,4.790)--(2.083,4.790)--(2.085,4.790)--(2.087,4.790)--(2.090,4.790)%
  --(2.092,4.790)--(2.094,4.790)--(2.096,4.790)--(2.098,4.790)--(2.101,4.790)--(2.103,4.790)%
  --(2.105,4.790)--(2.107,4.790)--(2.109,4.790)--(2.111,4.790)--(2.114,4.790)--(2.116,4.790)%
  --(2.118,4.790)--(2.120,4.790)--(2.122,4.790)--(2.125,4.790)--(2.127,4.790)--(2.129,4.790)%
  --(2.131,4.790)--(2.133,4.790)--(2.135,4.790)--(2.138,4.790)--(2.140,4.790)--(2.142,4.790)%
  --(2.144,4.790)--(2.146,4.790)--(2.149,4.790)--(2.151,4.790)--(2.153,4.790)--(2.155,4.790)%
  --(2.157,4.790)--(2.160,4.790)--(2.162,4.790)--(2.164,4.790)--(2.166,4.790)--(2.168,4.790)%
  --(2.170,4.790)--(2.173,4.790)--(2.175,4.790)--(2.177,4.790)--(2.179,4.790)--(2.181,4.790)%
  --(2.184,4.790)--(2.186,4.790)--(2.188,4.790)--(2.190,4.790)--(2.192,4.790)--(2.194,4.790)%
  --(2.197,4.790)--(2.199,4.790)--(2.201,4.790)--(2.203,4.790)--(2.205,4.790)--(2.208,4.790)%
  --(2.210,4.790)--(2.212,4.790)--(2.214,4.790)--(2.216,4.790)--(2.219,4.790)--(2.221,4.790)%
  --(2.223,4.790)--(2.225,4.790)--(2.227,4.790)--(2.229,4.790)--(2.232,4.790)--(2.234,4.790)%
  --(2.236,4.790)--(2.238,4.790)--(2.240,4.790)--(2.243,4.790)--(2.245,4.790)--(2.247,4.790)%
  --(2.249,4.790)--(2.251,4.790)--(2.253,4.790)--(2.256,4.790)--(2.258,4.790)--(2.260,4.790)%
  --(2.262,4.790)--(2.264,4.790)--(2.267,4.790)--(2.269,4.790)--(2.271,4.790)--(2.273,4.790)%
  --(2.275,4.790)--(2.278,4.790)--(2.280,4.790)--(2.282,4.790)--(2.284,4.790)--(2.286,4.790)%
  --(2.288,4.790)--(2.291,4.790)--(2.293,4.790)--(2.295,4.790)--(2.297,4.790)--(2.299,4.790)%
  --(2.302,4.790)--(2.304,4.790)--(2.306,4.790)--(2.308,4.790)--(2.310,4.790)--(2.312,4.790)%
  --(2.315,4.790)--(2.317,4.790)--(2.319,4.790)--(2.321,4.790)--(2.323,4.790)--(2.326,4.790)%
  --(2.328,4.790)--(2.330,4.790)--(2.332,4.790)--(2.334,4.790)--(2.337,4.790)--(2.339,4.790)%
  --(2.341,4.790)--(2.343,4.790)--(2.345,4.790)--(2.347,4.790)--(2.350,4.790)--(2.352,4.790)%
  --(2.354,4.790)--(2.356,4.790)--(2.358,4.790)--(2.361,4.790)--(2.363,4.790)--(2.365,4.790)%
  --(2.367,4.790)--(2.369,4.790)--(2.371,4.790)--(2.374,4.790)--(2.376,4.790)--(2.378,4.790)%
  --(2.380,4.790)--(2.382,4.790)--(2.385,4.790)--(2.387,4.790)--(2.389,4.790)--(2.391,4.790)%
  --(2.393,4.790)--(2.395,4.790)--(2.398,4.790)--(2.400,4.790)--(2.402,4.790)--(2.404,4.790)%
  --(2.406,4.790)--(2.409,4.790)--(2.411,4.790)--(2.413,4.790)--(2.415,4.790)--(2.417,4.790)%
  --(2.420,4.790)--(2.422,4.790)--(2.424,4.790)--(2.426,4.790)--(2.428,4.790)--(2.430,4.790)%
  --(2.433,4.790)--(2.435,4.790)--(2.437,4.790)--(2.439,4.790)--(2.441,4.790)--(2.444,4.790)%
  --(2.446,4.790)--(2.448,4.790)--(2.450,4.790)--(2.452,4.790)--(2.454,4.790)--(2.457,4.790)%
  --(2.459,4.790)--(2.461,4.790)--(2.463,4.790)--(2.465,4.790)--(2.468,4.790)--(2.470,4.790)%
  --(2.472,4.790)--(2.474,4.790)--(2.476,4.790)--(2.479,4.790)--(2.481,4.790)--(2.483,4.790)%
  --(2.485,4.790)--(2.487,4.790)--(2.489,4.790)--(2.492,4.790)--(2.494,4.790)--(2.496,4.790)%
  --(2.498,4.790)--(2.500,4.790)--(2.503,4.790)--(2.505,4.790)--(2.507,4.790)--(2.509,4.790)%
  --(2.511,4.790)--(2.513,4.790)--(2.516,4.790)--(2.518,4.790)--(2.520,4.790)--(2.522,4.790)%
  --(2.524,4.790)--(2.527,4.790)--(2.529,4.790)--(2.531,4.790)--(2.533,4.790)--(2.535,4.790)%
  --(2.538,4.790)--(2.540,4.790)--(2.542,4.790)--(2.544,4.790)--(2.546,4.790)--(2.548,4.790)%
  --(2.551,4.790)--(2.553,4.790)--(2.555,4.790)--(2.557,4.790)--(2.559,4.790)--(2.562,4.790)%
  --(2.564,4.790)--(2.566,4.790)--(2.568,4.790)--(2.570,4.790)--(2.572,4.790)--(2.575,4.790)%
  --(2.577,4.790)--(2.579,4.790)--(2.581,4.790)--(2.583,4.790)--(2.586,4.790)--(2.588,4.790)%
  --(2.590,4.790)--(2.592,4.790)--(2.594,4.790)--(2.597,4.790)--(2.599,4.790)--(2.601,4.790)%
  --(2.603,4.790)--(2.605,4.790)--(2.607,4.790)--(2.610,4.790)--(2.612,4.790)--(2.614,4.790)%
  --(2.616,4.790)--(2.618,4.790)--(2.621,4.790)--(2.623,4.790)--(2.625,4.790)--(2.627,4.790)%
  --(2.629,4.790)--(2.631,4.790)--(2.634,4.790)--(2.636,4.790)--(2.638,4.790)--(2.640,4.790)%
  --(2.642,4.790)--(2.645,4.790)--(2.647,4.790)--(2.649,4.790)--(2.651,4.790)--(2.653,4.790)%
  --(2.656,4.790)--(2.658,4.790)--(2.660,4.790)--(2.662,4.790)--(2.664,4.790)--(2.666,4.790)%
  --(2.669,4.790)--(2.671,4.790)--(2.673,4.790)--(2.675,4.790)--(2.677,4.790)--(2.680,4.790)%
  --(2.682,4.790)--(2.684,4.790)--(2.686,4.790)--(2.688,4.790)--(2.690,4.790)--(2.693,4.790)%
  --(2.695,4.790)--(2.697,4.790)--(2.699,4.790)--(2.701,4.790)--(2.704,4.790)--(2.706,4.790)%
  --(2.708,4.790)--(2.710,4.790)--(2.712,4.790)--(2.715,4.790)--(2.717,4.790)--(2.719,4.790)%
  --(2.721,4.790)--(2.723,4.790)--(2.725,4.790)--(2.728,4.790)--(2.730,4.790)--(2.732,4.790)%
  --(2.734,4.790)--(2.736,4.790)--(2.739,4.790)--(2.741,4.790)--(2.743,4.790)--(2.745,4.790)%
  --(2.747,4.790)--(2.749,4.790)--(2.752,4.790)--(2.754,4.790)--(2.756,4.790)--(2.758,4.790)%
  --(2.760,4.790)--(2.763,4.790)--(2.765,4.790)--(2.767,4.790)--(2.769,4.790)--(2.771,4.790)%
  --(2.774,4.790)--(2.776,4.790)--(2.778,4.790)--(2.780,4.790)--(2.782,4.790)--(2.784,4.790)%
  --(2.787,4.790)--(2.789,4.790)--(2.791,4.790)--(2.793,4.790)--(2.795,4.790)--(2.798,4.790)%
  --(2.800,4.790)--(2.802,4.790)--(2.804,4.790)--(2.806,4.790)--(2.808,4.790)--(2.811,4.790)%
  --(2.813,4.790)--(2.815,4.790)--(2.817,4.790)--(2.819,4.790)--(2.822,4.790)--(2.824,4.790)%
  --(2.826,4.790)--(2.828,4.790)--(2.830,4.790)--(2.833,4.790)--(2.835,4.790)--(2.837,4.790)%
  --(2.839,4.790)--(2.841,4.790)--(2.843,4.790)--(2.846,4.790)--(2.848,4.790)--(2.850,4.790)%
  --(2.852,4.790)--(2.854,4.790)--(2.857,4.790)--(2.859,4.790)--(2.861,4.790)--(2.863,4.790)%
  --(2.865,4.790)--(2.867,4.790)--(2.870,4.790)--(2.872,4.790)--(2.874,4.790)--(2.876,4.790)%
  --(2.878,4.790)--(2.881,4.790)--(2.883,4.790)--(2.885,4.790)--(2.887,4.790)--(2.889,4.790)%
  --(2.892,4.790)--(2.894,4.790)--(2.896,4.790)--(2.898,4.790)--(2.900,4.790)--(2.902,4.790)%
  --(2.905,4.790)--(2.907,4.790)--(2.909,4.790)--(2.911,4.790)--(2.913,4.790)--(2.916,4.790)%
  --(2.918,4.790)--(2.920,4.790)--(2.922,4.790)--(2.924,4.790)--(2.926,4.790)--(2.929,4.790)%
  --(2.931,4.790)--(2.933,4.790)--(2.935,4.790)--(2.937,4.790)--(2.940,4.790)--(2.942,4.790)%
  --(2.944,4.790)--(2.946,4.790)--(2.948,4.790)--(2.951,4.790)--(2.953,4.790)--(2.955,4.790)%
  --(2.957,4.790)--(2.959,4.790)--(2.961,4.790)--(2.964,4.790)--(2.966,4.790)--(2.968,4.790)%
  --(2.970,4.790)--(2.972,4.790)--(2.975,4.790)--(2.977,4.790)--(2.979,4.790)--(2.981,4.790)%
  --(2.983,4.790)--(2.985,4.790)--(2.988,4.790)--(2.990,4.790)--(2.992,4.790)--(2.994,4.790)%
  --(2.996,4.790)--(2.999,4.790)--(3.001,4.790)--(3.003,4.790)--(3.005,4.790)--(3.007,4.790)%
  --(3.009,4.790)--(3.012,4.790)--(3.014,4.790)--(3.016,4.790)--(3.018,4.790)--(3.020,4.790)%
  --(3.023,4.790)--(3.025,4.790)--(3.027,4.790)--(3.029,4.790)--(3.031,4.790)--(3.034,4.790)%
  --(3.036,4.790)--(3.038,4.790)--(3.040,4.790)--(3.042,4.790)--(3.044,4.790)--(3.047,4.790)%
  --(3.049,4.790)--(3.051,4.790)--(3.053,4.790)--(3.055,4.790)--(3.058,4.790)--(3.060,4.790)%
  --(3.062,4.790)--(3.064,4.790)--(3.066,4.790)--(3.068,4.790)--(3.071,4.790)--(3.073,4.790)%
  --(3.075,4.790)--(3.077,4.790)--(3.079,4.790)--(3.082,4.790)--(3.084,4.790)--(3.086,4.790)%
  --(3.088,4.790)--(3.090,4.790)--(3.093,4.790)--(3.095,4.790)--(3.097,4.790)--(3.099,4.790)%
  --(3.101,4.790)--(3.103,4.790)--(3.106,4.790)--(3.108,4.790)--(3.110,4.790)--(3.112,4.790)%
  --(3.114,4.790)--(3.117,4.790)--(3.119,4.790)--(3.121,4.790)--(3.123,4.790)--(3.125,4.790)%
  --(3.127,4.790)--(3.130,4.790)--(3.132,4.790)--(3.134,4.790)--(3.136,4.790)--(3.138,4.790)%
  --(3.141,4.790)--(3.143,4.790)--(3.145,4.790)--(3.147,4.790)--(3.149,4.790)--(3.152,4.790)%
  --(3.154,4.790)--(3.156,4.790)--(3.158,4.790)--(3.160,4.790)--(3.162,4.790)--(3.165,4.790)%
  --(3.167,4.790)--(3.169,4.790)--(3.171,4.790)--(3.173,4.790)--(3.176,4.790)--(3.178,4.790)%
  --(3.180,4.790)--(3.182,4.790)--(3.184,4.790)--(3.186,4.790)--(3.189,4.790)--(3.191,4.790)%
  --(3.193,4.790)--(3.195,4.790)--(3.197,4.790)--(3.200,4.790)--(3.202,4.790)--(3.204,4.790)%
  --(3.206,4.790)--(3.208,4.790)--(3.211,4.790)--(3.213,4.790)--(3.215,4.790)--(3.217,4.790)%
  --(3.219,4.790)--(3.221,4.790)--(3.224,4.790)--(3.226,4.790)--(3.228,4.790)--(3.230,4.790)%
  --(3.232,4.790)--(3.235,4.790)--(3.237,4.790)--(3.239,4.790)--(3.241,4.790)--(3.243,4.790)%
  --(3.245,4.790)--(3.248,4.790)--(3.250,4.790)--(3.252,4.790)--(3.254,4.790)--(3.256,4.790)%
  --(3.259,4.790)--(3.261,4.790)--(3.263,4.790)--(3.265,4.790)--(3.267,4.790)--(3.270,4.790)%
  --(3.272,4.790)--(3.274,4.790)--(3.276,4.790)--(3.278,4.790)--(3.280,4.790)--(3.283,4.790)%
  --(3.285,4.790)--(3.287,4.790)--(3.289,4.790)--(3.291,4.790)--(3.294,4.790)--(3.296,4.790)%
  --(3.298,4.790)--(3.300,4.790)--(3.302,4.790)--(3.304,4.790)--(3.307,4.790)--(3.309,4.790)%
  --(3.311,4.790)--(3.313,4.790)--(3.315,4.790)--(3.318,4.790)--(3.320,4.790)--(3.322,4.790)%
  --(3.324,4.790)--(3.326,4.790)--(3.329,4.790)--(3.331,4.790)--(3.333,4.790)--(3.335,4.790)%
  --(3.337,4.790)--(3.339,4.790)--(3.342,4.790)--(3.344,4.790)--(3.346,4.790)--(3.348,4.790)%
  --(3.350,4.790)--(3.353,4.790)--(3.355,4.790)--(3.357,4.790)--(3.359,4.790)--(3.361,4.790)%
  --(3.363,4.790)--(3.366,4.790)--(3.368,4.790)--(3.370,4.790)--(3.372,4.790)--(3.374,4.790)%
  --(3.377,4.790)--(3.379,4.790)--(3.381,4.790)--(3.383,4.790)--(3.385,4.790)--(3.388,4.790)%
  --(3.390,4.790)--(3.392,4.790)--(3.394,4.790)--(3.396,4.790)--(3.398,4.790)--(3.401,4.790)%
  --(3.403,4.790)--(3.405,4.790)--(3.407,4.790)--(3.409,4.790)--(3.412,4.790)--(3.414,4.790)%
  --(3.416,4.790)--(3.418,4.790)--(3.420,4.790)--(3.422,4.790)--(3.425,4.790)--(3.427,4.790)%
  --(3.429,4.790)--(3.431,4.790)--(3.433,4.790)--(3.436,4.790)--(3.438,4.790)--(3.440,4.790)%
  --(3.442,4.790)--(3.444,4.790)--(3.447,4.790)--(3.449,4.790)--(3.451,4.790)--(3.453,4.790)%
  --(3.455,4.790)--(3.457,4.790)--(3.460,4.790)--(3.462,4.790)--(3.464,4.790)--(3.466,4.790)%
  --(3.468,4.790)--(3.471,4.790)--(3.473,4.790)--(3.475,4.790)--(3.477,4.790)--(3.479,4.790)%
  --(3.481,4.790)--(3.484,4.790)--(3.486,4.790)--(3.488,4.790)--(3.490,4.790)--(3.492,4.790)%
  --(3.495,4.790)--(3.497,4.790)--(3.499,4.790)--(3.501,4.790)--(3.503,4.790)--(3.506,4.790)%
  --(3.508,4.790)--(3.510,4.790)--(3.512,4.790)--(3.514,4.790)--(3.516,4.790)--(3.519,4.790)%
  --(3.521,4.790)--(3.523,4.790)--(3.525,4.790)--(3.527,4.790)--(3.530,4.790)--(3.532,4.790)%
  --(3.534,4.790)--(3.536,4.790)--(3.538,4.790)--(3.540,4.790)--(3.543,4.790)--(3.545,4.790)%
  --(3.547,4.790)--(3.549,4.790)--(3.551,4.790)--(3.554,4.790)--(3.556,4.790)--(3.558,4.790)%
  --(3.560,4.790)--(3.562,4.790)--(3.564,4.790)--(3.567,4.790)--(3.569,4.790)--(3.571,4.790)%
  --(3.573,4.790)--(3.575,4.790)--(3.578,4.790)--(3.580,4.790)--(3.582,4.790)--(3.584,4.790)%
  --(3.586,4.790)--(3.589,4.790)--(3.591,4.790)--(3.593,4.790)--(3.595,4.790)--(3.597,4.790)%
  --(3.599,4.790)--(3.602,4.790)--(3.604,4.790)--(3.606,4.790)--(3.608,4.790)--(3.610,4.790)%
  --(3.613,4.790)--(3.615,4.790)--(3.617,4.790)--(3.619,4.790)--(3.621,4.790)--(3.623,4.790)%
  --(3.626,4.790)--(3.628,4.790)--(3.630,4.790)--(3.632,4.790)--(3.634,4.790)--(3.637,4.790)%
  --(3.639,4.790)--(3.641,4.790)--(3.643,4.790)--(3.645,4.790)--(3.648,4.790)--(3.650,4.790)%
  --(3.652,4.790)--(3.654,4.790)--(3.656,4.790)--(3.658,4.790)--(3.661,4.790)--(3.663,4.790)%
  --(3.665,4.790)--(3.667,4.790)--(3.669,4.790)--(3.672,4.790)--(3.674,4.790)--(3.676,4.790)%
  --(3.678,4.790)--(3.680,4.790)--(3.682,4.790)--(3.685,4.790)--(3.687,4.790)--(3.689,4.790)%
  --(3.691,4.790)--(3.693,4.790)--(3.696,4.790)--(3.698,4.790)--(3.700,4.790)--(3.702,4.790)%
  --(3.704,4.790)--(3.707,4.790)--(3.709,4.790)--(3.711,4.790)--(3.713,4.790)--(3.715,4.790)%
  --(3.717,4.790)--(3.720,4.790)--(3.722,4.790)--(3.724,4.790)--(3.726,4.790)--(3.728,4.790)%
  --(3.731,4.790)--(3.733,4.790)--(3.735,4.790)--(3.737,4.790)--(3.739,4.790)--(3.741,4.790)%
  --(3.744,4.790)--(3.746,4.790)--(3.748,4.790)--(3.750,4.790)--(3.752,4.790)--(3.755,4.790)%
  --(3.757,4.790)--(3.759,4.790)--(3.761,4.790)--(3.763,4.790)--(3.766,4.790)--(3.768,4.790)%
  --(3.770,4.790)--(3.772,4.790)--(3.774,4.790)--(3.776,4.790)--(3.779,4.790)--(3.781,4.790)%
  --(3.783,4.790)--(3.785,4.790)--(3.787,4.790)--(3.790,4.790)--(3.792,4.790)--(3.794,4.790)%
  --(3.796,4.790)--(3.798,4.790)--(3.800,4.790)--(3.803,4.790)--(3.805,4.790)--(3.807,4.790)%
  --(3.809,4.790)--(3.811,4.790)--(3.814,4.790)--(3.816,4.790)--(3.818,4.790)--(3.820,4.790)%
  --(3.822,4.790)--(3.825,4.790)--(3.827,4.790)--(3.829,4.790)--(3.831,4.790)--(3.833,4.790)%
  --(3.835,4.790)--(3.838,4.790)--(3.840,4.790)--(3.842,4.790)--(3.844,4.790)--(3.846,4.790)%
  --(3.849,4.790)--(3.851,4.790)--(3.853,4.790)--(3.855,4.790)--(3.857,4.790)--(3.859,4.790)%
  --(3.862,4.790)--(3.864,4.790)--(3.866,4.790)--(3.868,4.790)--(3.870,4.790)--(3.873,4.790)%
  --(3.875,4.790)--(3.877,4.790)--(3.879,4.790)--(3.881,4.790)--(3.884,4.790)--(3.886,4.790)%
  --(3.888,4.790)--(3.890,4.790)--(3.892,4.790)--(3.894,4.790)--(3.897,4.790)--(3.899,4.790)%
  --(3.901,4.790)--(3.903,4.790)--(3.905,4.790)--(3.908,4.790)--(3.910,4.790)--(3.912,4.790)%
  --(3.914,4.790)--(3.916,4.790)--(3.918,4.790)--(3.921,4.790)--(3.923,4.790)--(3.925,4.790)%
  --(3.927,4.790)--(3.929,4.790)--(3.932,4.790)--(3.934,4.790)--(3.936,4.790)--(3.938,4.790)%
  --(3.940,4.790)--(3.943,4.790)--(3.945,4.790)--(3.947,4.790)--(3.949,4.790)--(3.951,4.790)%
  --(3.953,4.790)--(3.956,4.790)--(3.958,4.790)--(3.960,4.790)--(3.962,4.790)--(3.964,4.790)%
  --(3.967,4.790)--(3.969,4.790)--(3.971,4.790)--(3.973,4.790)--(3.975,4.790)--(3.977,4.790)%
  --(3.980,4.790)--(3.982,4.790)--(3.984,4.790)--(3.986,4.790)--(3.988,4.790)--(3.991,4.790)%
  --(3.993,4.790)--(3.995,4.790)--(3.997,4.790)--(3.999,4.790)--(4.002,4.790)--(4.004,4.790)%
  --(4.006,4.790)--(4.008,4.790)--(4.010,4.790)--(4.012,4.790)--(4.015,4.790)--(4.017,4.790)%
  --(4.019,4.790)--(4.021,4.790)--(4.023,4.790)--(4.026,4.790)--(4.028,4.790)--(4.030,4.790)%
  --(4.032,4.790)--(4.034,4.790)--(4.036,4.790)--(4.039,4.790)--(4.041,4.790)--(4.043,4.790)%
  --(4.045,4.790)--(4.047,4.790)--(4.050,4.790)--(4.052,4.790)--(4.054,4.790)--(4.056,4.790)%
  --(4.058,4.790)--(4.061,4.790)--(4.063,4.790)--(4.065,4.790)--(4.067,4.790)--(4.069,4.790)%
  --(4.071,4.790)--(4.074,4.790)--(4.076,4.790)--(4.078,4.790)--(4.080,4.790)--(4.082,4.790)%
  --(4.085,4.790)--(4.087,4.790)--(4.089,4.790)--(4.091,4.790)--(4.093,4.790)--(4.095,4.790)%
  --(4.098,4.790)--(4.100,4.790)--(4.102,4.790)--(4.104,4.790)--(4.106,4.790)--(4.109,4.790)%
  --(4.111,4.790)--(4.113,4.790)--(4.115,4.790)--(4.117,4.790)--(4.120,4.790)--(4.122,4.790)%
  --(4.124,4.790)--(4.126,4.790)--(4.128,4.790)--(4.130,4.790)--(4.133,4.790)--(4.135,4.790)%
  --(4.137,4.790)--(4.139,4.790)--(4.141,4.790)--(4.144,4.790)--(4.146,4.790)--(4.148,4.790)%
  --(4.150,4.790)--(4.152,4.790)--(4.154,4.790)--(4.157,4.790)--(4.159,4.790)--(4.161,4.790)%
  --(4.163,4.790)--(4.165,4.790)--(4.168,4.790)--(4.170,4.790)--(4.172,4.790)--(4.174,4.790)%
  --(4.176,4.790)--(4.178,4.790)--(4.181,4.790)--(4.183,4.790)--(4.185,4.790)--(4.187,4.790)%
  --(4.189,4.790)--(4.192,4.790)--(4.194,4.790)--(4.196,4.790)--(4.198,4.790)--(4.200,4.790)%
  --(4.203,4.790)--(4.205,4.790)--(4.207,4.790)--(4.209,4.790)--(4.211,4.790)--(4.213,4.790)%
  --(4.216,4.790)--(4.218,4.790)--(4.220,4.790)--(4.222,4.790)--(4.224,4.790)--(4.227,4.790)%
  --(4.229,4.790)--(4.231,4.790)--(4.233,4.790)--(4.235,4.790)--(4.237,4.790)--(4.240,4.790)%
  --(4.242,4.790)--(4.244,4.790)--(4.246,4.790)--(4.248,4.790)--(4.251,4.790)--(4.253,4.790)%
  --(4.255,4.790)--(4.257,4.790)--(4.259,4.790)--(4.262,4.790)--(4.264,4.790)--(4.266,4.790)%
  --(4.268,4.790)--(4.270,4.790)--(4.272,4.790)--(4.275,4.790)--(4.277,4.790)--(4.279,4.790)%
  --(4.281,4.790)--(4.283,4.790)--(4.286,4.790)--(4.288,4.790)--(4.290,4.790)--(4.292,4.790)%
  --(4.294,4.790)--(4.296,4.790)--(4.299,4.790)--(4.301,4.790)--(4.303,4.790)--(4.305,4.790)%
  --(4.307,4.790)--(4.310,4.790)--(4.312,4.790)--(4.314,4.790)--(4.316,4.790)--(4.318,4.790)%
  --(4.321,4.790)--(4.323,4.790)--(4.325,4.790)--(4.327,4.790)--(4.329,4.790)--(4.331,4.790)%
  --(4.334,4.790)--(4.336,4.790)--(4.338,4.790)--(4.340,4.790)--(4.342,4.790)--(4.345,4.790)%
  --(4.347,4.790)--(4.349,4.790)--(4.351,4.790)--(4.353,4.790)--(4.355,4.790)--(4.358,4.790)%
  --(4.360,4.790)--(4.362,4.790)--(4.364,4.790)--(4.366,4.790)--(4.369,4.790)--(4.371,4.790)%
  --(4.373,4.790)--(4.375,4.790)--(4.377,4.790)--(4.380,4.790)--(4.382,4.790)--(4.384,4.790)%
  --(4.386,4.790)--(4.388,4.790)--(4.390,4.790)--(4.393,4.790)--(4.395,4.790)--(4.397,4.790)%
  --(4.399,4.790)--(4.401,4.790)--(4.404,4.790)--(4.406,4.790)--(4.408,4.790)--(4.410,4.790)%
  --(4.412,4.790)--(4.414,4.790)--(4.417,4.790)--(4.419,4.790)--(4.421,4.790)--(4.423,4.790)%
  --(4.425,4.790)--(4.428,4.790)--(4.430,4.790)--(4.432,4.790)--(4.434,4.790)--(4.436,4.790)%
  --(4.439,4.790)--(4.441,4.790)--(4.443,4.790)--(4.445,4.790)--(4.447,4.790)--(4.449,4.790)%
  --(4.452,4.790)--(4.454,4.790)--(4.456,4.790)--(4.458,4.790)--(4.460,4.790)--(4.463,4.790)%
  --(4.465,4.790)--(4.467,4.790)--(4.469,4.790)--(4.471,4.790)--(4.473,4.790)--(4.476,4.790)%
  --(4.478,4.790)--(4.480,4.790)--(4.482,4.790)--(4.484,4.790)--(4.487,4.790)--(4.489,4.790)%
  --(4.491,4.790)--(4.493,4.790)--(4.495,4.790)--(4.498,4.790)--(4.500,4.790)--(4.502,4.790)%
  --(4.504,4.790)--(4.506,4.790)--(4.508,4.790)--(4.511,4.790)--(4.513,4.790)--(4.515,4.790)%
  --(4.517,4.790)--(4.519,4.790)--(4.522,4.790)--(4.524,4.790)--(4.526,4.790)--(4.528,4.790)%
  --(4.530,4.790)--(4.532,4.790)--(4.535,4.790)--(4.537,4.790)--(4.539,4.790)--(4.541,4.790)%
  --(4.543,4.790)--(4.546,4.790)--(4.548,4.790)--(4.550,4.790)--(4.552,4.790)--(4.554,4.790)%
  --(4.557,4.790)--(4.559,4.790)--(4.561,4.790)--(4.563,4.790)--(4.565,4.790)--(4.567,4.790)%
  --(4.570,4.790)--(4.572,4.790)--(4.574,4.790)--(4.576,4.790)--(4.578,4.790)--(4.581,4.790)%
  --(4.583,4.790)--(4.585,4.790)--(4.587,4.790)--(4.589,4.790)--(4.591,4.790)--(4.594,4.790)%
  --(4.596,4.790)--(4.598,4.790)--(4.600,4.790)--(4.602,4.790)--(4.605,4.790)--(4.607,4.790)%
  --(4.609,4.790)--(4.611,4.790)--(4.613,4.790)--(4.616,4.790)--(4.618,4.790)--(4.620,4.790)%
  --(4.622,4.790)--(4.624,4.790)--(4.626,4.790)--(4.629,4.790)--(4.631,4.790)--(4.633,4.790)%
  --(4.635,4.790)--(4.637,4.790)--(4.640,4.790)--(4.642,4.790)--(4.644,4.790)--(4.646,4.790)%
  --(4.648,4.790)--(4.650,4.790)--(4.653,4.790)--(4.655,4.790)--(4.657,4.790)--(4.659,4.790)%
  --(4.661,4.790)--(4.664,4.790)--(4.666,4.790)--(4.668,4.790)--(4.670,4.790)--(4.672,4.790)%
  --(4.675,4.790)--(4.677,4.790)--(4.679,4.790)--(4.681,4.790)--(4.683,4.790)--(4.685,4.790)%
  --(4.688,4.790)--(4.690,4.790)--(4.692,4.790)--(4.694,4.790)--(4.696,4.790)--(4.699,4.790)%
  --(4.701,4.790)--(4.703,4.790)--(4.705,4.790)--(4.707,4.790)--(4.709,4.790)--(4.712,4.790)%
  --(4.714,4.790)--(4.716,4.790)--(4.718,4.790)--(4.720,4.790)--(4.723,4.790)--(4.725,4.790)%
  --(4.727,4.790)--(4.729,4.790)--(4.731,4.790)--(4.733,4.790)--(4.736,4.790)--(4.738,4.790)%
  --(4.740,4.790)--(4.742,4.790)--(4.744,4.790)--(4.747,4.790)--(4.749,4.790)--(4.751,4.790)%
  --(4.753,4.790)--(4.755,4.790)--(4.758,4.790)--(4.760,4.790)--(4.762,4.790)--(4.764,4.790)%
  --(4.766,4.790)--(4.768,4.790)--(4.771,4.790)--(4.773,4.790)--(4.775,4.790)--(4.777,4.790)%
  --(4.779,4.790)--(4.782,4.790)--(4.784,4.790)--(4.786,4.790)--(4.788,4.790)--(4.790,4.790)%
  --(4.792,4.790)--(4.795,4.790)--(4.797,4.790)--(4.799,4.790)--(4.801,4.790)--(4.803,4.790)%
  --(4.806,4.790)--(4.808,4.790)--(4.810,4.790)--(4.812,4.790)--(4.814,4.790)--(4.817,4.790)%
  --(4.819,4.790)--(4.821,4.790)--(4.823,4.790)--(4.825,4.790)--(4.827,4.790)--(4.830,4.790)%
  --(4.832,4.790)--(4.834,4.790)--(4.836,4.790)--(4.838,4.790)--(4.841,4.790)--(4.843,4.790)%
  --(4.845,4.790)--(4.847,4.790)--(4.849,4.790)--(4.851,4.790)--(4.854,4.790)--(4.856,4.790)%
  --(4.858,4.790)--(4.860,4.790)--(4.862,4.790)--(4.865,4.790)--(4.867,4.790)--(4.869,4.790)%
  --(4.871,4.790)--(4.873,4.790)--(4.876,4.790)--(4.878,4.790)--(4.880,4.790)--(4.882,4.790)%
  --(4.884,4.790)--(4.886,4.790)--(4.889,4.790)--(4.891,4.790)--(4.893,4.790)--(4.895,4.790)%
  --(4.897,4.790)--(4.900,4.790)--(4.902,4.790)--(4.904,4.790)--(4.906,4.790)--(4.908,4.790)%
  --(4.910,4.790)--(4.913,4.790)--(4.915,4.790)--(4.917,4.790)--(4.919,4.790)--(4.921,4.790)%
  --(4.924,4.790)--(4.926,4.790)--(4.928,4.790)--(4.930,4.790)--(4.932,4.790)--(4.935,4.790)%
  --(4.937,4.790)--(4.939,4.790)--(4.941,4.790)--(4.943,4.790)--(4.945,4.790)--(4.948,4.790)%
  --(4.950,4.790)--(4.952,4.790)--(4.954,4.790)--(4.956,4.790)--(4.959,4.790)--(4.961,4.790)%
  --(4.963,4.790)--(4.965,4.790)--(4.967,4.790)--(4.969,4.790)--(4.972,4.790)--(4.974,4.790)%
  --(4.976,4.790)--(4.978,4.790)--(4.980,4.790)--(4.983,4.790)--(4.985,4.790)--(4.987,4.790)%
  --(4.989,4.790)--(4.991,4.790)--(4.994,4.790)--(4.996,4.790)--(4.998,4.790)--(5.000,4.790)%
  --(5.002,4.790)--(5.004,4.790)--(5.007,4.790)--(5.009,4.790)--(5.011,4.790)--(5.013,4.790)%
  --(5.015,4.790)--(5.018,4.790)--(5.020,4.790)--(5.022,4.790)--(5.024,4.790)--(5.026,4.790)%
  --(5.028,4.790)--(5.031,4.790)--(5.033,4.790)--(5.035,4.790)--(5.037,4.790)--(5.039,4.790)%
  --(5.042,4.790)--(5.044,4.790)--(5.046,4.790)--(5.048,4.790)--(5.050,4.790)--(5.053,4.790)%
  --(5.055,4.790)--(5.057,4.790)--(5.059,4.790)--(5.061,4.790)--(5.063,4.790)--(5.066,4.790)%
  --(5.068,4.790)--(5.070,4.790)--(5.072,4.790)--(5.074,4.790)--(5.077,4.790)--(5.079,4.790)%
  --(5.081,4.790)--(5.083,4.790)--(5.085,4.790)--(5.087,4.790)--(5.090,4.790)--(5.092,4.790)%
  --(5.094,4.790)--(5.096,4.790)--(5.098,4.790)--(5.101,4.790)--(5.103,4.790)--(5.105,4.790)%
  --(5.107,4.790)--(5.109,4.790)--(5.112,4.790)--(5.114,4.790)--(5.116,4.790)--(5.118,4.790)%
  --(5.120,4.790)--(5.122,4.790)--(5.125,4.790)--(5.127,4.790)--(5.129,4.790)--(5.131,4.790)%
  --(5.133,4.790)--(5.136,4.790)--(5.138,4.790)--(5.140,4.790)--(5.142,4.790)--(5.144,4.790)%
  --(5.146,4.790)--(5.149,4.790)--(5.151,4.790)--(5.153,4.790)--(5.155,4.789)--(5.157,4.789)%
  --(5.160,4.789)--(5.162,4.789)--(5.164,4.789)--(5.166,4.789)--(5.168,4.789)--(5.171,4.789)%
  --(5.173,4.789)--(5.175,4.789)--(5.177,4.789)--(5.179,4.789)--(5.181,4.789)--(5.184,4.789)%
  --(5.186,4.789)--(5.188,4.789)--(5.190,4.789)--(5.192,4.789)--(5.195,4.789)--(5.197,4.789)%
  --(5.199,4.789)--(5.201,4.789)--(5.203,4.789)--(5.205,4.789)--(5.208,4.789)--(5.210,4.789)%
  --(5.212,4.789)--(5.214,4.789)--(5.216,4.789)--(5.219,4.789)--(5.221,4.789)--(5.223,4.789)%
  --(5.225,4.789)--(5.227,4.789)--(5.230,4.789)--(5.232,4.789)--(5.234,4.789)--(5.236,4.789)%
  --(5.238,4.789)--(5.240,4.789)--(5.243,4.789)--(5.245,4.789)--(5.247,4.789)--(5.249,4.789)%
  --(5.251,4.789)--(5.254,4.789)--(5.256,4.789)--(5.258,4.789)--(5.260,4.789)--(5.262,4.789)%
  --(5.264,4.789)--(5.267,4.789)--(5.269,4.789)--(5.271,4.789)--(5.273,4.789)--(5.275,4.789)%
  --(5.278,4.789)--(5.280,4.789)--(5.282,4.789)--(5.284,4.789)--(5.286,4.789)--(5.289,4.789)%
  --(5.291,4.789)--(5.293,4.789)--(5.295,4.789)--(5.297,4.789)--(5.299,4.789)--(5.302,4.789)%
  --(5.304,4.789)--(5.306,4.789)--(5.308,4.789)--(5.310,4.789)--(5.313,4.789)--(5.315,4.789)%
  --(5.317,4.789)--(5.319,4.789)--(5.321,4.789)--(5.323,4.789)--(5.326,4.789)--(5.328,4.788)%
  --(5.330,4.788)--(5.332,4.788)--(5.334,4.788)--(5.337,4.788)--(5.339,4.788)--(5.341,4.788)%
  --(5.343,4.788)--(5.345,4.788)--(5.347,4.788)--(5.350,4.788)--(5.352,4.788)--(5.354,4.788)%
  --(5.356,4.788)--(5.358,4.788)--(5.361,4.788)--(5.363,4.788)--(5.365,4.788)--(5.367,4.788)%
  --(5.369,4.788)--(5.372,4.788)--(5.374,4.788)--(5.376,4.788)--(5.378,4.788)--(5.380,4.788)%
  --(5.382,4.788)--(5.385,4.788)--(5.387,4.788)--(5.389,4.788)--(5.391,4.788)--(5.393,4.788)%
  --(5.396,4.788)--(5.398,4.788)--(5.400,4.788)--(5.402,4.788)--(5.404,4.788)--(5.406,4.788)%
  --(5.409,4.788)--(5.411,4.787)--(5.413,4.787)--(5.415,4.787)--(5.417,4.787)--(5.420,4.787)%
  --(5.422,4.787)--(5.424,4.787)--(5.426,4.787)--(5.428,4.787)--(5.431,4.787)--(5.433,4.787)%
  --(5.435,4.787)--(5.437,4.787)--(5.439,4.787)--(5.441,4.787)--(5.444,4.787)--(5.446,4.787)%
  --(5.448,4.787)--(5.450,4.787)--(5.452,4.787)--(5.455,4.787)--(5.457,4.787)--(5.459,4.787)%
  --(5.461,4.787)--(5.463,4.786)--(5.465,4.786)--(5.468,4.786)--(5.470,4.786)--(5.472,4.786)%
  --(5.474,4.786)--(5.476,4.786)--(5.479,4.786)--(5.481,4.786)--(5.483,4.786)--(5.485,4.786)%
  --(5.487,4.786)--(5.490,4.786)--(5.492,4.786)--(5.494,4.786)--(5.496,4.786)--(5.498,4.786)%
  --(5.500,4.786)--(5.503,4.786)--(5.505,4.785)--(5.507,4.785)--(5.509,4.785)--(5.511,4.785)%
  --(5.514,4.785)--(5.516,4.785)--(5.518,4.785)--(5.520,4.785)--(5.522,4.785)--(5.524,4.785)%
  --(5.527,4.785)--(5.529,4.785)--(5.531,4.785)--(5.533,4.785)--(5.535,4.784)--(5.538,4.784)%
  --(5.540,4.784)--(5.542,4.784)--(5.544,4.784)--(5.546,4.784)--(5.549,4.784)--(5.551,4.784)%
  --(5.553,4.784)--(5.555,4.784)--(5.557,4.784)--(5.559,4.784)--(5.562,4.784)--(5.564,4.783)%
  --(5.566,4.783)--(5.568,4.783)--(5.570,4.783)--(5.573,4.783)--(5.575,4.783)--(5.577,4.783)%
  --(5.579,4.783)--(5.581,4.783)--(5.583,4.783)--(5.586,4.782)--(5.588,4.782)--(5.590,4.782)%
  --(5.592,4.782)--(5.594,4.782)--(5.597,4.782)--(5.599,4.782)--(5.601,4.782)--(5.603,4.782)%
  --(5.605,4.782)--(5.608,4.781)--(5.610,4.781)--(5.612,4.781)--(5.614,4.781)--(5.616,4.781)%
  --(5.618,4.781)--(5.621,4.781)--(5.623,4.781)--(5.625,4.780)--(5.627,4.780)--(5.629,4.780)%
  --(5.632,4.780)--(5.634,4.780)--(5.636,4.780)--(5.638,4.780)--(5.640,4.780)--(5.642,4.779)%
  --(5.645,4.779)--(5.647,4.779)--(5.649,4.779)--(5.651,4.779)--(5.653,4.779)--(5.656,4.778)%
  --(5.658,4.778)--(5.660,4.778)--(5.662,4.778)--(5.664,4.778)--(5.667,4.778)--(5.669,4.778)%
  --(5.671,4.777)--(5.673,4.777)--(5.675,4.777)--(5.677,4.777)--(5.680,4.777)--(5.682,4.777)%
  --(5.684,4.776)--(5.686,4.776)--(5.688,4.776)--(5.691,4.776)--(5.693,4.776)--(5.695,4.775)%
  --(5.697,4.775)--(5.699,4.775)--(5.701,4.775)--(5.704,4.775)--(5.706,4.774)--(5.708,4.774)%
  --(5.710,4.774)--(5.712,4.774)--(5.715,4.774)--(5.717,4.773)--(5.719,4.773)--(5.721,4.773)%
  --(5.723,4.773)--(5.726,4.772)--(5.728,4.772)--(5.730,4.772)--(5.732,4.772)--(5.734,4.772)%
  --(5.736,4.771)--(5.739,4.771)--(5.741,4.771)--(5.743,4.771)--(5.745,4.770)--(5.747,4.770)%
  --(5.750,4.770)--(5.752,4.770)--(5.754,4.769)--(5.756,4.769)--(5.758,4.769)--(5.760,4.768)%
  --(5.763,4.768)--(5.765,4.768)--(5.767,4.768)--(5.769,4.767)--(5.771,4.767)--(5.774,4.767)%
  --(5.776,4.766)--(5.778,4.766)--(5.780,4.766)--(5.782,4.766)--(5.785,4.765)--(5.787,4.765)%
  --(5.789,4.765)--(5.791,4.764)--(5.793,4.764)--(5.795,4.764)--(5.798,4.763)--(5.800,4.763)%
  --(5.802,4.763)--(5.804,4.762)--(5.806,4.762)--(5.809,4.762)--(5.811,4.761)--(5.813,4.761)%
  --(5.815,4.760)--(5.817,4.760)--(5.819,4.760)--(5.822,4.759)--(5.824,4.759)--(5.826,4.759)%
  --(5.828,4.758)--(5.830,4.758)--(5.833,4.757)--(5.835,4.757)--(5.837,4.757)--(5.839,4.756)%
  --(5.841,4.756)--(5.844,4.755)--(5.846,4.755)--(5.848,4.754)--(5.850,4.754)--(5.852,4.754)%
  --(5.854,4.753)--(5.857,4.753)--(5.859,4.752)--(5.861,4.752)--(5.863,4.751)--(5.865,4.751)%
  --(5.868,4.750)--(5.870,4.750)--(5.872,4.749)--(5.874,4.749)--(5.876,4.748)--(5.878,4.748)%
  --(5.881,4.747)--(5.883,4.747)--(5.885,4.746)--(5.887,4.746)--(5.889,4.745)--(5.892,4.745)%
  --(5.894,4.744)--(5.896,4.744)--(5.898,4.743)--(5.900,4.743)--(5.902,4.742)--(5.905,4.742)%
  --(5.907,4.741)--(5.909,4.740)--(5.911,4.740)--(5.913,4.739)--(5.916,4.739)--(5.918,4.738)%
  --(5.920,4.738)--(5.922,4.737)--(5.924,4.736)--(5.927,4.736)--(5.929,4.735)--(5.931,4.734)%
  --(5.933,4.734)--(5.935,4.733)--(5.937,4.732)--(5.940,4.732)--(5.942,4.731)--(5.944,4.731)%
  --(5.946,4.730)--(5.948,4.729)--(5.951,4.728)--(5.953,4.728)--(5.955,4.727)--(5.957,4.726)%
  --(5.959,4.726)--(5.961,4.725)--(5.964,4.724)--(5.966,4.724)--(5.968,4.723)--(5.970,4.722)%
  --(5.972,4.721)--(5.975,4.721)--(5.977,4.720)--(5.979,4.719)--(5.981,4.718)--(5.983,4.718)%
  --(5.986,4.717)--(5.988,4.716)--(5.990,4.715)--(5.992,4.714)--(5.994,4.714)--(5.996,4.713)%
  --(5.999,4.712)--(6.001,4.711)--(6.003,4.710)--(6.005,4.709)--(6.007,4.709)--(6.010,4.708)%
  --(6.012,4.707)--(6.014,4.706)--(6.016,4.705)--(6.018,4.704)--(6.020,4.703)--(6.023,4.702)%
  --(6.025,4.702)--(6.027,4.701)--(6.029,4.700)--(6.031,4.699)--(6.034,4.698)--(6.036,4.697)%
  --(6.038,4.696)--(6.040,4.695)--(6.042,4.694)--(6.045,4.693)--(6.047,4.692)--(6.049,4.691)%
  --(6.051,4.690)--(6.053,4.689)--(6.055,4.688)--(6.058,4.687)--(6.060,4.686)--(6.062,4.685)%
  --(6.064,4.684)--(6.066,4.683)--(6.069,4.682)--(6.071,4.681)--(6.073,4.680)--(6.075,4.679)%
  --(6.077,4.678)--(6.079,4.677)--(6.082,4.676)--(6.084,4.675)--(6.086,4.674)--(6.088,4.673)%
  --(6.090,4.672)--(6.093,4.670)--(6.095,4.669)--(6.097,4.668)--(6.099,4.667)--(6.101,4.666)%
  --(6.104,4.665)--(6.106,4.664)--(6.108,4.663)--(6.110,4.661)--(6.112,4.660)--(6.114,4.659)%
  --(6.117,4.658)--(6.119,4.657)--(6.121,4.656)--(6.123,4.654)--(6.125,4.653)--(6.128,4.652)%
  --(6.130,4.651)--(6.132,4.650)--(6.134,4.648)--(6.136,4.647)--(6.138,4.646)--(6.141,4.645)%
  --(6.143,4.643)--(6.145,4.642)--(6.147,4.641)--(6.149,4.640)--(6.152,4.638)--(6.154,4.637)%
  --(6.156,4.636)--(6.158,4.635)--(6.160,4.633)--(6.163,4.632)--(6.165,4.631)--(6.167,4.629)%
  --(6.169,4.628)--(6.171,4.627)--(6.173,4.625)--(6.176,4.624)--(6.178,4.623)--(6.180,4.621)%
  --(6.182,4.620)--(6.184,4.619)--(6.187,4.617)--(6.189,4.616)--(6.191,4.615)--(6.193,4.613)%
  --(6.195,4.612)--(6.197,4.610)--(6.200,4.609)--(6.202,4.608)--(6.204,4.606)--(6.206,4.605)%
  --(6.208,4.603)--(6.211,4.602)--(6.213,4.601)--(6.215,4.599)--(6.217,4.598)--(6.219,4.596)%
  --(6.222,4.595)--(6.224,4.593)--(6.226,4.592)--(6.228,4.591)--(6.230,4.589)--(6.232,4.588)%
  --(6.235,4.586)--(6.237,4.585)--(6.239,4.583)--(6.241,4.582)--(6.243,4.580)--(6.246,4.579)%
  --(6.248,4.577)--(6.250,4.576)--(6.252,4.574)--(6.254,4.573)--(6.256,4.571)--(6.259,4.570)%
  --(6.261,4.568)--(6.263,4.567)--(6.265,4.565)--(6.267,4.564)--(6.270,4.562)--(6.272,4.561)%
  --(6.274,4.559)--(6.276,4.557)--(6.278,4.556)--(6.281,4.554)--(6.283,4.553)--(6.285,4.551)%
  --(6.287,4.550)--(6.289,4.548)--(6.291,4.546)--(6.294,4.545)--(6.296,4.543)--(6.298,4.542)%
  --(6.300,4.540)--(6.302,4.539)--(6.305,4.537)--(6.307,4.535)--(6.309,4.534)--(6.311,4.532)%
  --(6.313,4.531)--(6.315,4.529)--(6.318,4.527)--(6.320,4.526)--(6.322,4.524)--(6.324,4.522)%
  --(6.326,4.521)--(6.329,4.519)--(6.331,4.517)--(6.333,4.516)--(6.335,4.514)--(6.337,4.513)%
  --(6.340,4.511)--(6.342,4.509)--(6.344,4.508)--(6.346,4.506)--(6.348,4.504)--(6.350,4.503)%
  --(6.353,4.501)--(6.355,4.499)--(6.357,4.498)--(6.359,4.496)--(6.361,4.494)--(6.364,4.493)%
  --(6.366,4.491)--(6.368,4.489)--(6.370,4.487)--(6.372,4.486)--(6.374,4.484)--(6.377,4.482)%
  --(6.379,4.481)--(6.381,4.479)--(6.383,4.477)--(6.385,4.476)--(6.388,4.474)--(6.390,4.472)%
  --(6.392,4.470)--(6.394,4.469)--(6.396,4.467)--(6.399,4.465)--(6.401,4.464)--(6.403,4.462)%
  --(6.405,4.460)--(6.407,4.458)--(6.409,4.457)--(6.412,4.455)--(6.414,4.453)--(6.416,4.451)%
  --(6.418,4.450)--(6.420,4.448)--(6.423,4.446)--(6.425,4.444)--(6.427,4.443)--(6.429,4.441)%
  --(6.431,4.439)--(6.433,4.437)--(6.436,4.436)--(6.438,4.434)--(6.440,4.432)--(6.442,4.430)%
  --(6.444,4.429)--(6.447,4.427)--(6.449,4.425)--(6.451,4.423)--(6.453,4.422)--(6.455,4.420)%
  --(6.457,4.418)--(6.460,4.416)--(6.462,4.414)--(6.464,4.413)--(6.466,4.411)--(6.468,4.409)%
  --(6.471,4.407)--(6.473,4.406)--(6.475,4.404)--(6.477,4.402)--(6.479,4.400)--(6.482,4.398)%
  --(6.484,4.397)--(6.486,4.395)--(6.488,4.393)--(6.490,4.391)--(6.492,4.389)--(6.495,4.388)%
  --(6.497,4.386)--(6.499,4.384)--(6.501,4.382)--(6.503,4.380)--(6.506,4.379)--(6.508,4.377)%
  --(6.510,4.375)--(6.512,4.373)--(6.514,4.371)--(6.516,4.370)--(6.519,4.368)--(6.521,4.366)%
  --(6.523,4.364)--(6.525,4.362)--(6.527,4.360)--(6.530,4.359)--(6.532,4.357)--(6.534,4.355)%
  --(6.536,4.353)--(6.538,4.351)--(6.541,4.350)--(6.543,4.348)--(6.545,4.346)--(6.547,4.344)%
  --(6.549,4.342)--(6.551,4.340)--(6.554,4.339)--(6.556,4.337)--(6.558,4.335)--(6.560,4.333)%
  --(6.562,4.331)--(6.565,4.329)--(6.567,4.328)--(6.569,4.326)--(6.571,4.324)--(6.573,4.322)%
  --(6.575,4.320)--(6.578,4.318)--(6.580,4.317)--(6.582,4.315)--(6.584,4.313)--(6.586,4.311)%
  --(6.589,4.309)--(6.591,4.307)--(6.593,4.305)--(6.595,4.304)--(6.597,4.302)--(6.600,4.300)%
  --(6.602,4.298)--(6.604,4.296)--(6.606,4.294)--(6.608,4.293)--(6.610,4.291)--(6.613,4.289)%
  --(6.615,4.287)--(6.617,4.285)--(6.619,4.283)--(6.621,4.281)--(6.624,4.280)--(6.626,4.278)%
  --(6.628,4.276)--(6.630,4.274)--(6.632,4.272)--(6.634,4.270)--(6.637,4.268)--(6.639,4.267)%
  --(6.641,4.265)--(6.643,4.263)--(6.645,4.261)--(6.648,4.259)--(6.650,4.257)--(6.652,4.255)%
  --(6.654,4.254)--(6.656,4.252)--(6.659,4.250)--(6.661,4.248)--(6.663,4.246)--(6.665,4.244)%
  --(6.667,4.242)--(6.669,4.240)--(6.672,4.239)--(6.674,4.237)--(6.676,4.235)--(6.678,4.233)%
  --(6.680,4.231)--(6.683,4.229)--(6.685,4.227)--(6.687,4.225)--(6.689,4.224)--(6.691,4.222)%
  --(6.693,4.220)--(6.696,4.218)--(6.698,4.216)--(6.700,4.214)--(6.702,4.212)--(6.704,4.211)%
  --(6.707,4.209)--(6.709,4.207)--(6.711,4.205)--(6.713,4.203)--(6.715,4.201)--(6.718,4.199)%
  --(6.720,4.197)--(6.722,4.195)--(6.724,4.194)--(6.726,4.192)--(6.728,4.190)--(6.731,4.188)%
  --(6.733,4.186)--(6.735,4.184)--(6.737,4.182)--(6.739,4.180)--(6.742,4.179)--(6.744,4.177)%
  --(6.746,4.175)--(6.748,4.173)--(6.750,4.171)--(6.752,4.169)--(6.755,4.167)--(6.757,4.165)%
  --(6.759,4.164)--(6.761,4.162)--(6.763,4.160)--(6.766,4.158)--(6.768,4.156)--(6.770,4.154)%
  --(6.772,4.152)--(6.774,4.150)--(6.777,4.148)--(6.779,4.147)--(6.781,4.145)--(6.783,4.143)%
  --(6.785,4.141)--(6.787,4.139)--(6.790,4.137)--(6.792,4.135)--(6.794,4.133)--(6.796,4.131)%
  --(6.798,4.130)--(6.801,4.128)--(6.803,4.126)--(6.805,4.124)--(6.807,4.122)--(6.809,4.120)%
  --(6.811,4.118)--(6.814,4.116)--(6.816,4.114)--(6.818,4.113)--(6.820,4.111)--(6.822,4.109)%
  --(6.825,4.107)--(6.827,4.105)--(6.829,4.103)--(6.831,4.101)--(6.833,4.099)--(6.836,4.097)%
  --(6.838,4.096)--(6.840,4.094)--(6.842,4.092)--(6.844,4.090)--(6.846,4.088)--(6.849,4.086)%
  --(6.851,4.084)--(6.853,4.082)--(6.855,4.080)--(6.857,4.079)--(6.860,4.077)--(6.862,4.075)%
  --(6.864,4.073)--(6.866,4.071)--(6.868,4.069)--(6.870,4.067)--(6.873,4.065)--(6.875,4.063)%
  --(6.877,4.062)--(6.879,4.060)--(6.881,4.058)--(6.884,4.056)--(6.886,4.054)--(6.888,4.052)%
  --(6.890,4.050)--(6.892,4.048)--(6.895,4.046)--(6.897,4.044)--(6.899,4.043)--(6.901,4.041)%
  --(6.903,4.039)--(6.905,4.037)--(6.908,4.035)--(6.910,4.033)--(6.912,4.031)--(6.914,4.029)%
  --(6.916,4.027)--(6.919,4.026)--(6.921,4.024)--(6.923,4.022)--(6.925,4.020)--(6.927,4.018)%
  --(6.929,4.016)--(6.932,4.014)--(6.934,4.012)--(6.936,4.010)--(6.938,4.008)--(6.940,4.007)%
  --(6.943,4.005)--(6.945,4.003)--(6.947,4.001)--(6.949,3.999)--(6.951,3.997)--(6.954,3.995)%
  --(6.956,3.993)--(6.958,3.991)--(6.960,3.990)--(6.962,3.988)--(6.964,3.986)--(6.967,3.984)%
  --(6.969,3.982)--(6.971,3.980)--(6.973,3.978)--(6.975,3.976)--(6.978,3.974)--(6.980,3.972)%
  --(6.982,3.971)--(6.984,3.969)--(6.986,3.967)--(6.988,3.965)--(6.991,3.963)--(6.993,3.961)%
  --(6.995,3.959)--(6.997,3.957)--(6.999,3.955)--(7.002,3.953)--(7.004,3.952)--(7.006,3.950)%
  --(7.008,3.948)--(7.010,3.946)--(7.013,3.944)--(7.015,3.942)--(7.017,3.940)--(7.019,3.938)%
  --(7.021,3.936)--(7.023,3.934)--(7.026,3.933)--(7.028,3.931)--(7.030,3.929)--(7.032,3.927)%
  --(7.034,3.925)--(7.037,3.923)--(7.039,3.921)--(7.041,3.919)--(7.043,3.917)--(7.045,3.915)%
  --(7.047,3.914)--(7.050,3.912)--(7.052,3.910)--(7.054,3.908)--(7.056,3.906)--(7.058,3.904)%
  --(7.061,3.902)--(7.063,3.900)--(7.065,3.898)--(7.067,3.896)--(7.069,3.895)--(7.071,3.893)%
  --(7.074,3.891)--(7.076,3.889)--(7.078,3.887)--(7.080,3.885)--(7.082,3.883)--(7.085,3.881)%
  --(7.087,3.879)--(7.089,3.877)--(7.091,3.876)--(7.093,3.874)--(7.096,3.872)--(7.098,3.870)%
  --(7.100,3.868)--(7.102,3.866)--(7.104,3.864)--(7.106,3.862)--(7.109,3.860)--(7.111,3.858)%
  --(7.113,3.857)--(7.115,3.855)--(7.117,3.853)--(7.120,3.851)--(7.122,3.849)--(7.124,3.847)%
  --(7.126,3.845)--(7.128,3.843)--(7.130,3.841)--(7.133,3.839)--(7.135,3.838)--(7.137,3.836)%
  --(7.139,3.834)--(7.141,3.832)--(7.144,3.830)--(7.146,3.828)--(7.148,3.826)--(7.150,3.824)%
  --(7.152,3.822)--(7.155,3.820)--(7.157,3.819)--(7.159,3.817)--(7.161,3.815)--(7.163,3.813)%
  --(7.165,3.811)--(7.168,3.809)--(7.170,3.807)--(7.172,3.805)--(7.174,3.803)--(7.176,3.801)%
  --(7.179,3.800)--(7.181,3.798)--(7.183,3.796)--(7.185,3.794)--(7.187,3.792)--(7.189,3.790)%
  --(7.192,3.788)--(7.194,3.786)--(7.196,3.784)--(7.198,3.782)--(7.200,3.781)--(7.203,3.779)%
  --(7.205,3.777)--(7.207,3.775)--(7.209,3.773)--(7.211,3.771)--(7.214,3.769)--(7.216,3.767)%
  --(7.218,3.765)--(7.220,3.763)--(7.222,3.761)--(7.224,3.760)--(7.227,3.758)--(7.229,3.756)%
  --(7.231,3.754)--(7.233,3.752)--(7.235,3.750)--(7.238,3.748)--(7.240,3.746)--(7.242,3.744)%
  --(7.244,3.742)--(7.246,3.741)--(7.248,3.739)--(7.251,3.737)--(7.253,3.735)--(7.255,3.733)%
  --(7.257,3.731)--(7.259,3.729)--(7.262,3.727)--(7.264,3.725)--(7.266,3.723)--(7.268,3.722)%
  --(7.270,3.720)--(7.273,3.718)--(7.275,3.716)--(7.277,3.714)--(7.279,3.712)--(7.281,3.710)%
  --(7.283,3.708)--(7.286,3.706)--(7.288,3.704)--(7.290,3.703)--(7.292,3.701)--(7.294,3.699)%
  --(7.297,3.697)--(7.299,3.695)--(7.301,3.693)--(7.303,3.691)--(7.305,3.689)--(7.307,3.687)%
  --(7.310,3.685)--(7.312,3.684)--(7.314,3.682)--(7.316,3.680)--(7.318,3.678)--(7.321,3.676)%
  --(7.323,3.674)--(7.325,3.672)--(7.327,3.670)--(7.329,3.668)--(7.332,3.666)--(7.334,3.664)%
  --(7.336,3.663)--(7.338,3.661)--(7.340,3.659)--(7.342,3.657)--(7.345,3.655)--(7.347,3.653)%
  --(7.349,3.651)--(7.351,3.649)--(7.353,3.647)--(7.356,3.645)--(7.358,3.644)--(7.360,3.642)%
  --(7.362,3.640)--(7.364,3.638)--(7.366,3.636)--(7.369,3.634)--(7.371,3.632)--(7.373,3.630)%
  --(7.375,3.628)--(7.377,3.626)--(7.380,3.625)--(7.382,3.623)--(7.384,3.621)--(7.386,3.619)%
  --(7.388,3.617)--(7.391,3.615)--(7.393,3.613)--(7.395,3.611)--(7.397,3.609)--(7.399,3.607)%
  --(7.401,3.605)--(7.404,3.604)--(7.406,3.602)--(7.408,3.600)--(7.410,3.598)--(7.412,3.596)%
  --(7.415,3.594)--(7.417,3.592)--(7.419,3.590)--(7.421,3.588)--(7.423,3.586)--(7.425,3.585)%
  --(7.428,3.583)--(7.430,3.581)--(7.432,3.579)--(7.434,3.577)--(7.436,3.575)--(7.439,3.573)%
  --(7.441,3.571)--(7.443,3.569)--(7.445,3.567)--(7.447,3.566)--(7.450,3.564)--(7.452,3.562)%
  --(7.454,3.560)--(7.456,3.558)--(7.458,3.556)--(7.460,3.554)--(7.463,3.552)--(7.465,3.550)%
  --(7.467,3.548)--(7.469,3.547)--(7.471,3.545)--(7.474,3.543)--(7.476,3.541)--(7.478,3.539)%
  --(7.480,3.537)--(7.482,3.535)--(7.484,3.533)--(7.487,3.531)--(7.489,3.529)--(7.491,3.527)%
  --(7.493,3.526)--(7.495,3.524)--(7.498,3.522)--(7.500,3.520)--(7.502,3.518)--(7.504,3.516)%
  --(7.506,3.514)--(7.509,3.512)--(7.511,3.510)--(7.513,3.508)--(7.515,3.507)--(7.517,3.505)%
  --(7.519,3.503)--(7.522,3.501)--(7.524,3.499)--(7.526,3.497)--(7.528,3.495)--(7.530,3.493)%
  --(7.533,3.491)--(7.535,3.489)--(7.537,3.488)--(7.539,3.486)--(7.541,3.484)--(7.543,3.482)%
  --(7.546,3.480)--(7.548,3.478)--(7.550,3.476)--(7.552,3.474)--(7.554,3.472)--(7.557,3.470)%
  --(7.559,3.468)--(7.561,3.467)--(7.563,3.465)--(7.565,3.463)--(7.568,3.461)--(7.570,3.459)%
  --(7.572,3.457)--(7.574,3.455)--(7.576,3.453)--(7.578,3.451)--(7.581,3.449)--(7.583,3.448)%
  --(7.585,3.446)--(7.587,3.444)--(7.589,3.442)--(7.592,3.440)--(7.594,3.438)--(7.596,3.436)%
  --(7.598,3.434)--(7.600,3.432)--(7.602,3.430)--(7.605,3.429)--(7.607,3.427)--(7.609,3.425)%
  --(7.611,3.423)--(7.613,3.421)--(7.616,3.419)--(7.618,3.417)--(7.620,3.415)--(7.622,3.413)%
  --(7.624,3.411)--(7.626,3.410)--(7.629,3.408)--(7.631,3.406)--(7.633,3.404)--(7.635,3.402)%
  --(7.637,3.400)--(7.640,3.398)--(7.642,3.396)--(7.644,3.394)--(7.646,3.392)--(7.648,3.390)%
  --(7.651,3.389)--(7.653,3.387)--(7.655,3.385)--(7.657,3.383)--(7.659,3.381)--(7.661,3.379)%
  --(7.664,3.377)--(7.666,3.375)--(7.668,3.373)--(7.670,3.371)--(7.672,3.370)--(7.675,3.368)%
  --(7.677,3.366)--(7.679,3.364)--(7.681,3.362)--(7.683,3.360)--(7.685,3.358)--(7.688,3.356)%
  --(7.690,3.354)--(7.692,3.352)--(7.694,3.351)--(7.696,3.349)--(7.699,3.347)--(7.701,3.345)%
  --(7.703,3.343)--(7.705,3.341)--(7.707,3.339)--(7.710,3.337)--(7.712,3.335)--(7.714,3.333)%
  --(7.716,3.331)--(7.718,3.330)--(7.720,3.328)--(7.723,3.326)--(7.725,3.324)--(7.727,3.322)%
  --(7.729,3.320)--(7.731,3.318)--(7.734,3.316)--(7.736,3.314)--(7.738,3.312)--(7.740,3.311)%
  --(7.742,3.309)--(7.744,3.307)--(7.747,3.305)--(7.749,3.303)--(7.751,3.301)--(7.753,3.299)%
  --(7.755,3.297)--(7.758,3.295)--(7.760,3.293)--(7.762,3.292)--(7.764,3.290)--(7.766,3.288)%
  --(7.769,3.286)--(7.771,3.284)--(7.773,3.282)--(7.775,3.280)--(7.777,3.278)--(7.779,3.276)%
  --(7.782,3.274)--(7.784,3.272)--(7.786,3.271)--(7.788,3.269)--(7.790,3.267)--(7.793,3.265)%
  --(7.795,3.263)--(7.797,3.261)--(7.799,3.259)--(7.801,3.257)--(7.803,3.255)--(7.806,3.253)%
  --(7.808,3.252)--(7.810,3.250)--(7.812,3.248)--(7.814,3.246)--(7.817,3.244)--(7.819,3.242)%
  --(7.821,3.240)--(7.823,3.238)--(7.825,3.236)--(7.828,3.234)--(7.830,3.233)--(7.832,3.231)%
  --(7.834,3.229)--(7.836,3.227)--(7.838,3.225)--(7.841,3.223)--(7.843,3.221)--(7.845,3.219)%
  --(7.847,3.217)--(7.849,3.215)--(7.852,3.214)--(7.854,3.212)--(7.856,3.210)--(7.858,3.208)%
  --(7.860,3.206)--(7.862,3.204)--(7.865,3.202)--(7.867,3.200)--(7.869,3.198)--(7.871,3.196)%
  --(7.873,3.194)--(7.876,3.193)--(7.878,3.191)--(7.880,3.189)--(7.882,3.187)--(7.884,3.185)%
  --(7.887,3.183)--(7.889,3.181)--(7.891,3.179)--(7.893,3.177)--(7.895,3.175)--(7.897,3.174)%
  --(7.900,3.172)--(7.902,3.170)--(7.904,3.168)--(7.906,3.166)--(7.908,3.164)--(7.911,3.162)%
  --(7.913,3.160)--(7.915,3.158)--(7.917,3.156)--(7.919,3.155)--(7.921,3.153)--(7.924,3.151)%
  --(7.926,3.149)--(7.928,3.147)--(7.930,3.145)--(7.932,3.143)--(7.935,3.141)--(7.937,3.139)%
  --(7.939,3.137)--(7.941,3.135)--(7.943,3.134)--(7.946,3.132)--(7.948,3.130)--(7.950,3.128)%
  --(7.952,3.126)--(7.954,3.124)--(7.956,3.122)--(7.959,3.120)--(7.961,3.118)--(7.963,3.116)%
  --(7.965,3.115)--(7.967,3.113)--(7.970,3.111)--(7.972,3.109)--(7.974,3.107)--(7.976,3.105)%
  --(7.978,3.103)--(7.980,3.101)--(7.983,3.099)--(7.985,3.097)--(7.987,3.096)--(7.989,3.094)%
  --(7.991,3.092)--(7.994,3.090)--(7.996,3.088)--(7.998,3.086)--(8.000,3.084)--(8.002,3.082)%
  --(8.005,3.080)--(8.007,3.078)--(8.009,3.076)--(8.011,3.075)--(8.013,3.073)--(8.015,3.071)%
  --(8.018,3.069)--(8.020,3.067)--(8.022,3.065)--(8.024,3.063)--(8.026,3.061)--(8.029,3.059)%
  --(8.031,3.057)--(8.033,3.056)--(8.035,3.054)--(8.037,3.052)--(8.039,3.050)--(8.042,3.048)%
  --(8.044,3.046)--(8.046,3.044)--(8.048,3.042)--(8.050,3.040)--(8.053,3.038)--(8.055,3.037)%
  --(8.057,3.035)--(8.059,3.033)--(8.061,3.031)--(8.064,3.029)--(8.066,3.027)--(8.068,3.025)%
  --(8.070,3.023)--(8.072,3.021)--(8.074,3.019)--(8.077,3.018)--(8.079,3.016)--(8.081,3.014)%
  --(8.083,3.012)--(8.085,3.010)--(8.088,3.008)--(8.090,3.006)--(8.092,3.004)--(8.094,3.002)%
  --(8.096,3.000)--(8.098,2.998)--(8.101,2.997)--(8.103,2.995)--(8.105,2.993)--(8.107,2.991)%
  --(8.109,2.989)--(8.112,2.987)--(8.114,2.985)--(8.116,2.983)--(8.118,2.981)--(8.120,2.979)%
  --(8.123,2.978)--(8.125,2.976)--(8.127,2.974)--(8.129,2.972)--(8.131,2.970)--(8.133,2.968)%
  --(8.136,2.966)--(8.138,2.964)--(8.140,2.962)--(8.142,2.960)--(8.144,2.959)--(8.147,2.957)%
  --(8.149,2.955)--(8.151,2.953)--(8.153,2.951)--(8.155,2.949)--(8.157,2.947)--(8.160,2.945)%
  --(8.162,2.943)--(8.164,2.941)--(8.166,2.939)--(8.168,2.938)--(8.171,2.936)--(8.173,2.934)%
  --(8.175,2.932)--(8.177,2.930)--(8.179,2.928)--(8.182,2.926)--(8.184,2.924)--(8.186,2.922)%
  --(8.188,2.920)--(8.190,2.919)--(8.192,2.917)--(8.195,2.915)--(8.197,2.913)--(8.199,2.911)%
  --(8.201,2.909)--(8.203,2.907)--(8.206,2.905)--(8.208,2.903)--(8.210,2.901)--(8.212,2.900)%
  --(8.214,2.898)--(8.216,2.896)--(8.219,2.894)--(8.221,2.892)--(8.223,2.890)--(8.225,2.888)%
  --(8.227,2.886)--(8.230,2.884)--(8.232,2.882)--(8.234,2.880)--(8.236,2.879)--(8.238,2.877)%
  --(8.240,2.875)--(8.243,2.873)--(8.245,2.871)--(8.247,2.869)--(8.249,2.867)--(8.251,2.865)%
  --(8.254,2.863)--(8.256,2.861)--(8.258,2.860)--(8.260,2.858)--(8.262,2.856)--(8.265,2.854)%
  --(8.267,2.852)--(8.269,2.850)--(8.271,2.848)--(8.273,2.846)--(8.275,2.844)--(8.278,2.842)%
  --(8.280,2.841)--(8.282,2.839)--(8.284,2.837)--(8.286,2.835)--(8.289,2.833)--(8.291,2.831)%
  --(8.293,2.829)--(8.295,2.827)--(8.297,2.825)--(8.299,2.823)--(8.302,2.821)--(8.304,2.820)%
  --(8.306,2.818)--(8.308,2.816)--(8.310,2.814)--(8.313,2.812)--(8.315,2.810)--(8.317,2.808)%
  --(8.319,2.806)--(8.321,2.804)--(8.324,2.802)--(8.326,2.801)--(8.328,2.799)--(8.330,2.797)%
  --(8.332,2.795)--(8.334,2.793)--(8.337,2.791)--(8.339,2.789)--(8.341,2.787)--(8.343,2.785)%
  --(8.345,2.783)--(8.348,2.782)--(8.350,2.780)--(8.352,2.778)--(8.354,2.776)--(8.356,2.774)%
  --(8.358,2.772)--(8.361,2.770)--(8.363,2.768)--(8.365,2.766)--(8.367,2.764)--(8.369,2.763)%
  --(8.372,2.761)--(8.374,2.759)--(8.376,2.757)--(8.378,2.755)--(8.380,2.753)--(8.383,2.751)%
  --(8.385,2.749)--(8.387,2.747)--(8.389,2.745)--(8.391,2.743)--(8.393,2.742)--(8.396,2.740)%
  --(8.398,2.738)--(8.400,2.736)--(8.402,2.734)--(8.404,2.732)--(8.407,2.730)--(8.409,2.728)%
  --(8.411,2.726)--(8.413,2.724)--(8.415,2.723)--(8.417,2.721)--(8.420,2.719)--(8.422,2.717)%
  --(8.424,2.715)--(8.426,2.713)--(8.428,2.711)--(8.431,2.709)--(8.433,2.707)--(8.435,2.705)%
  --(8.437,2.704)--(8.439,2.702)--(8.442,2.700)--(8.444,2.698)--(8.446,2.696)--(8.448,2.694)%
  --(8.450,2.692)--(8.452,2.690)--(8.455,2.688)--(8.457,2.686)--(8.459,2.684)--(8.461,2.683)%
  --(8.463,2.681)--(8.466,2.679)--(8.468,2.677)--(8.470,2.675)--(8.472,2.673)--(8.474,2.671)%
  --(8.476,2.669)--(8.479,2.667)--(8.481,2.665)--(8.483,2.664)--(8.485,2.662)--(8.487,2.660)%
  --(8.490,2.658)--(8.492,2.656)--(8.494,2.654)--(8.496,2.652)--(8.498,2.650)--(8.501,2.648)%
  --(8.503,2.646)--(8.505,2.645)--(8.507,2.643)--(8.509,2.641)--(8.511,2.639)--(8.514,2.637)%
  --(8.516,2.635)--(8.518,2.633)--(8.520,2.631)--(8.522,2.629)--(8.525,2.627)--(8.527,2.625)%
  --(8.529,2.624)--(8.531,2.622)--(8.533,2.620)--(8.535,2.618)--(8.538,2.616)--(8.540,2.614)%
  --(8.542,2.612)--(8.544,2.610)--(8.546,2.608)--(8.549,2.606)--(8.551,2.605)--(8.553,2.603)%
  --(8.555,2.601)--(8.557,2.599)--(8.560,2.597)--(8.562,2.595)--(8.564,2.593)--(8.566,2.591)%
  --(8.568,2.589)--(8.570,2.587)--(8.573,2.586)--(8.575,2.584)--(8.577,2.582)--(8.579,2.580)%
  --(8.581,2.578)--(8.584,2.576)--(8.586,2.574)--(8.588,2.572)--(8.590,2.570)--(8.592,2.568)%
  --(8.594,2.566)--(8.597,2.565)--(8.599,2.563)--(8.601,2.561)--(8.603,2.559)--(8.605,2.557)%
  --(8.608,2.555)--(8.610,2.553)--(8.612,2.551)--(8.614,2.549)--(8.616,2.547)--(8.619,2.546)%
  --(8.621,2.544)--(8.623,2.542)--(8.625,2.540)--(8.627,2.538)--(8.629,2.536)--(8.632,2.534)%
  --(8.634,2.532)--(8.636,2.530)--(8.638,2.528)--(8.640,2.527)--(8.643,2.525)--(8.645,2.523)%
  --(8.647,2.521)--(8.649,2.519)--(8.651,2.517)--(8.653,2.515)--(8.656,2.513)--(8.658,2.511)%
  --(8.660,2.509)--(8.662,2.508)--(8.664,2.506)--(8.667,2.504)--(8.669,2.502)--(8.671,2.500)%
  --(8.673,2.498)--(8.675,2.496)--(8.678,2.494)--(8.680,2.492)--(8.682,2.490)--(8.684,2.488)%
  --(8.686,2.487)--(8.688,2.485)--(8.691,2.483)--(8.693,2.481)--(8.695,2.479)--(8.697,2.477)%
  --(8.699,2.475)--(8.702,2.473)--(8.704,2.471)--(8.706,2.469)--(8.708,2.468)--(8.710,2.466)%
  --(8.712,2.464)--(8.715,2.462)--(8.717,2.460)--(8.719,2.458)--(8.721,2.456)--(8.723,2.454)%
  --(8.726,2.452)--(8.728,2.450)--(8.730,2.449)--(8.732,2.447)--(8.734,2.445)--(8.737,2.443)%
  --(8.739,2.441)--(8.741,2.439)--(8.743,2.437)--(8.745,2.435)--(8.747,2.433)--(8.750,2.431)%
  --(8.752,2.429)--(8.754,2.428)--(8.756,2.426)--(8.758,2.424)--(8.761,2.422)--(8.763,2.420)%
  --(8.765,2.418)--(8.767,2.416)--(8.769,2.414)--(8.771,2.412)--(8.774,2.410)--(8.776,2.409)%
  --(8.778,2.407)--(8.780,2.405)--(8.782,2.403)--(8.785,2.401)--(8.787,2.399)--(8.789,2.397)%
  --(8.791,2.395)--(8.793,2.393)--(8.795,2.391)--(8.798,2.390)--(8.800,2.388)--(8.802,2.386)%
  --(8.804,2.384)--(8.806,2.382)--(8.809,2.380)--(8.811,2.378)--(8.813,2.376)--(8.815,2.374)%
  --(8.817,2.372)--(8.820,2.370)--(8.822,2.369)--(8.824,2.367)--(8.826,2.365)--(8.828,2.363)%
  --(8.830,2.361)--(8.833,2.359)--(8.835,2.357)--(8.837,2.355)--(8.839,2.353)--(8.841,2.351)%
  --(8.844,2.350)--(8.846,2.348)--(8.848,2.346)--(8.850,2.344)--(8.852,2.342)--(8.854,2.340)%
  --(8.857,2.338)--(8.859,2.336)--(8.861,2.334)--(8.863,2.332)--(8.865,2.331)--(8.868,2.329)%
  --(8.870,2.327)--(8.872,2.325)--(8.874,2.323)--(8.876,2.321)--(8.879,2.319)--(8.881,2.317)%
  --(8.883,2.315)--(8.885,2.313)--(8.887,2.311)--(8.889,2.310)--(8.892,2.308)--(8.894,2.306)%
  --(8.896,2.304)--(8.898,2.302)--(8.900,2.300)--(8.903,2.298)--(8.905,2.296)--(8.907,2.294)%
  --(8.909,2.292)--(8.911,2.291)--(8.913,2.289)--(8.916,2.287)--(8.918,2.285)--(8.920,2.283)%
  --(8.922,2.281)--(8.924,2.279)--(8.927,2.277)--(8.929,2.275)--(8.931,2.273)--(8.933,2.272)%
  --(8.935,2.270)--(8.938,2.268)--(8.940,2.266)--(8.942,2.264)--(8.944,2.262)--(8.946,2.260)%
  --(8.948,2.258)--(8.951,2.256)--(8.953,2.254)--(8.955,2.253)--(8.957,2.251)--(8.959,2.249)%
  --(8.962,2.247)--(8.964,2.245)--(8.966,2.243)--(8.968,2.241)--(8.970,2.239)--(8.972,2.237)%
  --(8.975,2.235)--(8.977,2.233)--(8.979,2.232)--(8.981,2.230)--(8.983,2.228)--(8.986,2.226)%
  --(8.988,2.224)--(8.990,2.222)--(8.992,2.220)--(8.994,2.218)--(8.997,2.216)--(8.999,2.214)%
  --(9.001,2.213)--(9.003,2.211)--(9.005,2.209)--(9.007,2.207)--(9.010,2.205)--(9.012,2.203)%
  --(9.014,2.201)--(9.016,2.199)--(9.018,2.197)--(9.021,2.195)--(9.023,2.194)--(9.025,2.192)%
  --(9.027,2.190)--(9.029,2.188)--(9.031,2.186)--(9.034,2.184)--(9.036,2.182)--(9.038,2.180)%
  --(9.040,2.178)--(9.042,2.176)--(9.045,2.174)--(9.047,2.173)--(9.049,2.171)--(9.051,2.169)%
  --(9.053,2.167)--(9.056,2.165)--(9.058,2.163)--(9.060,2.161)--(9.062,2.159)--(9.064,2.157)%
  --(9.066,2.155)--(9.069,2.154)--(9.071,2.152)--(9.073,2.150)--(9.075,2.148)--(9.077,2.146)%
  --(9.080,2.144)--(9.082,2.142)--(9.084,2.140)--(9.086,2.138)--(9.088,2.136)--(9.090,2.135)%
  --(9.093,2.133)--(9.095,2.131)--(9.097,2.129)--(9.099,2.127)--(9.101,2.125)--(9.104,2.123)%
  --(9.106,2.121)--(9.108,2.119)--(9.110,2.117)--(9.112,2.115)--(9.115,2.114)--(9.117,2.112)%
  --(9.119,2.110)--(9.121,2.108)--(9.123,2.106)--(9.125,2.104)--(9.128,2.102)--(9.130,2.100)%
  --(9.132,2.098)--(9.134,2.096)--(9.136,2.095)--(9.139,2.093)--(9.141,2.091)--(9.143,2.089)%
  --(9.145,2.087)--(9.147,2.085)--(9.149,2.083)--(9.152,2.081)--(9.154,2.079)--(9.156,2.077)%
  --(9.158,2.076)--(9.160,2.074)--(9.163,2.072)--(9.165,2.070)--(9.167,2.068)--(9.169,2.066)%
  --(9.171,2.064)--(9.174,2.062)--(9.176,2.060)--(9.178,2.058)--(9.180,2.057)--(9.182,2.055)%
  --(9.184,2.053)--(9.187,2.051)--(9.189,2.049)--(9.191,2.047)--(9.193,2.045)--(9.195,2.043)%
  --(9.198,2.041)--(9.200,2.039)--(9.202,2.037)--(9.204,2.036)--(9.206,2.034)--(9.208,2.032)%
  --(9.211,2.030)--(9.213,2.028)--(9.215,2.026)--(9.217,2.024)--(9.219,2.022)--(9.222,2.020)%
  --(9.224,2.018)--(9.226,2.017)--(9.228,2.015)--(9.230,2.013)--(9.233,2.011)--(9.235,2.009)%
  --(9.237,2.007)--(9.239,2.005)--(9.241,2.003)--(9.243,2.001)--(9.246,1.999)--(9.248,1.998)%
  --(9.250,1.996)--(9.252,1.994)--(9.254,1.992)--(9.257,1.990)--(9.259,1.988)--(9.261,1.986)%
  --(9.263,1.984)--(9.265,1.982)--(9.267,1.980)--(9.270,1.978)--(9.272,1.977)--(9.274,1.975)%
  --(9.276,1.973)--(9.278,1.971)--(9.281,1.969)--(9.283,1.967)--(9.285,1.965)--(9.287,1.963)%
  --(9.289,1.961)--(9.292,1.959)--(9.294,1.958)--(9.296,1.956)--(9.298,1.954)--(9.300,1.952)%
  --(9.302,1.950)--(9.305,1.948)--(9.307,1.946)--(9.309,1.944)--(9.311,1.942)--(9.313,1.940)%
  --(9.316,1.939)--(9.318,1.937)--(9.320,1.935)--(9.322,1.933)--(9.324,1.931)--(9.326,1.929)%
  --(9.329,1.927)--(9.331,1.925)--(9.333,1.923)--(9.335,1.921)--(9.337,1.919)--(9.340,1.918)%
  --(9.342,1.916)--(9.344,1.914)--(9.346,1.912)--(9.348,1.910)--(9.351,1.908)--(9.353,1.906)%
  --(9.355,1.904)--(9.357,1.902)--(9.359,1.900)--(9.361,1.899)--(9.364,1.897)--(9.366,1.895)%
  --(9.368,1.893)--(9.370,1.891)--(9.372,1.889)--(9.375,1.887)--(9.377,1.885)--(9.379,1.883)%
  --(9.381,1.881)--(9.383,1.880)--(9.385,1.878)--(9.388,1.876)--(9.390,1.874)--(9.392,1.872)%
  --(9.394,1.870)--(9.396,1.868)--(9.399,1.866)--(9.401,1.864)--(9.403,1.862)--(9.405,1.860)%
  --(9.407,1.859)--(9.409,1.857)--(9.412,1.855)--(9.414,1.853)--(9.416,1.851)--(9.418,1.849)%
  --(9.420,1.847)--(9.423,1.845)--(9.425,1.843)--(9.427,1.841)--(9.429,1.840)--(9.431,1.838)%
  --(9.434,1.836)--(9.436,1.834)--(9.438,1.832)--(9.440,1.830)--(9.442,1.828)--(9.444,1.826)%
  --(9.447,1.824)--(9.449,1.822)--(9.451,1.821)--(9.453,1.819)--(9.455,1.817)--(9.458,1.815)%
  --(9.460,1.813)--(9.462,1.811)--(9.464,1.809)--(9.466,1.807)--(9.468,1.805)--(9.471,1.803)%
  --(9.473,1.802)--(9.475,1.800)--(9.477,1.798)--(9.479,1.796)--(9.482,1.794)--(9.484,1.792)%
  --(9.486,1.790)--(9.488,1.788)--(9.490,1.786)--(9.493,1.784)--(9.495,1.782)--(9.497,1.781)%
  --(9.499,1.779)--(9.501,1.777)--(9.503,1.775)--(9.506,1.773)--(9.508,1.771)--(9.510,1.769)%
  --(9.512,1.767)--(9.514,1.765)--(9.517,1.763)--(9.519,1.762)--(9.521,1.760)--(9.523,1.758)%
  --(9.525,1.756)--(9.527,1.754)--(9.530,1.752)--(9.532,1.750)--(9.534,1.748)--(9.536,1.746)%
  --(9.538,1.744)--(9.541,1.743)--(9.543,1.741)--(9.545,1.739)--(9.547,1.737)--(9.549,1.735)%
  --(9.552,1.733)--(9.554,1.731)--(9.556,1.729)--(9.558,1.727)--(9.560,1.725)--(9.562,1.723)%
  --(9.565,1.722)--(9.567,1.720)--(9.569,1.718)--(9.571,1.716)--(9.573,1.714)--(9.576,1.712)%
  --(9.578,1.710)--(9.580,1.708)--(9.582,1.706)--(9.584,1.704)--(9.586,1.703)--(9.589,1.701)%
  --(9.591,1.699)--(9.593,1.697)--(9.595,1.695)--(9.597,1.693)--(9.600,1.691)--(9.602,1.689)%
  --(9.604,1.687)--(9.606,1.685)--(9.608,1.684)--(9.611,1.682)--(9.613,1.680)--(9.615,1.678)%
  --(9.617,1.676)--(9.619,1.674)--(9.621,1.672)--(9.624,1.670)--(9.626,1.668)--(9.628,1.666)%
  --(9.630,1.664)--(9.632,1.663)--(9.635,1.661)--(9.637,1.659)--(9.639,1.657)--(9.641,1.655)%
  --(9.643,1.653)--(9.645,1.651)--(9.648,1.649)--(9.650,1.647)--(9.652,1.645)--(9.654,1.644)%
  --(9.656,1.642)--(9.659,1.640)--(9.661,1.638)--(9.663,1.636)--(9.665,1.634)--(9.667,1.632)%
  --(9.670,1.630)--(9.672,1.628)--(9.674,1.626)--(9.676,1.625)--(9.678,1.623)--(9.680,1.621)%
  --(9.683,1.619)--(9.685,1.617)--(9.687,1.615)--(9.689,1.613)--(9.691,1.611)--(9.694,1.609)%
  --(9.696,1.607)--(9.698,1.605)--(9.700,1.604)--(9.702,1.602)--(9.704,1.600)--(9.707,1.598)%
  --(9.709,1.596)--(9.711,1.594)--(9.713,1.592)--(9.715,1.590)--(9.718,1.588)--(9.720,1.586)%
  --(9.722,1.585)--(9.724,1.583)--(9.726,1.581)--(9.729,1.579)--(9.731,1.577)--(9.733,1.575)%
  --(9.735,1.573)--(9.737,1.571)--(9.739,1.569)--(9.742,1.567)--(9.744,1.566)--(9.746,1.564)%
  --(9.748,1.562)--(9.750,1.560)--(9.753,1.558)--(9.755,1.556)--(9.757,1.554)--(9.759,1.552)%
  --(9.761,1.550)--(9.763,1.548)--(9.766,1.547)--(9.768,1.545)--(9.770,1.543)--(9.772,1.541)%
  --(9.774,1.539)--(9.777,1.537)--(9.779,1.535)--(9.781,1.533)--(9.783,1.531)--(9.785,1.529)%
  --(9.788,1.527)--(9.790,1.526)--(9.792,1.524)--(9.794,1.522)--(9.796,1.520)--(9.798,1.518)%
  --(9.801,1.516)--(9.803,1.514)--(9.805,1.512)--(9.807,1.510)--(9.809,1.508)--(9.812,1.507)%
  --(9.814,1.505)--(9.816,1.503)--(9.818,1.501)--(9.820,1.499)--(9.822,1.497)--(9.825,1.495)%
  --(9.827,1.493)--(9.829,1.491)--(9.831,1.489)--(9.833,1.488)--(9.836,1.486)--(9.838,1.484)%
  --(9.840,1.482)--(9.842,1.480)--(9.844,1.478)--(9.847,1.476)--(9.849,1.474)--(9.851,1.472)%
  --(9.853,1.470)--(9.855,1.468)--(9.857,1.467)--(9.860,1.465)--(9.862,1.463)--(9.864,1.461)%
  --(9.866,1.459)--(9.868,1.457)--(9.871,1.455)--(9.873,1.453)--(9.875,1.451)--(9.877,1.449)%
  --(9.879,1.448)--(9.881,1.446)--(9.884,1.444)--(9.886,1.442)--(9.888,1.440)--(9.890,1.438)%
  --(9.892,1.436)--(9.895,1.434)--(9.897,1.432)--(9.899,1.430)--(9.901,1.429)--(9.903,1.427)%
  --(9.906,1.425)--(9.908,1.423)--(9.910,1.421)--(9.912,1.419)--(9.914,1.417)--(9.916,1.415)%
  --(9.919,1.413)--(9.921,1.411)--(9.923,1.409)--(9.925,1.408)--(9.927,1.406)--(9.930,1.404)%
  --(9.932,1.402)--(9.934,1.400)--(9.936,1.398)--(9.938,1.396)--(9.940,1.394)--(9.943,1.392)%
  --(9.945,1.390)--(9.947,1.389)--(9.949,1.387)--(9.951,1.385)--(9.954,1.383)--(9.956,1.381)%
  --(9.958,1.379)--(9.960,1.377)--(9.962,1.375)--(9.964,1.373)--(9.967,1.371)--(9.969,1.370)%
  --(9.971,1.368)--(9.973,1.366)--(9.975,1.364)--(9.978,1.362)--(9.980,1.360)--(9.982,1.358)%
  --(9.984,1.356)--(9.986,1.354)--(9.989,1.352)--(9.991,1.350)--(9.993,1.349)--(9.995,1.347)%
  --(9.997,1.345)--(9.999,1.343)--(10.002,1.341)--(10.004,1.339)--(10.006,1.337)--(10.008,1.335)%
  --(10.010,1.333)--(10.013,1.331)--(10.015,1.330)--(10.017,1.328)--(10.019,1.326)--(10.021,1.324)%
  --(10.023,1.322)--(10.026,1.320)--(10.028,1.318)--(10.030,1.316)--(10.032,1.314)--(10.034,1.312)%
  --(10.037,1.311)--(10.039,1.309)--(10.041,1.307)--(10.043,1.305)--(10.045,1.303)--(10.048,1.301)%
  --(10.050,1.299)--(10.052,1.297)--(10.054,1.295)--(10.056,1.293)--(10.058,1.292)--(10.061,1.290)%
  --(10.063,1.288)--(10.065,1.286)--(10.067,1.284)--(10.069,1.282)--(10.072,1.280)--(10.074,1.278)%
  --(10.076,1.276)--(10.078,1.274)--(10.080,1.272)--(10.082,1.271)--(10.085,1.269)--(10.087,1.267)%
  --(10.089,1.265)--(10.091,1.263)--(10.093,1.261)--(10.096,1.259)--(10.098,1.257)--(10.100,1.255)%
  --(10.102,1.253)--(10.104,1.252)--(10.107,1.250)--(10.109,1.248)--(10.111,1.246)--(10.113,1.244)%
  --(10.115,1.242)--(10.117,1.240)--(10.120,1.238)--(10.122,1.236)--(10.124,1.234)--(10.126,1.233)%
  --(10.128,1.231)--(10.131,1.229)--(10.133,1.227)--(10.135,1.225)--(10.137,1.223)--(10.139,1.221)%
  --(10.141,1.219)--(10.144,1.217)--(10.146,1.215)--(10.148,1.213)--(10.150,1.212)--(10.152,1.210)%
  --(10.155,1.208)--(10.157,1.206)--(10.159,1.204)--(10.161,1.202)--(10.163,1.200)--(10.166,1.198)%
  --(10.168,1.196)--(10.170,1.194)--(10.172,1.193)--(10.174,1.191)--(10.176,1.189)--(10.179,1.187)%
  --(10.181,1.185)--(10.183,1.183)--(10.185,1.181)--(10.187,1.179)--(10.190,1.177)--(10.192,1.175)%
  --(10.194,1.174)--(10.196,1.172)--(10.198,1.170)--(10.200,1.168)--(10.203,1.166)--(10.205,1.164)%
  --(10.207,1.162)--(10.209,1.160)--(10.211,1.158)--(10.214,1.156)--(10.216,1.154)--(10.218,1.153)%
  --(10.220,1.151)--(10.222,1.149)--(10.225,1.147)--(10.227,1.145)--(10.229,1.143)--(10.231,1.141)%
  --(10.233,1.139)--(10.235,1.137)--(10.238,1.135)--(10.240,1.134)--(10.242,1.132);
\gpcolor{gp lt color border}
\gpsetlinetype{gp lt border}
\gpsetlinewidth{1.00}
\draw[gp path] (1.504,4.790)--(1.504,0.985)--(10.242,0.985)--(10.242,4.790)--cycle;
%% coordinates of the plot area
\gpdefrectangularnode{gp plot 1}{\pgfpoint{1.504cm}{0.985cm}}{\pgfpoint{10.242cm}{4.790cm}}
\end{tikzpicture}
%% gnuplot variables

\end{figure}

\begin{figure}[H]
	\centering
	\input{img/plot/bodePlotComposite.tex}
\end{figure}
