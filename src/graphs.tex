\subsection{Rectifier Circuit}
The rectifier circuit in Figure~\ref{fig:bridgeSchem} was tested in three
configurations: once with no filter capacitor, once with a~\SI{1}{\micro\farad}
capacitor, and once with a~\SI{100}{\micro\farad}.  Additionally, the
peak-to-peak voltage, average voltage, and minimum voltage were measured using
an oscilloscope in each configuration, a screenshot of which is shown in
Figure~\ref{fig:rectShots}.
%
\begin{figure}[H]
	\centering
	\subfloat[No capacitor]{\label{fig:noCapShot}\includegraphics[width=.3\textwidth]{img/shot/bridgeRectNoC.png}}
	\quad
	\subfloat[\SI{1}{\micro\farad} capacitor]{\label{fig:1uFCapShot}\includegraphics[width=.3\textwidth]{img/shot/bridgeRect1uF.png}}
	\quad
	\subfloat[\SI{100}{\micro\farad} capacitor]{\label{fig:100uFCapShot}\includegraphics[width=.3\textwidth]{img/shot/bridgeRect100uF.png}}

	\parbox{.9\textwidth}{
	\caption[Oscilloscope Screenshot --- Rectifier Circuit]{Various outputs of
	the rectifier circuit shown in Figure~\ref{fig:bridgeSchem}.  Note that the
	vertical scale in image~\subref{fig:100uFCapShot} is~\SI{.5}{\volt\per
	div}, while the others are~\SI{5}{\volt\per div}.}
	\label{fig:rectShots}}
\end{figure}
%
Data measured by the oscilloscope matches almost exactly the output expected
from the bridge rectifier shown in Figures~\ref{fig:bridgeRectOut}
and~\ref{fig:bridgeRectOutFilt}.  As the size of the filtering capacitor
increases, the output variation decreases.  For each rectifier circuit, the minimum, average, and maximum
values of each waveform were recorded and tabulated into
table~\ref{tab:rectData}.
%
\begin{table}[H]
	\centering
	\begin{tabular}{|c|c|c|c|}
	\hline
	\tbf{Capacitor (\si{\micro\farad}} &
		$\mathbf{V_{p-p}}$ \tbf{(\si{\volt})}&
			$\mathbf{V_{avg}}$ \tbf{(\si{\volt})}&
				$\mathbf{V_{min}}$ \tbf{(\si{\volt})} \\ \hline

	0         & 18.91	& 11.80 	& 0.0  \\ \hline
	1         & 17.34   & 12.13		& 1.6  \\ \hline
    100       & 1.56	& 18.03     & 17.2 \\ \hline
\end{tabular}

	\parbox{.6\textwidth}{
	\caption[Rectifier data]{Tabulated data from the experiments on the bridge
	rectifier.  Note the sharp decrease in peak-to-peak voltage introduced by
	the filtering capacitor.}
	\label{tab:rectData}
	}
\end{table}
%
The recorded values match what is expected of these circuits: as the capacitor
value increases, peak-to-peak voltage decreases sharply, the average value
increases, and the minimum value greatly increases (as a result of the previous
two changes).


\subsection{IC}
Once this procedure has been completed, a~\SI{2.2}{\kilo\ohm} resistor was
placed in parallel with both~$R_1$ and~$R_2$ to for an adequate load.  The
source voltage was subsequently swept from~6 to~\SI{30}{\volt} in order to test
the line regulation of the IC, or the amount of change in output per change in
input.  Measurements at each step were recorded and plotted in
Figure~\ref{fig:vrLineReg}.
%
\begin{figure}[H]
	\centering
	\begin{tikzpicture}[gnuplot]
%% generated with GNUPLOT 4.4p2 (Lua 5.1.4; terminal rev. 97, script rev. 96a)
%% Sat 08 Oct 2011 01:30:50 PM EDT
\gpsolidlines
\gpcolor{gp lt color border}
\gpsetlinetype{gp lt border}
\gpsetlinewidth{1.00}
\draw[gp path] (1.136,0.985)--(1.316,0.985);
\node[gp node right] at (0.952,0.985) { 0};
\draw[gp path] (1.136,1.433)--(1.316,1.433);
\node[gp node right] at (0.952,1.433) { 1};
\draw[gp path] (1.136,1.880)--(1.316,1.880);
\node[gp node right] at (0.952,1.880) { 2};
\draw[gp path] (1.136,2.328)--(1.316,2.328);
\node[gp node right] at (0.952,2.328) { 3};
\draw[gp path] (1.136,2.776)--(1.316,2.776);
\node[gp node right] at (0.952,2.776) { 4};
\draw[gp path] (1.136,3.223)--(1.316,3.223);
\node[gp node right] at (0.952,3.223) { 5};
\draw[gp path] (1.136,3.671)--(1.316,3.671);
\node[gp node right] at (0.952,3.671) { 6};
\draw[gp path] (1.136,4.119)--(1.316,4.119);
\node[gp node right] at (0.952,4.119) { 7};
\draw[gp path] (1.136,4.566)--(1.316,4.566);
\node[gp node right] at (0.952,4.566) { 8};
\draw[gp path] (1.136,0.985)--(1.136,1.165);
\node[gp node center] at (1.136,0.677) { 0};
\draw[gp path] (2.654,0.985)--(2.654,1.165);
\node[gp node center] at (2.654,0.677) { 5};
\draw[gp path] (4.171,0.985)--(4.171,1.165);
\node[gp node center] at (4.171,0.677) { 10};
\draw[gp path] (5.689,0.985)--(5.689,1.165);
\node[gp node center] at (5.689,0.677) { 15};
\draw[gp path] (7.207,0.985)--(7.207,1.165);
\node[gp node center] at (7.207,0.677) { 20};
\draw[gp path] (8.724,0.985)--(8.724,1.165);
\node[gp node center] at (8.724,0.677) { 25};
\draw[gp path] (10.242,0.985)--(10.242,1.165);
\node[gp node center] at (10.242,0.677) { 30};
\draw[gp path] (1.136,4.790)--(1.136,0.985)--(10.242,0.985);
\node[gp node center,rotate=-270] at (0.246,2.887) {Output Voltage, $V_\text{out}$ (V)};
\node[gp node center] at (5.689,0.215) {Input Voltage, $V_\text{in}$ (V)};
\node[gp node center] at (5.689,5.252) {Line Regulation: Voltage Regulator Only};
\gpcolor{gp lt color 0}
\gpsetlinetype{gp lt plot 0}
\draw[gp path] (2.957,3.026)--(3.261,3.460)--(3.564,3.899)--(3.868,4.324)--(4.171,4.593)%
  --(4.475,4.575)--(4.778,4.571)--(5.082,4.575)--(5.385,4.575)--(5.689,4.575)--(5.993,4.575)%
  --(6.296,4.575)--(6.600,4.580)--(6.903,4.580)--(7.207,4.580)--(7.510,4.584)--(7.814,4.584)%
  --(8.117,4.593)--(8.421,4.593)--(8.724,4.589)--(9.028,4.589)--(9.331,4.589)--(9.635,4.593)%
  --(9.938,4.593)--(10.242,4.593);
\gpsetpointsize{4.00}
\gppoint{gp mark 1}{(2.957,3.026)}
\gppoint{gp mark 1}{(3.261,3.460)}
\gppoint{gp mark 1}{(3.564,3.899)}
\gppoint{gp mark 1}{(3.868,4.324)}
\gppoint{gp mark 1}{(4.171,4.593)}
\gppoint{gp mark 1}{(4.475,4.575)}
\gppoint{gp mark 1}{(4.778,4.571)}
\gppoint{gp mark 1}{(5.082,4.575)}
\gppoint{gp mark 1}{(5.385,4.575)}
\gppoint{gp mark 1}{(5.689,4.575)}
\gppoint{gp mark 1}{(5.993,4.575)}
\gppoint{gp mark 1}{(6.296,4.575)}
\gppoint{gp mark 1}{(6.600,4.580)}
\gppoint{gp mark 1}{(6.903,4.580)}
\gppoint{gp mark 1}{(7.207,4.580)}
\gppoint{gp mark 1}{(7.510,4.584)}
\gppoint{gp mark 1}{(7.814,4.584)}
\gppoint{gp mark 1}{(8.117,4.593)}
\gppoint{gp mark 1}{(8.421,4.593)}
\gppoint{gp mark 1}{(8.724,4.589)}
\gppoint{gp mark 1}{(9.028,4.589)}
\gppoint{gp mark 1}{(9.331,4.589)}
\gppoint{gp mark 1}{(9.635,4.593)}
\gppoint{gp mark 1}{(9.938,4.593)}
\gppoint{gp mark 1}{(10.242,4.593)}
\gpcolor{gp lt color border}
\gpsetlinetype{gp lt border}
\draw[gp path] (1.136,4.790)--(1.136,0.985)--(10.242,0.985);
%% coordinates of the plot area
\gpdefrectangularnode{gp plot 1}{\pgfpoint{1.136cm}{0.985cm}}{\pgfpoint{10.242cm}{4.790cm}}
\end{tikzpicture}
%% gnuplot variables

	\parbox{4.25in}{
	\caption[Plot --- Line regulation of LM723]{Measured data for the line
	regulation tests on the LM723 circuit shown in Figure~\ref{fig:icSchem}.  The
	source voltage was varied from~6 to~\SI{30}{\volt}, while holding a constant
	load.}
	\label{fig:vrLineReg}
	}
\end{figure}
%
In the flat region of the graph (from inputs of~\SI{10}{\volt}
to~\SI{30}{\volt}), the maximum value is~\SI{8.06}{\volt} and the minimum value
is~\SI{8.01}{\volt}.  By using the formula for line regulation in
Equation~\eqref{eq:lineReg}
%
\begin{equation}
	\text{Line Regulation} = \frac{\Delta V_\text{out}}{V_\text{out}} \times 100 \qquad \text{,}
	\label{eq:lineReg}
\end{equation}
%
the line regulation of the IC regulator on its own is calculated to
be~\SI{5.02}{\percent}.

By a similar process, the load regulation (change in output per change in load)
of the IC was tested by using a static~\SI{18}{\volt} source and replacing the
load resistor for a decade box.  This resistance was varied from~\SI{100}{\ohm}
to~\SI{3.2}{\kilo\ohm}, and measurements were plotted in
Figure~\ref{fig:vrLoadReg}.
%
\begin{figure}[H]
	\centering
	\begin{tikzpicture}[gnuplot]
%% generated with GNUPLOT 4.4p2 (Lua 5.1.4; terminal rev. 97, script rev. 96a)
%% Fri 14 Oct 2011 11:40:11 AM EDT
\gpsolidlines
\gpcolor{gp lt color axes}
\gpsetlinetype{gp lt axes}
\gpsetlinewidth{1.00}
\draw[gp path] (1.872,0.985)--(10.242,0.985);
\gpcolor{gp lt color border}
\gpsetlinetype{gp lt border}
\draw[gp path] (1.872,0.985)--(2.052,0.985);
\node[gp node right] at (1.688,0.985) { 8.025};
\gpcolor{gp lt color axes}
\gpsetlinetype{gp lt axes}
\draw[gp path] (1.872,1.408)--(10.242,1.408);
\gpcolor{gp lt color border}
\gpsetlinetype{gp lt border}
\draw[gp path] (1.872,1.408)--(2.052,1.408);
\node[gp node right] at (1.688,1.408) { 8.03};
\gpcolor{gp lt color axes}
\gpsetlinetype{gp lt axes}
\draw[gp path] (1.872,1.831)--(10.242,1.831);
\gpcolor{gp lt color border}
\gpsetlinetype{gp lt border}
\draw[gp path] (1.872,1.831)--(2.052,1.831);
\node[gp node right] at (1.688,1.831) { 8.035};
\gpcolor{gp lt color axes}
\gpsetlinetype{gp lt axes}
\draw[gp path] (1.872,2.253)--(10.242,2.253);
\gpcolor{gp lt color border}
\gpsetlinetype{gp lt border}
\draw[gp path] (1.872,2.253)--(2.052,2.253);
\node[gp node right] at (1.688,2.253) { 8.04};
\gpcolor{gp lt color axes}
\gpsetlinetype{gp lt axes}
\draw[gp path] (1.872,2.676)--(10.242,2.676);
\gpcolor{gp lt color border}
\gpsetlinetype{gp lt border}
\draw[gp path] (1.872,2.676)--(2.052,2.676);
\node[gp node right] at (1.688,2.676) { 8.045};
\gpcolor{gp lt color axes}
\gpsetlinetype{gp lt axes}
\draw[gp path] (1.872,3.099)--(10.242,3.099);
\gpcolor{gp lt color border}
\gpsetlinetype{gp lt border}
\draw[gp path] (1.872,3.099)--(2.052,3.099);
\node[gp node right] at (1.688,3.099) { 8.05};
\gpcolor{gp lt color axes}
\gpsetlinetype{gp lt axes}
\draw[gp path] (1.872,3.522)--(10.242,3.522);
\gpcolor{gp lt color border}
\gpsetlinetype{gp lt border}
\draw[gp path] (1.872,3.522)--(2.052,3.522);
\node[gp node right] at (1.688,3.522) { 8.055};
\gpcolor{gp lt color axes}
\gpsetlinetype{gp lt axes}
\draw[gp path] (1.872,3.944)--(10.242,3.944);
\gpcolor{gp lt color border}
\gpsetlinetype{gp lt border}
\draw[gp path] (1.872,3.944)--(2.052,3.944);
\node[gp node right] at (1.688,3.944) { 8.06};
\gpcolor{gp lt color axes}
\gpsetlinetype{gp lt axes}
\draw[gp path] (1.872,4.367)--(10.242,4.367);
\gpcolor{gp lt color border}
\gpsetlinetype{gp lt border}
\draw[gp path] (1.872,4.367)--(2.052,4.367);
\node[gp node right] at (1.688,4.367) { 8.065};
\gpcolor{gp lt color axes}
\gpsetlinetype{gp lt axes}
\draw[gp path] (1.872,4.790)--(10.242,4.790);
\gpcolor{gp lt color border}
\gpsetlinetype{gp lt border}
\draw[gp path] (1.872,4.790)--(2.052,4.790);
\node[gp node right] at (1.688,4.790) { 8.07};
\gpcolor{gp lt color axes}
\gpsetlinetype{gp lt axes}
\draw[gp path] (1.872,0.985)--(1.872,4.790);
\gpcolor{gp lt color border}
\gpsetlinetype{gp lt border}
\draw[gp path] (1.872,0.985)--(1.872,1.165);
\node[gp node center] at (1.872,0.677) { 0.001};
\gpcolor{gp lt color axes}
\gpsetlinetype{gp lt axes}
\draw[gp path] (3.132,0.985)--(3.132,4.790);
\gpcolor{gp lt color border}
\gpsetlinetype{gp lt border}
\draw[gp path] (3.132,0.985)--(3.132,1.075);
\gpcolor{gp lt color axes}
\gpsetlinetype{gp lt axes}
\draw[gp path] (3.869,0.985)--(3.869,4.790);
\gpcolor{gp lt color border}
\gpsetlinetype{gp lt border}
\draw[gp path] (3.869,0.985)--(3.869,1.075);
\gpcolor{gp lt color axes}
\gpsetlinetype{gp lt axes}
\draw[gp path] (4.392,0.985)--(4.392,4.790);
\gpcolor{gp lt color border}
\gpsetlinetype{gp lt border}
\draw[gp path] (4.392,0.985)--(4.392,1.075);
\gpcolor{gp lt color axes}
\gpsetlinetype{gp lt axes}
\draw[gp path] (4.797,0.985)--(4.797,4.790);
\gpcolor{gp lt color border}
\gpsetlinetype{gp lt border}
\draw[gp path] (4.797,0.985)--(4.797,1.075);
\gpcolor{gp lt color axes}
\gpsetlinetype{gp lt axes}
\draw[gp path] (5.129,0.985)--(5.129,4.790);
\gpcolor{gp lt color border}
\gpsetlinetype{gp lt border}
\draw[gp path] (5.129,0.985)--(5.129,1.075);
\gpcolor{gp lt color axes}
\gpsetlinetype{gp lt axes}
\draw[gp path] (5.409,0.985)--(5.409,4.790);
\gpcolor{gp lt color border}
\gpsetlinetype{gp lt border}
\draw[gp path] (5.409,0.985)--(5.409,1.075);
\gpcolor{gp lt color axes}
\gpsetlinetype{gp lt axes}
\draw[gp path] (5.651,0.985)--(5.651,4.790);
\gpcolor{gp lt color border}
\gpsetlinetype{gp lt border}
\draw[gp path] (5.651,0.985)--(5.651,1.075);
\gpcolor{gp lt color axes}
\gpsetlinetype{gp lt axes}
\draw[gp path] (5.866,0.985)--(5.866,4.790);
\gpcolor{gp lt color border}
\gpsetlinetype{gp lt border}
\draw[gp path] (5.866,0.985)--(5.866,1.075);
\gpcolor{gp lt color axes}
\gpsetlinetype{gp lt axes}
\draw[gp path] (6.057,0.985)--(6.057,4.790);
\gpcolor{gp lt color border}
\gpsetlinetype{gp lt border}
\draw[gp path] (6.057,0.985)--(6.057,1.165);
\node[gp node center] at (6.057,0.677) { 0.01};
\gpcolor{gp lt color axes}
\gpsetlinetype{gp lt axes}
\draw[gp path] (7.317,0.985)--(7.317,4.790);
\gpcolor{gp lt color border}
\gpsetlinetype{gp lt border}
\draw[gp path] (7.317,0.985)--(7.317,1.075);
\gpcolor{gp lt color axes}
\gpsetlinetype{gp lt axes}
\draw[gp path] (8.054,0.985)--(8.054,4.790);
\gpcolor{gp lt color border}
\gpsetlinetype{gp lt border}
\draw[gp path] (8.054,0.985)--(8.054,1.075);
\gpcolor{gp lt color axes}
\gpsetlinetype{gp lt axes}
\draw[gp path] (8.577,0.985)--(8.577,4.790);
\gpcolor{gp lt color border}
\gpsetlinetype{gp lt border}
\draw[gp path] (8.577,0.985)--(8.577,1.075);
\gpcolor{gp lt color axes}
\gpsetlinetype{gp lt axes}
\draw[gp path] (8.982,0.985)--(8.982,4.790);
\gpcolor{gp lt color border}
\gpsetlinetype{gp lt border}
\draw[gp path] (8.982,0.985)--(8.982,1.075);
\gpcolor{gp lt color axes}
\gpsetlinetype{gp lt axes}
\draw[gp path] (9.314,0.985)--(9.314,4.790);
\gpcolor{gp lt color border}
\gpsetlinetype{gp lt border}
\draw[gp path] (9.314,0.985)--(9.314,1.075);
\gpcolor{gp lt color axes}
\gpsetlinetype{gp lt axes}
\draw[gp path] (9.594,0.985)--(9.594,4.790);
\gpcolor{gp lt color border}
\gpsetlinetype{gp lt border}
\draw[gp path] (9.594,0.985)--(9.594,1.075);
\gpcolor{gp lt color axes}
\gpsetlinetype{gp lt axes}
\draw[gp path] (9.836,0.985)--(9.836,4.790);
\gpcolor{gp lt color border}
\gpsetlinetype{gp lt border}
\draw[gp path] (9.836,0.985)--(9.836,1.075);
\gpcolor{gp lt color axes}
\gpsetlinetype{gp lt axes}
\draw[gp path] (10.051,0.985)--(10.051,4.790);
\gpcolor{gp lt color border}
\gpsetlinetype{gp lt border}
\draw[gp path] (10.051,0.985)--(10.051,1.075);
\gpcolor{gp lt color axes}
\gpsetlinetype{gp lt axes}
\draw[gp path] (10.242,0.985)--(10.242,4.790);
\gpcolor{gp lt color border}
\gpsetlinetype{gp lt border}
\draw[gp path] (10.242,0.985)--(10.242,1.165);
\node[gp node center] at (10.242,0.677) { 0.1};
\draw[gp path] (1.872,4.790)--(1.872,0.985)--(10.242,0.985)--(10.242,4.790)--cycle;
\node[gp node center,rotate=-270] at (0.246,2.887) {Output Voltage, $V_\text{out}$ (V)};
\node[gp node center] at (6.057,0.215) {Load Current, $I_\text{L}$ (A)};
\node[gp node center] at (6.057,5.252) {Load Regulation: Voltage Regulator Only};
\gpcolor{gp lt color 0}
\gpsetlinetype{gp lt plot 0}
\gpsetlinewidth{3.00}
\draw[gp path] (9.851,3.944)--(8.592,3.944)--(7.855,3.944)--(7.331,3.944)--(6.928,4.790)%
  --(6.596,4.790)--(6.318,4.790)--(6.074,4.790)--(5.859,4.790)--(5.668,4.790)--(5.494,3.944)%
  --(5.336,3.944)--(5.189,3.944)--(5.055,3.944)--(4.929,3.944)--(4.812,3.944)--(4.702,3.944)%
  --(4.598,3.944)--(4.500,3.944)--(4.406,3.944)--(4.319,4.790)--(4.234,4.790)--(4.153,3.944)%
  --(4.076,3.944)--(4.002,4.790)--(3.931,4.790)--(3.863,4.790)--(3.797,4.790)--(3.732,3.944)%
  --(3.671,4.790)--(3.613,4.790)--(3.555,4.790);
\gpsetpointsize{4.00}
\gppoint{gp mark 1}{(9.851,3.944)}
\gppoint{gp mark 1}{(8.592,3.944)}
\gppoint{gp mark 1}{(7.855,3.944)}
\gppoint{gp mark 1}{(7.331,3.944)}
\gppoint{gp mark 1}{(6.928,4.790)}
\gppoint{gp mark 1}{(6.596,4.790)}
\gppoint{gp mark 1}{(6.318,4.790)}
\gppoint{gp mark 1}{(6.074,4.790)}
\gppoint{gp mark 1}{(5.859,4.790)}
\gppoint{gp mark 1}{(5.668,4.790)}
\gppoint{gp mark 1}{(5.494,3.944)}
\gppoint{gp mark 1}{(5.336,3.944)}
\gppoint{gp mark 1}{(5.189,3.944)}
\gppoint{gp mark 1}{(5.055,3.944)}
\gppoint{gp mark 1}{(4.929,3.944)}
\gppoint{gp mark 1}{(4.812,3.944)}
\gppoint{gp mark 1}{(4.702,3.944)}
\gppoint{gp mark 1}{(4.598,3.944)}
\gppoint{gp mark 1}{(4.500,3.944)}
\gppoint{gp mark 1}{(4.406,3.944)}
\gppoint{gp mark 1}{(4.319,4.790)}
\gppoint{gp mark 1}{(4.234,4.790)}
\gppoint{gp mark 1}{(4.153,3.944)}
\gppoint{gp mark 1}{(4.076,3.944)}
\gppoint{gp mark 1}{(4.002,4.790)}
\gppoint{gp mark 1}{(3.931,4.790)}
\gppoint{gp mark 1}{(3.863,4.790)}
\gppoint{gp mark 1}{(3.797,4.790)}
\gppoint{gp mark 1}{(3.732,3.944)}
\gppoint{gp mark 1}{(3.671,4.790)}
\gppoint{gp mark 1}{(3.613,4.790)}
\gppoint{gp mark 1}{(3.555,4.790)}
\gpcolor{gp lt color border}
\gpsetlinetype{gp lt border}
\gpsetlinewidth{1.00}
\draw[gp path] (1.872,4.790)--(1.872,0.985)--(10.242,0.985)--(10.242,4.790)--cycle;
%% coordinates of the plot area
\gpdefrectangularnode{gp plot 1}{\pgfpoint{1.872cm}{0.985cm}}{\pgfpoint{10.242cm}{4.790cm}}
\end{tikzpicture}
%% gnuplot variables

	\parbox{4.25in}{
	\caption[Plot -- Load regulation of LM723]{Data recorded while testing the
	load regulation of the LM723 circuit depicted in Figure~\ref{fig:icSchem}.
	Unlike the line regulation tests, the source voltage was held constant
	while the input resistance varied from~100 to~\SI{3200}{\ohm}.}
	\label{fig:vrLoadReg}
	}
\end{figure}
%
Line regulation is calculated as in Equation~\eqref{eq:loadReg} by subtracting
the output at the minimum load from the output at maximum load and dividing the
difference by the output with a nominal load, or
%
\begin{equation}
	\text{Load Regulation} = \frac{V_\text{max R} - V_\text{min R}}{V_\text{nominal}} \times 100 \qquad \text{.}
	\label{eq:loadReg}
\end{equation}
%
In the case of the voltage regulator IC on its own, the load regulation is
equal to~$\frac{8.07-8.06}{8.07} \times 100$, or~\SI{0.123}{\percent}.


\begin{figure}[H]
	\centering
	\begin{tikzpicture}[gnuplot]
%% generated with GNUPLOT 4.4p2 (Lua 5.1.4; terminal rev. 97, script rev. 96a)
%% Sat 08 Oct 2011 01:30:50 PM EDT
\gpsolidlines
\gpcolor{gp lt color border}
\gpsetlinetype{gp lt border}
\gpsetlinewidth{1.00}
\draw[gp path] (1.872,0.985)--(2.052,0.985);
\node[gp node right] at (1.688,0.985) { 8.025};
\draw[gp path] (1.872,1.408)--(2.052,1.408);
\node[gp node right] at (1.688,1.408) { 8.03};
\draw[gp path] (1.872,1.831)--(2.052,1.831);
\node[gp node right] at (1.688,1.831) { 8.035};
\draw[gp path] (1.872,2.253)--(2.052,2.253);
\node[gp node right] at (1.688,2.253) { 8.04};
\draw[gp path] (1.872,2.676)--(2.052,2.676);
\node[gp node right] at (1.688,2.676) { 8.045};
\draw[gp path] (1.872,3.099)--(2.052,3.099);
\node[gp node right] at (1.688,3.099) { 8.05};
\draw[gp path] (1.872,3.522)--(2.052,3.522);
\node[gp node right] at (1.688,3.522) { 8.055};
\draw[gp path] (1.872,3.944)--(2.052,3.944);
\node[gp node right] at (1.688,3.944) { 8.06};
\draw[gp path] (1.872,4.367)--(2.052,4.367);
\node[gp node right] at (1.688,4.367) { 8.065};
\draw[gp path] (1.872,4.790)--(2.052,4.790);
\node[gp node right] at (1.688,4.790) { 8.07};
\draw[gp path] (1.872,0.985)--(1.872,1.165);
\node[gp node center] at (1.872,0.677) { 0.001};
\draw[gp path] (3.132,0.985)--(3.132,1.075);
\draw[gp path] (3.869,0.985)--(3.869,1.075);
\draw[gp path] (4.392,0.985)--(4.392,1.075);
\draw[gp path] (4.797,0.985)--(4.797,1.075);
\draw[gp path] (5.129,0.985)--(5.129,1.075);
\draw[gp path] (5.409,0.985)--(5.409,1.075);
\draw[gp path] (5.651,0.985)--(5.651,1.075);
\draw[gp path] (5.866,0.985)--(5.866,1.075);
\draw[gp path] (6.057,0.985)--(6.057,1.165);
\node[gp node center] at (6.057,0.677) { 0.01};
\draw[gp path] (7.317,0.985)--(7.317,1.075);
\draw[gp path] (8.054,0.985)--(8.054,1.075);
\draw[gp path] (8.577,0.985)--(8.577,1.075);
\draw[gp path] (8.982,0.985)--(8.982,1.075);
\draw[gp path] (9.314,0.985)--(9.314,1.075);
\draw[gp path] (9.594,0.985)--(9.594,1.075);
\draw[gp path] (9.836,0.985)--(9.836,1.075);
\draw[gp path] (10.051,0.985)--(10.051,1.075);
\draw[gp path] (10.242,0.985)--(10.242,1.165);
\node[gp node center] at (10.242,0.677) { 0.1};
\draw[gp path] (1.872,4.790)--(1.872,0.985)--(10.242,0.985);
\node[gp node center,rotate=-270] at (0.246,2.887) {Output Voltage, $V_\text{out}$ (V)};
\node[gp node center] at (6.057,0.215) {Load Current, $I_\text{L}$ (A)};
\node[gp node center] at (6.057,5.252) {Load Regulation: Voltage Regulator with AC Rectifier};
\gpcolor{gp lt color 0}
\gpsetlinetype{gp lt plot 0}
\draw[gp path] (9.843,1.408)--(8.588,3.099)--(7.856,4.790)--(7.330,3.944)--(6.923,3.099)%
  --(6.591,3.099)--(6.311,3.099)--(6.071,3.944)--(5.857,3.944)--(5.665,3.944)--(5.492,3.944)%
  --(5.334,3.944)--(5.188,3.944)--(5.053,3.944)--(4.928,3.944)--(4.811,3.944)--(4.701,3.944)%
  --(4.597,3.944)--(4.498,3.944)--(4.405,3.944);
\gpsetpointsize{4.00}
\gppoint{gp mark 1}{(9.843,1.408)}
\gppoint{gp mark 1}{(8.588,3.099)}
\gppoint{gp mark 1}{(7.856,4.790)}
\gppoint{gp mark 1}{(7.330,3.944)}
\gppoint{gp mark 1}{(6.923,3.099)}
\gppoint{gp mark 1}{(6.591,3.099)}
\gppoint{gp mark 1}{(6.311,3.099)}
\gppoint{gp mark 1}{(6.071,3.944)}
\gppoint{gp mark 1}{(5.857,3.944)}
\gppoint{gp mark 1}{(5.665,3.944)}
\gppoint{gp mark 1}{(5.492,3.944)}
\gppoint{gp mark 1}{(5.334,3.944)}
\gppoint{gp mark 1}{(5.188,3.944)}
\gppoint{gp mark 1}{(5.053,3.944)}
\gppoint{gp mark 1}{(4.928,3.944)}
\gppoint{gp mark 1}{(4.811,3.944)}
\gppoint{gp mark 1}{(4.701,3.944)}
\gppoint{gp mark 1}{(4.597,3.944)}
\gppoint{gp mark 1}{(4.498,3.944)}
\gppoint{gp mark 1}{(4.405,3.944)}
\gpcolor{gp lt color border}
\gpsetlinetype{gp lt border}
\draw[gp path] (1.872,4.790)--(1.872,0.985)--(10.242,0.985);
%% coordinates of the plot area
\gpdefrectangularnode{gp plot 1}{\pgfpoint{1.872cm}{0.985cm}}{\pgfpoint{10.242cm}{4.790cm}}
\end{tikzpicture}
%% gnuplot variables

	\caption{}
	\label{fig:fullLoadReg}
\end{figure}
