\subsection{Initial Testing}
Upon beginning the experiment, students were first instructed to test the provided circuit to ensure that it is working.  While powered, the pin labeled~$V_\text{CC}$ was to measure~\SI{5}{\volt}, the LEDs were to measure~\SI{3.2}{\volt}, and pin 3 of the DAC was to measure around~\SI{-14}{\volt}.  These values were recorded and are tabulated in Table~\ref{tab:initTest}.
%
\begin{table}[H]
	\centering
	\begin{tabular}{|c|c|c|c|}
	\hline
	\tbf{Location} & \tbf{Expected Value (\si{\volt})} &
		\tbf{Measured Value (\si{\volt})} & \tbf{Error (\si{\percent})} \\ \hline
	$V_\text{CC}$  & 5.000 & 5.120 & 2.4  \\ \hline
	$V_\text{LED}$ & 3.200 & 3.158 & -1.3 \\ \hline
	$V_\text{EE}$  &-14.00 &-14.75 & 5.4 \\ \hline
\end{tabular}

	\parbox{.6\textwidth}{
	\caption[Measured Initial Values]{List of values recorded during the initial testing phases.  The voltages were within reasonable limits, allowing the student to safely assume that the board was operating correctly.}
	\label{tab:initTest}}
\end{table}
%
Next, the student shorted the input terminal to ground, ensuring that all eight LEDs from the ADC were lit.  The voltage at the DAC's output was measured at~\SI{-1.4}{\milli\volt}~--- reasonably close to the expected value of zero volts, implying proper operation.

\subsection{ADC/DAC DC Test}
The input short to ground was removed in place of the in-lab DC voltage supply.  With the current limited to~\SI{0.2}{\ampere}, the voltage was varied from zero to five volts in increments of one, measuring the output voltage at each step and tabulating all recorded voltages in Table~\ref{tab:dcSweep}.
%
\begin{table}[H]
	\centering
	\begin{tabular}{|c|c|c|c|}
	\hline
	\tbf{Input (\si{\volt})} &
		\tbf{Expected Output (\si{\volt})} &
		\tbf{Measured Output (\si{\volt})} & \tbf{Error (\si{\percent})} \\ \hline
	0  & 0  & 0      & 0 \\ \hline
	1  & -1 & -0.957 & -4.26 \\ \hline
	2  & -2 & -1.901 & -4.95 \\ \hline
	3  & -3 & -2.865 & -4.50 \\ \hline
	4  & -4 & -3.742 & -6.45 \\ \hline
	5  & -5 & -4.300 & -14.0 \\ \hline
\end{tabular}

	\parbox{.6\textwidth}{
	\caption[DC Sweep Results]{Data recorded from a DC sweep performed with all eight DIP switches closed.}
	\label{tab:dcSweep}}
\end{table}
%
As is clearly visible in the recorded values, the output tended to be slightly higher than the expected output (note that since the output was expected to be negative as previously stated, the error is still calculated to be negative as well).  Since all eight of the switches between the ADC and DAC were closed for this sweep, the output voltages were close to their theoretical values.  Additionally, theoretical values were simply the negative value of the input and were not directly calculated.  This is a result of the known-working ADC being used to create a valid series of bits for the DAC's input.

Next, the input voltage was locked at one volt and a series of DIP switches between the ADC and DAC were opened so that students could observe the effects of a changing digital input.  The switches associated with bits~0,~1,~3,~5, and~7 were opened one at a time, while recording the output voltage corresponding to each configuration.  Table~\ref{tab:1vDIPs} contains the measured voltages for each state.
%
\begin{table}[H]
	\centering
	\begin{tabular}{|c|c|}
	\hline
	\tbf{Open Bit} & \tbf{Meaured Output (\si{\milli\volt})} \\ \hline
	None & -957  \\ \hline
	0    & -970  \\ \hline
	1    & -954  \\ \hline
	3    & -1106 \\ \hline
	5    & -953  \\ \hline
	7    & -3390 \\ \hline
\end{tabular}

	\parbox{.6\textwidth}{
	\caption[\SI{1}{\volt}DC DIP Switches]{Measured output for various DIP switch configurations with a~\SI{1}{\volt} input.}
	\label{tab:1vDIPs}}
\end{table}
%
For a one volt input, the associated ADC output is~\texttt{0011 0010}.  The DAC is logic-one based but the LEDs are logic-zero activated, so each bit passes through an inverter before entering the DAC (after leaving the DIP switches).  This means that opening a switch will only have an effect on bits that are zero as they pass through the switches~(i.e. bits~0,~2,~3,~6, and~7).  Note the drastic effect that bit seven has on the output voltage, as it is the most-significant bit.

The DIP switch test was repeated with an input of five volts.  The results of this test are tabulated in Table~\ref{tab:5vDIPs}.
%
\begin{table}[H]
	\centering
	\begin{tabular}{|c|c|}
	\hline
	\tbf{Open Bit} & \tbf{Meaured Output (\si{\volt})} \\ \hline
	None &  4.313 \\ \hline
	0    &  4.313 \\ \hline
	1    &  4.314 \\ \hline
	3    &  4.471 \\ \hline
	5    &  4.311 \\ \hline
	7    &  4.313 \\ \hline
\end{tabular}
\\
	\parbox{.6\textwidth}{
	\caption[\SI{5}{\volt}DC DIP Switches]{Output measured for several DIP switch states for a~\SI{5}{\volt} input to the ADC.}
	\label{tab:5vDIPs}}
\end{table}
%
The ADC output associated with a~\SI{5}{\volt} input in this case was~\texttt{1110 0010}; as with the~\SI{1}{\volt} test, only bits of value zero (as output by the ADC) are effected by a change in DIP switch.  In this case, only bits zero and three have the potential to change the output voltage of the DAC.  While a change in output is visible for bit three, there is a negligible change due to bit zero, as it is the least significant.

\subsection{ADC/DAC AC Test}
Once the properties of a DAC had been examined for a static input, a sinusoidal input was used in place of the DC source to test the system's performance under varying conditions.  With the in-lab function generator providing a~\SI{400}{\hertz} signal ranging from zero to~\SI{4}{\volt} as shown in the upper signal~(A1) of Figure~\ref{fig:pt4a}, the DAC produced the lower waveform~(A2).
%
\begin{figure}[H]
	\centering
	\includegraphics[width=.6\textwidth]{img/shot/pt4ashot.png}
	\parbox{.6\textwidth}{
	\caption[\SI{400}{\hertz} Sine Wave --- All Closed]{DAC output for a~\SI{400}{\hertz} sine wave input.  Note that the output does resemble the input (albeit of the opposite polarity) despite its quantized nature.}
	\label{fig:pt4a}}
\end{figure}
%
As can be seen in the above image, the output waveform greatly resembles the input waveform.  While the output is a negative representation of the input, this is expected behavior.

Just as the DC system was tested for various DIP switch configurations, the AC system was tested.  For each of bits zero, five, and seven, a single switch was opened, and the output recaptured (displayed in Figure~\ref{fig:pt4b}.
%
\begin{figure}[H]
	\centering
	\subfloat[Bit zero open] {\label{subfig:pt4b_bit0}\includegraphics[width=.3\textwidth]{img/shot/pt4b_bit0.png}}
	\quad
	\subfloat[Bit five open] {\label{subfig:pt4b_bit5}\includegraphics[width=.3\textwidth]{img/shot/pt4b_bit5.png}}
	\quad
	\subfloat[Bit seven open]{\label{subfig:pt4b_bit7}\includegraphics[width=.3\textwidth]{img/shot/pt4b_bit7.png}}

	\parbox{.8\textwidth}{
	\caption[\SI{400}{\hertz} Sine Wave --- DIP Switches]{Results of varying the open DIP switch.  Note that, where bit five is opened in~\subref{subfig:pt4b_bit5}, the upper portions of the sinusoid are not represented.  Similarly,~\subref{subfig:pt4b_bit7} shows a nearly complete loss of a recognizable signal when bit seven is opened.}
	\label{fig:pt4b}}
\end{figure}
%
\emph{Because the changing shape of the output is the intended focus of this test, screenshots for each configuration are shown next to each other for easy comparison.  Full-sized versions are available in the Appendix, and specific measurements from each are aggregated in Table~\ref{tab:pt4b}.}
%
\begin{table}[H]
	\centering
	\begin{tabular}{|c|c|}
	\hline
	\tbf{Open Bit} & \tbf{Meaured Voltage (\si{\volt}, P-P)}\\ \hline
	None & 3.750  \\ \hline
	0    & 3.750  \\ \hline
	5    & 3.812  \\ \hline
	7    & 2.437  \\ \hline
\end{tabular}

	\parbox{.6\textwidth}{
	\caption[\SI{400}{\hertz} DIP Switches]{Measured output of a varying state of DIP switch positions for a~\SI{400}{\hertz} sinusoidal input.}
	\label{tab:pt4b}}
\end{table}
%
While in the DC case the output voltage distorted, in the AC case the effects are exaggerated.  The frequency of the output appears to remain roughly constant (this was difficult to measure due to the oscilloscope limitations), but the peak-to-peak voltage accuracy degrades as more-significant bits are opened via DIP switch.  This, coupled with the loss of accurate value representation, produces an increasingly distorted signal that becomes almost unrecognizable in the case of the most-significant bit's opening.
