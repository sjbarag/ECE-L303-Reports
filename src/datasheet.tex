\subsection{Rectifier}
The half wave rectifier was built according to the provided schematic (recreated
in Figure~\ref{fig:schem1}, where the value of the resistor was provided
as~\SI{1}{\kilo\ohm} in the instructions.  The input voltage was calculated by
multiplying the specified average voltage of~\SI{5.6}{\volt}DC by~$\pi$ as
shown in Equation~\ref{eq:v_avg}.  This resulted in a peak voltage
of~\SI{17.6}{\volt}DC.
%
\begin{equation}
	V_\text{avg} = \frac{V_p}{\pi}
	\label{eq:v_avg}
\end{equation}
%
To calculate the AC signal's root-mean-square (RMS) value, the peak voltage was
then divided by~$\sqrt{2}$, as shown in Equation~\ref{eq:rms}.
%
\begin{equation}
	V_\text{RMS} = \frac{V_p}{\sqrt{2}}
	\label{eq:rms}
\end{equation}
%
The resulting calculations provided a designed input of \SI{12.2}{\volt} (RMS).
Using an oscilloscope in combination with a variac transformer, the input was
adjusted to be as close to this value as possible. Data for the half-wave
rectifier was captured by the oscilloscope.  Due to the large number of
measurements taken, it is available by request to the author.

\subsection{Logic Indicator}
The logic indicator was built entirely to the specifications provided by the
course instructors.  Using a \SI{300}{\ohm} resistor in series with the LED
allows for the required current of \SI{15}{\milli\ampere} to flow when the
input voltage is between four and five volts.  That this resistance is correct
can be confirmed via Ohm's law.

In testing the logic indicator, the DC input voltage was stepped from zero to
five volts, while recording the brightness of the LED subjectively at each
step.  The results are listed in Table~\ref{tab:ckt2data}.
%
\begin{figure}[H]
	\centering
	\begin{tabular}{|c|c|c|}
\hline
\tbf{Input (V)} & \tbf{Output State} & \tbf{Comment} \\ \hline
0				& Off				& ---				\\ \hline
1				& Off				& ---				\\ \hline
2				& On				& Dim				\\ \hline
3				& On				& Brighter			\\ \hline
4				& On				& Full Brightness 	\\ \hline
5				& On				& ---				\\ \hline
\end{tabular}

	\caption{Brightness observations for the circuit shown in
		Figure~\ref{fig:schem2}.}
	\label{tab:ckt2data}
\end{figure}
%
It is interesting to note that for inputs of both~\SI{2}{\volt}
and~\SI{3}{\volt}, the diode appears to be on, albeit dim.  The~\SI{3}{\volt}
case, while brighter than at two volts, is not yet at the full brightness of
the diode.  Full brightness for this diode occurs in the
designed~4--\SI{5}{\volt} range, showing that the final design is adequate in
functioning as a logic level indicator.

\subsection{Voltage Regulator}
As with the rectifier and logic indicator, the voltage regulator was built
according to the provided design.  The value of the~\SI{400}{\ohm} resistor is
mostly irrelevant, as its sole purpose is to drop the difference between the
zener diode's reverse breakdown voltage and the input voltage.  As long as the
resistor is large enough to limit the current through the diode to safe levels
(as indicated by the part's datasheet), its specific value does not affect the
output drastically.  To confirm that a given resistor ensures diode safety,
Ohm's law again becomes useful.
%
\begin{figure}[H]
	\centering
	\begin{tikzpicture}[gnuplot]
%% generated with GNUPLOT 4.4p2 (Lua 5.1.4; terminal rev. 97, script rev. 96a)
%% Wed 28 Sep 2011 01:27:40 PM EDT
\gpsolidlines
\gpcolor{gp lt color border}
\gpsetlinetype{gp lt border}
\gpsetlinewidth{1.00}
\draw[gp path] (1.320,0.985)--(1.500,0.985);
\node[gp node right] at (1.136,0.985) { 4};
\draw[gp path] (1.320,1.608)--(1.500,1.608);
\node[gp node right] at (1.136,1.608) { 5};
\draw[gp path] (1.320,2.231)--(1.500,2.231);
\node[gp node right] at (1.136,2.231) { 6};
\draw[gp path] (1.320,2.854)--(1.500,2.854);
\node[gp node right] at (1.136,2.854) { 7};
\draw[gp path] (1.320,3.477)--(1.500,3.477);
\node[gp node right] at (1.136,3.477) { 8};
\draw[gp path] (1.320,4.100)--(1.500,4.100);
\node[gp node right] at (1.136,4.100) { 9};
\draw[gp path] (1.320,4.723)--(1.500,4.723);
\node[gp node right] at (1.136,4.723) { 10};
\draw[gp path] (1.320,5.346)--(1.500,5.346);
\node[gp node right] at (1.136,5.346) { 11};
\draw[gp path] (1.320,0.985)--(1.320,1.165);
\node[gp node center] at (1.320,0.677) { 5};
\draw[gp path] (3.551,0.985)--(3.551,1.165);
\node[gp node center] at (3.551,0.677) { 10};
\draw[gp path] (5.781,0.985)--(5.781,1.165);
\node[gp node center] at (5.781,0.677) { 15};
\draw[gp path] (8.012,0.985)--(8.012,1.165);
\node[gp node center] at (8.012,0.677) { 20};
\draw[gp path] (10.242,0.985)--(10.242,1.165);
\node[gp node center] at (10.242,0.677) { 25};
\draw[gp path] (1.320,5.346)--(1.320,0.985)--(10.242,0.985);
\node[gp node center,rotate=-270] at (0.246,3.165) {Output Voltage, $V_{out}$ (V)};
\node[gp node center] at (5.781,0.215) {Input Voltage, $V_{in}$ (V)};
\gpcolor{gp lt color 0}
\gpsetlinetype{gp lt plot 0}
\draw[gp path] (1.320,1.606)--(1.766,2.229)--(2.212,2.853)--(2.658,3.476)--(3.104,4.099)%
  --(3.551,4.568)--(3.997,4.605)--(4.443,4.630)--(4.889,4.686)--(5.335,4.734)--(5.781,4.778)%
  --(6.227,4.821)--(6.673,4.866)--(7.119,4.917)--(7.565,4.965)--(8.012,4.997)--(8.458,4.966)%
  --(8.904,4.929)--(9.350,4.890)--(9.796,4.847)--(10.242,4.810);
\gpsetpointsize{4.00}
\gppoint{gp mark 1}{(1.320,1.606)}
\gppoint{gp mark 1}{(1.766,2.229)}
\gppoint{gp mark 1}{(2.212,2.853)}
\gppoint{gp mark 1}{(2.658,3.476)}
\gppoint{gp mark 1}{(3.104,4.099)}
\gppoint{gp mark 1}{(3.551,4.568)}
\gppoint{gp mark 1}{(3.997,4.605)}
\gppoint{gp mark 1}{(4.443,4.630)}
\gppoint{gp mark 1}{(4.889,4.686)}
\gppoint{gp mark 1}{(5.335,4.734)}
\gppoint{gp mark 1}{(5.781,4.778)}
\gppoint{gp mark 1}{(6.227,4.821)}
\gppoint{gp mark 1}{(6.673,4.866)}
\gppoint{gp mark 1}{(7.119,4.917)}
\gppoint{gp mark 1}{(7.565,4.965)}
\gppoint{gp mark 1}{(8.012,4.997)}
\gppoint{gp mark 1}{(8.458,4.966)}
\gppoint{gp mark 1}{(8.904,4.929)}
\gppoint{gp mark 1}{(9.350,4.890)}
\gppoint{gp mark 1}{(9.796,4.847)}
\gppoint{gp mark 1}{(10.242,4.810)}
\gpcolor{gp lt color border}
\gpsetlinetype{gp lt border}
\draw[gp path] (1.320,5.346)--(1.320,0.985)--(10.242,0.985);
%% coordinates of the plot area
\gpdefrectangularnode{gp plot 1}{\pgfpoint{1.320cm}{0.985cm}}{\pgfpoint{10.242cm}{5.346cm}}
\end{tikzpicture}
%% gnuplot variables

	\caption{Output measurements of the zener diode voltage
		regulator described in Figure~\ref{fig:schem3}.}
	\label{tab:ckt3data}
\end{figure}
%
The reverse breakdown voltage of the zener diode was tested in a curve tracer
prior to the experiment, the results of which are shown in
Figure~\ref{fig:zenerCurve}.  For this specific diode, $V_z$ is
approximately~\SI{1.9}{\volt}.  As was expected, the data captured indicates
that the voltage was regulated at roughly~\SI{10.2}{\volt}.  The zener diode is
not a perfect regulator however, hence the variation in output voltage.  This
is mostly due to variances as the package temperature increases and the innate
properties of the zener diode as a device.

\subsection{Constant Current Source}
\begin{figure}[H]
	\centering
	\begin{tikzpicture}[gnuplot]
%% generated with GNUPLOT 4.4p2 (Lua 5.1.4; terminal rev. 97, script rev. 96a)
%% Wed 28 Sep 2011 10:44:45 AM EDT
\gpsolidlines
\gpcolor{gp lt color border}
\gpsetlinetype{gp lt border}
\gpsetlinewidth{1.00}
\draw[gp path] (1.504,0.985)--(1.684,0.985);
\node[gp node right] at (1.320,0.985) { 3};
\draw[gp path] (1.504,1.619)--(1.684,1.619);
\node[gp node right] at (1.320,1.619) { 3.5};
\draw[gp path] (1.504,2.253)--(1.684,2.253);
\node[gp node right] at (1.320,2.253) { 4};
\draw[gp path] (1.504,2.888)--(1.684,2.888);
\node[gp node right] at (1.320,2.888) { 4.5};
\draw[gp path] (1.504,3.522)--(1.684,3.522);
\node[gp node right] at (1.320,3.522) { 5};
\draw[gp path] (1.504,4.156)--(1.684,4.156);
\node[gp node right] at (1.320,4.156) { 5.5};
\draw[gp path] (1.504,4.790)--(1.684,4.790);
\node[gp node right] at (1.320,4.790) { 6};
\draw[gp path] (1.504,0.985)--(1.504,1.165);
\node[gp node center] at (1.504,0.677) { 0};
\draw[gp path] (2.378,0.985)--(2.378,1.165);
\node[gp node center] at (2.378,0.677) { 0.5};
\draw[gp path] (3.252,0.985)--(3.252,1.165);
\node[gp node center] at (3.252,0.677) { 1};
\draw[gp path] (4.125,0.985)--(4.125,1.165);
\node[gp node center] at (4.125,0.677) { 1.5};
\draw[gp path] (4.999,0.985)--(4.999,1.165);
\node[gp node center] at (4.999,0.677) { 2};
\draw[gp path] (5.873,0.985)--(5.873,1.165);
\node[gp node center] at (5.873,0.677) { 2.5};
\draw[gp path] (6.747,0.985)--(6.747,1.165);
\node[gp node center] at (6.747,0.677) { 3};
\draw[gp path] (7.621,0.985)--(7.621,1.165);
\node[gp node center] at (7.621,0.677) { 3.5};
\draw[gp path] (8.494,0.985)--(8.494,1.165);
\node[gp node center] at (8.494,0.677) { 4};
\draw[gp path] (9.368,0.985)--(9.368,1.165);
\node[gp node center] at (9.368,0.677) { 4.5};
\draw[gp path] (10.242,0.985)--(10.242,1.165);
\node[gp node center] at (10.242,0.677) { 5};
\draw[gp path] (1.504,4.790)--(1.504,0.985)--(10.242,0.985);
\node[gp node center,rotate=-270] at (0.246,2.887) {Measured Current, $I_{out}$ (\si{\milli\ampere})};
\node[gp node center] at (5.873,0.215) {Load Resistance, $R$ (\si{\kilo\ohm})};
\node[gp node center] at (5.873,5.252) {Measured current from a JFET-based constant current source};
\node[gp node right] at (3.712,1.627) {16V source};
\gpcolor{gp lt color 0}
\gpsetlinetype{gp lt plot 0}
\draw[gp path] (3.896,1.627)--(4.812,1.627);
\draw[gp path] (1.591,4.384)--(1.679,4.371)--(1.854,4.371)--(2.028,4.371)--(2.203,4.371)%
  --(2.378,4.384)--(2.553,4.384)--(2.727,4.384)--(2.902,4.384)--(3.077,4.384)--(3.252,4.384)%
  --(3.426,4.384)--(3.601,4.384)--(3.776,4.384)--(3.951,4.397)--(4.125,4.397)--(4.300,4.384)%
  --(4.475,4.371)--(4.650,4.346)--(4.824,4.333)--(4.999,4.333)--(5.174,4.308)--(5.349,4.245)%
  --(5.523,4.169)--(5.698,4.067)--(5.873,3.940)--(6.048,3.801)--(6.223,3.648)--(6.397,3.496)%
  --(6.572,3.331)--(6.747,3.179)--(6.922,3.027)--(7.096,2.875)--(7.271,2.735)--(7.446,2.608)%
  --(7.621,2.469)--(7.795,2.342)--(7.970,2.228)--(8.145,2.101)--(8.320,1.987)--(8.494,1.886)%
  --(8.669,1.784)--(8.844,1.683)--(9.019,1.581)--(9.193,1.492)--(9.368,1.404)--(9.543,1.327)%
  --(9.718,1.239)--(9.892,1.163)--(10.067,1.086)--(10.242,1.010);
\gpsetpointsize{4.00}
\gppoint{gp mark 1}{(1.591,4.384)}
\gppoint{gp mark 1}{(1.679,4.371)}
\gppoint{gp mark 1}{(1.854,4.371)}
\gppoint{gp mark 1}{(2.028,4.371)}
\gppoint{gp mark 1}{(2.203,4.371)}
\gppoint{gp mark 1}{(2.378,4.384)}
\gppoint{gp mark 1}{(2.553,4.384)}
\gppoint{gp mark 1}{(2.727,4.384)}
\gppoint{gp mark 1}{(2.902,4.384)}
\gppoint{gp mark 1}{(3.077,4.384)}
\gppoint{gp mark 1}{(3.252,4.384)}
\gppoint{gp mark 1}{(3.426,4.384)}
\gppoint{gp mark 1}{(3.601,4.384)}
\gppoint{gp mark 1}{(3.776,4.384)}
\gppoint{gp mark 1}{(3.951,4.397)}
\gppoint{gp mark 1}{(4.125,4.397)}
\gppoint{gp mark 1}{(4.300,4.384)}
\gppoint{gp mark 1}{(4.475,4.371)}
\gppoint{gp mark 1}{(4.650,4.346)}
\gppoint{gp mark 1}{(4.824,4.333)}
\gppoint{gp mark 1}{(4.999,4.333)}
\gppoint{gp mark 1}{(5.174,4.308)}
\gppoint{gp mark 1}{(5.349,4.245)}
\gppoint{gp mark 1}{(5.523,4.169)}
\gppoint{gp mark 1}{(5.698,4.067)}
\gppoint{gp mark 1}{(5.873,3.940)}
\gppoint{gp mark 1}{(6.048,3.801)}
\gppoint{gp mark 1}{(6.223,3.648)}
\gppoint{gp mark 1}{(6.397,3.496)}
\gppoint{gp mark 1}{(6.572,3.331)}
\gppoint{gp mark 1}{(6.747,3.179)}
\gppoint{gp mark 1}{(6.922,3.027)}
\gppoint{gp mark 1}{(7.096,2.875)}
\gppoint{gp mark 1}{(7.271,2.735)}
\gppoint{gp mark 1}{(7.446,2.608)}
\gppoint{gp mark 1}{(7.621,2.469)}
\gppoint{gp mark 1}{(7.795,2.342)}
\gppoint{gp mark 1}{(7.970,2.228)}
\gppoint{gp mark 1}{(8.145,2.101)}
\gppoint{gp mark 1}{(8.320,1.987)}
\gppoint{gp mark 1}{(8.494,1.886)}
\gppoint{gp mark 1}{(8.669,1.784)}
\gppoint{gp mark 1}{(8.844,1.683)}
\gppoint{gp mark 1}{(9.019,1.581)}
\gppoint{gp mark 1}{(9.193,1.492)}
\gppoint{gp mark 1}{(9.368,1.404)}
\gppoint{gp mark 1}{(9.543,1.327)}
\gppoint{gp mark 1}{(9.718,1.239)}
\gppoint{gp mark 1}{(9.892,1.163)}
\gppoint{gp mark 1}{(10.067,1.086)}
\gppoint{gp mark 1}{(10.242,1.010)}
\gppoint{gp mark 1}{(4.354,1.627)}
\gpcolor{gp lt color border}
\node[gp node right] at (3.712,1.319) {32V source };
\gpcolor{gp lt color 1}
\gpsetlinetype{gp lt plot 1}
\draw[gp path] (3.896,1.319)--(4.812,1.319);
\draw[gp path] (1.591,4.004)--(1.679,4.004)--(1.854,4.016)--(2.028,4.029)--(2.203,4.029)%
  --(2.378,4.029)--(2.553,4.042)--(2.727,4.054)--(2.902,4.067)--(3.077,4.067)--(3.252,4.105)%
  --(3.426,4.105)--(3.601,4.105)--(3.776,4.105)--(3.951,4.118)--(4.125,4.130)--(4.300,4.143)%
  --(4.475,4.156)--(4.650,4.169)--(4.824,4.169)--(4.999,4.207)--(5.174,4.232)--(5.349,4.232)%
  --(5.523,4.232)--(5.698,4.232)--(5.873,4.232)--(6.048,4.245)--(6.223,4.257)--(6.397,4.270)%
  --(6.572,4.283)--(6.747,4.295)--(6.922,4.308)--(7.096,4.321)--(7.271,4.333)--(7.446,4.333)%
  --(7.621,4.346)--(7.795,4.346)--(7.970,4.346)--(8.145,4.359)--(8.320,4.371)--(8.494,4.371)%
  --(8.669,4.371)--(8.844,4.371)--(9.019,4.359)--(9.193,4.346)--(9.368,4.333)--(9.543,4.333)%
  --(9.718,4.308)--(9.892,4.283)--(10.067,4.257)--(10.242,4.232);
\gppoint{gp mark 2}{(1.591,4.004)}
\gppoint{gp mark 2}{(1.679,4.004)}
\gppoint{gp mark 2}{(1.854,4.016)}
\gppoint{gp mark 2}{(2.028,4.029)}
\gppoint{gp mark 2}{(2.203,4.029)}
\gppoint{gp mark 2}{(2.378,4.029)}
\gppoint{gp mark 2}{(2.553,4.042)}
\gppoint{gp mark 2}{(2.727,4.054)}
\gppoint{gp mark 2}{(2.902,4.067)}
\gppoint{gp mark 2}{(3.077,4.067)}
\gppoint{gp mark 2}{(3.252,4.105)}
\gppoint{gp mark 2}{(3.426,4.105)}
\gppoint{gp mark 2}{(3.601,4.105)}
\gppoint{gp mark 2}{(3.776,4.105)}
\gppoint{gp mark 2}{(3.951,4.118)}
\gppoint{gp mark 2}{(4.125,4.130)}
\gppoint{gp mark 2}{(4.300,4.143)}
\gppoint{gp mark 2}{(4.475,4.156)}
\gppoint{gp mark 2}{(4.650,4.169)}
\gppoint{gp mark 2}{(4.824,4.169)}
\gppoint{gp mark 2}{(4.999,4.207)}
\gppoint{gp mark 2}{(5.174,4.232)}
\gppoint{gp mark 2}{(5.349,4.232)}
\gppoint{gp mark 2}{(5.523,4.232)}
\gppoint{gp mark 2}{(5.698,4.232)}
\gppoint{gp mark 2}{(5.873,4.232)}
\gppoint{gp mark 2}{(6.048,4.245)}
\gppoint{gp mark 2}{(6.223,4.257)}
\gppoint{gp mark 2}{(6.397,4.270)}
\gppoint{gp mark 2}{(6.572,4.283)}
\gppoint{gp mark 2}{(6.747,4.295)}
\gppoint{gp mark 2}{(6.922,4.308)}
\gppoint{gp mark 2}{(7.096,4.321)}
\gppoint{gp mark 2}{(7.271,4.333)}
\gppoint{gp mark 2}{(7.446,4.333)}
\gppoint{gp mark 2}{(7.621,4.346)}
\gppoint{gp mark 2}{(7.795,4.346)}
\gppoint{gp mark 2}{(7.970,4.346)}
\gppoint{gp mark 2}{(8.145,4.359)}
\gppoint{gp mark 2}{(8.320,4.371)}
\gppoint{gp mark 2}{(8.494,4.371)}
\gppoint{gp mark 2}{(8.669,4.371)}
\gppoint{gp mark 2}{(8.844,4.371)}
\gppoint{gp mark 2}{(9.019,4.359)}
\gppoint{gp mark 2}{(9.193,4.346)}
\gppoint{gp mark 2}{(9.368,4.333)}
\gppoint{gp mark 2}{(9.543,4.333)}
\gppoint{gp mark 2}{(9.718,4.308)}
\gppoint{gp mark 2}{(9.892,4.283)}
\gppoint{gp mark 2}{(10.067,4.257)}
\gppoint{gp mark 2}{(10.242,4.232)}
\gppoint{gp mark 2}{(4.354,1.319)}
\gpcolor{gp lt color border}
\gpsetlinetype{gp lt border}
\draw[gp path] (1.504,4.790)--(1.504,0.985)--(10.242,0.985);
%% coordinates of the plot area
\gpdefrectangularnode{gp plot 1}{\pgfpoint{1.504cm}{0.985cm}}{\pgfpoint{10.242cm}{4.790cm}}
\end{tikzpicture}
%% gnuplot variables

	\caption{Measured current with a variable resistance.  The experiment was
		performed with a source voltage of both~\SI{16}{\volt}
		and~\SI{32}{\volt}, as depicted in Figure~\ref{fig:schem4}}
	\label{tab:ckt4data}
\end{figure}
