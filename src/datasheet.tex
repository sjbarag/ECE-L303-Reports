\subsection{Rectifier}
The half wave rectifier was buit according to the provided schematic (recreated
in Figure~\ref{fig:schem1}, where the value of the resistor was provided
as~\SI{1}{\kilo\ohm} in the instructions.  The input voltage was calculated by
multiplying the specified average voltage of~\SI{5.6}{\volt}DC by~$\pi$ as
shown in Equation~\ref{eq:v_avg}.  This resulted in a peak voltage
of~\SI{17.6}{\volt}DC.
%
\begin{equation}
	V_\text{avg} = \frac{V_p}{\pi}
	\label{eq:v_avg}
\end{equation}
%
To calculate the AC signal's root-mean-square (RMS) value, the peak voltage was
then divided by~$\sqrt{2}$, as shown in Equation~\ref{eq:rms}.
%
\begin{equation}
	V_\text{RMS} = \frac{V_p}{\sqrt{2}}
	\label{eq:rms}
\end{equation}
%
The resulting calculations provided a designed input of \SI{12.2}{\volt} (RMS).
Using an oscilloscope in combination with a variac transformer, the input was
adjusted to be as close to this value as possible. Data for the half-wave
rectifier was captured by the oscilloscope.  Due to the large number of
measurements taken, it is available by request to the author.

\subsection{Logic Indicator}
\begin{figure}[H]
	\centering
	\begin{tabular}{|c|c|c|}
\hline
\tbf{Input (V)} & \tbf{Output State} & \tbf{Comment} \\ \hline
0				& Off				& ---				\\ \hline
1				& Off				& ---				\\ \hline
2				& On				& Dim				\\ \hline
3				& On				& Brighter			\\ \hline
4				& On				& Full Brightness 	\\ \hline
5				& On				& ---				\\ \hline
\end{tabular}

	\caption{Brightness observations for the circuit shown in
		Figure~\ref{fig:schem2}.}
	\label{tab:ckt2data}
\end{figure}

\subsection{Voltage Regulator}
\begin{figure}[H]
	\centering
	\begin{tabular}{|c|c|}
\hline
\tbf{Input (V)} & \tbf{Output (V)} \\ \hline
5			& 4.997  	\\ \hline
6			& 5.997  	\\ \hline
7			& 6.998  	\\ \hline
8			& 7.999  	\\ \hline
9			& 8.999  	\\ \hline
10			& 9.751  	\\ \hline
11			& 9.810  	\\ \hline
12			& 9.850  	\\ \hline
13			& 9.940  	\\ \hline
14			& 10.017 	\\ \hline
15			& 10.089 	\\ \hline
16			& 10.157 	\\ \hline
17			& 10.230 	\\ \hline
18			& 10.312 	\\ \hline
19			& 10.388 	\\ \hline
20			& 10.440 	\\ \hline
21			& 10.390 	\\ \hline
22			& 10.331 	\\ \hline
23			& 10.268 	\\ \hline
24			& 10.199 	\\ \hline
25			& 10.140 	\\ \hline
\end{tabular}

	\caption{Output measurements of the zener diode voltage
		regulator described in Figure~\ref{fig:schem3}.}
	\label{tab:ckt3data}
\end{figure}

\subsection{Constant Current Source}
\begin{figure}[H]
	\centering
	\begin{tikzpicture}[gnuplot]
%% generated with GNUPLOT 4.4p2 (Lua 5.1.4; terminal rev. 97, script rev. 96a)
%% Wed 28 Sep 2011 01:27:40 PM EDT
\gpsolidlines
\gpcolor{gp lt color border}
\gpsetlinetype{gp lt border}
\gpsetlinewidth{1.00}
\draw[gp path] (1.504,0.985)--(1.684,0.985);
\node[gp node right] at (1.320,0.985) { 3};
\draw[gp path] (1.504,1.712)--(1.684,1.712);
\node[gp node right] at (1.320,1.712) { 3.5};
\draw[gp path] (1.504,2.439)--(1.684,2.439);
\node[gp node right] at (1.320,2.439) { 4};
\draw[gp path] (1.504,3.166)--(1.684,3.166);
\node[gp node right] at (1.320,3.166) { 4.5};
\draw[gp path] (1.504,3.892)--(1.684,3.892);
\node[gp node right] at (1.320,3.892) { 5};
\draw[gp path] (1.504,4.619)--(1.684,4.619);
\node[gp node right] at (1.320,4.619) { 5.5};
\draw[gp path] (1.504,5.346)--(1.684,5.346);
\node[gp node right] at (1.320,5.346) { 6};
\draw[gp path] (1.504,0.985)--(1.504,1.165);
\node[gp node center] at (1.504,0.677) { 0.01};
\draw[gp path] (2.381,0.985)--(2.381,1.075);
\draw[gp path] (2.894,0.985)--(2.894,1.075);
\draw[gp path] (3.258,0.985)--(3.258,1.075);
\draw[gp path] (3.540,0.985)--(3.540,1.075);
\draw[gp path] (3.770,0.985)--(3.770,1.075);
\draw[gp path] (3.965,0.985)--(3.965,1.075);
\draw[gp path] (4.134,0.985)--(4.134,1.075);
\draw[gp path] (4.283,0.985)--(4.283,1.075);
\draw[gp path] (4.417,0.985)--(4.417,1.165);
\node[gp node center] at (4.417,0.677) { 0.1};
\draw[gp path] (5.293,0.985)--(5.293,1.075);
\draw[gp path] (5.806,0.985)--(5.806,1.075);
\draw[gp path] (6.170,0.985)--(6.170,1.075);
\draw[gp path] (6.453,0.985)--(6.453,1.075);
\draw[gp path] (6.683,0.985)--(6.683,1.075);
\draw[gp path] (6.878,0.985)--(6.878,1.075);
\draw[gp path] (7.047,0.985)--(7.047,1.075);
\draw[gp path] (7.196,0.985)--(7.196,1.075);
\draw[gp path] (7.329,0.985)--(7.329,1.165);
\node[gp node center] at (7.329,0.677) { 1};
\draw[gp path] (8.206,0.985)--(8.206,1.075);
\draw[gp path] (8.719,0.985)--(8.719,1.075);
\draw[gp path] (9.083,0.985)--(9.083,1.075);
\draw[gp path] (9.365,0.985)--(9.365,1.075);
\draw[gp path] (9.596,0.985)--(9.596,1.075);
\draw[gp path] (9.791,0.985)--(9.791,1.075);
\draw[gp path] (9.960,0.985)--(9.960,1.075);
\draw[gp path] (10.109,0.985)--(10.109,1.075);
\draw[gp path] (10.242,0.985)--(10.242,1.165);
\node[gp node center] at (10.242,0.677) { 10};
\draw[gp path] (1.504,5.346)--(1.504,0.985)--(10.242,0.985);
\node[gp node center,rotate=-270] at (0.246,3.165) {Measured Current, $I_{out}$ (\si{\milli\ampere})};
\node[gp node center] at (5.873,0.215) {Load Resistance, $R$ (\si{\kilo\ohm})};
\node[gp node right] at (3.712,1.627) {16V source};
\gpcolor{gp lt color 0}
\gpsetlinetype{gp lt plot 0}
\draw[gp path] (3.896,1.627)--(4.812,1.627);
\draw[gp path] (3.540,4.881)--(4.417,4.866)--(5.293,4.866)--(5.806,4.866)--(6.170,4.866)%
  --(6.453,4.881)--(6.683,4.881)--(6.878,4.881)--(7.047,4.881)--(7.196,4.881)--(7.329,4.881)%
  --(7.450,4.881)--(7.560,4.881)--(7.661,4.881)--(7.755,4.895)--(7.842,4.895)--(7.924,4.881)%
  --(8.001,4.866)--(8.073,4.837)--(8.141,4.823)--(8.206,4.823)--(8.268,4.794)--(8.327,4.721)%
  --(8.383,4.634)--(8.437,4.517)--(8.488,4.372)--(8.538,4.212)--(8.586,4.038)--(8.632,3.863)%
  --(8.676,3.674)--(8.719,3.500)--(8.761,3.325)--(8.801,3.151)--(8.840,2.991)--(8.877,2.846)%
  --(8.914,2.686)--(8.950,2.540)--(8.984,2.410)--(9.018,2.264)--(9.051,2.133)--(9.083,2.017)%
  --(9.114,1.901)--(9.145,1.785)--(9.174,1.668)--(9.203,1.566)--(9.232,1.465)--(9.260,1.377)%
  --(9.287,1.276)--(9.314,1.189)--(9.340,1.101)--(9.365,1.014);
\gpsetpointsize{4.00}
\gppoint{gp mark 1}{(3.540,4.881)}
\gppoint{gp mark 1}{(4.417,4.866)}
\gppoint{gp mark 1}{(5.293,4.866)}
\gppoint{gp mark 1}{(5.806,4.866)}
\gppoint{gp mark 1}{(6.170,4.866)}
\gppoint{gp mark 1}{(6.453,4.881)}
\gppoint{gp mark 1}{(6.683,4.881)}
\gppoint{gp mark 1}{(6.878,4.881)}
\gppoint{gp mark 1}{(7.047,4.881)}
\gppoint{gp mark 1}{(7.196,4.881)}
\gppoint{gp mark 1}{(7.329,4.881)}
\gppoint{gp mark 1}{(7.450,4.881)}
\gppoint{gp mark 1}{(7.560,4.881)}
\gppoint{gp mark 1}{(7.661,4.881)}
\gppoint{gp mark 1}{(7.755,4.895)}
\gppoint{gp mark 1}{(7.842,4.895)}
\gppoint{gp mark 1}{(7.924,4.881)}
\gppoint{gp mark 1}{(8.001,4.866)}
\gppoint{gp mark 1}{(8.073,4.837)}
\gppoint{gp mark 1}{(8.141,4.823)}
\gppoint{gp mark 1}{(8.206,4.823)}
\gppoint{gp mark 1}{(8.268,4.794)}
\gppoint{gp mark 1}{(8.327,4.721)}
\gppoint{gp mark 1}{(8.383,4.634)}
\gppoint{gp mark 1}{(8.437,4.517)}
\gppoint{gp mark 1}{(8.488,4.372)}
\gppoint{gp mark 1}{(8.538,4.212)}
\gppoint{gp mark 1}{(8.586,4.038)}
\gppoint{gp mark 1}{(8.632,3.863)}
\gppoint{gp mark 1}{(8.676,3.674)}
\gppoint{gp mark 1}{(8.719,3.500)}
\gppoint{gp mark 1}{(8.761,3.325)}
\gppoint{gp mark 1}{(8.801,3.151)}
\gppoint{gp mark 1}{(8.840,2.991)}
\gppoint{gp mark 1}{(8.877,2.846)}
\gppoint{gp mark 1}{(8.914,2.686)}
\gppoint{gp mark 1}{(8.950,2.540)}
\gppoint{gp mark 1}{(8.984,2.410)}
\gppoint{gp mark 1}{(9.018,2.264)}
\gppoint{gp mark 1}{(9.051,2.133)}
\gppoint{gp mark 1}{(9.083,2.017)}
\gppoint{gp mark 1}{(9.114,1.901)}
\gppoint{gp mark 1}{(9.145,1.785)}
\gppoint{gp mark 1}{(9.174,1.668)}
\gppoint{gp mark 1}{(9.203,1.566)}
\gppoint{gp mark 1}{(9.232,1.465)}
\gppoint{gp mark 1}{(9.260,1.377)}
\gppoint{gp mark 1}{(9.287,1.276)}
\gppoint{gp mark 1}{(9.314,1.189)}
\gppoint{gp mark 1}{(9.340,1.101)}
\gppoint{gp mark 1}{(9.365,1.014)}
\gppoint{gp mark 1}{(4.354,1.627)}
\gpcolor{gp lt color border}
\node[gp node right] at (3.712,1.319) {32V source };
\gpcolor{gp lt color 1}
\gpsetlinetype{gp lt plot 1}
\draw[gp path] (3.896,1.319)--(4.812,1.319);
\draw[gp path] (3.540,4.445)--(4.417,4.445)--(5.293,4.459)--(5.806,4.474)--(6.170,4.474)%
  --(6.453,4.474)--(6.683,4.488)--(6.878,4.503)--(7.047,4.517)--(7.196,4.517)--(7.329,4.561)%
  --(7.450,4.561)--(7.560,4.561)--(7.661,4.561)--(7.755,4.576)--(7.842,4.590)--(7.924,4.605)%
  --(8.001,4.619)--(8.073,4.634)--(8.141,4.634)--(8.206,4.677)--(8.268,4.706)--(8.327,4.706)%
  --(8.383,4.706)--(8.437,4.706)--(8.488,4.706)--(8.538,4.721)--(8.586,4.735)--(8.632,4.750)%
  --(8.676,4.765)--(8.719,4.779)--(8.761,4.794)--(8.801,4.808)--(8.840,4.823)--(8.877,4.823)%
  --(8.914,4.837)--(8.950,4.837)--(8.984,4.837)--(9.018,4.852)--(9.051,4.866)--(9.083,4.866)%
  --(9.114,4.866)--(9.145,4.866)--(9.174,4.852)--(9.203,4.837)--(9.232,4.823)--(9.260,4.823)%
  --(9.287,4.794)--(9.314,4.765)--(9.340,4.735)--(9.365,4.706);
\gppoint{gp mark 2}{(3.540,4.445)}
\gppoint{gp mark 2}{(4.417,4.445)}
\gppoint{gp mark 2}{(5.293,4.459)}
\gppoint{gp mark 2}{(5.806,4.474)}
\gppoint{gp mark 2}{(6.170,4.474)}
\gppoint{gp mark 2}{(6.453,4.474)}
\gppoint{gp mark 2}{(6.683,4.488)}
\gppoint{gp mark 2}{(6.878,4.503)}
\gppoint{gp mark 2}{(7.047,4.517)}
\gppoint{gp mark 2}{(7.196,4.517)}
\gppoint{gp mark 2}{(7.329,4.561)}
\gppoint{gp mark 2}{(7.450,4.561)}
\gppoint{gp mark 2}{(7.560,4.561)}
\gppoint{gp mark 2}{(7.661,4.561)}
\gppoint{gp mark 2}{(7.755,4.576)}
\gppoint{gp mark 2}{(7.842,4.590)}
\gppoint{gp mark 2}{(7.924,4.605)}
\gppoint{gp mark 2}{(8.001,4.619)}
\gppoint{gp mark 2}{(8.073,4.634)}
\gppoint{gp mark 2}{(8.141,4.634)}
\gppoint{gp mark 2}{(8.206,4.677)}
\gppoint{gp mark 2}{(8.268,4.706)}
\gppoint{gp mark 2}{(8.327,4.706)}
\gppoint{gp mark 2}{(8.383,4.706)}
\gppoint{gp mark 2}{(8.437,4.706)}
\gppoint{gp mark 2}{(8.488,4.706)}
\gppoint{gp mark 2}{(8.538,4.721)}
\gppoint{gp mark 2}{(8.586,4.735)}
\gppoint{gp mark 2}{(8.632,4.750)}
\gppoint{gp mark 2}{(8.676,4.765)}
\gppoint{gp mark 2}{(8.719,4.779)}
\gppoint{gp mark 2}{(8.761,4.794)}
\gppoint{gp mark 2}{(8.801,4.808)}
\gppoint{gp mark 2}{(8.840,4.823)}
\gppoint{gp mark 2}{(8.877,4.823)}
\gppoint{gp mark 2}{(8.914,4.837)}
\gppoint{gp mark 2}{(8.950,4.837)}
\gppoint{gp mark 2}{(8.984,4.837)}
\gppoint{gp mark 2}{(9.018,4.852)}
\gppoint{gp mark 2}{(9.051,4.866)}
\gppoint{gp mark 2}{(9.083,4.866)}
\gppoint{gp mark 2}{(9.114,4.866)}
\gppoint{gp mark 2}{(9.145,4.866)}
\gppoint{gp mark 2}{(9.174,4.852)}
\gppoint{gp mark 2}{(9.203,4.837)}
\gppoint{gp mark 2}{(9.232,4.823)}
\gppoint{gp mark 2}{(9.260,4.823)}
\gppoint{gp mark 2}{(9.287,4.794)}
\gppoint{gp mark 2}{(9.314,4.765)}
\gppoint{gp mark 2}{(9.340,4.735)}
\gppoint{gp mark 2}{(9.365,4.706)}
\gppoint{gp mark 2}{(4.354,1.319)}
\gpcolor{gp lt color border}
\gpsetlinetype{gp lt border}
\draw[gp path] (1.504,5.346)--(1.504,0.985)--(10.242,0.985);
%% coordinates of the plot area
\gpdefrectangularnode{gp plot 1}{\pgfpoint{1.504cm}{0.985cm}}{\pgfpoint{10.242cm}{5.346cm}}
\end{tikzpicture}
%% gnuplot variables

	\caption{Measured current with a variable resistance.  The experiment was
		performed with a source voltage of both~\SI{16}{\volt}
		and~\SI{32}{\volt}, as depicted in Figure~\ref{fig:schem4}}
	\label{tab:ckt4data}
\end{figure}
