A free-running oscillator can be created by using the schematic shown in
Figure~\ref{fig:free_run}, where the switch is fixed closed.
%
\begin{figure}[H]
	\centering
	%\includegraphics[width=.6\textwidth]{img/shot/free_run.pdf}
	A free-running oscillator can be created by using the schematic shown in
Figure~\ref{fig:free_run}, where the switch is fixed closed.
%
\begin{figure}[H]
	\centering
	\includegraphics[width=.6\textwidth]{img/shot/free_run.pdf}
	\caption{}
	\label{fig:free_run}
\end{figure}
%
Using this circuit, the period of the output signal can be calculated by using~\eqref{eq:t0}:
%
\begin{equation}
	T_0 = \frac{3}{2} R_A C_T \left[ 1 + \frac{R_B}{2R_A - R_B} \right] \quad \text{,}
	\label{eq:t0}
\end{equation}
%
where only~$R_A$ and~$C_T$ are external to the provided circuit and available
to modify.  In the case that~$R_A = R_B$, the duty cycle of the square wave
output becomes exactly~\SI{50}{\percent}, and the frequency calculation
simiplifies to the~\eqref{eq:t0_even}.
%
\begin{equation}
	T_0 = 3 R_A C_T
	\label{eq:t0_even}
\end{equation}
%
Testing the possible output frequencies is possible by fixing~$C_T$ at the suggested value of~\SI{3300}{\pico\farad} and varying~$R_A$.

	\parbox{.6\textwidth}{
	\caption{Free-running oscilloscope circuit schematic, as provided in Figure~1 of the
	Intersil datasheet for an ICL8038 integrated circuit [1].}
	\label{fig:free_run}}
\end{figure}
%
Using this circuit, the period of the output signal can be calculated by using~\eqref{eq:t0}:
%
\begin{equation}
	T_0 = \frac{3}{2} R_A C_T \left[ 1 + \frac{R_B}{2R_A - R_B} \right] \quad \text{,}
	\label{eq:t0}
\end{equation}
%
where only~$R_A$ and~$C_T$ are external to the provided circuit and available
to modify.  In the case that~$R_A = R_B$, the duty cycle of the square wave
output becomes exactly~\SI{50}{\percent}, and the frequency calculation
simiplifies to the~\eqref{eq:t0_even}.
%
\begin{equation}
	T_0 = 3 R_A C_T
	\label{eq:t0_even}
\end{equation}
%
Testing the possible output frequencies is possible by fixing~$C_T$ at the
suggested value of~\SI{3300}{\pico\farad} and varying~$R_A$.  In order to view
all three cases for $R_A$~(where~$R_A < R_B$,~$R_A = R_B$, and~$R_A > R_B$), it
should be varied from~\SI{1}{\kilo\ohm} to~\SI{20}{\kilo\ohm} in steps
of~\SI{2}{\kilo\ohm} while measuring the output frequency and duty cycle at
each stage.  This allows several steps before and after the critical point
of~$R_A = R_B$, so that students may view the effects of a varying resistance.

Similarly, $R_A$ can be fixed to~\SI{10}{\kilo\ohm}~(the same as~$R_B$) while
varying~$C_T$.  This ensures a constant duty cycle of~\SI{50}{\percent} over
different values for the timing capacitor.  It is suggested that~$C_T$ is
varied from~\SI{100}{\pico\farad} to~\SI{100}{\micro\farad} in steps of an
order of magnitude, while using an oscilloscope to measure the output frequency
at each step.  By~\eqref{eq:t0_even}, the frequency will vary linearly with
changing capacitance.

\begin{figure}[H]
	\centering
	\begin{tikzpicture}[gnuplot]
%% generated with GNUPLOT 4.4p2 (Lua 5.1.4; terminal rev. 97, script rev. 96a)
%% Thu 17 Nov 2011 11:24:14 PM EST
\gpsolidlines
\gpcolor{gp lt color axes}
\gpsetlinetype{gp lt axes}
\gpsetlinewidth{1.00}
\draw[gp path] (1.688,0.985)--(9.014,0.985);
\gpcolor{gp lt color border}
\gpsetlinetype{gp lt border}
\draw[gp path] (1.688,0.985)--(1.868,0.985);
\node[gp node right] at (1.504,0.985) { 0.1};
\draw[gp path] (1.688,1.271)--(1.778,1.271);
\draw[gp path] (1.688,1.439)--(1.778,1.439);
\draw[gp path] (1.688,1.558)--(1.778,1.558);
\draw[gp path] (1.688,1.650)--(1.778,1.650);
\draw[gp path] (1.688,1.725)--(1.778,1.725);
\draw[gp path] (1.688,1.789)--(1.778,1.789);
\draw[gp path] (1.688,1.844)--(1.778,1.844);
\draw[gp path] (1.688,1.893)--(1.778,1.893);
\gpcolor{gp lt color axes}
\gpsetlinetype{gp lt axes}
\draw[gp path] (1.688,1.936)--(9.014,1.936);
\gpcolor{gp lt color border}
\gpsetlinetype{gp lt border}
\draw[gp path] (1.688,1.936)--(1.868,1.936);
\node[gp node right] at (1.504,1.936) { 1};
\draw[gp path] (1.688,2.223)--(1.778,2.223);
\draw[gp path] (1.688,2.390)--(1.778,2.390);
\draw[gp path] (1.688,2.509)--(1.778,2.509);
\draw[gp path] (1.688,2.601)--(1.778,2.601);
\draw[gp path] (1.688,2.676)--(1.778,2.676);
\draw[gp path] (1.688,2.740)--(1.778,2.740);
\draw[gp path] (1.688,2.795)--(1.778,2.795);
\draw[gp path] (1.688,2.844)--(1.778,2.844);
\gpcolor{gp lt color axes}
\gpsetlinetype{gp lt axes}
\draw[gp path] (1.688,2.888)--(5.522,2.888);
\draw[gp path] (8.830,2.888)--(9.014,2.888);
\gpcolor{gp lt color border}
\gpsetlinetype{gp lt border}
\draw[gp path] (1.688,2.888)--(1.868,2.888);
\node[gp node right] at (1.504,2.888) { 10};
\draw[gp path] (1.688,3.174)--(1.778,3.174);
\draw[gp path] (1.688,3.341)--(1.778,3.341);
\draw[gp path] (1.688,3.460)--(1.778,3.460);
\draw[gp path] (1.688,3.552)--(1.778,3.552);
\draw[gp path] (1.688,3.628)--(1.778,3.628);
\draw[gp path] (1.688,3.691)--(1.778,3.691);
\draw[gp path] (1.688,3.747)--(1.778,3.747);
\draw[gp path] (1.688,3.795)--(1.778,3.795);
\gpcolor{gp lt color axes}
\gpsetlinetype{gp lt axes}
\draw[gp path] (1.688,3.839)--(9.014,3.839);
\gpcolor{gp lt color border}
\gpsetlinetype{gp lt border}
\draw[gp path] (1.688,3.839)--(1.868,3.839);
\node[gp node right] at (1.504,3.839) { 100};
\draw[gp path] (1.688,4.125)--(1.778,4.125);
\draw[gp path] (1.688,4.293)--(1.778,4.293);
\draw[gp path] (1.688,4.411)--(1.778,4.411);
\draw[gp path] (1.688,4.504)--(1.778,4.504);
\draw[gp path] (1.688,4.579)--(1.778,4.579);
\draw[gp path] (1.688,4.643)--(1.778,4.643);
\draw[gp path] (1.688,4.698)--(1.778,4.698);
\draw[gp path] (1.688,4.746)--(1.778,4.746);
\gpcolor{gp lt color axes}
\gpsetlinetype{gp lt axes}
\draw[gp path] (1.688,4.790)--(9.014,4.790);
\gpcolor{gp lt color border}
\gpsetlinetype{gp lt border}
\draw[gp path] (1.688,4.790)--(1.868,4.790);
\node[gp node right] at (1.504,4.790) { 1000};
\gpcolor{gp lt color axes}
\gpsetlinetype{gp lt axes}
\draw[gp path] (1.688,0.985)--(1.688,4.790);
\gpcolor{gp lt color border}
\gpsetlinetype{gp lt border}
\draw[gp path] (1.688,0.985)--(1.688,1.165);
\node[gp node center] at (1.688,0.677) { 0.0001};
\draw[gp path] (2.157,0.985)--(2.157,1.075);
\draw[gp path] (2.778,0.985)--(2.778,1.075);
\draw[gp path] (3.096,0.985)--(3.096,1.075);
\gpcolor{gp lt color axes}
\gpsetlinetype{gp lt axes}
\draw[gp path] (3.247,0.985)--(3.247,4.790);
\gpcolor{gp lt color border}
\gpsetlinetype{gp lt border}
\draw[gp path] (3.247,0.985)--(3.247,1.165);
\node[gp node center] at (3.247,0.677) { 0.001};
\draw[gp path] (3.716,0.985)--(3.716,1.075);
\draw[gp path] (4.337,0.985)--(4.337,1.075);
\draw[gp path] (4.655,0.985)--(4.655,1.075);
\gpcolor{gp lt color axes}
\gpsetlinetype{gp lt axes}
\draw[gp path] (4.806,0.985)--(4.806,4.790);
\gpcolor{gp lt color border}
\gpsetlinetype{gp lt border}
\draw[gp path] (4.806,0.985)--(4.806,1.165);
\node[gp node center] at (4.806,0.677) { 0.01};
\draw[gp path] (5.275,0.985)--(5.275,1.075);
\draw[gp path] (5.896,0.985)--(5.896,1.075);
\draw[gp path] (6.214,0.985)--(6.214,1.075);
\gpcolor{gp lt color axes}
\gpsetlinetype{gp lt axes}
\draw[gp path] (6.365,0.985)--(6.365,2.425);
\draw[gp path] (6.365,3.349)--(6.365,4.790);
\gpcolor{gp lt color border}
\gpsetlinetype{gp lt border}
\draw[gp path] (6.365,0.985)--(6.365,1.165);
\node[gp node center] at (6.365,0.677) { 0.1};
\draw[gp path] (6.835,0.985)--(6.835,1.075);
\draw[gp path] (7.455,0.985)--(7.455,1.075);
\draw[gp path] (7.773,0.985)--(7.773,1.075);
\gpcolor{gp lt color axes}
\gpsetlinetype{gp lt axes}
\draw[gp path] (7.924,0.985)--(7.924,2.425);
\draw[gp path] (7.924,3.349)--(7.924,4.790);
\gpcolor{gp lt color border}
\gpsetlinetype{gp lt border}
\draw[gp path] (7.924,0.985)--(7.924,1.165);
\node[gp node center] at (7.924,0.677) { 1};
\draw[gp path] (8.394,0.985)--(8.394,1.075);
\draw[gp path] (9.014,0.985)--(9.014,1.075);
\draw[gp path] (9.014,0.985)--(8.834,0.985);
\node[gp node left] at (9.198,0.985) { 0};
\draw[gp path] (9.014,1.936)--(8.834,1.936);
\node[gp node left] at (9.198,1.936) { 15};
\draw[gp path] (9.014,2.888)--(8.834,2.888);
\node[gp node left] at (9.198,2.888) { 30};
\draw[gp path] (9.014,3.839)--(8.834,3.839);
\node[gp node left] at (9.198,3.839) { 45};
\draw[gp path] (9.014,4.790)--(8.834,4.790);
\node[gp node left] at (9.198,4.790) { 60};
\draw[gp path] (1.688,4.790)--(1.688,0.985)--(9.014,0.985)--(9.014,4.790)--cycle;
\node[gp node center,rotate=-270] at (0.246,2.887) {Output Frequency, $f$ (\si{\kilo\hertz})};
\node[gp node center,rotate=-270] at (10.087,2.887) {Output Duty Cycle, (\si{\percent})};
\node[gp node center] at (5.351,0.215) {Capacitance, $C_T$ (\si{\micro\farad})};
\node[gp node center] at (5.351,5.252) {Free-running Oscillator: Varying Capacitor};
\draw[gp path] (5.522,2.425)--(5.522,3.349)--(8.830,3.349)--(8.830,2.425)--cycle;
\draw[gp path] (5.522,3.349)--(8.830,3.349);
\node[gp node right] at (7.546,3.041) {$f$};
\gpcolor{gp lt color 0}
\gpsetlinetype{gp lt plot 0}
\gpsetlinewidth{3.00}
\draw[gp path] (7.730,3.041)--(8.646,3.041);
\draw[gp path] (1.688,3.998)--(2.157,3.851)--(2.496,3.735)--(3.247,3.237)--(3.716,3.019)%
  --(4.055,2.919)--(4.806,2.402)--(5.340,2.128)--(5.615,1.974)--(6.365,1.492)--(6.899,1.182)%
  --(7.174,1.032)--(7.924,1.494)--(8.458,1.169)--(8.733,1.021);
\gpsetpointsize{4.00}
\gppoint{gp mark 8}{(1.688,3.998)}
\gppoint{gp mark 8}{(2.157,3.851)}
\gppoint{gp mark 8}{(2.496,3.735)}
\gppoint{gp mark 8}{(3.247,3.237)}
\gppoint{gp mark 8}{(3.716,3.019)}
\gppoint{gp mark 8}{(4.055,2.919)}
\gppoint{gp mark 8}{(4.806,2.402)}
\gppoint{gp mark 8}{(5.340,2.128)}
\gppoint{gp mark 8}{(5.615,1.974)}
\gppoint{gp mark 8}{(6.365,1.492)}
\gppoint{gp mark 8}{(6.899,1.182)}
\gppoint{gp mark 8}{(7.174,1.032)}
\gppoint{gp mark 8}{(7.924,1.494)}
\gppoint{gp mark 8}{(8.458,1.169)}
\gppoint{gp mark 8}{(8.733,1.021)}
\gppoint{gp mark 8}{(8.188,3.041)}
\gpcolor{gp lt color border}
\node[gp node right] at (7.546,2.733) {Duty Cycle};
\gpcolor{gp lt color 1}
\gpsetlinetype{gp lt plot 1}
\draw[gp path] (7.730,2.733)--(8.646,2.733);
\draw[gp path] (1.688,3.084)--(2.157,3.433)--(2.496,3.604)--(3.247,3.985)--(3.716,4.048)%
  --(4.055,4.073)--(4.806,4.149)--(5.340,4.143)--(5.615,4.149)--(6.365,4.156)--(6.899,4.149)%
  --(7.174,4.181)--(7.924,4.156)--(8.458,4.188)--(8.733,4.162);
\gppoint{gp mark 5}{(1.688,3.084)}
\gppoint{gp mark 5}{(2.157,3.433)}
\gppoint{gp mark 5}{(2.496,3.604)}
\gppoint{gp mark 5}{(3.247,3.985)}
\gppoint{gp mark 5}{(3.716,4.048)}
\gppoint{gp mark 5}{(4.055,4.073)}
\gppoint{gp mark 5}{(4.806,4.149)}
\gppoint{gp mark 5}{(5.340,4.143)}
\gppoint{gp mark 5}{(5.615,4.149)}
\gppoint{gp mark 5}{(6.365,4.156)}
\gppoint{gp mark 5}{(6.899,4.149)}
\gppoint{gp mark 5}{(7.174,4.181)}
\gppoint{gp mark 5}{(7.924,4.156)}
\gppoint{gp mark 5}{(8.458,4.188)}
\gppoint{gp mark 5}{(8.733,4.162)}
\gppoint{gp mark 5}{(8.188,2.733)}
\gpcolor{gp lt color border}
\gpsetlinetype{gp lt border}
\gpsetlinewidth{1.00}
\draw[gp path] (1.688,4.790)--(1.688,0.985)--(9.014,0.985)--(9.014,4.790)--cycle;
%% coordinates of the plot area
\gpdefrectangularnode{gp plot 1}{\pgfpoint{1.688cm}{0.985cm}}{\pgfpoint{9.014cm}{4.790cm}}
\end{tikzpicture}
%% gnuplot variables

\end{figure}

\begin{figure}[H]
	\centering
	\begin{tikzpicture}[gnuplot]
%% generated with GNUPLOT 4.4p2 (Lua 5.1.4; terminal rev. 97, script rev. 96a)
%% Thu 17 Nov 2011 08:55:20 PM EST
\gpsolidlines
\gpcolor{gp lt color axes}
\gpsetlinetype{gp lt axes}
\gpsetlinewidth{1.00}
\draw[gp path] (1.320,0.985)--(9.014,0.985);
\gpcolor{gp lt color border}
\gpsetlinetype{gp lt border}
\draw[gp path] (1.320,0.985)--(1.500,0.985);
\node[gp node right] at (1.136,0.985) { 0};
\gpcolor{gp lt color axes}
\gpsetlinetype{gp lt axes}
\draw[gp path] (1.320,1.461)--(9.014,1.461);
\gpcolor{gp lt color border}
\gpsetlinetype{gp lt border}
\draw[gp path] (1.320,1.461)--(1.500,1.461);
\node[gp node right] at (1.136,1.461) { 2};
\gpcolor{gp lt color axes}
\gpsetlinetype{gp lt axes}
\draw[gp path] (1.320,1.936)--(9.014,1.936);
\gpcolor{gp lt color border}
\gpsetlinetype{gp lt border}
\draw[gp path] (1.320,1.936)--(1.500,1.936);
\node[gp node right] at (1.136,1.936) { 4};
\gpcolor{gp lt color axes}
\gpsetlinetype{gp lt axes}
\draw[gp path] (1.320,2.412)--(9.014,2.412);
\gpcolor{gp lt color border}
\gpsetlinetype{gp lt border}
\draw[gp path] (1.320,2.412)--(1.500,2.412);
\node[gp node right] at (1.136,2.412) { 6};
\gpcolor{gp lt color axes}
\gpsetlinetype{gp lt axes}
\draw[gp path] (1.320,2.888)--(9.014,2.888);
\gpcolor{gp lt color border}
\gpsetlinetype{gp lt border}
\draw[gp path] (1.320,2.888)--(1.500,2.888);
\node[gp node right] at (1.136,2.888) { 8};
\gpcolor{gp lt color axes}
\gpsetlinetype{gp lt axes}
\draw[gp path] (1.320,3.363)--(9.014,3.363);
\gpcolor{gp lt color border}
\gpsetlinetype{gp lt border}
\draw[gp path] (1.320,3.363)--(1.500,3.363);
\node[gp node right] at (1.136,3.363) { 10};
\gpcolor{gp lt color axes}
\gpsetlinetype{gp lt axes}
\draw[gp path] (1.320,3.839)--(1.504,3.839);
\draw[gp path] (4.812,3.839)--(9.014,3.839);
\gpcolor{gp lt color border}
\gpsetlinetype{gp lt border}
\draw[gp path] (1.320,3.839)--(1.500,3.839);
\node[gp node right] at (1.136,3.839) { 12};
\gpcolor{gp lt color axes}
\gpsetlinetype{gp lt axes}
\draw[gp path] (1.320,4.314)--(1.504,4.314);
\draw[gp path] (4.812,4.314)--(9.014,4.314);
\gpcolor{gp lt color border}
\gpsetlinetype{gp lt border}
\draw[gp path] (1.320,4.314)--(1.500,4.314);
\node[gp node right] at (1.136,4.314) { 14};
\gpcolor{gp lt color axes}
\gpsetlinetype{gp lt axes}
\draw[gp path] (1.320,4.790)--(9.014,4.790);
\gpcolor{gp lt color border}
\gpsetlinetype{gp lt border}
\draw[gp path] (1.320,4.790)--(1.500,4.790);
\node[gp node right] at (1.136,4.790) { 16};
\gpcolor{gp lt color axes}
\gpsetlinetype{gp lt axes}
\draw[gp path] (1.320,0.985)--(1.320,4.790);
\gpcolor{gp lt color border}
\gpsetlinetype{gp lt border}
\draw[gp path] (1.320,0.985)--(1.320,1.165);
\node[gp node center] at (1.320,0.677) { 0};
\gpcolor{gp lt color axes}
\gpsetlinetype{gp lt axes}
\draw[gp path] (2.089,0.985)--(2.089,3.686);
\draw[gp path] (2.089,4.610)--(2.089,4.790);
\gpcolor{gp lt color border}
\gpsetlinetype{gp lt border}
\draw[gp path] (2.089,0.985)--(2.089,1.165);
\node[gp node center] at (2.089,0.677) { 2};
\gpcolor{gp lt color axes}
\gpsetlinetype{gp lt axes}
\draw[gp path] (2.859,0.985)--(2.859,3.686);
\draw[gp path] (2.859,4.610)--(2.859,4.790);
\gpcolor{gp lt color border}
\gpsetlinetype{gp lt border}
\draw[gp path] (2.859,0.985)--(2.859,1.165);
\node[gp node center] at (2.859,0.677) { 4};
\gpcolor{gp lt color axes}
\gpsetlinetype{gp lt axes}
\draw[gp path] (3.628,0.985)--(3.628,3.686);
\draw[gp path] (3.628,4.610)--(3.628,4.790);
\gpcolor{gp lt color border}
\gpsetlinetype{gp lt border}
\draw[gp path] (3.628,0.985)--(3.628,1.165);
\node[gp node center] at (3.628,0.677) { 6};
\gpcolor{gp lt color axes}
\gpsetlinetype{gp lt axes}
\draw[gp path] (4.398,0.985)--(4.398,3.686);
\draw[gp path] (4.398,4.610)--(4.398,4.790);
\gpcolor{gp lt color border}
\gpsetlinetype{gp lt border}
\draw[gp path] (4.398,0.985)--(4.398,1.165);
\node[gp node center] at (4.398,0.677) { 8};
\gpcolor{gp lt color axes}
\gpsetlinetype{gp lt axes}
\draw[gp path] (5.167,0.985)--(5.167,4.790);
\gpcolor{gp lt color border}
\gpsetlinetype{gp lt border}
\draw[gp path] (5.167,0.985)--(5.167,1.165);
\node[gp node center] at (5.167,0.677) { 10};
\gpcolor{gp lt color axes}
\gpsetlinetype{gp lt axes}
\draw[gp path] (5.936,0.985)--(5.936,4.790);
\gpcolor{gp lt color border}
\gpsetlinetype{gp lt border}
\draw[gp path] (5.936,0.985)--(5.936,1.165);
\node[gp node center] at (5.936,0.677) { 12};
\gpcolor{gp lt color axes}
\gpsetlinetype{gp lt axes}
\draw[gp path] (6.706,0.985)--(6.706,4.790);
\gpcolor{gp lt color border}
\gpsetlinetype{gp lt border}
\draw[gp path] (6.706,0.985)--(6.706,1.165);
\node[gp node center] at (6.706,0.677) { 14};
\gpcolor{gp lt color axes}
\gpsetlinetype{gp lt axes}
\draw[gp path] (7.475,0.985)--(7.475,4.790);
\gpcolor{gp lt color border}
\gpsetlinetype{gp lt border}
\draw[gp path] (7.475,0.985)--(7.475,1.165);
\node[gp node center] at (7.475,0.677) { 16};
\gpcolor{gp lt color axes}
\gpsetlinetype{gp lt axes}
\draw[gp path] (8.245,0.985)--(8.245,4.790);
\gpcolor{gp lt color border}
\gpsetlinetype{gp lt border}
\draw[gp path] (8.245,0.985)--(8.245,1.165);
\node[gp node center] at (8.245,0.677) { 18};
\gpcolor{gp lt color axes}
\gpsetlinetype{gp lt axes}
\draw[gp path] (9.014,0.985)--(9.014,4.790);
\gpcolor{gp lt color border}
\gpsetlinetype{gp lt border}
\draw[gp path] (9.014,0.985)--(9.014,1.165);
\node[gp node center] at (9.014,0.677) { 20};
\draw[gp path] (9.014,0.985)--(8.834,0.985);
\node[gp node left] at (9.198,0.985) { 0};
\draw[gp path] (9.014,1.461)--(8.834,1.461);
\node[gp node left] at (9.198,1.461) { 10};
\draw[gp path] (9.014,1.936)--(8.834,1.936);
\node[gp node left] at (9.198,1.936) { 20};
\draw[gp path] (9.014,2.412)--(8.834,2.412);
\node[gp node left] at (9.198,2.412) { 30};
\draw[gp path] (9.014,2.888)--(8.834,2.888);
\node[gp node left] at (9.198,2.888) { 40};
\draw[gp path] (9.014,3.363)--(8.834,3.363);
\node[gp node left] at (9.198,3.363) { 50};
\draw[gp path] (9.014,3.839)--(8.834,3.839);
\node[gp node left] at (9.198,3.839) { 60};
\draw[gp path] (9.014,4.314)--(8.834,4.314);
\node[gp node left] at (9.198,4.314) { 70};
\draw[gp path] (9.014,4.790)--(8.834,4.790);
\node[gp node left] at (9.198,4.790) { 80};
\draw[gp path] (1.320,4.790)--(1.320,0.985)--(9.014,0.985)--(9.014,4.790)--cycle;
\node[gp node center,rotate=-270] at (0.246,2.887) {Output Frequency, $f$ (\si{\kilo\hertz})};
\node[gp node center,rotate=-270] at (10.087,2.887) {Output Duty Cycle, (\si{\percent})};
\node[gp node center] at (5.167,0.215) {Resistance, $R_A$ (\si{\kilo\ohm})};
\node[gp node center] at (5.167,5.252) {Free-running Oscillator: Varying Resistor};
\draw[gp path] (1.504,3.686)--(1.504,4.610)--(4.812,4.610)--(4.812,3.686)--cycle;
\draw[gp path] (1.504,4.610)--(4.812,4.610);
\node[gp node right] at (3.528,4.302) {$f$};
\gpcolor{gp lt color 0}
\gpsetlinetype{gp lt plot 0}
\gpsetlinewidth{3.00}
\draw[gp path] (3.712,4.302)--(4.628,4.302);
\draw[gp path] (3.628,2.405)--(4.398,3.382)--(5.167,3.534)--(5.936,3.449)--(6.706,3.311)%
  --(7.475,3.152)--(8.245,2.996)--(9.014,2.852);
\gpsetpointsize{4.00}
\gppoint{gp mark 8}{(3.628,2.405)}
\gppoint{gp mark 8}{(4.398,3.382)}
\gppoint{gp mark 8}{(5.167,3.534)}
\gppoint{gp mark 8}{(5.936,3.449)}
\gppoint{gp mark 8}{(6.706,3.311)}
\gppoint{gp mark 8}{(7.475,3.152)}
\gppoint{gp mark 8}{(8.245,2.996)}
\gppoint{gp mark 8}{(9.014,2.852)}
\gppoint{gp mark 8}{(4.170,4.302)}
\gpcolor{gp lt color border}
\node[gp node right] at (3.528,3.994) {Duty Cycle};
\gpcolor{gp lt color 1}
\gpsetlinetype{gp lt plot 1}
\draw[gp path] (3.712,3.994)--(4.628,3.994);
\draw[gp path] (3.628,1.746)--(4.398,2.721)--(5.167,3.325)--(5.936,3.729)--(6.706,4.015)%
  --(7.475,4.234)--(8.245,4.405)--(9.014,4.547);
\gppoint{gp mark 5}{(3.628,1.746)}
\gppoint{gp mark 5}{(4.398,2.721)}
\gppoint{gp mark 5}{(5.167,3.325)}
\gppoint{gp mark 5}{(5.936,3.729)}
\gppoint{gp mark 5}{(6.706,4.015)}
\gppoint{gp mark 5}{(7.475,4.234)}
\gppoint{gp mark 5}{(8.245,4.405)}
\gppoint{gp mark 5}{(9.014,4.547)}
\gppoint{gp mark 5}{(4.170,3.994)}
\gpcolor{gp lt color border}
\gpsetlinetype{gp lt border}
\gpsetlinewidth{1.00}
\draw[gp path] (1.320,4.790)--(1.320,0.985)--(9.014,0.985)--(9.014,4.790)--cycle;
%% coordinates of the plot area
\gpdefrectangularnode{gp plot 1}{\pgfpoint{1.320cm}{0.985cm}}{\pgfpoint{9.014cm}{4.790cm}}
\end{tikzpicture}
%% gnuplot variables

\end{figure}


%www.intersil.com/data/FN/FN2864.pdf
