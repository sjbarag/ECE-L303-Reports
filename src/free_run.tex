A free-running oscillator can be created by using the schematic shown in
Figure~\ref{fig:free_run}, where the switch is fixed closed.
%
\begin{figure}[H]
	\centering
	%\includegraphics[width=.6\textwidth]{img/shot/free_run.pdf}
	A free-running oscillator can be created by using the schematic shown in
Figure~\ref{fig:free_run}, where the switch is fixed closed.
%
\begin{figure}[H]
	\centering
	\includegraphics[width=.6\textwidth]{img/shot/free_run.pdf}
	\caption{}
	\label{fig:free_run}
\end{figure}
%
Using this circuit, the period of the output signal can be calculated by using~\eqref{eq:t0}:
%
\begin{equation}
	T_0 = \frac{3}{2} R_A C_T \left[ 1 + \frac{R_B}{2R_A - R_B} \right] \quad \text{,}
	\label{eq:t0}
\end{equation}
%
where only~$R_A$ and~$C_T$ are external to the provided circuit and available
to modify.  In the case that~$R_A = R_B$, the duty cycle of the square wave
output becomes exactly~\SI{50}{\percent}, and the frequency calculation
simiplifies to the~\eqref{eq:t0_even}.
%
\begin{equation}
	T_0 = 3 R_A C_T
	\label{eq:t0_even}
\end{equation}
%
Testing the possible output frequencies is possible by fixing~$C_T$ at the suggested value of~\SI{3300}{\pico\farad} and varying~$R_A$.

	\parbox{.6\textwidth}{
	\caption{Free-running oscilloscope circuit schematic, as provided in Figure~1 of the
	Intersil datasheet for an ICL8038 integrated circuit [1].}
	\label{fig:free_run}}
\end{figure}
%
Using this circuit, the period of the output signal can be calculated by using~\eqref{eq:t0}:
%
\begin{equation}
	T_0 = \frac{3}{2} R_A C_T \left[ 1 + \frac{R_B}{2R_A - R_B} \right] \quad \text{,}
	\label{eq:t0}
\end{equation}
%
where only~$R_A$ and~$C_T$ are external to the provided circuit and available
to modify.  In the case that~$R_A = R_B$, the duty cycle of the square wave
output becomes exactly~\SI{50}{\percent}, and the frequency calculation
simiplifies to the~\eqref{eq:t0_even}.
%
\begin{equation}
	T_0 = 3 R_A C_T
	\label{eq:t0_even}
\end{equation}
%
Testing the possible output frequencies is possible by fixing~$C_T$ at the
suggested value of~\SI{3300}{\pico\farad} and varying~$R_A$.  In order to view
all three cases for $R_A$~(where~$R_A < R_B$,~$R_A = R_B$, and~$R_A > R_B$), it
should be varied from~\SI{1}{\kilo\ohm} to~\SI{20}{\kilo\ohm} in steps
of~\SI{2}{\kilo\ohm} while measuring the output frequency and duty cycle at
each stage.  This allows several steps before and after the critical point
of~$R_A = R_B$, so that students may view the effects of a varying resistance.

Similarly, $R_A$ can be fixed to~\SI{10}{\kilo\ohm}~(the same as~$R_B$) while
varying~$C_T$.  This ensures a constant duty cycle of~\SI{50}{\percent} over
different values for the timing capacitor.  It is suggested that~$C_T$ is
varied from~\SI{100}{\pico\farad} to~\SI{100}{\micro\farad} in steps of an
order of magnitude, while using an oscilloscope to measure the output frequency
at each step.  By~\eqref{eq:t0_even}, the frequency will vary linearly with
changing capacitance.

\begin{figure}[H]
	\centering
	A free-running oscillator can be created by using the schematic shown in
Figure~\ref{fig:free_run}, where the switch is fixed closed.
%
\begin{figure}[H]
	\centering
	\includegraphics[width=.6\textwidth]{img/shot/free_run.pdf}
	\caption{}
	\label{fig:free_run}
\end{figure}
%
Using this circuit, the period of the output signal can be calculated by using~\eqref{eq:t0}:
%
\begin{equation}
	T_0 = \frac{3}{2} R_A C_T \left[ 1 + \frac{R_B}{2R_A - R_B} \right] \quad \text{,}
	\label{eq:t0}
\end{equation}
%
where only~$R_A$ and~$C_T$ are external to the provided circuit and available
to modify.  In the case that~$R_A = R_B$, the duty cycle of the square wave
output becomes exactly~\SI{50}{\percent}, and the frequency calculation
simiplifies to the~\eqref{eq:t0_even}.
%
\begin{equation}
	T_0 = 3 R_A C_T
	\label{eq:t0_even}
\end{equation}
%
Testing the possible output frequencies is possible by fixing~$C_T$ at the suggested value of~\SI{3300}{\pico\farad} and varying~$R_A$.

\end{figure}


%www.intersil.com/data/FN/FN2864.pdf
