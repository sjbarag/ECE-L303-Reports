Finally, the voltage-controlled oscillator was fed a frequency modulated
sinusoid.  The results of the DC sweep should imply the expected result for
this test, in which the duty cycle of the output oscillates.
%
\begin{figure}[H]
	\centering
	\begin{circuitikz}
	\draw[ultra thick] (-2, -1.5) rectangle (2, 1.5);
	\draw[/tikz/circuitikz/bipoles/length=1cm]
	(-1.5, 1.5) node[below] {4} to [R, l=\SI{10}{\kilo\ohm}] ++(0, 1.5)
		to [short, -o] (4, 3) node[right] {$V+$}
	(0, 1.5) node[below] {5} to [R, l=\SI{+10}{\kilo\ohm}] ++(0, 1.5)
	(1.5, 1.5) node[below] {6} to [short] ++(0, 1.5)
	(3, 3) to [R, l_=\SI{10}{\kilo\ohm}] ++(0, -1.5) to [short] ++(0, -.5)

	(1.5, -1.5) node[above] {12} to [R, l=\SI{82}{\kilo\ohm}] ++(0, -1.5)
	(0, -1.5) node[above] {11} to [short] ++(0, -1.5)
	(-1.5, -1.5) node[above] {10} to [C, l=\SI{3.3}{\micro\farad}] ++(0, -1.5)
		to [short, -o] (4, -3) node[right] {$V-$};



%	(-2, 0) node[right] {8} to [short] ++(-.5, 0)
%		to [short] ++(0, 1) to [short] ++(.5, 0) node[right] {7};

	% short stuff on the right side
	\draw[/tikz/circuitikz/bipoles/length=.75cm]
	(2, 1) node[left] {9} to [short, -o] ++(2, 0)
	(2, .5) node[left] {3} to [short, -o] ++(2, 0)
		(3, .5) to [R, l=$R_\text{TRI}$] ++(0, -1) node[ground] {}
	(2, -1) node[left] {2} to [short, -o] ++(2, 0)
		(3, -1) to [R, l=$R_\text{SINE}$] ++(0, -1) node[ground] {}

	% short stuff on the left side
	(-2, 1) node[right] {7} to [short] ++(-0.5, 0)
		to [R, l_=\SI{10}{\kilo\ohm}] ++(0, -1) to [C, l_=\SI{10}{\micro\farad}, -o] ++(0, -.75) node[below] {FM}
	(-2, 0) node[right] {8} to [short] ++(-.5, 0);

	% square wave
	\draw[thick]
	(4.2, .85) -- ++(.1, 0) -- ++(0, .3)
		-- ++(.2, 0) -- ++(0, -.3) -- ++(.2, 0) -- ++(0, .3)
		-- ++(.2, 0) -- ++(0, -.3) -- ++(.1, 0);

	% triangle
	\draw[thick]
	(4.2, .5) -- ++(.1, .15) -- ++(.2, -.3) -- ++(.2, .3) -- ++(.2, -.3) --  ++(.1, .15);

	% sinusoid
	\draw[thick]
	(4.2, -1) sin ++(.1, .15) cos ++(.1, -.15) sin ++(.1, -.15) cos ++(.1, .15)
		sin ++(.1, .15) cos ++(.1, -.15) sin ++(.1, -.15) cos ++(.1, .15);
\end{circuitikz}

	\parbox{.6\textwidth}{
	\caption{Schematic for a frequency-modulated voltage-controlled oscillator
	as provided in the Intersil datasheet for an ICL8038 IC [1].  All
	previously-used components are unchanged, and the new resistor and
	capacitor are~\SI{10}{\kilo\ohm} and~\SI{10}{\micro\farad}, respectively.}
	\label{fig:freq_mod}}
\end{figure}
%
For this experiment all previously-used components are unchanged, and the new
resistor and capacitor are~\SI{10}{\kilo\ohm} and~\SI{10}{\micro\farad},
respectively.  In this case, the amplitude of the input signal determines the
frequency of the output, which will vary from~\SI{1}{\volt} to~\SI{4}{\volt} in
one-volt increments.  For each of these amplitudes, the input frequency will
vary from~\SI{1}{\hertz} to~\SI{4}{\hertz}.  This allows students to view the
effects of a varying amplitude on the output frequency, while observing the
effects of a varying input frequency on the rate of change of the output's duty
cycle.

\begin{figure}[H]
	\centering
	\begin{circuitikz}
	\draw[ultra thick] (-2, -1.5) rectangle (2, 1.5);
	\draw[/tikz/circuitikz/bipoles/length=1cm]
	(-1.5, 1.5) node[below] {4} to [R, l=\SI{10}{\kilo\ohm}] ++(0, 1.5)
		to [short, -o] (4, 3) node[right] {$V+$}
	(0, 1.5) node[below] {5} to [R, l=\SI{+10}{\kilo\ohm}] ++(0, 1.5)
	(1.5, 1.5) node[below] {6} to [short] ++(0, 1.5)
	(3, 3) to [R, l_=\SI{10}{\kilo\ohm}] ++(0, -1.5) to [short] ++(0, -.5)

	(1.5, -1.5) node[above] {12} to [R, l=\SI{82}{\kilo\ohm}] ++(0, -1.5)
	(0, -1.5) node[above] {11} to [short] ++(0, -1.5)
	(-1.5, -1.5) node[above] {10} to [C, l=\SI{3.3}{\micro\farad}] ++(0, -1.5)
		to [short, -o] (4, -3) node[right] {$V-$};



%	(-2, 0) node[right] {8} to [short] ++(-.5, 0)
%		to [short] ++(0, 1) to [short] ++(.5, 0) node[right] {7};

	% short stuff on the right side
	\draw[/tikz/circuitikz/bipoles/length=.75cm]
	(2, 1) node[left] {9} to [short, -o] ++(2, 0)
	(2, .5) node[left] {3} to [short, -o] ++(2, 0)
		(3, .5) to [R, l=$R_\text{TRI}$] ++(0, -1) node[ground] {}
	(2, -1) node[left] {2} to [short, -o] ++(2, 0)
		(3, -1) to [R, l=$R_\text{SINE}$] ++(0, -1) node[ground] {}

	% short stuff on the left side
	(-2, 1) node[right] {7} to [short] ++(-0.5, 0)
		to [R, l_=\SI{10}{\kilo\ohm}] ++(0, -1) to [C, l_=\SI{10}{\micro\farad}, -o] ++(0, -.75) node[below] {FM}
	(-2, 0) node[right] {8} to [short] ++(-.5, 0);

	% square wave
	\draw[thick]
	(4.2, .85) -- ++(.1, 0) -- ++(0, .3)
		-- ++(.2, 0) -- ++(0, -.3) -- ++(.2, 0) -- ++(0, .3)
		-- ++(.2, 0) -- ++(0, -.3) -- ++(.1, 0);

	% triangle
	\draw[thick]
	(4.2, .5) -- ++(.1, .15) -- ++(.2, -.3) -- ++(.2, .3) -- ++(.2, -.3) --  ++(.1, .15);

	% sinusoid
	\draw[thick]
	(4.2, -1) sin ++(.1, .15) cos ++(.1, -.15) sin ++(.1, -.15) cos ++(.1, .15)
		sin ++(.1, .15) cos ++(.1, -.15) sin ++(.1, -.15) cos ++(.1, .15);
\end{circuitikz}

\end{figure}
