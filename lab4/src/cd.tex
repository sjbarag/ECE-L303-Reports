The generation of a sine wave signal is based on the properties of the feedback
loop generally described in Figure~\ref{fig:blockDiag}.  When the forward and
feedback circuits are balanced properly, an evenly formed sine wave will be
produced.  However, when they are out of balance, the two stages will not
oscillate at an even rate and will create a malformed signal.
%
\begin{figure}[H]
	\centering
	\begin{tikzpicture}[node distance=2cm, auto,>=latex']
	\tikzstyle{function}=[circle, draw, node distance=1cm]
	\tikzstyle{stage}=[rectangle, minimum width=5em, minimum height=3em,
		text centered, draw]
	\tikzstyle{line}=[draw, thin, -latex']

	% nodes
	\node (in) {$V_1$};
	\node [function, right of=in] (sum) {+};
	\node [stage, right of=sum] (As) {$A(s)$};
	\node [stage, below of=As]  (Bs) {$B(s)$};
	\node [right of=As] (out) {$V_2$};

	% paths
	\path [line] (in) -- (sum);
	\path [line] (sum) -- node {$V_e$} (As);
	\path [line] (As) node[above=14pt] {Forward Circuit} -- node [name=junc] {} (out);
	\path [line] (junc) |- (Bs);
	\path [line] (Bs) node[below=14pt] {Feedback Circuit} -| node[near end] {$V_f$} (sum);
\end{tikzpicture}
\\
	\parbox{.6\textwidth}{
	\caption[Feedback Block Diagram]{Block diagram of the feedback system being
		used to create a sine wave generator.  Signal~$A(s)$ will later be
		represeted by the opamp and resistors~$R_3$ and~$R_4$;~$B(s)$ will be
		provided by the two resistor-capacitor combinations (one series and one
		parallel) forming the positive feedback loop.}
	\label{fig:blockDiag}}
\end{figure}
%
To build the actual oscillator, signal~$V_1$ was replaced by a constant voltage
source.  Blocks~$A(s)$ and~$B(s)$ were created by separate parts of the opamp
circuit shown in Figure~\ref{fig:weinBridgeSchem}.
%
\begin{figure}[H]
	\centering
	\begin{circuitikz}
	\draw (0,0) node[op amp] (opamp) {}
	% supplies
	(opamp.up) ++(0, .25) node[above] {\SI{25}{\volt}} -- (opamp.up)
	(opamp.down) node[ground] {}

	% input
	(opamp.-) to [R, l_=$R_4$] ++(-2, 0)
	to [short] ++(0, -1) node[ground] {}

	% negative feedback
	(opamp.-) to [short] ++(0, 1.5) coordinate (topleft)
	(opamp.out) to [short] ++($(0, 14pt) + (0, 1.5)$)
	to [R, l_=$R_3$] (topleft)

	% positive feedback
	(opamp.out) to [R, l^=$R$] ++(0, -1.5) coordinate (r-c)
	to [C, l^=$C$] ++(0, -1) coordinate (bottomright)
	(opamp.+) node[left] {$V_f$}
	(opamp.+) to [short, -*] ++($(0, 14pt) + (0, -2.5)$) coordinate (zp)
	node[above left] {$Z_p$}
	to [short] (bottomright)
	(r-c) node[right] {$Z_s$}

	% -> R || C -> gnd
	(zp) to [short] ++(-1, 0)
	to [R, l_=$R$] ++(0, -1.5)
	to [short] ++(1, 0) node[ground] {}
	(zp) to [C, l^=$C$] ++(0, -1.5)

	% output
	(opamp.out) to [short, *-o] ++(.5, 0) node[above] {$V_2$};
\end{circuitikz}
\\
	\parbox{.6\textwidth}{
	\caption[Wein Bridge Feedback Oscillator]{Circuit schematic for the Wein
		Bridge feedback oscillator, as provided in the lab four instructions.}
	\label{fig:weinBridgeSchem}}
\end{figure}
%
The negative feedback loop formed by resistors~$R_3$ and~$R_4$ represents
the~``forward circuit''~$A(s)$ in the block diagram.  Similarly, the positive
feedback loop comprised of the two~RC combinations is represented by
the~``feedback circuit''~$B(s)$ in the block diagram.  Because of the
properties of an operational amplifier, we show the device as the summing
function in Figure~\ref{fig:blockDiag}.

In order to properly balance such a circuit, the lab instructions provide
that~$K$, the forward voltage gain of~$A(s)$, must be equal to three.
Because~$K$ is also equal to one plus the ratio of~$R_3$ to~$R_4$, the
resulting ratio of~$R_3$ to~$R_4$ is defined by
%
\begin{align}
	K = 3 &= 1 + \frac{R_3}{R_4} \nonumber
\intertext{or }
	R_3 &= 2 R_4 \quad \text{.} \label{eq:r3-r4}
\intertext{It is is also provided that resistors~$R_1$,~$R_2$, and~$R_4$ will
all have a common resistance~$R$, so that}
	R_1 = R_2 &= R_4 = R \quad \text{,} \nonumber
\intertext{thus}
	R_3 &= 2 R	\label{eq:r3-r} \quad \text{.}
\end{align}
%
The resistance of~$R$ was given as~\SI{10}{\kilo\ohm} in the lab instructions,
with the specific tolerance of~1\%, and~$R_3$ was replaced with
a~\SI{25}{\kilo\ohm} potentiometer, allowing for a precise balance of
resistance.  These details are significant because any difference in resistance
will decrease the symmetry of oscillation, resulting in a malformed --- or even
non-existent --- output sine waveform.

The frequency of oscillation was provided to students as being calculated by
%
\begin{equation}
	f_0 = \frac{1}{2 \pi R C} (\si{\hertz}) \quad \text{,}
	\label{eq:frequency}
\end{equation}
%
where~$R$ is as described above and~$C$ represents the value of either of the
two capacitors included in Figure~\ref{fig:weinBridgeSchem}.  As the design
goal for this experiment was~$f_0 =
$\SI{3.3}{\kilo\hertz},~\eqref{eq:frequency} can be used to determine the
correct capacitance required:
%
\begin{align*}
	C &= \frac{1}{2 \pi R f_0} \\
	  &= \frac{1}{2 \pi \left(10\times10^3\right) \left(3.3\times10^3\right)} \\
	  &= \SI{4.822}{\nano\farad} \quad \text{.}
\end{align*}
%
Students were tasked with finding a pair of capacitors that would match this
value as close as possible, using parallel combinations of different sizes if
necessary.
