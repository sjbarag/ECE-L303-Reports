In conclusion, a functioning --- albeit slightly limited --- sine wave
generator was successfully constructed.  While it can output a wide range of
amplitudes while changing the frequency by
just~\SI{0.027}{\kilo\hertz\per\volt}, the frequency of the signal is very much
fixed limited by the constant-value elements that were used.  Despite changing
the load resistance, no significant variance in output frequency was achieved,
though this did serve to modulate the output amplitude.  It should be noted
that the circuit is not safe for loads less than~\SI{1}{\kilo\ohm}, as the
feedback voltage is not large enough to maintain a stable comparison by the
opamp.

In order to vary the frequency, either all of the capcitors or all of the
fixed-value resistors in this circuit would need to be adjusted simultaneously.
However, the constructed system produced a sinusoidal signal with a frequency
just~\SI{2.87}{\percent} below the designed value at~\SI{3.205}{\kilo\hertz},
likely as a result of the use of a non-ideal operational amplifier and the
small capacitance that exists in the coaxial oscilloscope cables, as well as
the minor difference between ideally-matched element values.
