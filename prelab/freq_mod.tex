Finally, the voltage-controlled oscillator was fed a frequency modulated
sinusoid.  The results of the DC sweep should imply the expected result for
this test, in which the duty cycle of the output oscillates.
%
\begin{figure}[H]
	\centering
	\includegraphics[width=.6\textwidth]{img/shot/freq_mod.pdf}
	\parbox{.6\textwidth}{
	\caption{Provided schematic for a frequency-modulated voltage-controlled
	oscillator.  All previously-used components are unchanged, and the new
	resistor and capacitor are~\SI{10}{\kilo\ohm} and~\SI{10}{\micro\farad},
	respectively.}
	\label{fig:freq_mod}}
\end{figure}
%
For this experiment all previously-used components are unchanged, and the new
resistor and capacitor are~\SI{10}{\kilo\ohm} and~\SI{10}{\micro\farad},
respectively.  In this case, the amplitude of the input signal determines the
frequency of the output, which will vary from~\SI{1}{\volt} to~\SI{4}{\volt} in
one-volt increments.  For each of these amplitudes, the input frequency will
vary from~\SI{1}{\hertz} to~\SI{4}{\hertz}.  This allows students to view the
effects of a varying amplitude on the output frequency, while observing the
effects of a varying input frequency on the rate of change of the output's duty
cycle.
